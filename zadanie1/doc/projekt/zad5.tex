\section{Dobór parametrów regulatorów PID i DMC}
\subsection{Regulator PID}

Do doboru nastaw regulatora PID zastosowano metodę inżynierską, 
która polega na przeprowadzeniu doświadczeń i analizy uzyskanych wyników. 
Na podstawie wyciągniętych wniosków modyfikowane są nastawy regulatora. 
Parametry  dobierane są metodą prób i błędów, aż do osiągnięcia oczekiwanych wyników. 
Jako pierwszy dobierany był parametr wzmocnienia członu proporcjonalnego, 
poprzez obserwację zachowania się uchybu regulacji w stanie ustalonym oraz przeregulowanie. 
Zmniejszając stopniowo wzmocnienie zmniejszano przeregulowanie, a uchyb zwiększał się. 
Ostatecznie dobrano wartość parametru ??. 
Parametr  zwiększając go co likwidowało uchyb regulacji,  
zbyt duży czas zdwojenia zwiększał czas regulacji. 
Ostatnim elementem strojenia jest wyznaczenie parametru czasu wyprzedzenia , 
również metodą prób i błędów, tak by zminimalizować czas regulacji. 
Po wykonaniu tej czynności kończy się proces strojenia regulatora.

\subsection{Regulator DMC}

Do doboru nastaw regulatora DMC zastosowano metodę inżynierską, 
która polega na przeprowadzeniu doświadczeń i analizy uzyskanych wyników. 
Na podstawie wyciągniętych wniosków modyfikowane są nastawy regulatora. 
Parametry  dobierane są metodą prób i błędów, aż do osiągnięcia oczekiwanych wyników.
