\section{Symulacyjne wyznaczenie odpowiedzi skokowej procesu}

Odpowiedzi skokowych procesu zostały wyznaczone symulacyjnie dla pięciu zmian sygnału sterującego. 
Uwzględnione zostały ograniczenia wartości tego sygnału Umin=\num{0.1}, Umax=\num{1.5}. 
Początkowa wartość sterowania wynosiła Upp=\num{0.8}.

\begin{figure}[H]
    \centering
    \includegraphics[scale=0.8]{../projekt/zad2/odp_skok.pdf}
    \caption{ odpSkok }
\end{figure}

Dzięki uzyskanym odpowiedziom skokowym otrzymano charakterystykę statyczną y(u)
\begin{figure}[H]
    \centering
    \includegraphics[scale=0.8]{../projekt/zad2/char_stat.pdf}
    \caption{char stat}
\end{figure}

Wniosek: 
Na podstawie wykresu charakterystyki statycznej można ustalić, 
że właściwości statyczne procesu są liniowe. 
Wzmocnienie statyczne procesu określone zostało dzięki obliczeniu współczynnika nachylenia 
wykresu charakterystyki statycznej, wynosi on K=\num{1.0305}.
