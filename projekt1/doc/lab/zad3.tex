\section{Przekształcenie otrzymanej odpowiedzi skokowej}
Przekształcić jedną z otrzymanych odpowiedzi w taki sposób, aby otrzymać odpowiedź
skokową wykorzystywaną w algorytmie DMC, tzn. zestaw liczb s1, s2, . . . (przy skoku
jednostkowym sygnału sterującego: od chwili k = 0 włącznie sygnał sterujący ma
wartość 1, w przeszłości jest zerowy). Zamieścić rysunek odpowiedzi skokowej. Należy
wykonać aproksymację odpowiedzi skokowej używając w tym celu członu inercyjnego
drugiego rzędu z opóźnieniem (szczegóły w opisie znajdującym się na stronie przedmiotu). W celu doboru parametrów modelu wykorzystać optymalizację. Zamieścić rysunek
porównujący odpowiedź skokową oryginalną i wersję aproksymowaną. Uzasadnić wybór parametrów optymalizacji.