\section{Testy klasycznych regulatorów PID i DMC}
\label{lab:zad3}

Dla trajektorii zmian sygnałów zadanych: Tpp, Tpp + 5, Tpp + 15, Tpp przetestowac
regulatory z laboratorium 1 (tj. wykorzystywane dla obiektu liniowego). Omówic wyniki.
Jakosc regulacji ocenic jakosciowo (na podstawie rysunków przebiegów sygnałów)
oraz ilosciowo, wyznaczajac wskaznik jakosci regulacji. Zamiescic wyniki pomiarów
(przebiegi sygnałów wejsciowych i wyjsciowych procesu oraz wartosci wskaznika E).

%\begin{figure}[H] 
%    \centering
%    \input{projekt/figure/zad1charstat_u_y_z.tex}
%    \caption{Punkt pracy obiektu symulacji}
%    \label{projekt:zad1:figure:charstat_u_y_z}
%\end{figure}


\subsection{Klasyczny algorytm PID}
\label{lab:zad3:PID}


\newpage

\subsection{Klasyczny algorytm DMC}
\label{lab:zad3:DMC}



\newpage
