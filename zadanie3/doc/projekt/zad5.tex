\section{Implementacja rozmytych algorytmów PID i DMC}
\label{projekt:zad5}

Poprzez zastosowanie wielu lokalnych regulatorów dla poszczególnych punktów
pracy oraz funkcji przynależności zaimplementowano rozmyte algorytmy PID oraz DMC.

\label{projekt:zad5:fuzzyFunctions}
\subsection{Funkcje przynależności}

Wybór regulatora dokonywany jest dzięki funkcji przynależności. 
Jako funkcja przynależności posłużyła nam funkcja trapezoidalna. 
Funkcja ta zawiera współczynniki, które
definiują jak duży wpływ na dane wyjście ma dany regulator w badanym
obszarze. Dobór tych współczynników został przeprowadzony poprzez
obserwacje zachowania obiektu. Dla każdego z lokalnych regulatorów PID oraz
DMC należało dobrać parametry osobno w jego punkcie pracy, w tym celu zostały
przeprowadzone eksperymenty, parametry dobrane zostały metodą inżynierską.
Ostatecznie regulatory zostały połączone w funkcji. Dzięki wykorzystaniu
rozmytego algorytmu PID oraz DMC możemy zniwelować problem nieliniowości
obiektu. Rozmywanie dokonywane jest na podstawie charakterystyki statycznej,
dobór parametrów na podstawie aktualnej wartości sygnału wyjściowego y.
Kształt trapezoidalny funkcji przynależności został wybrany na podstawie
eksperymentów, dla trapezoidalnej funkcji wagi zmieniają się proporcjonalnie.
Kształt trapezu pozwala na utrzymanie wartości 1 co w przypadku trójkątnej nie
jest możliwe, natomiast na ramionach trapezu można dobrze uzupełniać funkcje
przynależności równie dobrze jak w trójkątnej.

\newpage
\subsection{Wykresy funkcji przynależności}

\begin{figure}[H] 
    \centering
    \input{projekt/figure/Wykresy_Przynal/przynalN2.tex}
    \caption{Funkcje rozmycia dla 2 regulatorów lokalnych}
    \label{lab:zad4:fuzzyFunction:2:figure}
\end{figure}

\begin{figure}[H] 
    \centering
    \input{projekt/figure/Wykresy_Przynal/przynalN3.tex}
    \caption{Funkcje rozmycia dla 3 regulatorów lokalnych}
    \label{lab:zad4:fuzzyFunction:3:figure}
\end{figure}

\begin{figure}[H] 
    \centering
    \input{projekt/figure/Wykresy_Przynal/przynalN4.tex}
    \caption{Funkcje rozmycia dla 4 regulatorów lokalnych}
    \label{lab:zad4:fuzzyFunction:4:figure}
\end{figure}

\begin{figure}[H] 
    \centering
    \input{projekt/figure/Wykresy_Przynal/przynalN5.tex}
    \caption{Funkcje rozmycia dla 5 regulatorów lokalnych}
    \label{lab:zad4:fuzzyFunction:5:figure}
\end{figure}

\newpage

\subsection{Rozmyty algorytm PID}
\label{projekt:zad5:PID}

\lstinputlisting[
    firstline=71, 
    lastline=87, 
    caption="Wyznaczanie nastaw rozmytych regulatorów PID"]
    {../lab/main.m}
%\newpage

\lstinputlisting[
    firstline=198, 
    lastline=229, 
    caption="Wyznaczenie wartości sterowania u"]
    {../lab/main.m}
%\newpage


\newpage

\subsection{Rozmyty algorytm DMC}
\label{projekt:zad5:DMC}


\lstinputlisting[
    firstline=88, 
    lastline=114, 
    caption="Wyznaczanie nastaw rozmytych regulatorów DMC"]
    {../lab/main.m}
%\newpage

\lstinputlisting[
    firstline=230, 
    lastline=260, 
    caption="Wyznaczenie wartości sterowania u"]
    {../lab/main.m}
%\newpage

\newpage
