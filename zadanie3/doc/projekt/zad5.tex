\section{Implementacja rozmytych algorytmów PID i DMC}
\label{projekt:zad5}


W tym samym programie zaimplementowac i omówic rozmyty algorytm PID i rozmyty
algorytm DMC w najprostszej wersji analitycznej. Uzasadnic wybór zmiennej,
na podstawie której dokonywane jest rozmywanie. Uzasadnic wybór i kształt funkcji
przynaleznosci.

%\begin{figure}[H] 
%    \centering
%    \input{projekt/figure/zad1charstat_u_y_z.tex}
%    \caption{Punkt pracy obiektu symulacji}
%    \label{projekt:zad1:figure:charstat_u_y_z}
%\end{figure}


\subsection{Funkcje przynależności}
\label{projekt:zad5:fuzzyFunctions}

\begin{figure}[H] 
    \centering
    \input{projekt/figure/Wykresy_Przynal/przynalN2.tex}
    \caption{Funkcje rozmycia dla 2 regulatorów lokalnych}
    \label{lab:zad4:fuzzyFunction:2:figure}
\end{figure}

\begin{figure}[H] 
    \centering
    \input{projekt/figure/Wykresy_Przynal/przynalN3.tex}
    \caption{Funkcje rozmycia dla 3 regulatorów lokalnych}
    \label{lab:zad4:fuzzyFunction:3:figure}
\end{figure}

\begin{figure}[H] 
    \centering
    \input{projekt/figure/Wykresy_Przynal/przynalN4.tex}
    \caption{Funkcje rozmycia dla 4 regulatorów lokalnych}
    \label{lab:zad4:fuzzyFunction:4:figure}
\end{figure}

\begin{figure}[H] 
    \centering
    \input{projekt/figure/Wykresy_Przynal/przynalN5.tex}
    \caption{Funkcje rozmycia dla 5 regulatorów lokalnych}
    \label{lab:zad4:fuzzyFunction:5:figure}
\end{figure}

\newpage

\subsection{Rozmyty algorytm PID}
\label{projekt:zad5:PID}



\newpage

\subsection{Rozmyty algorytm DMC}
\label{projekt:zad5:DMC}



\newpage
