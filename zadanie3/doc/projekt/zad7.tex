\section{Dobór parametrów lambda lokalnych regulatorów DMC}
\label{projekt:zad7}

\subsection{Rozmyty regulator DMC}
\label{projekt:zad7:PID}

\begin{figure}[H] 
   \centering
   % This file was created by matlab2tikz.
%
\definecolor{mycolor1}{rgb}{0.00000,0.44700,0.74100}%
\definecolor{mycolor2}{rgb}{0.85000,0.32500,0.09800}%
%
\begin{tikzpicture}

\begin{axis}[%
width=4.521in,
height=1.493in,
at={(0.758in,2.554in)},
scale only axis,
xmin=1,
xmax=1800,
xlabel style={font=\color{white!15!black}},
xlabel={k},
ymin=-4.551,
ymax=0.1,
ylabel style={font=\color{white!15!black}},
ylabel={y},
axis background/.style={fill=white},
title style={font=\bfseries, align=center},
title={E=222.9899\\[1ex]N= [70         20]\\[1ex]$\text{N}_\text{u}\text{= [30          5]}$\\[1ex]lambda= [25         10]},
xmajorgrids,
ymajorgrids,
legend style={legend cell align=left, align=left, draw=white!15!black}
]
\addplot [color=mycolor1]
  table[row sep=crcr]{%
1	0\\
2	0\\
3	0\\
4	0\\
5	0\\
6	0\\
7	0\\
8	0\\
9	0\\
10	0\\
11	0\\
12	0\\
13	0\\
14	0\\
15	0\\
16	0\\
17	0\\
18	0\\
19	0\\
20	0\\
21	0\\
22	0\\
23	0\\
24	0\\
25	-0.031566\\
26	-0.14924\\
27	-0.3872\\
28	-0.7244\\
29	-1.1124\\
30	-1.5162\\
31	-1.9104\\
32	-2.2732\\
33	-2.5865\\
34	-2.8359\\
35	-3.0115\\
36	-3.1088\\
37	-3.1287\\
38	-3.0778\\
39	-2.9677\\
40	-2.8142\\
41	-2.635\\
42	-2.4482\\
43	-2.27\\
44	-2.1129\\
45	-1.9856\\
46	-1.893\\
47	-1.8359\\
48	-1.8125\\
49	-1.8186\\
50	-1.8483\\
51	-1.8948\\
52	-1.951\\
53	-2.01\\
54	-2.0659\\
55	-2.1137\\
56	-2.1499\\
57	-2.1726\\
58	-2.1812\\
59	-2.1765\\
60	-2.1605\\
61	-2.1359\\
62	-2.1056\\
63	-2.0729\\
64	-2.0408\\
65	-2.0116\\
66	-1.9873\\
67	-1.969\\
68	-1.9573\\
69	-1.9521\\
70	-1.9527\\
71	-1.9581\\
72	-1.967\\
73	-1.9781\\
74	-1.9901\\
75	-2.0016\\
76	-2.0118\\
77	-2.02\\
78	-2.0256\\
79	-2.0285\\
80	-2.0288\\
81	-2.0268\\
82	-2.0231\\
83	-2.0181\\
84	-2.0125\\
85	-2.0068\\
86	-2.0015\\
87	-1.9969\\
88	-1.9934\\
89	-1.9911\\
90	-1.9899\\
91	-1.9899\\
92	-1.9908\\
93	-1.9924\\
94	-1.9945\\
95	-1.9968\\
96	-1.999\\
97	-2.0011\\
98	-2.0028\\
99	-2.004\\
100	-2.0047\\
101	-2.005\\
102	-2.0048\\
103	-2.0042\\
104	-2.0034\\
105	-2.0024\\
106	-2.0014\\
107	-2.0004\\
108	-1.9995\\
109	-1.9988\\
110	-1.9983\\
111	-1.9981\\
112	-1.998\\
113	-1.9981\\
114	-1.9984\\
115	-1.9988\\
116	-1.9992\\
117	-1.9997\\
118	-2.0001\\
119	-2.0004\\
120	-2.0007\\
121	-2.0008\\
122	-2.0009\\
123	-2.0009\\
124	-2.0008\\
125	-2.0007\\
126	-2.0005\\
127	-2.0003\\
128	-2.0001\\
129	-1.9999\\
130	-1.9998\\
131	-1.9997\\
132	-1.9996\\
133	-1.9996\\
134	-1.9996\\
135	-1.9997\\
136	-1.9998\\
137	-1.9998\\
138	-1.9999\\
139	-2\\
140	-2.0001\\
141	-2.0001\\
142	-2.0001\\
143	-2.0002\\
144	-2.0002\\
145	-2.0002\\
146	-2.0001\\
147	-2.0001\\
148	-2.0001\\
149	-2\\
150	-2\\
151	-2\\
152	-1.9999\\
153	-1.9999\\
154	-1.9999\\
155	-1.9999\\
156	-1.9999\\
157	-2\\
158	-2\\
159	-2\\
160	-2\\
161	-2\\
162	-2\\
163	-2\\
164	-2\\
165	-2\\
166	-2\\
167	-2\\
168	-2\\
169	-2\\
170	-2\\
171	-2\\
172	-2\\
173	-2\\
174	-2\\
175	-2\\
176	-2\\
177	-2\\
178	-2\\
179	-2\\
180	-2\\
181	-2\\
182	-2\\
183	-2\\
184	-2\\
185	-2\\
186	-2\\
187	-2\\
188	-2\\
189	-2\\
190	-2\\
191	-2\\
192	-2\\
193	-2\\
194	-2\\
195	-2\\
196	-2\\
197	-2\\
198	-2\\
199	-2\\
200	-2\\
201	-2\\
202	-2\\
203	-2\\
204	-2\\
205	-2.0801\\
206	-2.2463\\
207	-2.4603\\
208	-2.6956\\
209	-2.9345\\
210	-3.163\\
211	-3.3695\\
212	-3.5435\\
213	-3.6786\\
214	-3.7739\\
215	-3.8329\\
216	-3.8624\\
217	-3.8713\\
218	-3.8686\\
219	-3.8626\\
220	-3.8604\\
221	-3.8671\\
222	-3.8861\\
223	-3.9187\\
224	-3.9645\\
225	-4.022\\
226	-4.0885\\
227	-4.1599\\
228	-4.2317\\
229	-4.3006\\
230	-4.3642\\
231	-4.4202\\
232	-4.4669\\
233	-4.5033\\
234	-4.529\\
235	-4.5446\\
236	-4.551\\
237	-4.5495\\
238	-4.5419\\
239	-4.5301\\
240	-4.5157\\
241	-4.5005\\
242	-4.4859\\
243	-4.4731\\
244	-4.4629\\
245	-4.4559\\
246	-4.4522\\
247	-4.4518\\
248	-4.4543\\
249	-4.4592\\
250	-4.466\\
251	-4.474\\
252	-4.4825\\
253	-4.4908\\
254	-4.4986\\
255	-4.5053\\
256	-4.5106\\
257	-4.5145\\
258	-4.5167\\
259	-4.5175\\
260	-4.517\\
261	-4.5153\\
262	-4.5128\\
263	-4.5097\\
264	-4.5064\\
265	-4.5031\\
266	-4.5\\
267	-4.4973\\
268	-4.4952\\
269	-4.4937\\
270	-4.4928\\
271	-4.4926\\
272	-4.4929\\
273	-4.4936\\
274	-4.4947\\
275	-4.496\\
276	-4.4974\\
277	-4.4987\\
278	-4.5\\
279	-4.5011\\
280	-4.502\\
281	-4.5026\\
282	-4.503\\
283	-4.5031\\
284	-4.503\\
285	-4.5027\\
286	-4.5022\\
287	-4.5017\\
288	-4.5011\\
289	-4.5005\\
290	-4.5\\
291	-4.4995\\
292	-4.4992\\
293	-4.4989\\
294	-4.4988\\
295	-4.4987\\
296	-4.4988\\
297	-4.4989\\
298	-4.4991\\
299	-4.4993\\
300	-4.4996\\
301	-4.4998\\
302	-4.5\\
303	-4.5002\\
304	-4.5004\\
305	-4.5005\\
306	-4.5005\\
307	-4.5005\\
308	-4.5005\\
309	-4.5005\\
310	-4.5004\\
311	-4.5003\\
312	-4.5002\\
313	-4.5001\\
314	-4.5\\
315	-4.4999\\
316	-4.4999\\
317	-4.4998\\
318	-4.4998\\
319	-4.4998\\
320	-4.4998\\
321	-4.4998\\
322	-4.4998\\
323	-4.4999\\
324	-4.4999\\
325	-4.5\\
326	-4.5\\
327	-4.5\\
328	-4.5001\\
329	-4.5001\\
330	-4.5001\\
331	-4.5001\\
332	-4.5001\\
333	-4.5001\\
334	-4.5001\\
335	-4.5\\
336	-4.5\\
337	-4.5\\
338	-4.5\\
339	-4.5\\
340	-4.5\\
341	-4.5\\
342	-4.5\\
343	-4.5\\
344	-4.5\\
345	-4.5\\
346	-4.5\\
347	-4.5\\
348	-4.5\\
349	-4.5\\
350	-4.5\\
351	-4.5\\
352	-4.5\\
353	-4.5\\
354	-4.5\\
355	-4.5\\
356	-4.5\\
357	-4.5\\
358	-4.5\\
359	-4.5\\
360	-4.5\\
361	-4.5\\
362	-4.5\\
363	-4.5\\
364	-4.5\\
365	-4.5\\
366	-4.5\\
367	-4.5\\
368	-4.5\\
369	-4.5\\
370	-4.5\\
371	-4.5\\
372	-4.5\\
373	-4.5\\
374	-4.5\\
375	-4.5\\
376	-4.5\\
377	-4.5\\
378	-4.5\\
379	-4.5\\
380	-4.5\\
381	-4.5\\
382	-4.5\\
383	-4.5\\
384	-4.5\\
385	-4.5\\
386	-4.5\\
387	-4.5\\
388	-4.5\\
389	-4.5\\
390	-4.5\\
391	-4.5\\
392	-4.5\\
393	-4.5\\
394	-4.5\\
395	-4.5\\
396	-4.5\\
397	-4.5\\
398	-4.5\\
399	-4.5\\
400	-4.5\\
401	-4.5\\
402	-4.5\\
403	-4.5\\
404	-4.5\\
405	-4.433\\
406	-4.2895\\
407	-4.0712\\
408	-3.8044\\
409	-3.5147\\
410	-3.2303\\
411	-2.9734\\
412	-2.7606\\
413	-2.6016\\
414	-2.5002\\
415	-2.4553\\
416	-2.4618\\
417	-2.5116\\
418	-2.5949\\
419	-2.7006\\
420	-2.8176\\
421	-2.9355\\
422	-3.0451\\
423	-3.1391\\
424	-3.2123\\
425	-3.2617\\
426	-3.2864\\
427	-3.2878\\
428	-3.2686\\
429	-3.233\\
430	-3.1858\\
431	-3.1323\\
432	-3.0773\\
433	-3.0252\\
434	-2.9796\\
435	-2.9429\\
436	-2.9167\\
437	-2.9013\\
438	-2.8963\\
439	-2.9006\\
440	-2.9125\\
441	-2.9299\\
442	-2.9507\\
443	-2.9727\\
444	-2.9941\\
445	-3.0133\\
446	-3.029\\
447	-3.0406\\
448	-3.0476\\
449	-3.0502\\
450	-3.0486\\
451	-3.0437\\
452	-3.0363\\
453	-3.0271\\
454	-3.0173\\
455	-3.0076\\
456	-2.9987\\
457	-2.9912\\
458	-2.9856\\
459	-2.9818\\
460	-2.9801\\
461	-2.9801\\
462	-2.9817\\
463	-2.9846\\
464	-2.9882\\
465	-2.9923\\
466	-2.9964\\
467	-3.0002\\
468	-3.0035\\
469	-3.0061\\
470	-3.0078\\
471	-3.0088\\
472	-3.0089\\
473	-3.0083\\
474	-3.0072\\
475	-3.0057\\
476	-3.0039\\
477	-3.0021\\
478	-3.0004\\
479	-2.9989\\
480	-2.9977\\
481	-2.9969\\
482	-2.9964\\
483	-2.9962\\
484	-2.9964\\
485	-2.9968\\
486	-2.9974\\
487	-2.9982\\
488	-2.9989\\
489	-2.9997\\
490	-3.0003\\
491	-3.0009\\
492	-3.0013\\
493	-3.0015\\
494	-3.0016\\
495	-3.0016\\
496	-3.0014\\
497	-3.0012\\
498	-3.0009\\
499	-3.0006\\
500	-3.0002\\
501	-2.9999\\
502	-2.9997\\
503	-2.9995\\
504	-2.9994\\
505	-2.9993\\
506	-2.9993\\
507	-2.9994\\
508	-2.9995\\
509	-2.9996\\
510	-2.9997\\
511	-2.9999\\
512	-3\\
513	-3.0001\\
514	-3.0002\\
515	-3.0003\\
516	-3.0003\\
517	-3.0003\\
518	-3.0003\\
519	-3.0002\\
520	-3.0002\\
521	-3.0001\\
522	-3.0001\\
523	-3\\
524	-3\\
525	-2.9999\\
526	-2.9999\\
527	-2.9999\\
528	-2.9999\\
529	-2.9999\\
530	-2.9999\\
531	-2.9999\\
532	-2.9999\\
533	-3\\
534	-3\\
535	-3\\
536	-3\\
537	-3\\
538	-3.0001\\
539	-3.0001\\
540	-3.0001\\
541	-3\\
542	-3\\
543	-3\\
544	-3\\
545	-3\\
546	-3\\
547	-3\\
548	-3\\
549	-3\\
550	-3\\
551	-3\\
552	-3\\
553	-3\\
554	-3\\
555	-3\\
556	-3\\
557	-3\\
558	-3\\
559	-3\\
560	-3\\
561	-3\\
562	-3\\
563	-3\\
564	-3\\
565	-3\\
566	-3\\
567	-3\\
568	-3\\
569	-3\\
570	-3\\
571	-3\\
572	-3\\
573	-3\\
574	-3\\
575	-3\\
576	-3\\
577	-3\\
578	-3\\
579	-3\\
580	-3\\
581	-3\\
582	-3\\
583	-3\\
584	-3\\
585	-3\\
586	-3\\
587	-3\\
588	-3\\
589	-3\\
590	-3\\
591	-3\\
592	-3\\
593	-3\\
594	-3\\
595	-3\\
596	-3\\
597	-3\\
598	-3\\
599	-3\\
600	-3\\
601	-3\\
602	-3\\
603	-3\\
604	-3\\
605	-2.9275\\
606	-2.7774\\
607	-2.5534\\
608	-2.2884\\
609	-2.0117\\
610	-1.7526\\
611	-1.5309\\
612	-1.3591\\
613	-1.2423\\
614	-1.1802\\
615	-1.168\\
616	-1.198\\
617	-1.2603\\
618	-1.3441\\
619	-1.4381\\
620	-1.5318\\
621	-1.6164\\
622	-1.6849\\
623	-1.7327\\
624	-1.7578\\
625	-1.7608\\
626	-1.744\\
627	-1.7116\\
628	-1.6686\\
629	-1.6203\\
630	-1.5717\\
631	-1.5271\\
632	-1.4897\\
633	-1.4615\\
634	-1.4435\\
635	-1.4354\\
636	-1.4363\\
637	-1.4445\\
638	-1.458\\
639	-1.4746\\
640	-1.4922\\
641	-1.5089\\
642	-1.5233\\
643	-1.5341\\
644	-1.541\\
645	-1.5438\\
646	-1.5427\\
647	-1.5385\\
648	-1.5319\\
649	-1.5238\\
650	-1.5152\\
651	-1.5069\\
652	-1.4996\\
653	-1.4938\\
654	-1.4897\\
655	-1.4874\\
656	-1.4869\\
657	-1.4878\\
658	-1.49\\
659	-1.4928\\
660	-1.496\\
661	-1.4993\\
662	-1.5022\\
663	-1.5045\\
664	-1.5062\\
665	-1.5071\\
666	-1.5073\\
667	-1.5069\\
668	-1.506\\
669	-1.5047\\
670	-1.5032\\
671	-1.5017\\
672	-1.5003\\
673	-1.4991\\
674	-1.4983\\
675	-1.4977\\
676	-1.4975\\
677	-1.4975\\
678	-1.4978\\
679	-1.4983\\
680	-1.4989\\
681	-1.4995\\
682	-1.5\\
683	-1.5005\\
684	-1.5009\\
685	-1.5011\\
686	-1.5013\\
687	-1.5012\\
688	-1.5011\\
689	-1.5009\\
690	-1.5007\\
691	-1.5004\\
692	-1.5001\\
693	-1.4999\\
694	-1.4997\\
695	-1.4996\\
696	-1.4995\\
697	-1.4995\\
698	-1.4996\\
699	-1.4996\\
700	-1.4997\\
701	-1.4998\\
702	-1.4999\\
703	-1.5\\
704	-1.5001\\
705	-1.5002\\
706	-1.5002\\
707	-1.5002\\
708	-1.5002\\
709	-1.5002\\
710	-1.5001\\
711	-1.5001\\
712	-1.5\\
713	-1.5\\
714	-1.5\\
715	-1.4999\\
716	-1.4999\\
717	-1.4999\\
718	-1.4999\\
719	-1.4999\\
720	-1.4999\\
721	-1.5\\
722	-1.5\\
723	-1.5\\
724	-1.5\\
725	-1.5\\
726	-1.5\\
727	-1.5\\
728	-1.5\\
729	-1.5\\
730	-1.5\\
731	-1.5\\
732	-1.5\\
733	-1.5\\
734	-1.5\\
735	-1.5\\
736	-1.5\\
737	-1.5\\
738	-1.5\\
739	-1.5\\
740	-1.5\\
741	-1.5\\
742	-1.5\\
743	-1.5\\
744	-1.5\\
745	-1.5\\
746	-1.5\\
747	-1.5\\
748	-1.5\\
749	-1.5\\
750	-1.5\\
751	-1.5\\
752	-1.5\\
753	-1.5\\
754	-1.5\\
755	-1.5\\
756	-1.5\\
757	-1.5\\
758	-1.5\\
759	-1.5\\
760	-1.5\\
761	-1.5\\
762	-1.5\\
763	-1.5\\
764	-1.5\\
765	-1.5\\
766	-1.5\\
767	-1.5\\
768	-1.5\\
769	-1.5\\
770	-1.5\\
771	-1.5\\
772	-1.5\\
773	-1.5\\
774	-1.5\\
775	-1.5\\
776	-1.5\\
777	-1.5\\
778	-1.5\\
779	-1.5\\
780	-1.5\\
781	-1.5\\
782	-1.5\\
783	-1.5\\
784	-1.5\\
785	-1.5\\
786	-1.5\\
787	-1.5\\
788	-1.5\\
789	-1.5\\
790	-1.5\\
791	-1.5\\
792	-1.5\\
793	-1.5\\
794	-1.5\\
795	-1.5\\
796	-1.5\\
797	-1.5\\
798	-1.5\\
799	-1.5\\
800	-1.5\\
801	-1.5\\
802	-1.5\\
803	-1.5\\
804	-1.5\\
805	-1.5662\\
806	-1.7134\\
807	-1.9287\\
808	-2.1876\\
809	-2.4631\\
810	-2.7357\\
811	-2.9898\\
812	-3.2117\\
813	-3.3913\\
814	-3.5229\\
815	-3.6053\\
816	-3.6416\\
817	-3.638\\
818	-3.6029\\
819	-3.5462\\
820	-3.4777\\
821	-3.4065\\
822	-3.3405\\
823	-3.2855\\
824	-3.2457\\
825	-3.2231\\
826	-3.2181\\
827	-3.2295\\
828	-3.2549\\
829	-3.2913\\
830	-3.335\\
831	-3.3823\\
832	-3.4297\\
833	-3.4741\\
834	-3.5128\\
835	-3.5442\\
836	-3.567\\
837	-3.581\\
838	-3.5864\\
839	-3.5842\\
840	-3.5757\\
841	-3.5624\\
842	-3.5461\\
843	-3.5284\\
844	-3.5108\\
845	-3.4948\\
846	-3.4813\\
847	-3.4709\\
848	-3.4641\\
849	-3.4608\\
850	-3.4608\\
851	-3.4637\\
852	-3.4689\\
853	-3.4756\\
854	-3.4832\\
855	-3.491\\
856	-3.4983\\
857	-3.5048\\
858	-3.51\\
859	-3.5138\\
860	-3.516\\
861	-3.5168\\
862	-3.5162\\
863	-3.5145\\
864	-3.5119\\
865	-3.5089\\
866	-3.5056\\
867	-3.5023\\
868	-3.4994\\
869	-3.4968\\
870	-3.4949\\
871	-3.4936\\
872	-3.493\\
873	-3.493\\
874	-3.4935\\
875	-3.4944\\
876	-3.4956\\
877	-3.4969\\
878	-3.4983\\
879	-3.4997\\
880	-3.5008\\
881	-3.5018\\
882	-3.5025\\
883	-3.5029\\
884	-3.503\\
885	-3.5029\\
886	-3.5026\\
887	-3.5022\\
888	-3.5016\\
889	-3.501\\
890	-3.5004\\
891	-3.4999\\
892	-3.4994\\
893	-3.4991\\
894	-3.4988\\
895	-3.4987\\
896	-3.4987\\
897	-3.4988\\
898	-3.499\\
899	-3.4992\\
900	-3.4994\\
901	-3.4997\\
902	-3.4999\\
903	-3.5001\\
904	-3.5003\\
905	-3.5004\\
906	-3.5005\\
907	-3.5005\\
908	-3.5005\\
909	-3.5005\\
910	-3.5004\\
911	-3.5003\\
912	-3.5002\\
913	-3.5001\\
914	-3.5\\
915	-3.4999\\
916	-3.4998\\
917	-3.4998\\
918	-3.4998\\
919	-3.4998\\
920	-3.4998\\
921	-3.4998\\
922	-3.4999\\
923	-3.4999\\
924	-3.4999\\
925	-3.5\\
926	-3.5\\
927	-3.5001\\
928	-3.5001\\
929	-3.5001\\
930	-3.5001\\
931	-3.5001\\
932	-3.5001\\
933	-3.5001\\
934	-3.5001\\
935	-3.5\\
936	-3.5\\
937	-3.5\\
938	-3.5\\
939	-3.5\\
940	-3.5\\
941	-3.5\\
942	-3.5\\
943	-3.5\\
944	-3.5\\
945	-3.5\\
946	-3.5\\
947	-3.5\\
948	-3.5\\
949	-3.5\\
950	-3.5\\
951	-3.5\\
952	-3.5\\
953	-3.5\\
954	-3.5\\
955	-3.5\\
956	-3.5\\
957	-3.5\\
958	-3.5\\
959	-3.5\\
960	-3.5\\
961	-3.5\\
962	-3.5\\
963	-3.5\\
964	-3.5\\
965	-3.5\\
966	-3.5\\
967	-3.5\\
968	-3.5\\
969	-3.5\\
970	-3.5\\
971	-3.5\\
972	-3.5\\
973	-3.5\\
974	-3.5\\
975	-3.5\\
976	-3.5\\
977	-3.5\\
978	-3.5\\
979	-3.5\\
980	-3.5\\
981	-3.5\\
982	-3.5\\
983	-3.5\\
984	-3.5\\
985	-3.5\\
986	-3.5\\
987	-3.5\\
988	-3.5\\
989	-3.5\\
990	-3.5\\
991	-3.5\\
992	-3.5\\
993	-3.5\\
994	-3.5\\
995	-3.5\\
996	-3.5\\
997	-3.5\\
998	-3.5\\
999	-3.5\\
1000	-3.5\\
1001	-3.5\\
1002	-3.5\\
1003	-3.5\\
1004	-3.5\\
1005	-3.3771\\
1006	-3.1245\\
1007	-2.7436\\
1008	-2.3041\\
1009	-1.8633\\
1010	-1.4715\\
1011	-1.1551\\
1012	-0.92548\\
1013	-0.78294\\
1014	-0.72196\\
1015	-0.73262\\
1016	-0.80155\\
1017	-0.91287\\
1018	-1.0494\\
1019	-1.1942\\
1020	-1.3318\\
1021	-1.4493\\
1022	-1.5374\\
1023	-1.591\\
1024	-1.6087\\
1025	-1.5933\\
1026	-1.5505\\
1027	-1.4883\\
1028	-1.4154\\
1029	-1.3402\\
1030	-1.2701\\
1031	-1.2104\\
1032	-1.1648\\
1033	-1.1347\\
1034	-1.12\\
1035	-1.1193\\
1036	-1.13\\
1037	-1.1489\\
1038	-1.1727\\
1039	-1.1981\\
1040	-1.2221\\
1041	-1.2424\\
1042	-1.2574\\
1043	-1.2664\\
1044	-1.2692\\
1045	-1.2663\\
1046	-1.2589\\
1047	-1.2483\\
1048	-1.2358\\
1049	-1.2229\\
1050	-1.2109\\
1051	-1.2006\\
1052	-1.1928\\
1053	-1.1876\\
1054	-1.1851\\
1055	-1.1851\\
1056	-1.187\\
1057	-1.1904\\
1058	-1.1946\\
1059	-1.199\\
1060	-1.2032\\
1061	-1.2068\\
1062	-1.2094\\
1063	-1.2109\\
1064	-1.2114\\
1065	-1.2109\\
1066	-1.2096\\
1067	-1.2078\\
1068	-1.2057\\
1069	-1.2035\\
1070	-1.2015\\
1071	-1.1998\\
1072	-1.1984\\
1073	-1.1976\\
1074	-1.1972\\
1075	-1.1972\\
1076	-1.1976\\
1077	-1.1982\\
1078	-1.199\\
1079	-1.1997\\
1080	-1.2005\\
1081	-1.2011\\
1082	-1.2015\\
1083	-1.2018\\
1084	-1.2019\\
1085	-1.2018\\
1086	-1.2016\\
1087	-1.2013\\
1088	-1.2009\\
1089	-1.2005\\
1090	-1.2002\\
1091	-1.1999\\
1092	-1.1997\\
1093	-1.1995\\
1094	-1.1995\\
1095	-1.1995\\
1096	-1.1996\\
1097	-1.1997\\
1098	-1.1998\\
1099	-1.1999\\
1100	-1.2001\\
1101	-1.2002\\
1102	-1.2003\\
1103	-1.2003\\
1104	-1.2003\\
1105	-1.2003\\
1106	-1.2003\\
1107	-1.2002\\
1108	-1.2001\\
1109	-1.2001\\
1110	-1.2\\
1111	-1.2\\
1112	-1.1999\\
1113	-1.1999\\
1114	-1.1999\\
1115	-1.1999\\
1116	-1.1999\\
1117	-1.1999\\
1118	-1.2\\
1119	-1.2\\
1120	-1.2\\
1121	-1.2\\
1122	-1.2\\
1123	-1.2\\
1124	-1.2001\\
1125	-1.2\\
1126	-1.2\\
1127	-1.2\\
1128	-1.2\\
1129	-1.2\\
1130	-1.2\\
1131	-1.2\\
1132	-1.2\\
1133	-1.2\\
1134	-1.2\\
1135	-1.2\\
1136	-1.2\\
1137	-1.2\\
1138	-1.2\\
1139	-1.2\\
1140	-1.2\\
1141	-1.2\\
1142	-1.2\\
1143	-1.2\\
1144	-1.2\\
1145	-1.2\\
1146	-1.2\\
1147	-1.2\\
1148	-1.2\\
1149	-1.2\\
1150	-1.2\\
1151	-1.2\\
1152	-1.2\\
1153	-1.2\\
1154	-1.2\\
1155	-1.2\\
1156	-1.2\\
1157	-1.2\\
1158	-1.2\\
1159	-1.2\\
1160	-1.2\\
1161	-1.2\\
1162	-1.2\\
1163	-1.2\\
1164	-1.2\\
1165	-1.2\\
1166	-1.2\\
1167	-1.2\\
1168	-1.2\\
1169	-1.2\\
1170	-1.2\\
1171	-1.2\\
1172	-1.2\\
1173	-1.2\\
1174	-1.2\\
1175	-1.2\\
1176	-1.2\\
1177	-1.2\\
1178	-1.2\\
1179	-1.2\\
1180	-1.2\\
1181	-1.2\\
1182	-1.2\\
1183	-1.2\\
1184	-1.2\\
1185	-1.2\\
1186	-1.2\\
1187	-1.2\\
1188	-1.2\\
1189	-1.2\\
1190	-1.2\\
1191	-1.2\\
1192	-1.2\\
1193	-1.2\\
1194	-1.2\\
1195	-1.2\\
1196	-1.2\\
1197	-1.2\\
1198	-1.2\\
1199	-1.2\\
1200	-1.2\\
1201	-1.2\\
1202	-1.2\\
1203	-1.2\\
1204	-1.2\\
1205	-1.1602\\
1206	-1.0818\\
1207	-0.97257\\
1208	-0.85049\\
1209	-0.73004\\
1210	-0.62296\\
1211	-0.53612\\
1212	-0.47261\\
1213	-0.4324\\
1214	-0.4132\\
1215	-0.41116\\
1216	-0.42148\\
1217	-0.43905\\
1218	-0.45894\\
1219	-0.477\\
1220	-0.49013\\
1221	-0.49652\\
1222	-0.49566\\
1223	-0.48812\\
1224	-0.4753\\
1225	-0.45902\\
1226	-0.4412\\
1227	-0.42359\\
1228	-0.40756\\
1229	-0.39405\\
1230	-0.3835\\
1231	-0.37593\\
1232	-0.37106\\
1233	-0.36834\\
1234	-0.36713\\
1235	-0.36676\\
1236	-0.36663\\
1237	-0.36627\\
1238	-0.36537\\
1239	-0.36376\\
1240	-0.36144\\
1241	-0.3585\\
1242	-0.35513\\
1243	-0.35153\\
1244	-0.34792\\
1245	-0.34447\\
1246	-0.34132\\
1247	-0.33854\\
1248	-0.33617\\
1249	-0.33419\\
1250	-0.33255\\
1251	-0.33117\\
1252	-0.32998\\
1253	-0.3289\\
1254	-0.32787\\
1255	-0.32684\\
1256	-0.32578\\
1257	-0.32469\\
1258	-0.32356\\
1259	-0.32242\\
1260	-0.32128\\
1261	-0.32016\\
1262	-0.31909\\
1263	-0.31808\\
1264	-0.31713\\
1265	-0.31626\\
1266	-0.31546\\
1267	-0.31473\\
1268	-0.31405\\
1269	-0.31343\\
1270	-0.31284\\
1271	-0.31228\\
1272	-0.31174\\
1273	-0.31122\\
1274	-0.31072\\
1275	-0.31023\\
1276	-0.30975\\
1277	-0.30929\\
1278	-0.30885\\
1279	-0.30843\\
1280	-0.30802\\
1281	-0.30764\\
1282	-0.30728\\
1283	-0.30694\\
1284	-0.30661\\
1285	-0.30631\\
1286	-0.30602\\
1287	-0.30575\\
1288	-0.30549\\
1289	-0.30524\\
1290	-0.305\\
1291	-0.30477\\
1292	-0.30455\\
1293	-0.30434\\
1294	-0.30414\\
1295	-0.30395\\
1296	-0.30377\\
1297	-0.30359\\
1298	-0.30343\\
1299	-0.30327\\
1300	-0.30312\\
1301	-0.30297\\
1302	-0.30284\\
1303	-0.30271\\
1304	-0.30258\\
1305	-0.30247\\
1306	-0.30235\\
1307	-0.30225\\
1308	-0.30214\\
1309	-0.30204\\
1310	-0.30195\\
1311	-0.30186\\
1312	-0.30178\\
1313	-0.30169\\
1314	-0.30162\\
1315	-0.30154\\
1316	-0.30147\\
1317	-0.3014\\
1318	-0.30134\\
1319	-0.30128\\
1320	-0.30122\\
1321	-0.30116\\
1322	-0.30111\\
1323	-0.30106\\
1324	-0.30101\\
1325	-0.30097\\
1326	-0.30092\\
1327	-0.30088\\
1328	-0.30084\\
1329	-0.3008\\
1330	-0.30076\\
1331	-0.30073\\
1332	-0.3007\\
1333	-0.30066\\
1334	-0.30063\\
1335	-0.3006\\
1336	-0.30058\\
1337	-0.30055\\
1338	-0.30053\\
1339	-0.3005\\
1340	-0.30048\\
1341	-0.30046\\
1342	-0.30044\\
1343	-0.30042\\
1344	-0.3004\\
1345	-0.30038\\
1346	-0.30036\\
1347	-0.30034\\
1348	-0.30033\\
1349	-0.30031\\
1350	-0.3003\\
1351	-0.30029\\
1352	-0.30027\\
1353	-0.30026\\
1354	-0.30025\\
1355	-0.30024\\
1356	-0.30023\\
1357	-0.30022\\
1358	-0.30021\\
1359	-0.3002\\
1360	-0.30019\\
1361	-0.30018\\
1362	-0.30017\\
1363	-0.30016\\
1364	-0.30016\\
1365	-0.30015\\
1366	-0.30014\\
1367	-0.30014\\
1368	-0.30013\\
1369	-0.30012\\
1370	-0.30012\\
1371	-0.30011\\
1372	-0.30011\\
1373	-0.3001\\
1374	-0.3001\\
1375	-0.30009\\
1376	-0.30009\\
1377	-0.30008\\
1378	-0.30008\\
1379	-0.30008\\
1380	-0.30007\\
1381	-0.30007\\
1382	-0.30007\\
1383	-0.30006\\
1384	-0.30006\\
1385	-0.30006\\
1386	-0.30006\\
1387	-0.30005\\
1388	-0.30005\\
1389	-0.30005\\
1390	-0.30005\\
1391	-0.30004\\
1392	-0.30004\\
1393	-0.30004\\
1394	-0.30004\\
1395	-0.30004\\
1396	-0.30003\\
1397	-0.30003\\
1398	-0.30003\\
1399	-0.30003\\
1400	-0.30003\\
1401	-0.30003\\
1402	-0.30003\\
1403	-0.30003\\
1404	-0.30002\\
1405	-0.2918\\
1406	-0.27545\\
1407	-0.25355\\
1408	-0.22962\\
1409	-0.20662\\
1410	-0.18663\\
1411	-0.17075\\
1412	-0.15927\\
1413	-0.15182\\
1414	-0.14761\\
1415	-0.14564\\
1416	-0.14488\\
1417	-0.14442\\
1418	-0.14357\\
1419	-0.1419\\
1420	-0.13926\\
1421	-0.13569\\
1422	-0.1314\\
1423	-0.12666\\
1424	-0.12177\\
1425	-0.11698\\
1426	-0.11249\\
1427	-0.10841\\
1428	-0.10479\\
1429	-0.1016\\
1430	-0.09878\\
1431	-0.096248\\
1432	-0.093914\\
1433	-0.09147\\
1434	-0.088652\\
1435	-0.085401\\
1436	-0.081759\\
1437	-0.077631\\
1438	-0.072963\\
1439	-0.067853\\
1440	-0.062464\\
1441	-0.05682\\
1442	-0.050947\\
1443	-0.044846\\
1444	-0.038567\\
1445	-0.032162\\
1446	-0.025722\\
1447	-0.019337\\
1448	-0.013116\\
1449	-0.0072166\\
1450	-0.0017877\\
1451	0.0030547\\
1452	0.0072348\\
1453	0.010714\\
1454	0.013482\\
1455	0.015555\\
1456	0.016963\\
1457	0.017752\\
1458	0.017977\\
1459	0.017696\\
1460	0.016975\\
1461	0.015879\\
1462	0.014475\\
1463	0.012829\\
1464	0.011007\\
1465	0.0090701\\
1466	0.0070792\\
1467	0.0050904\\
1468	0.0031555\\
1469	0.0013209\\
1470	-0.0003727\\
1471	-0.0018915\\
1472	-0.0032087\\
1473	-0.0043051\\
1474	-0.0051693\\
1475	-0.0057974\\
1476	-0.0061928\\
1477	-0.0063656\\
1478	-0.0063317\\
1479	-0.0061122\\
1480	-0.0057319\\
1481	-0.0052182\\
1482	-0.0046001\\
1483	-0.003907\\
1484	-0.0031676\\
1485	-0.0024091\\
1486	-0.0016564\\
1487	-0.00093141\\
1488	-0.00025302\\
1489	0.00036351\\
1490	0.00090644\\
1491	0.0013675\\
1492	0.0017418\\
1493	0.0020275\\
1494	0.0022255\\
1495	0.0023392\\
1496	0.002374\\
1497	0.002337\\
1498	0.0022367\\
1499	0.0020822\\
1500	0.0018835\\
1501	0.0016508\\
1502	0.001394\\
1503	0.0011229\\
1504	0.00084666\\
1505	0.00057374\\
1506	0.00031162\\
1507	6.6744e-05\\
1508	-0.00015553\\
1509	-0.00035104\\
1510	-0.00051678\\
1511	-0.0006509\\
1512	-0.00075266\\
1513	-0.00082231\\
1514	-0.00086103\\
1515	-0.00087077\\
1516	-0.00085415\\
1517	-0.00081429\\
1518	-0.00075469\\
1519	-0.00067908\\
1520	-0.00059128\\
1521	-0.00049509\\
1522	-0.00039416\\
1523	-0.00029192\\
1524	-0.00019149\\
1525	-9.5611e-05\\
1526	-6.624e-06\\
1527	7.3579e-05\\
1528	0.00014355\\
1529	0.0002023\\
1530	0.00024924\\
1531	0.00028422\\
1532	0.00030742\\
1533	0.00031936\\
1534	0.00032083\\
1535	0.00031285\\
1536	0.00029661\\
1537	0.00027339\\
1538	0.00024459\\
1539	0.00021159\\
1540	0.00017578\\
1541	0.00013847\\
1542	0.0001009\\
1543	6.4181e-05\\
1544	2.9295e-05\\
1545	-2.9313e-06\\
1546	-3.183e-05\\
1547	-5.6897e-05\\
1548	-7.779e-05\\
1549	-9.4321e-05\\
1550	-0.00010645\\
1551	-0.00011426\\
1552	-0.00011797\\
1553	-0.00011788\\
1554	-0.00011437\\
1555	-0.00010789\\
1556	-9.8928e-05\\
1557	-8.7993e-05\\
1558	-7.5602e-05\\
1559	-6.2262e-05\\
1560	-4.8455e-05\\
1561	-3.4631e-05\\
1562	-2.1194e-05\\
1563	-8.4932e-06\\
1564	3.1749e-06\\
1565	1.3576e-05\\
1566	2.2535e-05\\
1567	2.9937e-05\\
1568	3.5724e-05\\
1569	3.9892e-05\\
1570	4.248e-05\\
1571	4.3575e-05\\
1572	4.3294e-05\\
1573	4.1784e-05\\
1574	3.9213e-05\\
1575	3.5762e-05\\
1576	3.162e-05\\
1577	2.6977e-05\\
1578	2.2018e-05\\
1579	1.6919e-05\\
1580	1.1842e-05\\
1581	6.9316e-06\\
1582	2.3147e-06\\
1583	-1.9047e-06\\
1584	-5.644e-06\\
1585	-8.8429e-06\\
1586	-1.1463e-05\\
1587	-1.3486e-05\\
1588	-1.4914e-05\\
1589	-1.5765e-05\\
1590	-1.6073e-05\\
1591	-1.5882e-05\\
1592	-1.525e-05\\
1593	-1.4238e-05\\
1594	-1.2914e-05\\
1595	-1.1349e-05\\
1596	-9.6117e-06\\
1597	-7.77e-06\\
1598	-5.8881e-06\\
1599	-4.0246e-06\\
1600	-2.2319e-06\\
1601	-5.5483e-07\\
1602	9.695e-07\\
1603	2.3122e-06\\
1604	3.4525e-06\\
1605	0.0010817\\
1606	0.0043126\\
1607	0.0096419\\
1608	0.016545\\
1609	0.024378\\
1610	0.03256\\
1611	0.04063\\
1612	0.04826\\
1613	0.055241\\
1614	0.061462\\
1615	0.06689\\
1616	0.071544\\
1617	0.075482\\
1618	0.078778\\
1619	0.081515\\
1620	0.083775\\
1621	0.085638\\
1622	0.087171\\
1623	0.088435\\
1624	0.089483\\
1625	0.090355\\
1626	0.091087\\
1627	0.091708\\
1628	0.09224\\
1629	0.0927\\
1630	0.093104\\
1631	0.093462\\
1632	0.093783\\
1633	0.094073\\
1634	0.094338\\
1635	0.094582\\
1636	0.094809\\
1637	0.09502\\
1638	0.095218\\
1639	0.095404\\
1640	0.095581\\
1641	0.095748\\
1642	0.095906\\
1643	0.096057\\
1644	0.096201\\
1645	0.096338\\
1646	0.096469\\
1647	0.096594\\
1648	0.096714\\
1649	0.096829\\
1650	0.096939\\
1651	0.097045\\
1652	0.097146\\
1653	0.097244\\
1654	0.097338\\
1655	0.097428\\
1656	0.097514\\
1657	0.097597\\
1658	0.097677\\
1659	0.097754\\
1660	0.097829\\
1661	0.0979\\
1662	0.097969\\
1663	0.098035\\
1664	0.098099\\
1665	0.098161\\
1666	0.09822\\
1667	0.098278\\
1668	0.098333\\
1669	0.098386\\
1670	0.098438\\
1671	0.098488\\
1672	0.098536\\
1673	0.098582\\
1674	0.098627\\
1675	0.09867\\
1676	0.098712\\
1677	0.098752\\
1678	0.098792\\
1679	0.098829\\
1680	0.098866\\
1681	0.098901\\
1682	0.098935\\
1683	0.098968\\
1684	0.099\\
1685	0.099031\\
1686	0.099061\\
1687	0.09909\\
1688	0.099118\\
1689	0.099145\\
1690	0.099171\\
1691	0.099197\\
1692	0.099221\\
1693	0.099245\\
1694	0.099268\\
1695	0.09929\\
1696	0.099312\\
1697	0.099333\\
1698	0.099353\\
1699	0.099373\\
1700	0.099392\\
1701	0.09941\\
1702	0.099428\\
1703	0.099445\\
1704	0.099462\\
1705	0.099478\\
1706	0.099494\\
1707	0.099509\\
1708	0.099524\\
1709	0.099538\\
1710	0.099552\\
1711	0.099565\\
1712	0.099579\\
1713	0.099591\\
1714	0.099603\\
1715	0.099615\\
1716	0.099627\\
1717	0.099638\\
1718	0.099649\\
1719	0.099659\\
1720	0.099669\\
1721	0.099679\\
1722	0.099689\\
1723	0.099698\\
1724	0.099707\\
1725	0.099716\\
1726	0.099724\\
1727	0.099732\\
1728	0.09974\\
1729	0.099748\\
1730	0.099756\\
1731	0.099763\\
1732	0.09977\\
1733	0.099777\\
1734	0.099783\\
1735	0.09979\\
1736	0.099796\\
1737	0.099802\\
1738	0.099808\\
1739	0.099814\\
1740	0.099819\\
1741	0.099824\\
1742	0.09983\\
1743	0.099835\\
1744	0.09984\\
1745	0.099844\\
1746	0.099849\\
1747	0.099853\\
1748	0.099858\\
1749	0.099862\\
1750	0.099866\\
1751	0.09987\\
1752	0.099874\\
1753	0.099877\\
1754	0.099881\\
1755	0.099885\\
1756	0.099888\\
1757	0.099891\\
1758	0.099894\\
1759	0.099898\\
1760	0.099901\\
1761	0.099904\\
1762	0.099906\\
1763	0.099909\\
1764	0.099912\\
1765	0.099914\\
1766	0.099917\\
1767	0.099919\\
1768	0.099922\\
1769	0.099924\\
1770	0.099926\\
1771	0.099928\\
1772	0.099931\\
1773	0.099933\\
1774	0.099935\\
1775	0.099936\\
1776	0.099938\\
1777	0.09994\\
1778	0.099942\\
1779	0.099944\\
1780	0.099945\\
1781	0.099947\\
1782	0.099948\\
1783	0.09995\\
1784	0.099951\\
1785	0.099953\\
1786	0.099954\\
1787	0.099956\\
1788	0.099957\\
1789	0.099958\\
1790	0.099959\\
1791	0.099961\\
1792	0.099962\\
1793	0.099963\\
1794	0.099964\\
1795	0.099965\\
1796	0.099966\\
1797	0.099967\\
1798	0.099968\\
1799	0.099969\\
1800	0.09997\\
};
\addlegendentry{Wyjście y}

\addplot [color=mycolor2, dashed]
  table[row sep=crcr]{%
1	0\\
2	0\\
3	0\\
4	0\\
5	0\\
6	0\\
7	0\\
8	0\\
9	0\\
10	0\\
11	0\\
12	0\\
13	0\\
14	0\\
15	0\\
16	0\\
17	0\\
18	0\\
19	0\\
20	-2\\
21	-2\\
22	-2\\
23	-2\\
24	-2\\
25	-2\\
26	-2\\
27	-2\\
28	-2\\
29	-2\\
30	-2\\
31	-2\\
32	-2\\
33	-2\\
34	-2\\
35	-2\\
36	-2\\
37	-2\\
38	-2\\
39	-2\\
40	-2\\
41	-2\\
42	-2\\
43	-2\\
44	-2\\
45	-2\\
46	-2\\
47	-2\\
48	-2\\
49	-2\\
50	-2\\
51	-2\\
52	-2\\
53	-2\\
54	-2\\
55	-2\\
56	-2\\
57	-2\\
58	-2\\
59	-2\\
60	-2\\
61	-2\\
62	-2\\
63	-2\\
64	-2\\
65	-2\\
66	-2\\
67	-2\\
68	-2\\
69	-2\\
70	-2\\
71	-2\\
72	-2\\
73	-2\\
74	-2\\
75	-2\\
76	-2\\
77	-2\\
78	-2\\
79	-2\\
80	-2\\
81	-2\\
82	-2\\
83	-2\\
84	-2\\
85	-2\\
86	-2\\
87	-2\\
88	-2\\
89	-2\\
90	-2\\
91	-2\\
92	-2\\
93	-2\\
94	-2\\
95	-2\\
96	-2\\
97	-2\\
98	-2\\
99	-2\\
100	-2\\
101	-2\\
102	-2\\
103	-2\\
104	-2\\
105	-2\\
106	-2\\
107	-2\\
108	-2\\
109	-2\\
110	-2\\
111	-2\\
112	-2\\
113	-2\\
114	-2\\
115	-2\\
116	-2\\
117	-2\\
118	-2\\
119	-2\\
120	-2\\
121	-2\\
122	-2\\
123	-2\\
124	-2\\
125	-2\\
126	-2\\
127	-2\\
128	-2\\
129	-2\\
130	-2\\
131	-2\\
132	-2\\
133	-2\\
134	-2\\
135	-2\\
136	-2\\
137	-2\\
138	-2\\
139	-2\\
140	-2\\
141	-2\\
142	-2\\
143	-2\\
144	-2\\
145	-2\\
146	-2\\
147	-2\\
148	-2\\
149	-2\\
150	-2\\
151	-2\\
152	-2\\
153	-2\\
154	-2\\
155	-2\\
156	-2\\
157	-2\\
158	-2\\
159	-2\\
160	-2\\
161	-2\\
162	-2\\
163	-2\\
164	-2\\
165	-2\\
166	-2\\
167	-2\\
168	-2\\
169	-2\\
170	-2\\
171	-2\\
172	-2\\
173	-2\\
174	-2\\
175	-2\\
176	-2\\
177	-2\\
178	-2\\
179	-2\\
180	-2\\
181	-2\\
182	-2\\
183	-2\\
184	-2\\
185	-2\\
186	-2\\
187	-2\\
188	-2\\
189	-2\\
190	-2\\
191	-2\\
192	-2\\
193	-2\\
194	-2\\
195	-2\\
196	-2\\
197	-2\\
198	-2\\
199	-2\\
200	-4.5\\
201	-4.5\\
202	-4.5\\
203	-4.5\\
204	-4.5\\
205	-4.5\\
206	-4.5\\
207	-4.5\\
208	-4.5\\
209	-4.5\\
210	-4.5\\
211	-4.5\\
212	-4.5\\
213	-4.5\\
214	-4.5\\
215	-4.5\\
216	-4.5\\
217	-4.5\\
218	-4.5\\
219	-4.5\\
220	-4.5\\
221	-4.5\\
222	-4.5\\
223	-4.5\\
224	-4.5\\
225	-4.5\\
226	-4.5\\
227	-4.5\\
228	-4.5\\
229	-4.5\\
230	-4.5\\
231	-4.5\\
232	-4.5\\
233	-4.5\\
234	-4.5\\
235	-4.5\\
236	-4.5\\
237	-4.5\\
238	-4.5\\
239	-4.5\\
240	-4.5\\
241	-4.5\\
242	-4.5\\
243	-4.5\\
244	-4.5\\
245	-4.5\\
246	-4.5\\
247	-4.5\\
248	-4.5\\
249	-4.5\\
250	-4.5\\
251	-4.5\\
252	-4.5\\
253	-4.5\\
254	-4.5\\
255	-4.5\\
256	-4.5\\
257	-4.5\\
258	-4.5\\
259	-4.5\\
260	-4.5\\
261	-4.5\\
262	-4.5\\
263	-4.5\\
264	-4.5\\
265	-4.5\\
266	-4.5\\
267	-4.5\\
268	-4.5\\
269	-4.5\\
270	-4.5\\
271	-4.5\\
272	-4.5\\
273	-4.5\\
274	-4.5\\
275	-4.5\\
276	-4.5\\
277	-4.5\\
278	-4.5\\
279	-4.5\\
280	-4.5\\
281	-4.5\\
282	-4.5\\
283	-4.5\\
284	-4.5\\
285	-4.5\\
286	-4.5\\
287	-4.5\\
288	-4.5\\
289	-4.5\\
290	-4.5\\
291	-4.5\\
292	-4.5\\
293	-4.5\\
294	-4.5\\
295	-4.5\\
296	-4.5\\
297	-4.5\\
298	-4.5\\
299	-4.5\\
300	-4.5\\
301	-4.5\\
302	-4.5\\
303	-4.5\\
304	-4.5\\
305	-4.5\\
306	-4.5\\
307	-4.5\\
308	-4.5\\
309	-4.5\\
310	-4.5\\
311	-4.5\\
312	-4.5\\
313	-4.5\\
314	-4.5\\
315	-4.5\\
316	-4.5\\
317	-4.5\\
318	-4.5\\
319	-4.5\\
320	-4.5\\
321	-4.5\\
322	-4.5\\
323	-4.5\\
324	-4.5\\
325	-4.5\\
326	-4.5\\
327	-4.5\\
328	-4.5\\
329	-4.5\\
330	-4.5\\
331	-4.5\\
332	-4.5\\
333	-4.5\\
334	-4.5\\
335	-4.5\\
336	-4.5\\
337	-4.5\\
338	-4.5\\
339	-4.5\\
340	-4.5\\
341	-4.5\\
342	-4.5\\
343	-4.5\\
344	-4.5\\
345	-4.5\\
346	-4.5\\
347	-4.5\\
348	-4.5\\
349	-4.5\\
350	-4.5\\
351	-4.5\\
352	-4.5\\
353	-4.5\\
354	-4.5\\
355	-4.5\\
356	-4.5\\
357	-4.5\\
358	-4.5\\
359	-4.5\\
360	-4.5\\
361	-4.5\\
362	-4.5\\
363	-4.5\\
364	-4.5\\
365	-4.5\\
366	-4.5\\
367	-4.5\\
368	-4.5\\
369	-4.5\\
370	-4.5\\
371	-4.5\\
372	-4.5\\
373	-4.5\\
374	-4.5\\
375	-4.5\\
376	-4.5\\
377	-4.5\\
378	-4.5\\
379	-4.5\\
380	-4.5\\
381	-4.5\\
382	-4.5\\
383	-4.5\\
384	-4.5\\
385	-4.5\\
386	-4.5\\
387	-4.5\\
388	-4.5\\
389	-4.5\\
390	-4.5\\
391	-4.5\\
392	-4.5\\
393	-4.5\\
394	-4.5\\
395	-4.5\\
396	-4.5\\
397	-4.5\\
398	-4.5\\
399	-4.5\\
400	-3\\
401	-3\\
402	-3\\
403	-3\\
404	-3\\
405	-3\\
406	-3\\
407	-3\\
408	-3\\
409	-3\\
410	-3\\
411	-3\\
412	-3\\
413	-3\\
414	-3\\
415	-3\\
416	-3\\
417	-3\\
418	-3\\
419	-3\\
420	-3\\
421	-3\\
422	-3\\
423	-3\\
424	-3\\
425	-3\\
426	-3\\
427	-3\\
428	-3\\
429	-3\\
430	-3\\
431	-3\\
432	-3\\
433	-3\\
434	-3\\
435	-3\\
436	-3\\
437	-3\\
438	-3\\
439	-3\\
440	-3\\
441	-3\\
442	-3\\
443	-3\\
444	-3\\
445	-3\\
446	-3\\
447	-3\\
448	-3\\
449	-3\\
450	-3\\
451	-3\\
452	-3\\
453	-3\\
454	-3\\
455	-3\\
456	-3\\
457	-3\\
458	-3\\
459	-3\\
460	-3\\
461	-3\\
462	-3\\
463	-3\\
464	-3\\
465	-3\\
466	-3\\
467	-3\\
468	-3\\
469	-3\\
470	-3\\
471	-3\\
472	-3\\
473	-3\\
474	-3\\
475	-3\\
476	-3\\
477	-3\\
478	-3\\
479	-3\\
480	-3\\
481	-3\\
482	-3\\
483	-3\\
484	-3\\
485	-3\\
486	-3\\
487	-3\\
488	-3\\
489	-3\\
490	-3\\
491	-3\\
492	-3\\
493	-3\\
494	-3\\
495	-3\\
496	-3\\
497	-3\\
498	-3\\
499	-3\\
500	-3\\
501	-3\\
502	-3\\
503	-3\\
504	-3\\
505	-3\\
506	-3\\
507	-3\\
508	-3\\
509	-3\\
510	-3\\
511	-3\\
512	-3\\
513	-3\\
514	-3\\
515	-3\\
516	-3\\
517	-3\\
518	-3\\
519	-3\\
520	-3\\
521	-3\\
522	-3\\
523	-3\\
524	-3\\
525	-3\\
526	-3\\
527	-3\\
528	-3\\
529	-3\\
530	-3\\
531	-3\\
532	-3\\
533	-3\\
534	-3\\
535	-3\\
536	-3\\
537	-3\\
538	-3\\
539	-3\\
540	-3\\
541	-3\\
542	-3\\
543	-3\\
544	-3\\
545	-3\\
546	-3\\
547	-3\\
548	-3\\
549	-3\\
550	-3\\
551	-3\\
552	-3\\
553	-3\\
554	-3\\
555	-3\\
556	-3\\
557	-3\\
558	-3\\
559	-3\\
560	-3\\
561	-3\\
562	-3\\
563	-3\\
564	-3\\
565	-3\\
566	-3\\
567	-3\\
568	-3\\
569	-3\\
570	-3\\
571	-3\\
572	-3\\
573	-3\\
574	-3\\
575	-3\\
576	-3\\
577	-3\\
578	-3\\
579	-3\\
580	-3\\
581	-3\\
582	-3\\
583	-3\\
584	-3\\
585	-3\\
586	-3\\
587	-3\\
588	-3\\
589	-3\\
590	-3\\
591	-3\\
592	-3\\
593	-3\\
594	-3\\
595	-3\\
596	-3\\
597	-3\\
598	-3\\
599	-3\\
600	-1.5\\
601	-1.5\\
602	-1.5\\
603	-1.5\\
604	-1.5\\
605	-1.5\\
606	-1.5\\
607	-1.5\\
608	-1.5\\
609	-1.5\\
610	-1.5\\
611	-1.5\\
612	-1.5\\
613	-1.5\\
614	-1.5\\
615	-1.5\\
616	-1.5\\
617	-1.5\\
618	-1.5\\
619	-1.5\\
620	-1.5\\
621	-1.5\\
622	-1.5\\
623	-1.5\\
624	-1.5\\
625	-1.5\\
626	-1.5\\
627	-1.5\\
628	-1.5\\
629	-1.5\\
630	-1.5\\
631	-1.5\\
632	-1.5\\
633	-1.5\\
634	-1.5\\
635	-1.5\\
636	-1.5\\
637	-1.5\\
638	-1.5\\
639	-1.5\\
640	-1.5\\
641	-1.5\\
642	-1.5\\
643	-1.5\\
644	-1.5\\
645	-1.5\\
646	-1.5\\
647	-1.5\\
648	-1.5\\
649	-1.5\\
650	-1.5\\
651	-1.5\\
652	-1.5\\
653	-1.5\\
654	-1.5\\
655	-1.5\\
656	-1.5\\
657	-1.5\\
658	-1.5\\
659	-1.5\\
660	-1.5\\
661	-1.5\\
662	-1.5\\
663	-1.5\\
664	-1.5\\
665	-1.5\\
666	-1.5\\
667	-1.5\\
668	-1.5\\
669	-1.5\\
670	-1.5\\
671	-1.5\\
672	-1.5\\
673	-1.5\\
674	-1.5\\
675	-1.5\\
676	-1.5\\
677	-1.5\\
678	-1.5\\
679	-1.5\\
680	-1.5\\
681	-1.5\\
682	-1.5\\
683	-1.5\\
684	-1.5\\
685	-1.5\\
686	-1.5\\
687	-1.5\\
688	-1.5\\
689	-1.5\\
690	-1.5\\
691	-1.5\\
692	-1.5\\
693	-1.5\\
694	-1.5\\
695	-1.5\\
696	-1.5\\
697	-1.5\\
698	-1.5\\
699	-1.5\\
700	-1.5\\
701	-1.5\\
702	-1.5\\
703	-1.5\\
704	-1.5\\
705	-1.5\\
706	-1.5\\
707	-1.5\\
708	-1.5\\
709	-1.5\\
710	-1.5\\
711	-1.5\\
712	-1.5\\
713	-1.5\\
714	-1.5\\
715	-1.5\\
716	-1.5\\
717	-1.5\\
718	-1.5\\
719	-1.5\\
720	-1.5\\
721	-1.5\\
722	-1.5\\
723	-1.5\\
724	-1.5\\
725	-1.5\\
726	-1.5\\
727	-1.5\\
728	-1.5\\
729	-1.5\\
730	-1.5\\
731	-1.5\\
732	-1.5\\
733	-1.5\\
734	-1.5\\
735	-1.5\\
736	-1.5\\
737	-1.5\\
738	-1.5\\
739	-1.5\\
740	-1.5\\
741	-1.5\\
742	-1.5\\
743	-1.5\\
744	-1.5\\
745	-1.5\\
746	-1.5\\
747	-1.5\\
748	-1.5\\
749	-1.5\\
750	-1.5\\
751	-1.5\\
752	-1.5\\
753	-1.5\\
754	-1.5\\
755	-1.5\\
756	-1.5\\
757	-1.5\\
758	-1.5\\
759	-1.5\\
760	-1.5\\
761	-1.5\\
762	-1.5\\
763	-1.5\\
764	-1.5\\
765	-1.5\\
766	-1.5\\
767	-1.5\\
768	-1.5\\
769	-1.5\\
770	-1.5\\
771	-1.5\\
772	-1.5\\
773	-1.5\\
774	-1.5\\
775	-1.5\\
776	-1.5\\
777	-1.5\\
778	-1.5\\
779	-1.5\\
780	-1.5\\
781	-1.5\\
782	-1.5\\
783	-1.5\\
784	-1.5\\
785	-1.5\\
786	-1.5\\
787	-1.5\\
788	-1.5\\
789	-1.5\\
790	-1.5\\
791	-1.5\\
792	-1.5\\
793	-1.5\\
794	-1.5\\
795	-1.5\\
796	-1.5\\
797	-1.5\\
798	-1.5\\
799	-1.5\\
800	-3.5\\
801	-3.5\\
802	-3.5\\
803	-3.5\\
804	-3.5\\
805	-3.5\\
806	-3.5\\
807	-3.5\\
808	-3.5\\
809	-3.5\\
810	-3.5\\
811	-3.5\\
812	-3.5\\
813	-3.5\\
814	-3.5\\
815	-3.5\\
816	-3.5\\
817	-3.5\\
818	-3.5\\
819	-3.5\\
820	-3.5\\
821	-3.5\\
822	-3.5\\
823	-3.5\\
824	-3.5\\
825	-3.5\\
826	-3.5\\
827	-3.5\\
828	-3.5\\
829	-3.5\\
830	-3.5\\
831	-3.5\\
832	-3.5\\
833	-3.5\\
834	-3.5\\
835	-3.5\\
836	-3.5\\
837	-3.5\\
838	-3.5\\
839	-3.5\\
840	-3.5\\
841	-3.5\\
842	-3.5\\
843	-3.5\\
844	-3.5\\
845	-3.5\\
846	-3.5\\
847	-3.5\\
848	-3.5\\
849	-3.5\\
850	-3.5\\
851	-3.5\\
852	-3.5\\
853	-3.5\\
854	-3.5\\
855	-3.5\\
856	-3.5\\
857	-3.5\\
858	-3.5\\
859	-3.5\\
860	-3.5\\
861	-3.5\\
862	-3.5\\
863	-3.5\\
864	-3.5\\
865	-3.5\\
866	-3.5\\
867	-3.5\\
868	-3.5\\
869	-3.5\\
870	-3.5\\
871	-3.5\\
872	-3.5\\
873	-3.5\\
874	-3.5\\
875	-3.5\\
876	-3.5\\
877	-3.5\\
878	-3.5\\
879	-3.5\\
880	-3.5\\
881	-3.5\\
882	-3.5\\
883	-3.5\\
884	-3.5\\
885	-3.5\\
886	-3.5\\
887	-3.5\\
888	-3.5\\
889	-3.5\\
890	-3.5\\
891	-3.5\\
892	-3.5\\
893	-3.5\\
894	-3.5\\
895	-3.5\\
896	-3.5\\
897	-3.5\\
898	-3.5\\
899	-3.5\\
900	-3.5\\
901	-3.5\\
902	-3.5\\
903	-3.5\\
904	-3.5\\
905	-3.5\\
906	-3.5\\
907	-3.5\\
908	-3.5\\
909	-3.5\\
910	-3.5\\
911	-3.5\\
912	-3.5\\
913	-3.5\\
914	-3.5\\
915	-3.5\\
916	-3.5\\
917	-3.5\\
918	-3.5\\
919	-3.5\\
920	-3.5\\
921	-3.5\\
922	-3.5\\
923	-3.5\\
924	-3.5\\
925	-3.5\\
926	-3.5\\
927	-3.5\\
928	-3.5\\
929	-3.5\\
930	-3.5\\
931	-3.5\\
932	-3.5\\
933	-3.5\\
934	-3.5\\
935	-3.5\\
936	-3.5\\
937	-3.5\\
938	-3.5\\
939	-3.5\\
940	-3.5\\
941	-3.5\\
942	-3.5\\
943	-3.5\\
944	-3.5\\
945	-3.5\\
946	-3.5\\
947	-3.5\\
948	-3.5\\
949	-3.5\\
950	-3.5\\
951	-3.5\\
952	-3.5\\
953	-3.5\\
954	-3.5\\
955	-3.5\\
956	-3.5\\
957	-3.5\\
958	-3.5\\
959	-3.5\\
960	-3.5\\
961	-3.5\\
962	-3.5\\
963	-3.5\\
964	-3.5\\
965	-3.5\\
966	-3.5\\
967	-3.5\\
968	-3.5\\
969	-3.5\\
970	-3.5\\
971	-3.5\\
972	-3.5\\
973	-3.5\\
974	-3.5\\
975	-3.5\\
976	-3.5\\
977	-3.5\\
978	-3.5\\
979	-3.5\\
980	-3.5\\
981	-3.5\\
982	-3.5\\
983	-3.5\\
984	-3.5\\
985	-3.5\\
986	-3.5\\
987	-3.5\\
988	-3.5\\
989	-3.5\\
990	-3.5\\
991	-3.5\\
992	-3.5\\
993	-3.5\\
994	-3.5\\
995	-3.5\\
996	-3.5\\
997	-3.5\\
998	-3.5\\
999	-3.5\\
1000	-1.2\\
1001	-1.2\\
1002	-1.2\\
1003	-1.2\\
1004	-1.2\\
1005	-1.2\\
1006	-1.2\\
1007	-1.2\\
1008	-1.2\\
1009	-1.2\\
1010	-1.2\\
1011	-1.2\\
1012	-1.2\\
1013	-1.2\\
1014	-1.2\\
1015	-1.2\\
1016	-1.2\\
1017	-1.2\\
1018	-1.2\\
1019	-1.2\\
1020	-1.2\\
1021	-1.2\\
1022	-1.2\\
1023	-1.2\\
1024	-1.2\\
1025	-1.2\\
1026	-1.2\\
1027	-1.2\\
1028	-1.2\\
1029	-1.2\\
1030	-1.2\\
1031	-1.2\\
1032	-1.2\\
1033	-1.2\\
1034	-1.2\\
1035	-1.2\\
1036	-1.2\\
1037	-1.2\\
1038	-1.2\\
1039	-1.2\\
1040	-1.2\\
1041	-1.2\\
1042	-1.2\\
1043	-1.2\\
1044	-1.2\\
1045	-1.2\\
1046	-1.2\\
1047	-1.2\\
1048	-1.2\\
1049	-1.2\\
1050	-1.2\\
1051	-1.2\\
1052	-1.2\\
1053	-1.2\\
1054	-1.2\\
1055	-1.2\\
1056	-1.2\\
1057	-1.2\\
1058	-1.2\\
1059	-1.2\\
1060	-1.2\\
1061	-1.2\\
1062	-1.2\\
1063	-1.2\\
1064	-1.2\\
1065	-1.2\\
1066	-1.2\\
1067	-1.2\\
1068	-1.2\\
1069	-1.2\\
1070	-1.2\\
1071	-1.2\\
1072	-1.2\\
1073	-1.2\\
1074	-1.2\\
1075	-1.2\\
1076	-1.2\\
1077	-1.2\\
1078	-1.2\\
1079	-1.2\\
1080	-1.2\\
1081	-1.2\\
1082	-1.2\\
1083	-1.2\\
1084	-1.2\\
1085	-1.2\\
1086	-1.2\\
1087	-1.2\\
1088	-1.2\\
1089	-1.2\\
1090	-1.2\\
1091	-1.2\\
1092	-1.2\\
1093	-1.2\\
1094	-1.2\\
1095	-1.2\\
1096	-1.2\\
1097	-1.2\\
1098	-1.2\\
1099	-1.2\\
1100	-1.2\\
1101	-1.2\\
1102	-1.2\\
1103	-1.2\\
1104	-1.2\\
1105	-1.2\\
1106	-1.2\\
1107	-1.2\\
1108	-1.2\\
1109	-1.2\\
1110	-1.2\\
1111	-1.2\\
1112	-1.2\\
1113	-1.2\\
1114	-1.2\\
1115	-1.2\\
1116	-1.2\\
1117	-1.2\\
1118	-1.2\\
1119	-1.2\\
1120	-1.2\\
1121	-1.2\\
1122	-1.2\\
1123	-1.2\\
1124	-1.2\\
1125	-1.2\\
1126	-1.2\\
1127	-1.2\\
1128	-1.2\\
1129	-1.2\\
1130	-1.2\\
1131	-1.2\\
1132	-1.2\\
1133	-1.2\\
1134	-1.2\\
1135	-1.2\\
1136	-1.2\\
1137	-1.2\\
1138	-1.2\\
1139	-1.2\\
1140	-1.2\\
1141	-1.2\\
1142	-1.2\\
1143	-1.2\\
1144	-1.2\\
1145	-1.2\\
1146	-1.2\\
1147	-1.2\\
1148	-1.2\\
1149	-1.2\\
1150	-1.2\\
1151	-1.2\\
1152	-1.2\\
1153	-1.2\\
1154	-1.2\\
1155	-1.2\\
1156	-1.2\\
1157	-1.2\\
1158	-1.2\\
1159	-1.2\\
1160	-1.2\\
1161	-1.2\\
1162	-1.2\\
1163	-1.2\\
1164	-1.2\\
1165	-1.2\\
1166	-1.2\\
1167	-1.2\\
1168	-1.2\\
1169	-1.2\\
1170	-1.2\\
1171	-1.2\\
1172	-1.2\\
1173	-1.2\\
1174	-1.2\\
1175	-1.2\\
1176	-1.2\\
1177	-1.2\\
1178	-1.2\\
1179	-1.2\\
1180	-1.2\\
1181	-1.2\\
1182	-1.2\\
1183	-1.2\\
1184	-1.2\\
1185	-1.2\\
1186	-1.2\\
1187	-1.2\\
1188	-1.2\\
1189	-1.2\\
1190	-1.2\\
1191	-1.2\\
1192	-1.2\\
1193	-1.2\\
1194	-1.2\\
1195	-1.2\\
1196	-1.2\\
1197	-1.2\\
1198	-1.2\\
1199	-1.2\\
1200	-0.3\\
1201	-0.3\\
1202	-0.3\\
1203	-0.3\\
1204	-0.3\\
1205	-0.3\\
1206	-0.3\\
1207	-0.3\\
1208	-0.3\\
1209	-0.3\\
1210	-0.3\\
1211	-0.3\\
1212	-0.3\\
1213	-0.3\\
1214	-0.3\\
1215	-0.3\\
1216	-0.3\\
1217	-0.3\\
1218	-0.3\\
1219	-0.3\\
1220	-0.3\\
1221	-0.3\\
1222	-0.3\\
1223	-0.3\\
1224	-0.3\\
1225	-0.3\\
1226	-0.3\\
1227	-0.3\\
1228	-0.3\\
1229	-0.3\\
1230	-0.3\\
1231	-0.3\\
1232	-0.3\\
1233	-0.3\\
1234	-0.3\\
1235	-0.3\\
1236	-0.3\\
1237	-0.3\\
1238	-0.3\\
1239	-0.3\\
1240	-0.3\\
1241	-0.3\\
1242	-0.3\\
1243	-0.3\\
1244	-0.3\\
1245	-0.3\\
1246	-0.3\\
1247	-0.3\\
1248	-0.3\\
1249	-0.3\\
1250	-0.3\\
1251	-0.3\\
1252	-0.3\\
1253	-0.3\\
1254	-0.3\\
1255	-0.3\\
1256	-0.3\\
1257	-0.3\\
1258	-0.3\\
1259	-0.3\\
1260	-0.3\\
1261	-0.3\\
1262	-0.3\\
1263	-0.3\\
1264	-0.3\\
1265	-0.3\\
1266	-0.3\\
1267	-0.3\\
1268	-0.3\\
1269	-0.3\\
1270	-0.3\\
1271	-0.3\\
1272	-0.3\\
1273	-0.3\\
1274	-0.3\\
1275	-0.3\\
1276	-0.3\\
1277	-0.3\\
1278	-0.3\\
1279	-0.3\\
1280	-0.3\\
1281	-0.3\\
1282	-0.3\\
1283	-0.3\\
1284	-0.3\\
1285	-0.3\\
1286	-0.3\\
1287	-0.3\\
1288	-0.3\\
1289	-0.3\\
1290	-0.3\\
1291	-0.3\\
1292	-0.3\\
1293	-0.3\\
1294	-0.3\\
1295	-0.3\\
1296	-0.3\\
1297	-0.3\\
1298	-0.3\\
1299	-0.3\\
1300	-0.3\\
1301	-0.3\\
1302	-0.3\\
1303	-0.3\\
1304	-0.3\\
1305	-0.3\\
1306	-0.3\\
1307	-0.3\\
1308	-0.3\\
1309	-0.3\\
1310	-0.3\\
1311	-0.3\\
1312	-0.3\\
1313	-0.3\\
1314	-0.3\\
1315	-0.3\\
1316	-0.3\\
1317	-0.3\\
1318	-0.3\\
1319	-0.3\\
1320	-0.3\\
1321	-0.3\\
1322	-0.3\\
1323	-0.3\\
1324	-0.3\\
1325	-0.3\\
1326	-0.3\\
1327	-0.3\\
1328	-0.3\\
1329	-0.3\\
1330	-0.3\\
1331	-0.3\\
1332	-0.3\\
1333	-0.3\\
1334	-0.3\\
1335	-0.3\\
1336	-0.3\\
1337	-0.3\\
1338	-0.3\\
1339	-0.3\\
1340	-0.3\\
1341	-0.3\\
1342	-0.3\\
1343	-0.3\\
1344	-0.3\\
1345	-0.3\\
1346	-0.3\\
1347	-0.3\\
1348	-0.3\\
1349	-0.3\\
1350	-0.3\\
1351	-0.3\\
1352	-0.3\\
1353	-0.3\\
1354	-0.3\\
1355	-0.3\\
1356	-0.3\\
1357	-0.3\\
1358	-0.3\\
1359	-0.3\\
1360	-0.3\\
1361	-0.3\\
1362	-0.3\\
1363	-0.3\\
1364	-0.3\\
1365	-0.3\\
1366	-0.3\\
1367	-0.3\\
1368	-0.3\\
1369	-0.3\\
1370	-0.3\\
1371	-0.3\\
1372	-0.3\\
1373	-0.3\\
1374	-0.3\\
1375	-0.3\\
1376	-0.3\\
1377	-0.3\\
1378	-0.3\\
1379	-0.3\\
1380	-0.3\\
1381	-0.3\\
1382	-0.3\\
1383	-0.3\\
1384	-0.3\\
1385	-0.3\\
1386	-0.3\\
1387	-0.3\\
1388	-0.3\\
1389	-0.3\\
1390	-0.3\\
1391	-0.3\\
1392	-0.3\\
1393	-0.3\\
1394	-0.3\\
1395	-0.3\\
1396	-0.3\\
1397	-0.3\\
1398	-0.3\\
1399	-0.3\\
1400	0\\
1401	0\\
1402	0\\
1403	0\\
1404	0\\
1405	0\\
1406	0\\
1407	0\\
1408	0\\
1409	0\\
1410	0\\
1411	0\\
1412	0\\
1413	0\\
1414	0\\
1415	0\\
1416	0\\
1417	0\\
1418	0\\
1419	0\\
1420	0\\
1421	0\\
1422	0\\
1423	0\\
1424	0\\
1425	0\\
1426	0\\
1427	0\\
1428	0\\
1429	0\\
1430	0\\
1431	0\\
1432	0\\
1433	0\\
1434	0\\
1435	0\\
1436	0\\
1437	0\\
1438	0\\
1439	0\\
1440	0\\
1441	0\\
1442	0\\
1443	0\\
1444	0\\
1445	0\\
1446	0\\
1447	0\\
1448	0\\
1449	0\\
1450	0\\
1451	0\\
1452	0\\
1453	0\\
1454	0\\
1455	0\\
1456	0\\
1457	0\\
1458	0\\
1459	0\\
1460	0\\
1461	0\\
1462	0\\
1463	0\\
1464	0\\
1465	0\\
1466	0\\
1467	0\\
1468	0\\
1469	0\\
1470	0\\
1471	0\\
1472	0\\
1473	0\\
1474	0\\
1475	0\\
1476	0\\
1477	0\\
1478	0\\
1479	0\\
1480	0\\
1481	0\\
1482	0\\
1483	0\\
1484	0\\
1485	0\\
1486	0\\
1487	0\\
1488	0\\
1489	0\\
1490	0\\
1491	0\\
1492	0\\
1493	0\\
1494	0\\
1495	0\\
1496	0\\
1497	0\\
1498	0\\
1499	0\\
1500	0\\
1501	0\\
1502	0\\
1503	0\\
1504	0\\
1505	0\\
1506	0\\
1507	0\\
1508	0\\
1509	0\\
1510	0\\
1511	0\\
1512	0\\
1513	0\\
1514	0\\
1515	0\\
1516	0\\
1517	0\\
1518	0\\
1519	0\\
1520	0\\
1521	0\\
1522	0\\
1523	0\\
1524	0\\
1525	0\\
1526	0\\
1527	0\\
1528	0\\
1529	0\\
1530	0\\
1531	0\\
1532	0\\
1533	0\\
1534	0\\
1535	0\\
1536	0\\
1537	0\\
1538	0\\
1539	0\\
1540	0\\
1541	0\\
1542	0\\
1543	0\\
1544	0\\
1545	0\\
1546	0\\
1547	0\\
1548	0\\
1549	0\\
1550	0\\
1551	0\\
1552	0\\
1553	0\\
1554	0\\
1555	0\\
1556	0\\
1557	0\\
1558	0\\
1559	0\\
1560	0\\
1561	0\\
1562	0\\
1563	0\\
1564	0\\
1565	0\\
1566	0\\
1567	0\\
1568	0\\
1569	0\\
1570	0\\
1571	0\\
1572	0\\
1573	0\\
1574	0\\
1575	0\\
1576	0\\
1577	0\\
1578	0\\
1579	0\\
1580	0\\
1581	0\\
1582	0\\
1583	0\\
1584	0\\
1585	0\\
1586	0\\
1587	0\\
1588	0\\
1589	0\\
1590	0\\
1591	0\\
1592	0\\
1593	0\\
1594	0\\
1595	0\\
1596	0\\
1597	0\\
1598	0\\
1599	0\\
1600	0.1\\
1601	0.1\\
1602	0.1\\
1603	0.1\\
1604	0.1\\
1605	0.1\\
1606	0.1\\
1607	0.1\\
1608	0.1\\
1609	0.1\\
1610	0.1\\
1611	0.1\\
1612	0.1\\
1613	0.1\\
1614	0.1\\
1615	0.1\\
1616	0.1\\
1617	0.1\\
1618	0.1\\
1619	0.1\\
1620	0.1\\
1621	0.1\\
1622	0.1\\
1623	0.1\\
1624	0.1\\
1625	0.1\\
1626	0.1\\
1627	0.1\\
1628	0.1\\
1629	0.1\\
1630	0.1\\
1631	0.1\\
1632	0.1\\
1633	0.1\\
1634	0.1\\
1635	0.1\\
1636	0.1\\
1637	0.1\\
1638	0.1\\
1639	0.1\\
1640	0.1\\
1641	0.1\\
1642	0.1\\
1643	0.1\\
1644	0.1\\
1645	0.1\\
1646	0.1\\
1647	0.1\\
1648	0.1\\
1649	0.1\\
1650	0.1\\
1651	0.1\\
1652	0.1\\
1653	0.1\\
1654	0.1\\
1655	0.1\\
1656	0.1\\
1657	0.1\\
1658	0.1\\
1659	0.1\\
1660	0.1\\
1661	0.1\\
1662	0.1\\
1663	0.1\\
1664	0.1\\
1665	0.1\\
1666	0.1\\
1667	0.1\\
1668	0.1\\
1669	0.1\\
1670	0.1\\
1671	0.1\\
1672	0.1\\
1673	0.1\\
1674	0.1\\
1675	0.1\\
1676	0.1\\
1677	0.1\\
1678	0.1\\
1679	0.1\\
1680	0.1\\
1681	0.1\\
1682	0.1\\
1683	0.1\\
1684	0.1\\
1685	0.1\\
1686	0.1\\
1687	0.1\\
1688	0.1\\
1689	0.1\\
1690	0.1\\
1691	0.1\\
1692	0.1\\
1693	0.1\\
1694	0.1\\
1695	0.1\\
1696	0.1\\
1697	0.1\\
1698	0.1\\
1699	0.1\\
1700	0.1\\
1701	0.1\\
1702	0.1\\
1703	0.1\\
1704	0.1\\
1705	0.1\\
1706	0.1\\
1707	0.1\\
1708	0.1\\
1709	0.1\\
1710	0.1\\
1711	0.1\\
1712	0.1\\
1713	0.1\\
1714	0.1\\
1715	0.1\\
1716	0.1\\
1717	0.1\\
1718	0.1\\
1719	0.1\\
1720	0.1\\
1721	0.1\\
1722	0.1\\
1723	0.1\\
1724	0.1\\
1725	0.1\\
1726	0.1\\
1727	0.1\\
1728	0.1\\
1729	0.1\\
1730	0.1\\
1731	0.1\\
1732	0.1\\
1733	0.1\\
1734	0.1\\
1735	0.1\\
1736	0.1\\
1737	0.1\\
1738	0.1\\
1739	0.1\\
1740	0.1\\
1741	0.1\\
1742	0.1\\
1743	0.1\\
1744	0.1\\
1745	0.1\\
1746	0.1\\
1747	0.1\\
1748	0.1\\
1749	0.1\\
1750	0.1\\
1751	0.1\\
1752	0.1\\
1753	0.1\\
1754	0.1\\
1755	0.1\\
1756	0.1\\
1757	0.1\\
1758	0.1\\
1759	0.1\\
1760	0.1\\
1761	0.1\\
1762	0.1\\
1763	0.1\\
1764	0.1\\
1765	0.1\\
1766	0.1\\
1767	0.1\\
1768	0.1\\
1769	0.1\\
1770	0.1\\
1771	0.1\\
1772	0.1\\
1773	0.1\\
1774	0.1\\
1775	0.1\\
1776	0.1\\
1777	0.1\\
1778	0.1\\
1779	0.1\\
1780	0.1\\
1781	0.1\\
1782	0.1\\
1783	0.1\\
1784	0.1\\
1785	0.1\\
1786	0.1\\
1787	0.1\\
1788	0.1\\
1789	0.1\\
1790	0.1\\
1791	0.1\\
1792	0.1\\
1793	0.1\\
1794	0.1\\
1795	0.1\\
1796	0.1\\
1797	0.1\\
1798	0.1\\
1799	0.1\\
1800	0.1\\
};
\addlegendentry{$\text{Wartość zadana y}_{\text{zad}}$}

\end{axis}

\begin{axis}[%
width=4.521in,
height=1.493in,
at={(0.758in,0.481in)},
scale only axis,
xmin=1,
xmax=1800,
xlabel style={font=\color{white!15!black}},
xlabel={k},
ymin=-1,
ymax=0.5,
ylabel style={font=\color{white!15!black}},
ylabel={u},
axis background/.style={fill=white},
xmajorgrids,
ymajorgrids,
legend style={legend cell align=left, align=left, draw=white!15!black}
]
\addplot [color=mycolor1]
  table[row sep=crcr]{%
1	0\\
2	0\\
3	0\\
4	0\\
5	0\\
6	0\\
7	0\\
8	0\\
9	0\\
10	0\\
11	0\\
12	0\\
13	0\\
14	0\\
15	0\\
16	0\\
17	0\\
18	0\\
19	0\\
20	-0.3399\\
21	-0.6634\\
22	-0.97925\\
23	-1\\
24	-1\\
25	-1\\
26	-0.98614\\
27	-0.95007\\
28	-0.89951\\
29	-0.83738\\
30	-0.77061\\
31	-0.70354\\
32	-0.64083\\
33	-0.58575\\
34	-0.54098\\
35	-0.50822\\
36	-0.4884\\
37	-0.48149\\
38	-0.48663\\
39	-0.502\\
40	-0.52505\\
41	-0.55267\\
42	-0.58158\\
43	-0.60877\\
44	-0.63178\\
45	-0.64897\\
46	-0.65956\\
47	-0.66354\\
48	-0.66155\\
49	-0.65464\\
50	-0.64415\\
51	-0.63146\\
52	-0.61792\\
53	-0.60474\\
54	-0.59291\\
55	-0.58319\\
56	-0.57604\\
57	-0.57167\\
58	-0.57004\\
59	-0.5709\\
60	-0.5738\\
61	-0.57821\\
62	-0.58351\\
63	-0.5891\\
64	-0.59445\\
65	-0.59911\\
66	-0.60277\\
67	-0.60525\\
68	-0.6065\\
69	-0.6066\\
70	-0.6057\\
71	-0.60402\\
72	-0.60182\\
73	-0.59937\\
74	-0.5969\\
75	-0.59462\\
76	-0.59271\\
77	-0.59125\\
78	-0.59032\\
79	-0.58991\\
80	-0.58998\\
81	-0.59046\\
82	-0.59124\\
83	-0.5922\\
84	-0.59325\\
85	-0.59427\\
86	-0.59519\\
87	-0.59593\\
88	-0.59646\\
89	-0.59677\\
90	-0.59686\\
91	-0.59675\\
92	-0.59648\\
93	-0.5961\\
94	-0.59566\\
95	-0.5952\\
96	-0.59477\\
97	-0.5944\\
98	-0.59411\\
99	-0.59391\\
100	-0.59381\\
101	-0.5938\\
102	-0.59387\\
103	-0.594\\
104	-0.59417\\
105	-0.59437\\
106	-0.59456\\
107	-0.59474\\
108	-0.59489\\
109	-0.595\\
110	-0.59506\\
111	-0.59509\\
112	-0.59508\\
113	-0.59504\\
114	-0.59498\\
115	-0.5949\\
116	-0.59481\\
117	-0.59473\\
118	-0.59466\\
119	-0.5946\\
120	-0.59456\\
121	-0.59454\\
122	-0.59453\\
123	-0.59454\\
124	-0.59456\\
125	-0.59459\\
126	-0.59463\\
127	-0.59466\\
128	-0.5947\\
129	-0.59472\\
130	-0.59475\\
131	-0.59476\\
132	-0.59477\\
133	-0.59477\\
134	-0.59476\\
135	-0.59475\\
136	-0.59474\\
137	-0.59472\\
138	-0.59471\\
139	-0.59469\\
140	-0.59468\\
141	-0.59467\\
142	-0.59467\\
143	-0.59467\\
144	-0.59467\\
145	-0.59467\\
146	-0.59468\\
147	-0.59468\\
148	-0.59469\\
149	-0.5947\\
150	-0.5947\\
151	-0.59471\\
152	-0.59471\\
153	-0.59471\\
154	-0.59471\\
155	-0.59471\\
156	-0.59471\\
157	-0.59471\\
158	-0.5947\\
159	-0.5947\\
160	-0.5947\\
161	-0.5947\\
162	-0.59469\\
163	-0.59469\\
164	-0.59469\\
165	-0.59469\\
166	-0.59469\\
167	-0.59469\\
168	-0.59469\\
169	-0.5947\\
170	-0.5947\\
171	-0.5947\\
172	-0.5947\\
173	-0.5947\\
174	-0.5947\\
175	-0.5947\\
176	-0.5947\\
177	-0.5947\\
178	-0.5947\\
179	-0.5947\\
180	-0.5947\\
181	-0.5947\\
182	-0.5947\\
183	-0.5947\\
184	-0.5947\\
185	-0.5947\\
186	-0.5947\\
187	-0.5947\\
188	-0.5947\\
189	-0.5947\\
190	-0.5947\\
191	-0.5947\\
192	-0.5947\\
193	-0.5947\\
194	-0.5947\\
195	-0.5947\\
196	-0.5947\\
197	-0.5947\\
198	-0.5947\\
199	-0.5947\\
200	-1\\
201	-1\\
202	-1\\
203	-1\\
204	-1\\
205	-0.98856\\
206	-0.96673\\
207	-0.93239\\
208	-0.901\\
209	-0.87435\\
210	-0.85682\\
211	-0.84801\\
212	-0.84802\\
213	-0.8555\\
214	-0.86914\\
215	-0.88716\\
216	-0.90782\\
217	-0.92933\\
218	-0.9501\\
219	-0.96879\\
220	-0.9844\\
221	-0.99629\\
222	-1\\
223	-1\\
224	-1\\
225	-0.9972\\
226	-0.99195\\
227	-0.98509\\
228	-0.97755\\
229	-0.97004\\
230	-0.96309\\
231	-0.95713\\
232	-0.95243\\
233	-0.94913\\
234	-0.94727\\
235	-0.94679\\
236	-0.94753\\
237	-0.94927\\
238	-0.95178\\
239	-0.95476\\
240	-0.95796\\
241	-0.96111\\
242	-0.96402\\
243	-0.96651\\
244	-0.96846\\
245	-0.96982\\
246	-0.97058\\
247	-0.97075\\
248	-0.97042\\
249	-0.96967\\
250	-0.96862\\
251	-0.96737\\
252	-0.96605\\
253	-0.96474\\
254	-0.96354\\
255	-0.96251\\
256	-0.96171\\
257	-0.96115\\
258	-0.96084\\
259	-0.96078\\
260	-0.96092\\
261	-0.96123\\
262	-0.96167\\
263	-0.96219\\
264	-0.96274\\
265	-0.96329\\
266	-0.96378\\
267	-0.96421\\
268	-0.96454\\
269	-0.96477\\
270	-0.96489\\
271	-0.96492\\
272	-0.96486\\
273	-0.96472\\
274	-0.96454\\
275	-0.96432\\
276	-0.96409\\
277	-0.96386\\
278	-0.96366\\
279	-0.96348\\
280	-0.96335\\
281	-0.96325\\
282	-0.9632\\
283	-0.96319\\
284	-0.96322\\
285	-0.96328\\
286	-0.96335\\
287	-0.96344\\
288	-0.96354\\
289	-0.96363\\
290	-0.96372\\
291	-0.96379\\
292	-0.96385\\
293	-0.96389\\
294	-0.96391\\
295	-0.96391\\
296	-0.9639\\
297	-0.96387\\
298	-0.96384\\
299	-0.9638\\
300	-0.96376\\
301	-0.96373\\
302	-0.96369\\
303	-0.96366\\
304	-0.96364\\
305	-0.96362\\
306	-0.96361\\
307	-0.96361\\
308	-0.96362\\
309	-0.96363\\
310	-0.96364\\
311	-0.96366\\
312	-0.96367\\
313	-0.96369\\
314	-0.9637\\
315	-0.96372\\
316	-0.96373\\
317	-0.96373\\
318	-0.96374\\
319	-0.96374\\
320	-0.96373\\
321	-0.96373\\
322	-0.96372\\
323	-0.96372\\
324	-0.96371\\
325	-0.9637\\
326	-0.9637\\
327	-0.96369\\
328	-0.96369\\
329	-0.96369\\
330	-0.96368\\
331	-0.96368\\
332	-0.96368\\
333	-0.96369\\
334	-0.96369\\
335	-0.96369\\
336	-0.96369\\
337	-0.9637\\
338	-0.9637\\
339	-0.9637\\
340	-0.9637\\
341	-0.9637\\
342	-0.96371\\
343	-0.96371\\
344	-0.96371\\
345	-0.9637\\
346	-0.9637\\
347	-0.9637\\
348	-0.9637\\
349	-0.9637\\
350	-0.9637\\
351	-0.9637\\
352	-0.9637\\
353	-0.9637\\
354	-0.9637\\
355	-0.9637\\
356	-0.9637\\
357	-0.9637\\
358	-0.9637\\
359	-0.9637\\
360	-0.9637\\
361	-0.9637\\
362	-0.9637\\
363	-0.9637\\
364	-0.9637\\
365	-0.9637\\
366	-0.9637\\
367	-0.9637\\
368	-0.9637\\
369	-0.9637\\
370	-0.9637\\
371	-0.9637\\
372	-0.9637\\
373	-0.9637\\
374	-0.9637\\
375	-0.9637\\
376	-0.9637\\
377	-0.9637\\
378	-0.9637\\
379	-0.9637\\
380	-0.9637\\
381	-0.9637\\
382	-0.9637\\
383	-0.9637\\
384	-0.9637\\
385	-0.9637\\
386	-0.9637\\
387	-0.9637\\
388	-0.9637\\
389	-0.9637\\
390	-0.9637\\
391	-0.9637\\
392	-0.9637\\
393	-0.9637\\
394	-0.9637\\
395	-0.9637\\
396	-0.9637\\
397	-0.9637\\
398	-0.9637\\
399	-0.9637\\
400	-0.71541\\
401	-0.6661\\
402	-0.57616\\
403	-0.55368\\
404	-0.53245\\
405	-0.54245\\
406	-0.56341\\
407	-0.60146\\
408	-0.64697\\
409	-0.69763\\
410	-0.74685\\
411	-0.79101\\
412	-0.82643\\
413	-0.85144\\
414	-0.86538\\
415	-0.86888\\
416	-0.86327\\
417	-0.85044\\
418	-0.83247\\
419	-0.81148\\
420	-0.78944\\
421	-0.76808\\
422	-0.74883\\
423	-0.73277\\
424	-0.7206\\
425	-0.71269\\
426	-0.70906\\
427	-0.70941\\
428	-0.7132\\
429	-0.71967\\
430	-0.72797\\
431	-0.73719\\
432	-0.74648\\
433	-0.75507\\
434	-0.76239\\
435	-0.76802\\
436	-0.77176\\
437	-0.77358\\
438	-0.77361\\
439	-0.77211\\
440	-0.76941\\
441	-0.76588\\
442	-0.76192\\
443	-0.75788\\
444	-0.75408\\
445	-0.75078\\
446	-0.74816\\
447	-0.74631\\
448	-0.74529\\
449	-0.74504\\
450	-0.74548\\
451	-0.74648\\
452	-0.74789\\
453	-0.74954\\
454	-0.75127\\
455	-0.75293\\
456	-0.7544\\
457	-0.7556\\
458	-0.75648\\
459	-0.757\\
460	-0.75718\\
461	-0.75705\\
462	-0.75667\\
463	-0.75609\\
464	-0.7554\\
465	-0.75466\\
466	-0.75394\\
467	-0.75328\\
468	-0.75273\\
469	-0.75232\\
470	-0.75206\\
471	-0.75194\\
472	-0.75196\\
473	-0.7521\\
474	-0.75232\\
475	-0.75261\\
476	-0.75292\\
477	-0.75324\\
478	-0.75353\\
479	-0.75378\\
480	-0.75397\\
481	-0.7541\\
482	-0.75416\\
483	-0.75417\\
484	-0.75412\\
485	-0.75403\\
486	-0.75391\\
487	-0.75378\\
488	-0.75365\\
489	-0.75352\\
490	-0.75341\\
491	-0.75332\\
492	-0.75326\\
493	-0.75322\\
494	-0.75321\\
495	-0.75323\\
496	-0.75326\\
497	-0.75331\\
498	-0.75336\\
499	-0.75342\\
500	-0.75348\\
501	-0.75353\\
502	-0.75357\\
503	-0.7536\\
504	-0.75362\\
505	-0.75362\\
506	-0.75362\\
507	-0.75361\\
508	-0.75359\\
509	-0.75356\\
510	-0.75354\\
511	-0.75351\\
512	-0.75349\\
513	-0.75347\\
514	-0.75346\\
515	-0.75345\\
516	-0.75345\\
517	-0.75345\\
518	-0.75345\\
519	-0.75346\\
520	-0.75347\\
521	-0.75348\\
522	-0.75349\\
523	-0.7535\\
524	-0.75351\\
525	-0.75352\\
526	-0.75352\\
527	-0.75352\\
528	-0.75352\\
529	-0.75352\\
530	-0.75352\\
531	-0.75351\\
532	-0.75351\\
533	-0.7535\\
534	-0.7535\\
535	-0.7535\\
536	-0.75349\\
537	-0.75349\\
538	-0.75349\\
539	-0.75349\\
540	-0.75349\\
541	-0.75349\\
542	-0.75349\\
543	-0.75349\\
544	-0.7535\\
545	-0.7535\\
546	-0.7535\\
547	-0.7535\\
548	-0.7535\\
549	-0.7535\\
550	-0.7535\\
551	-0.7535\\
552	-0.7535\\
553	-0.7535\\
554	-0.7535\\
555	-0.7535\\
556	-0.7535\\
557	-0.7535\\
558	-0.7535\\
559	-0.7535\\
560	-0.7535\\
561	-0.7535\\
562	-0.7535\\
563	-0.7535\\
564	-0.7535\\
565	-0.7535\\
566	-0.7535\\
567	-0.7535\\
568	-0.7535\\
569	-0.7535\\
570	-0.7535\\
571	-0.7535\\
572	-0.7535\\
573	-0.7535\\
574	-0.7535\\
575	-0.7535\\
576	-0.7535\\
577	-0.7535\\
578	-0.7535\\
579	-0.7535\\
580	-0.7535\\
581	-0.7535\\
582	-0.7535\\
583	-0.7535\\
584	-0.7535\\
585	-0.7535\\
586	-0.7535\\
587	-0.7535\\
588	-0.7535\\
589	-0.7535\\
590	-0.7535\\
591	-0.7535\\
592	-0.7535\\
593	-0.7535\\
594	-0.7535\\
595	-0.7535\\
596	-0.7535\\
597	-0.7535\\
598	-0.7535\\
599	-0.7535\\
600	-0.50521\\
601	-0.4559\\
602	-0.36596\\
603	-0.34348\\
604	-0.32225\\
605	-0.33316\\
606	-0.3554\\
607	-0.39495\\
608	-0.44081\\
609	-0.48977\\
610	-0.53446\\
611	-0.57127\\
612	-0.59706\\
613	-0.61116\\
614	-0.61403\\
615	-0.60737\\
616	-0.59337\\
617	-0.5745\\
618	-0.55317\\
619	-0.53156\\
620	-0.5115\\
621	-0.49441\\
622	-0.48126\\
623	-0.47257\\
624	-0.46844\\
625	-0.46857\\
626	-0.47232\\
627	-0.4788\\
628	-0.48699\\
629	-0.49585\\
630	-0.50441\\
631	-0.51188\\
632	-0.5177\\
633	-0.52156\\
634	-0.5234\\
635	-0.52334\\
636	-0.52169\\
637	-0.51884\\
638	-0.51523\\
639	-0.51129\\
640	-0.50743\\
641	-0.50399\\
642	-0.50119\\
643	-0.4992\\
644	-0.49807\\
645	-0.49776\\
646	-0.49817\\
647	-0.49916\\
648	-0.50055\\
649	-0.50213\\
650	-0.50373\\
651	-0.5052\\
652	-0.50642\\
653	-0.50732\\
654	-0.50785\\
655	-0.50803\\
656	-0.50788\\
657	-0.50748\\
658	-0.5069\\
659	-0.50622\\
660	-0.50552\\
661	-0.50485\\
662	-0.50429\\
663	-0.50386\\
664	-0.50358\\
665	-0.50345\\
666	-0.50347\\
667	-0.5036\\
668	-0.50383\\
669	-0.5041\\
670	-0.50439\\
671	-0.50467\\
672	-0.50492\\
673	-0.50511\\
674	-0.50524\\
675	-0.50531\\
676	-0.50531\\
677	-0.50526\\
678	-0.50517\\
679	-0.50506\\
680	-0.50493\\
681	-0.50481\\
682	-0.5047\\
683	-0.50461\\
684	-0.50455\\
685	-0.50451\\
686	-0.5045\\
687	-0.50452\\
688	-0.50455\\
689	-0.50459\\
690	-0.50465\\
691	-0.5047\\
692	-0.50475\\
693	-0.50479\\
694	-0.50482\\
695	-0.50483\\
696	-0.50484\\
697	-0.50484\\
698	-0.50482\\
699	-0.5048\\
700	-0.50478\\
701	-0.50476\\
702	-0.50474\\
703	-0.50472\\
704	-0.50471\\
705	-0.5047\\
706	-0.50469\\
707	-0.50469\\
708	-0.5047\\
709	-0.50471\\
710	-0.50472\\
711	-0.50472\\
712	-0.50473\\
713	-0.50474\\
714	-0.50475\\
715	-0.50475\\
716	-0.50475\\
717	-0.50476\\
718	-0.50475\\
719	-0.50475\\
720	-0.50475\\
721	-0.50474\\
722	-0.50474\\
723	-0.50474\\
724	-0.50473\\
725	-0.50473\\
726	-0.50473\\
727	-0.50473\\
728	-0.50473\\
729	-0.50473\\
730	-0.50473\\
731	-0.50473\\
732	-0.50474\\
733	-0.50474\\
734	-0.50474\\
735	-0.50474\\
736	-0.50474\\
737	-0.50474\\
738	-0.50474\\
739	-0.50474\\
740	-0.50474\\
741	-0.50474\\
742	-0.50474\\
743	-0.50474\\
744	-0.50474\\
745	-0.50474\\
746	-0.50474\\
747	-0.50474\\
748	-0.50474\\
749	-0.50474\\
750	-0.50474\\
751	-0.50474\\
752	-0.50474\\
753	-0.50474\\
754	-0.50474\\
755	-0.50474\\
756	-0.50474\\
757	-0.50474\\
758	-0.50474\\
759	-0.50474\\
760	-0.50474\\
761	-0.50474\\
762	-0.50474\\
763	-0.50474\\
764	-0.50474\\
765	-0.50474\\
766	-0.50474\\
767	-0.50474\\
768	-0.50474\\
769	-0.50474\\
770	-0.50474\\
771	-0.50474\\
772	-0.50474\\
773	-0.50474\\
774	-0.50474\\
775	-0.50474\\
776	-0.50474\\
777	-0.50474\\
778	-0.50474\\
779	-0.50474\\
780	-0.50474\\
781	-0.50474\\
782	-0.50474\\
783	-0.50474\\
784	-0.50474\\
785	-0.50474\\
786	-0.50474\\
787	-0.50474\\
788	-0.50474\\
789	-0.50474\\
790	-0.50474\\
791	-0.50474\\
792	-0.50474\\
793	-0.50474\\
794	-0.50474\\
795	-0.50474\\
796	-0.50474\\
797	-0.50474\\
798	-0.50474\\
799	-0.50474\\
800	-0.83578\\
801	-0.90154\\
802	-1\\
803	-1\\
804	-1\\
805	-0.9905\\
806	-0.97061\\
807	-0.93525\\
808	-0.89611\\
809	-0.8556\\
810	-0.81884\\
811	-0.78737\\
812	-0.76302\\
813	-0.7463\\
814	-0.73746\\
815	-0.73606\\
816	-0.74133\\
817	-0.75206\\
818	-0.76681\\
819	-0.78396\\
820	-0.80191\\
821	-0.81919\\
822	-0.83456\\
823	-0.84712\\
824	-0.85631\\
825	-0.86193\\
826	-0.86408\\
827	-0.86312\\
828	-0.85957\\
829	-0.85407\\
830	-0.84732\\
831	-0.83999\\
832	-0.83268\\
833	-0.82593\\
834	-0.82015\\
835	-0.81561\\
836	-0.81249\\
837	-0.81081\\
838	-0.81051\\
839	-0.81142\\
840	-0.81331\\
841	-0.81591\\
842	-0.81893\\
843	-0.82208\\
844	-0.82512\\
845	-0.82784\\
846	-0.83007\\
847	-0.83171\\
848	-0.83273\\
849	-0.83314\\
850	-0.83299\\
851	-0.83237\\
852	-0.8314\\
853	-0.83018\\
854	-0.82885\\
855	-0.82752\\
856	-0.82629\\
857	-0.82523\\
858	-0.8244\\
859	-0.82383\\
860	-0.82352\\
861	-0.82346\\
862	-0.82363\\
863	-0.82397\\
864	-0.82444\\
865	-0.82498\\
866	-0.82555\\
867	-0.82611\\
868	-0.8266\\
869	-0.827\\
870	-0.8273\\
871	-0.82749\\
872	-0.82757\\
873	-0.82754\\
874	-0.82743\\
875	-0.82726\\
876	-0.82704\\
877	-0.8268\\
878	-0.82656\\
879	-0.82634\\
880	-0.82614\\
881	-0.82599\\
882	-0.82589\\
883	-0.82583\\
884	-0.82582\\
885	-0.82585\\
886	-0.82591\\
887	-0.82599\\
888	-0.82609\\
889	-0.8262\\
890	-0.8263\\
891	-0.82638\\
892	-0.82646\\
893	-0.82651\\
894	-0.82655\\
895	-0.82656\\
896	-0.82656\\
897	-0.82654\\
898	-0.82651\\
899	-0.82647\\
900	-0.82642\\
901	-0.82638\\
902	-0.82634\\
903	-0.82631\\
904	-0.82628\\
905	-0.82626\\
906	-0.82625\\
907	-0.82625\\
908	-0.82625\\
909	-0.82626\\
910	-0.82628\\
911	-0.82629\\
912	-0.82631\\
913	-0.82633\\
914	-0.82635\\
915	-0.82636\\
916	-0.82637\\
917	-0.82638\\
918	-0.82638\\
919	-0.82638\\
920	-0.82638\\
921	-0.82637\\
922	-0.82636\\
923	-0.82636\\
924	-0.82635\\
925	-0.82634\\
926	-0.82633\\
927	-0.82633\\
928	-0.82633\\
929	-0.82632\\
930	-0.82632\\
931	-0.82632\\
932	-0.82633\\
933	-0.82633\\
934	-0.82633\\
935	-0.82634\\
936	-0.82634\\
937	-0.82634\\
938	-0.82634\\
939	-0.82635\\
940	-0.82635\\
941	-0.82635\\
942	-0.82635\\
943	-0.82635\\
944	-0.82635\\
945	-0.82634\\
946	-0.82634\\
947	-0.82634\\
948	-0.82634\\
949	-0.82634\\
950	-0.82634\\
951	-0.82634\\
952	-0.82634\\
953	-0.82634\\
954	-0.82634\\
955	-0.82634\\
956	-0.82634\\
957	-0.82634\\
958	-0.82634\\
959	-0.82634\\
960	-0.82634\\
961	-0.82634\\
962	-0.82634\\
963	-0.82634\\
964	-0.82634\\
965	-0.82634\\
966	-0.82634\\
967	-0.82634\\
968	-0.82634\\
969	-0.82634\\
970	-0.82634\\
971	-0.82634\\
972	-0.82634\\
973	-0.82634\\
974	-0.82634\\
975	-0.82634\\
976	-0.82634\\
977	-0.82634\\
978	-0.82634\\
979	-0.82634\\
980	-0.82634\\
981	-0.82634\\
982	-0.82634\\
983	-0.82634\\
984	-0.82634\\
985	-0.82634\\
986	-0.82634\\
987	-0.82634\\
988	-0.82634\\
989	-0.82634\\
990	-0.82634\\
991	-0.82634\\
992	-0.82634\\
993	-0.82634\\
994	-0.82634\\
995	-0.82634\\
996	-0.82634\\
997	-0.82634\\
998	-0.82634\\
999	-0.82634\\
1000	-0.44564\\
1001	-0.37002\\
1002	-0.23212\\
1003	-0.19764\\
1004	-0.1651\\
1005	-0.18377\\
1006	-0.22197\\
1007	-0.29025\\
1008	-0.3688\\
1009	-0.45047\\
1010	-0.52149\\
1011	-0.57591\\
1012	-0.60963\\
1013	-0.62321\\
1014	-0.61898\\
1015	-0.6009\\
1016	-0.57321\\
1017	-0.54014\\
1018	-0.50544\\
1019	-0.47226\\
1020	-0.44307\\
1021	-0.41966\\
1022	-0.4031\\
1023	-0.39383\\
1024	-0.39161\\
1025	-0.39562\\
1026	-0.40456\\
1027	-0.41678\\
1028	-0.43053\\
1029	-0.4441\\
1030	-0.4561\\
1031	-0.46551\\
1032	-0.47176\\
1033	-0.47472\\
1034	-0.47462\\
1035	-0.47195\\
1036	-0.46738\\
1037	-0.46165\\
1038	-0.45546\\
1039	-0.44946\\
1040	-0.44417\\
1041	-0.43996\\
1042	-0.43705\\
1043	-0.43549\\
1044	-0.43522\\
1045	-0.43604\\
1046	-0.43769\\
1047	-0.43987\\
1048	-0.44227\\
1049	-0.44462\\
1050	-0.44669\\
1051	-0.44831\\
1052	-0.44939\\
1053	-0.44991\\
1054	-0.4499\\
1055	-0.44945\\
1056	-0.44866\\
1057	-0.44767\\
1058	-0.44661\\
1059	-0.44557\\
1060	-0.44467\\
1061	-0.44395\\
1062	-0.44346\\
1063	-0.44321\\
1064	-0.44317\\
1065	-0.44333\\
1066	-0.44363\\
1067	-0.44401\\
1068	-0.44443\\
1069	-0.44483\\
1070	-0.44519\\
1071	-0.44546\\
1072	-0.44565\\
1073	-0.44574\\
1074	-0.44573\\
1075	-0.44565\\
1076	-0.44552\\
1077	-0.44535\\
1078	-0.44516\\
1079	-0.44499\\
1080	-0.44483\\
1081	-0.44471\\
1082	-0.44463\\
1083	-0.44459\\
1084	-0.44459\\
1085	-0.44461\\
1086	-0.44467\\
1087	-0.44473\\
1088	-0.44481\\
1089	-0.44488\\
1090	-0.44494\\
1091	-0.44499\\
1092	-0.44502\\
1093	-0.44503\\
1094	-0.44503\\
1095	-0.44502\\
1096	-0.44499\\
1097	-0.44496\\
1098	-0.44493\\
1099	-0.4449\\
1100	-0.44487\\
1101	-0.44485\\
1102	-0.44484\\
1103	-0.44483\\
1104	-0.44483\\
1105	-0.44484\\
1106	-0.44485\\
1107	-0.44486\\
1108	-0.44487\\
1109	-0.44489\\
1110	-0.4449\\
1111	-0.4449\\
1112	-0.44491\\
1113	-0.44491\\
1114	-0.44491\\
1115	-0.44491\\
1116	-0.4449\\
1117	-0.4449\\
1118	-0.44489\\
1119	-0.44489\\
1120	-0.44488\\
1121	-0.44488\\
1122	-0.44488\\
1123	-0.44488\\
1124	-0.44488\\
1125	-0.44488\\
1126	-0.44488\\
1127	-0.44488\\
1128	-0.44488\\
1129	-0.44489\\
1130	-0.44489\\
1131	-0.44489\\
1132	-0.44489\\
1133	-0.44489\\
1134	-0.44489\\
1135	-0.44489\\
1136	-0.44489\\
1137	-0.44489\\
1138	-0.44489\\
1139	-0.44489\\
1140	-0.44489\\
1141	-0.44489\\
1142	-0.44489\\
1143	-0.44488\\
1144	-0.44489\\
1145	-0.44489\\
1146	-0.44489\\
1147	-0.44489\\
1148	-0.44489\\
1149	-0.44489\\
1150	-0.44489\\
1151	-0.44489\\
1152	-0.44489\\
1153	-0.44489\\
1154	-0.44489\\
1155	-0.44489\\
1156	-0.44489\\
1157	-0.44489\\
1158	-0.44489\\
1159	-0.44489\\
1160	-0.44489\\
1161	-0.44489\\
1162	-0.44489\\
1163	-0.44489\\
1164	-0.44489\\
1165	-0.44489\\
1166	-0.44489\\
1167	-0.44489\\
1168	-0.44489\\
1169	-0.44489\\
1170	-0.44489\\
1171	-0.44489\\
1172	-0.44489\\
1173	-0.44489\\
1174	-0.44489\\
1175	-0.44489\\
1176	-0.44489\\
1177	-0.44489\\
1178	-0.44489\\
1179	-0.44489\\
1180	-0.44489\\
1181	-0.44489\\
1182	-0.44489\\
1183	-0.44489\\
1184	-0.44489\\
1185	-0.44489\\
1186	-0.44489\\
1187	-0.44489\\
1188	-0.44489\\
1189	-0.44489\\
1190	-0.44489\\
1191	-0.44489\\
1192	-0.44489\\
1193	-0.44489\\
1194	-0.44489\\
1195	-0.44489\\
1196	-0.44489\\
1197	-0.44489\\
1198	-0.44489\\
1199	-0.44489\\
1200	-0.29592\\
1201	-0.26633\\
1202	-0.21237\\
1203	-0.19887\\
1204	-0.18614\\
1205	-0.19208\\
1206	-0.20337\\
1207	-0.22232\\
1208	-0.24214\\
1209	-0.26103\\
1210	-0.27558\\
1211	-0.28476\\
1212	-0.28799\\
1213	-0.28587\\
1214	-0.27942\\
1215	-0.27006\\
1216	-0.25921\\
1217	-0.24818\\
1218	-0.23804\\
1219	-0.22956\\
1220	-0.22319\\
1221	-0.21904\\
1222	-0.21696\\
1223	-0.21658\\
1224	-0.21742\\
1225	-0.21894\\
1226	-0.22062\\
1227	-0.22205\\
1228	-0.22295\\
1229	-0.22315\\
1230	-0.22265\\
1231	-0.22151\\
1232	-0.21989\\
1233	-0.21797\\
1234	-0.21593\\
1235	-0.21396\\
1236	-0.21216\\
1237	-0.21064\\
1238	-0.20941\\
1239	-0.20849\\
1240	-0.20783\\
1241	-0.20738\\
1242	-0.20706\\
1243	-0.20682\\
1244	-0.20658\\
1245	-0.20632\\
1246	-0.20601\\
1247	-0.20562\\
1248	-0.20518\\
1249	-0.20468\\
1250	-0.20416\\
1251	-0.20362\\
1252	-0.2031\\
1253	-0.20261\\
1254	-0.20216\\
1255	-0.20175\\
1256	-0.20139\\
1257	-0.20108\\
1258	-0.2008\\
1259	-0.20055\\
1260	-0.20032\\
1261	-0.2001\\
1262	-0.19989\\
1263	-0.19969\\
1264	-0.19948\\
1265	-0.19927\\
1266	-0.19907\\
1267	-0.19887\\
1268	-0.19867\\
1269	-0.19848\\
1270	-0.1983\\
1271	-0.19813\\
1272	-0.19797\\
1273	-0.19782\\
1274	-0.19768\\
1275	-0.19755\\
1276	-0.19742\\
1277	-0.19731\\
1278	-0.1972\\
1279	-0.19709\\
1280	-0.19699\\
1281	-0.19689\\
1282	-0.1968\\
1283	-0.19671\\
1284	-0.19662\\
1285	-0.19654\\
1286	-0.19646\\
1287	-0.19638\\
1288	-0.19631\\
1289	-0.19624\\
1290	-0.19617\\
1291	-0.19611\\
1292	-0.19605\\
1293	-0.19599\\
1294	-0.19594\\
1295	-0.19589\\
1296	-0.19584\\
1297	-0.19579\\
1298	-0.19575\\
1299	-0.1957\\
1300	-0.19566\\
1301	-0.19562\\
1302	-0.19559\\
1303	-0.19555\\
1304	-0.19552\\
1305	-0.19548\\
1306	-0.19545\\
1307	-0.19542\\
1308	-0.1954\\
1309	-0.19537\\
1310	-0.19534\\
1311	-0.19532\\
1312	-0.1953\\
1313	-0.19527\\
1314	-0.19525\\
1315	-0.19523\\
1316	-0.19521\\
1317	-0.19519\\
1318	-0.19518\\
1319	-0.19516\\
1320	-0.19514\\
1321	-0.19513\\
1322	-0.19511\\
1323	-0.1951\\
1324	-0.19509\\
1325	-0.19507\\
1326	-0.19506\\
1327	-0.19505\\
1328	-0.19504\\
1329	-0.19503\\
1330	-0.19502\\
1331	-0.19501\\
1332	-0.195\\
1333	-0.19499\\
1334	-0.19498\\
1335	-0.19498\\
1336	-0.19497\\
1337	-0.19496\\
1338	-0.19495\\
1339	-0.19495\\
1340	-0.19494\\
1341	-0.19494\\
1342	-0.19493\\
1343	-0.19492\\
1344	-0.19492\\
1345	-0.19491\\
1346	-0.19491\\
1347	-0.1949\\
1348	-0.1949\\
1349	-0.1949\\
1350	-0.19489\\
1351	-0.19489\\
1352	-0.19489\\
1353	-0.19488\\
1354	-0.19488\\
1355	-0.19488\\
1356	-0.19487\\
1357	-0.19487\\
1358	-0.19487\\
1359	-0.19486\\
1360	-0.19486\\
1361	-0.19486\\
1362	-0.19486\\
1363	-0.19486\\
1364	-0.19485\\
1365	-0.19485\\
1366	-0.19485\\
1367	-0.19485\\
1368	-0.19485\\
1369	-0.19484\\
1370	-0.19484\\
1371	-0.19484\\
1372	-0.19484\\
1373	-0.19484\\
1374	-0.19484\\
1375	-0.19484\\
1376	-0.19483\\
1377	-0.19483\\
1378	-0.19483\\
1379	-0.19483\\
1380	-0.19483\\
1381	-0.19483\\
1382	-0.19483\\
1383	-0.19483\\
1384	-0.19483\\
1385	-0.19483\\
1386	-0.19483\\
1387	-0.19482\\
1388	-0.19482\\
1389	-0.19482\\
1390	-0.19482\\
1391	-0.19482\\
1392	-0.19482\\
1393	-0.19482\\
1394	-0.19482\\
1395	-0.19482\\
1396	-0.19482\\
1397	-0.19482\\
1398	-0.19482\\
1399	-0.19482\\
1400	-0.14516\\
1401	-0.1353\\
1402	-0.11731\\
1403	-0.11281\\
1404	-0.10857\\
1405	-0.10971\\
1406	-0.11169\\
1407	-0.11498\\
1408	-0.11767\\
1409	-0.11949\\
1410	-0.11981\\
1411	-0.11868\\
1412	-0.11622\\
1413	-0.11279\\
1414	-0.10878\\
1415	-0.10458\\
1416	-0.10052\\
1417	-0.096829\\
1418	-0.093649\\
1419	-0.091007\\
1420	-0.08886\\
1421	-0.087108\\
1422	-0.085627\\
1423	-0.084294\\
1424	-0.083\\
1425	-0.081668\\
1426	-0.080252\\
1427	-0.078739\\
1428	-0.074968\\
1429	-0.07122\\
1430	-0.067513\\
1431	-0.063869\\
1432	-0.058386\\
1433	-0.053081\\
1434	-0.048011\\
1435	-0.043219\\
1436	-0.036994\\
1437	-0.031174\\
1438	-0.024226\\
1439	-0.01785\\
1440	-0.010691\\
1441	-0.0042532\\
1442	0.002567\\
1443	0.0084896\\
1444	0.013494\\
1445	0.017565\\
1446	0.020703\\
1447	0.022922\\
1448	0.024252\\
1449	0.024749\\
1450	0.024491\\
1451	0.023571\\
1452	0.022095\\
1453	0.020169\\
1454	0.017903\\
1455	0.015397\\
1456	0.012747\\
1457	0.010043\\
1458	0.0073606\\
1459	0.0047698\\
1460	0.0023289\\
1461	8.6335e-05\\
1462	-0.0019191\\
1463	-0.0036586\\
1464	-0.0051123\\
1465	-0.0062696\\
1466	-0.0071283\\
1467	-0.0076938\\
1468	-0.007979\\
1469	-0.0080029\\
1470	-0.0077898\\
1471	-0.0073686\\
1472	-0.0067713\\
1473	-0.0060322\\
1474	-0.0051867\\
1475	-0.0042698\\
1476	-0.0033158\\
1477	-0.0023566\\
1478	-0.0014214\\
1479	-0.00053577\\
1480	0.00027867\\
1481	0.0010046\\
1482	0.0016292\\
1483	0.0021441\\
1484	0.002545\\
1485	0.0028316\\
1486	0.003007\\
1487	0.0030774\\
1488	0.0030516\\
1489	0.0029402\\
1490	0.0027551\\
1491	0.0025093\\
1492	0.0022162\\
1493	0.0018891\\
1494	0.0015408\\
1495	0.0011837\\
1496	0.00082878\\
1497	0.0004861\\
1498	0.00016424\\
1499	-0.00012967\\
1500	-0.00039001\\
1501	-0.0006127\\
1502	-0.00079512\\
1503	-0.00093607\\
1504	-0.0010356\\
1505	-0.0010951\\
1506	-0.0011168\\
1507	-0.0011038\\
1508	-0.00106\\
1509	-0.00098981\\
1510	-0.0008978\\
1511	-0.00078888\\
1512	-0.0006679\\
1513	-0.00053959\\
1514	-0.00040843\\
1515	-0.00027852\\
1516	-0.00015353\\
1517	-3.6613e-05\\
1518	6.9638e-05\\
1519	0.00016319\\
1520	0.00024261\\
1521	0.00030698\\
1522	0.00035596\\
1523	0.00038967\\
1524	0.00040867\\
1525	0.00041388\\
1526	0.00040656\\
1527	0.00038817\\
1528	0.00036037\\
1529	0.00032491\\
1530	0.00028361\\
1531	0.00023824\\
1532	0.00019053\\
1533	0.00014211\\
1534	9.4444e-05\\
1535	4.8845e-05\\
1536	6.4257e-06\\
1537	-3.1905e-05\\
1538	-6.5446e-05\\
1539	-9.3708e-05\\
1540	-0.0001164\\
1541	-0.00013343\\
1542	-0.00014487\\
1543	-0.00015095\\
1544	-0.00015203\\
1545	-0.00014859\\
1546	-0.00014118\\
1547	-0.00013041\\
1548	-0.00011694\\
1549	-0.00010142\\
1550	-8.4519e-05\\
1551	-6.6863e-05\\
1552	-4.9041e-05\\
1553	-3.1587e-05\\
1554	-1.4972e-05\\
1555	4.0567e-07\\
1556	1.4224e-05\\
1557	2.6239e-05\\
1558	3.6282e-05\\
1559	4.426e-05\\
1560	5.015e-05\\
1561	5.399e-05\\
1562	5.5875e-05\\
1563	5.5947e-05\\
1564	5.4388e-05\\
1565	5.1406e-05\\
1566	4.7231e-05\\
1567	4.2106e-05\\
1568	3.6273e-05\\
1569	2.9975e-05\\
1570	2.344e-05\\
1571	1.6883e-05\\
1572	1.0496e-05\\
1573	4.447e-06\\
1574	-1.1219e-06\\
1575	-6.0974e-06\\
1576	-1.0395e-05\\
1577	-1.3958e-05\\
1578	-1.6756e-05\\
1579	-1.8787e-05\\
1580	-2.0067e-05\\
1581	-2.0635e-05\\
1582	-2.0547e-05\\
1583	-1.9871e-05\\
1584	-1.8686e-05\\
1585	-1.7078e-05\\
1586	-1.5135e-05\\
1587	-1.2948e-05\\
1588	-1.0606e-05\\
1589	-8.1909e-06\\
1590	-5.7813e-06\\
1591	-3.4463e-06\\
1592	-1.2463e-06\\
1593	7.6843e-07\\
1594	2.558e-06\\
1595	4.0931e-06\\
1596	5.3548e-06\\
1597	6.3339e-06\\
1598	7.0305e-06\\
1599	7.4527e-06\\
1600	0.017003\\
1601	0.033177\\
1602	0.04897\\
1603	0.064374\\
1604	0.079399\\
1605	0.09393\\
1606	0.10771\\
1607	0.1205\\
1608	0.13212\\
1609	0.14246\\
1610	0.15154\\
1611	0.15939\\
1612	0.16613\\
1613	0.17188\\
1614	0.17677\\
1615	0.18093\\
1616	0.18448\\
1617	0.18754\\
1618	0.19018\\
1619	0.19249\\
1620	0.19452\\
1621	0.19634\\
1622	0.19798\\
1623	0.19947\\
1624	0.20084\\
1625	0.20211\\
1626	0.2033\\
1627	0.20441\\
1628	0.20545\\
1629	0.20644\\
1630	0.20738\\
1631	0.20828\\
1632	0.20913\\
1633	0.20995\\
1634	0.21073\\
1635	0.21148\\
1636	0.2122\\
1637	0.21289\\
1638	0.21355\\
1639	0.21419\\
1640	0.21481\\
1641	0.2154\\
1642	0.21597\\
1643	0.21652\\
1644	0.21705\\
1645	0.21756\\
1646	0.21805\\
1647	0.21853\\
1648	0.21898\\
1649	0.21943\\
1650	0.21985\\
1651	0.22027\\
1652	0.22067\\
1653	0.22105\\
1654	0.22142\\
1655	0.22178\\
1656	0.22213\\
1657	0.22246\\
1658	0.22279\\
1659	0.2231\\
1660	0.22341\\
1661	0.2237\\
1662	0.22398\\
1663	0.22426\\
1664	0.22453\\
1665	0.22478\\
1666	0.22503\\
1667	0.22527\\
1668	0.22551\\
1669	0.22573\\
1670	0.22595\\
1671	0.22616\\
1672	0.22637\\
1673	0.22656\\
1674	0.22676\\
1675	0.22694\\
1676	0.22712\\
1677	0.2273\\
1678	0.22747\\
1679	0.22763\\
1680	0.22779\\
1681	0.22794\\
1682	0.22809\\
1683	0.22824\\
1684	0.22838\\
1685	0.22851\\
1686	0.22864\\
1687	0.22877\\
1688	0.2289\\
1689	0.22902\\
1690	0.22913\\
1691	0.22924\\
1692	0.22935\\
1693	0.22946\\
1694	0.22956\\
1695	0.22966\\
1696	0.22976\\
1697	0.22985\\
1698	0.22994\\
1699	0.23003\\
1700	0.23011\\
1701	0.2302\\
1702	0.23028\\
1703	0.23035\\
1704	0.23043\\
1705	0.2305\\
1706	0.23057\\
1707	0.23064\\
1708	0.23071\\
1709	0.23077\\
1710	0.23084\\
1711	0.2309\\
1712	0.23096\\
1713	0.23101\\
1714	0.23107\\
1715	0.23112\\
1716	0.23118\\
1717	0.23123\\
1718	0.23128\\
1719	0.23132\\
1720	0.23137\\
1721	0.23142\\
1722	0.23146\\
1723	0.2315\\
1724	0.23154\\
1725	0.23158\\
1726	0.23162\\
1727	0.23166\\
1728	0.23169\\
1729	0.23173\\
1730	0.23176\\
1731	0.2318\\
1732	0.23183\\
1733	0.23186\\
1734	0.23189\\
1735	0.23192\\
1736	0.23195\\
1737	0.23198\\
1738	0.232\\
1739	0.23203\\
1740	0.23206\\
1741	0.23208\\
1742	0.2321\\
1743	0.23213\\
1744	0.23215\\
1745	0.23217\\
1746	0.23219\\
1747	0.23221\\
1748	0.23223\\
1749	0.23225\\
1750	0.23227\\
1751	0.23229\\
1752	0.23231\\
1753	0.23232\\
1754	0.23234\\
1755	0.23236\\
1756	0.23237\\
1757	0.23239\\
1758	0.2324\\
1759	0.23242\\
1760	0.23243\\
1761	0.23245\\
1762	0.23246\\
1763	0.23247\\
1764	0.23248\\
1765	0.2325\\
1766	0.23251\\
1767	0.23252\\
1768	0.23253\\
1769	0.23254\\
1770	0.23255\\
1771	0.23256\\
1772	0.23257\\
1773	0.23258\\
1774	0.23259\\
1775	0.2326\\
1776	0.23261\\
1777	0.23261\\
1778	0.23262\\
1779	0.23263\\
1780	0.23264\\
1781	0.23265\\
1782	0.23265\\
1783	0.23266\\
1784	0.23267\\
1785	0.23267\\
1786	0.23268\\
1787	0.23269\\
1788	0.23269\\
1789	0.2327\\
1790	0.2327\\
1791	0.23271\\
1792	0.23271\\
1793	0.23272\\
1794	0.23272\\
1795	0.23273\\
1796	0.23273\\
1797	0.23274\\
1798	0.23274\\
1799	0.23275\\
1800	0.23275\\
};
\addlegendentry{Sterowanie u}

\end{axis}
\end{tikzpicture}%
   \caption{Dwa regulatory lokalne DMC}
   \label{projekt:zad7:DMC:2:figure}
\end{figure}

\begin{figure}[H] 
   \centering
   \input{projekt/figure/z7DMC3.tex}
   \caption{Trzy regulatory lokalne DMC}
   \label{projekt:zad7:DMC:3:figure}
\end{figure}

\begin{figure}[H] 
   \centering
   \input{projekt/figure/z7DMC4.tex}
   \caption{Cztery regulatory lokalne DMC}
   \label{projekt:zad7:DMC:4:figure}
\end{figure}

\begin{figure}[H] 
   \centering
   \input{projekt/figure/z7DMC5.tex}
   \caption{Pięć regulatorów lokalnych DMC}
   \label{projekt:zad7:DMC:5:figure}
\end{figure}

\subsection{Wnioski}
Ostatnim krokiem w dostrajaniu regulatora rozmytego było
wyznaczenie współczynnika kary lambda dla wszystkich regulatorów
lokalnych, za pomocą którego można zapewnić kompromis pomiędzy
szybkością regulacji a postacią sygnału sterującego. Ponownie był on
wyznaczany metodą testowania. Spośród wszystkich regulatorów
najlepszym względem współczynnika jakości regulacji okazał się być
rozmyty regulator DMC o 4 regulatorach lokalnych oraz nastawach:


\newpage
