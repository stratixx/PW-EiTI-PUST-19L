\section{Dobór parametrów lambda lokalnych regulatorów DMC}
\label{projekt:zad7}

\subsection{Rozmyty regulator DMC}
\label{projekt:zad7:PID}

\begin{figure}[H] 
   \centering
   % This file was created by matlab2tikz.
%
\definecolor{mycolor1}{rgb}{0.00000,0.44700,0.74100}%
\definecolor{mycolor2}{rgb}{0.85000,0.32500,0.09800}%
%
\begin{tikzpicture}

\begin{axis}[%
width=4.521in,
height=1.493in,
at={(0.758in,2.554in)},
scale only axis,
xmin=1,
xmax=1800,
xlabel style={font=\color{white!15!black}},
xlabel={k},
ymin=-4.551,
ymax=0.1,
ylabel style={font=\color{white!15!black}},
ylabel={y},
axis background/.style={fill=white},
title style={font=\bfseries, align=center},
title={E=222.9899\\[1ex]N= [70         20]\\[1ex]$\text{N}_\text{u}\text{= [30          5]}$\\[1ex]lambda= [25         10]},
xmajorgrids,
ymajorgrids,
legend style={legend cell align=left, align=left, draw=white!15!black}
]
\addplot [color=mycolor1]
  table[row sep=crcr]{%
1	0\\
2	0\\
3	0\\
4	0\\
5	0\\
6	0\\
7	0\\
8	0\\
9	0\\
10	0\\
11	0\\
12	0\\
13	0\\
14	0\\
15	0\\
16	0\\
17	0\\
18	0\\
19	0\\
20	0\\
21	0\\
22	0\\
23	0\\
24	0\\
25	-0.031566\\
26	-0.14924\\
27	-0.3872\\
28	-0.7244\\
29	-1.1124\\
30	-1.5162\\
31	-1.9104\\
32	-2.2732\\
33	-2.5865\\
34	-2.8359\\
35	-3.0115\\
36	-3.1088\\
37	-3.1287\\
38	-3.0778\\
39	-2.9677\\
40	-2.8142\\
41	-2.635\\
42	-2.4482\\
43	-2.27\\
44	-2.1129\\
45	-1.9856\\
46	-1.893\\
47	-1.8359\\
48	-1.8125\\
49	-1.8186\\
50	-1.8483\\
51	-1.8948\\
52	-1.951\\
53	-2.01\\
54	-2.0659\\
55	-2.1137\\
56	-2.1499\\
57	-2.1726\\
58	-2.1812\\
59	-2.1765\\
60	-2.1605\\
61	-2.1359\\
62	-2.1056\\
63	-2.0729\\
64	-2.0408\\
65	-2.0116\\
66	-1.9873\\
67	-1.969\\
68	-1.9573\\
69	-1.9521\\
70	-1.9527\\
71	-1.9581\\
72	-1.967\\
73	-1.9781\\
74	-1.9901\\
75	-2.0016\\
76	-2.0118\\
77	-2.02\\
78	-2.0256\\
79	-2.0285\\
80	-2.0288\\
81	-2.0268\\
82	-2.0231\\
83	-2.0181\\
84	-2.0125\\
85	-2.0068\\
86	-2.0015\\
87	-1.9969\\
88	-1.9934\\
89	-1.9911\\
90	-1.9899\\
91	-1.9899\\
92	-1.9908\\
93	-1.9924\\
94	-1.9945\\
95	-1.9968\\
96	-1.999\\
97	-2.0011\\
98	-2.0028\\
99	-2.004\\
100	-2.0047\\
101	-2.005\\
102	-2.0048\\
103	-2.0042\\
104	-2.0034\\
105	-2.0024\\
106	-2.0014\\
107	-2.0004\\
108	-1.9995\\
109	-1.9988\\
110	-1.9983\\
111	-1.9981\\
112	-1.998\\
113	-1.9981\\
114	-1.9984\\
115	-1.9988\\
116	-1.9992\\
117	-1.9997\\
118	-2.0001\\
119	-2.0004\\
120	-2.0007\\
121	-2.0008\\
122	-2.0009\\
123	-2.0009\\
124	-2.0008\\
125	-2.0007\\
126	-2.0005\\
127	-2.0003\\
128	-2.0001\\
129	-1.9999\\
130	-1.9998\\
131	-1.9997\\
132	-1.9996\\
133	-1.9996\\
134	-1.9996\\
135	-1.9997\\
136	-1.9998\\
137	-1.9998\\
138	-1.9999\\
139	-2\\
140	-2.0001\\
141	-2.0001\\
142	-2.0001\\
143	-2.0002\\
144	-2.0002\\
145	-2.0002\\
146	-2.0001\\
147	-2.0001\\
148	-2.0001\\
149	-2\\
150	-2\\
151	-2\\
152	-1.9999\\
153	-1.9999\\
154	-1.9999\\
155	-1.9999\\
156	-1.9999\\
157	-2\\
158	-2\\
159	-2\\
160	-2\\
161	-2\\
162	-2\\
163	-2\\
164	-2\\
165	-2\\
166	-2\\
167	-2\\
168	-2\\
169	-2\\
170	-2\\
171	-2\\
172	-2\\
173	-2\\
174	-2\\
175	-2\\
176	-2\\
177	-2\\
178	-2\\
179	-2\\
180	-2\\
181	-2\\
182	-2\\
183	-2\\
184	-2\\
185	-2\\
186	-2\\
187	-2\\
188	-2\\
189	-2\\
190	-2\\
191	-2\\
192	-2\\
193	-2\\
194	-2\\
195	-2\\
196	-2\\
197	-2\\
198	-2\\
199	-2\\
200	-2\\
201	-2\\
202	-2\\
203	-2\\
204	-2\\
205	-2.0801\\
206	-2.2463\\
207	-2.4603\\
208	-2.6956\\
209	-2.9345\\
210	-3.163\\
211	-3.3695\\
212	-3.5435\\
213	-3.6786\\
214	-3.7739\\
215	-3.8329\\
216	-3.8624\\
217	-3.8713\\
218	-3.8686\\
219	-3.8626\\
220	-3.8604\\
221	-3.8671\\
222	-3.8861\\
223	-3.9187\\
224	-3.9645\\
225	-4.022\\
226	-4.0885\\
227	-4.1599\\
228	-4.2317\\
229	-4.3006\\
230	-4.3642\\
231	-4.4202\\
232	-4.4669\\
233	-4.5033\\
234	-4.529\\
235	-4.5446\\
236	-4.551\\
237	-4.5495\\
238	-4.5419\\
239	-4.5301\\
240	-4.5157\\
241	-4.5005\\
242	-4.4859\\
243	-4.4731\\
244	-4.4629\\
245	-4.4559\\
246	-4.4522\\
247	-4.4518\\
248	-4.4543\\
249	-4.4592\\
250	-4.466\\
251	-4.474\\
252	-4.4825\\
253	-4.4908\\
254	-4.4986\\
255	-4.5053\\
256	-4.5106\\
257	-4.5145\\
258	-4.5167\\
259	-4.5175\\
260	-4.517\\
261	-4.5153\\
262	-4.5128\\
263	-4.5097\\
264	-4.5064\\
265	-4.5031\\
266	-4.5\\
267	-4.4973\\
268	-4.4952\\
269	-4.4937\\
270	-4.4928\\
271	-4.4926\\
272	-4.4929\\
273	-4.4936\\
274	-4.4947\\
275	-4.496\\
276	-4.4974\\
277	-4.4987\\
278	-4.5\\
279	-4.5011\\
280	-4.502\\
281	-4.5026\\
282	-4.503\\
283	-4.5031\\
284	-4.503\\
285	-4.5027\\
286	-4.5022\\
287	-4.5017\\
288	-4.5011\\
289	-4.5005\\
290	-4.5\\
291	-4.4995\\
292	-4.4992\\
293	-4.4989\\
294	-4.4988\\
295	-4.4987\\
296	-4.4988\\
297	-4.4989\\
298	-4.4991\\
299	-4.4993\\
300	-4.4996\\
301	-4.4998\\
302	-4.5\\
303	-4.5002\\
304	-4.5004\\
305	-4.5005\\
306	-4.5005\\
307	-4.5005\\
308	-4.5005\\
309	-4.5005\\
310	-4.5004\\
311	-4.5003\\
312	-4.5002\\
313	-4.5001\\
314	-4.5\\
315	-4.4999\\
316	-4.4999\\
317	-4.4998\\
318	-4.4998\\
319	-4.4998\\
320	-4.4998\\
321	-4.4998\\
322	-4.4998\\
323	-4.4999\\
324	-4.4999\\
325	-4.5\\
326	-4.5\\
327	-4.5\\
328	-4.5001\\
329	-4.5001\\
330	-4.5001\\
331	-4.5001\\
332	-4.5001\\
333	-4.5001\\
334	-4.5001\\
335	-4.5\\
336	-4.5\\
337	-4.5\\
338	-4.5\\
339	-4.5\\
340	-4.5\\
341	-4.5\\
342	-4.5\\
343	-4.5\\
344	-4.5\\
345	-4.5\\
346	-4.5\\
347	-4.5\\
348	-4.5\\
349	-4.5\\
350	-4.5\\
351	-4.5\\
352	-4.5\\
353	-4.5\\
354	-4.5\\
355	-4.5\\
356	-4.5\\
357	-4.5\\
358	-4.5\\
359	-4.5\\
360	-4.5\\
361	-4.5\\
362	-4.5\\
363	-4.5\\
364	-4.5\\
365	-4.5\\
366	-4.5\\
367	-4.5\\
368	-4.5\\
369	-4.5\\
370	-4.5\\
371	-4.5\\
372	-4.5\\
373	-4.5\\
374	-4.5\\
375	-4.5\\
376	-4.5\\
377	-4.5\\
378	-4.5\\
379	-4.5\\
380	-4.5\\
381	-4.5\\
382	-4.5\\
383	-4.5\\
384	-4.5\\
385	-4.5\\
386	-4.5\\
387	-4.5\\
388	-4.5\\
389	-4.5\\
390	-4.5\\
391	-4.5\\
392	-4.5\\
393	-4.5\\
394	-4.5\\
395	-4.5\\
396	-4.5\\
397	-4.5\\
398	-4.5\\
399	-4.5\\
400	-4.5\\
401	-4.5\\
402	-4.5\\
403	-4.5\\
404	-4.5\\
405	-4.433\\
406	-4.2895\\
407	-4.0712\\
408	-3.8044\\
409	-3.5147\\
410	-3.2303\\
411	-2.9734\\
412	-2.7606\\
413	-2.6016\\
414	-2.5002\\
415	-2.4553\\
416	-2.4618\\
417	-2.5116\\
418	-2.5949\\
419	-2.7006\\
420	-2.8176\\
421	-2.9355\\
422	-3.0451\\
423	-3.1391\\
424	-3.2123\\
425	-3.2617\\
426	-3.2864\\
427	-3.2878\\
428	-3.2686\\
429	-3.233\\
430	-3.1858\\
431	-3.1323\\
432	-3.0773\\
433	-3.0252\\
434	-2.9796\\
435	-2.9429\\
436	-2.9167\\
437	-2.9013\\
438	-2.8963\\
439	-2.9006\\
440	-2.9125\\
441	-2.9299\\
442	-2.9507\\
443	-2.9727\\
444	-2.9941\\
445	-3.0133\\
446	-3.029\\
447	-3.0406\\
448	-3.0476\\
449	-3.0502\\
450	-3.0486\\
451	-3.0437\\
452	-3.0363\\
453	-3.0271\\
454	-3.0173\\
455	-3.0076\\
456	-2.9987\\
457	-2.9912\\
458	-2.9856\\
459	-2.9818\\
460	-2.9801\\
461	-2.9801\\
462	-2.9817\\
463	-2.9846\\
464	-2.9882\\
465	-2.9923\\
466	-2.9964\\
467	-3.0002\\
468	-3.0035\\
469	-3.0061\\
470	-3.0078\\
471	-3.0088\\
472	-3.0089\\
473	-3.0083\\
474	-3.0072\\
475	-3.0057\\
476	-3.0039\\
477	-3.0021\\
478	-3.0004\\
479	-2.9989\\
480	-2.9977\\
481	-2.9969\\
482	-2.9964\\
483	-2.9962\\
484	-2.9964\\
485	-2.9968\\
486	-2.9974\\
487	-2.9982\\
488	-2.9989\\
489	-2.9997\\
490	-3.0003\\
491	-3.0009\\
492	-3.0013\\
493	-3.0015\\
494	-3.0016\\
495	-3.0016\\
496	-3.0014\\
497	-3.0012\\
498	-3.0009\\
499	-3.0006\\
500	-3.0002\\
501	-2.9999\\
502	-2.9997\\
503	-2.9995\\
504	-2.9994\\
505	-2.9993\\
506	-2.9993\\
507	-2.9994\\
508	-2.9995\\
509	-2.9996\\
510	-2.9997\\
511	-2.9999\\
512	-3\\
513	-3.0001\\
514	-3.0002\\
515	-3.0003\\
516	-3.0003\\
517	-3.0003\\
518	-3.0003\\
519	-3.0002\\
520	-3.0002\\
521	-3.0001\\
522	-3.0001\\
523	-3\\
524	-3\\
525	-2.9999\\
526	-2.9999\\
527	-2.9999\\
528	-2.9999\\
529	-2.9999\\
530	-2.9999\\
531	-2.9999\\
532	-2.9999\\
533	-3\\
534	-3\\
535	-3\\
536	-3\\
537	-3\\
538	-3.0001\\
539	-3.0001\\
540	-3.0001\\
541	-3\\
542	-3\\
543	-3\\
544	-3\\
545	-3\\
546	-3\\
547	-3\\
548	-3\\
549	-3\\
550	-3\\
551	-3\\
552	-3\\
553	-3\\
554	-3\\
555	-3\\
556	-3\\
557	-3\\
558	-3\\
559	-3\\
560	-3\\
561	-3\\
562	-3\\
563	-3\\
564	-3\\
565	-3\\
566	-3\\
567	-3\\
568	-3\\
569	-3\\
570	-3\\
571	-3\\
572	-3\\
573	-3\\
574	-3\\
575	-3\\
576	-3\\
577	-3\\
578	-3\\
579	-3\\
580	-3\\
581	-3\\
582	-3\\
583	-3\\
584	-3\\
585	-3\\
586	-3\\
587	-3\\
588	-3\\
589	-3\\
590	-3\\
591	-3\\
592	-3\\
593	-3\\
594	-3\\
595	-3\\
596	-3\\
597	-3\\
598	-3\\
599	-3\\
600	-3\\
601	-3\\
602	-3\\
603	-3\\
604	-3\\
605	-2.9275\\
606	-2.7774\\
607	-2.5534\\
608	-2.2884\\
609	-2.0117\\
610	-1.7526\\
611	-1.5309\\
612	-1.3591\\
613	-1.2423\\
614	-1.1802\\
615	-1.168\\
616	-1.198\\
617	-1.2603\\
618	-1.3441\\
619	-1.4381\\
620	-1.5318\\
621	-1.6164\\
622	-1.6849\\
623	-1.7327\\
624	-1.7578\\
625	-1.7608\\
626	-1.744\\
627	-1.7116\\
628	-1.6686\\
629	-1.6203\\
630	-1.5717\\
631	-1.5271\\
632	-1.4897\\
633	-1.4615\\
634	-1.4435\\
635	-1.4354\\
636	-1.4363\\
637	-1.4445\\
638	-1.458\\
639	-1.4746\\
640	-1.4922\\
641	-1.5089\\
642	-1.5233\\
643	-1.5341\\
644	-1.541\\
645	-1.5438\\
646	-1.5427\\
647	-1.5385\\
648	-1.5319\\
649	-1.5238\\
650	-1.5152\\
651	-1.5069\\
652	-1.4996\\
653	-1.4938\\
654	-1.4897\\
655	-1.4874\\
656	-1.4869\\
657	-1.4878\\
658	-1.49\\
659	-1.4928\\
660	-1.496\\
661	-1.4993\\
662	-1.5022\\
663	-1.5045\\
664	-1.5062\\
665	-1.5071\\
666	-1.5073\\
667	-1.5069\\
668	-1.506\\
669	-1.5047\\
670	-1.5032\\
671	-1.5017\\
672	-1.5003\\
673	-1.4991\\
674	-1.4983\\
675	-1.4977\\
676	-1.4975\\
677	-1.4975\\
678	-1.4978\\
679	-1.4983\\
680	-1.4989\\
681	-1.4995\\
682	-1.5\\
683	-1.5005\\
684	-1.5009\\
685	-1.5011\\
686	-1.5013\\
687	-1.5012\\
688	-1.5011\\
689	-1.5009\\
690	-1.5007\\
691	-1.5004\\
692	-1.5001\\
693	-1.4999\\
694	-1.4997\\
695	-1.4996\\
696	-1.4995\\
697	-1.4995\\
698	-1.4996\\
699	-1.4996\\
700	-1.4997\\
701	-1.4998\\
702	-1.4999\\
703	-1.5\\
704	-1.5001\\
705	-1.5002\\
706	-1.5002\\
707	-1.5002\\
708	-1.5002\\
709	-1.5002\\
710	-1.5001\\
711	-1.5001\\
712	-1.5\\
713	-1.5\\
714	-1.5\\
715	-1.4999\\
716	-1.4999\\
717	-1.4999\\
718	-1.4999\\
719	-1.4999\\
720	-1.4999\\
721	-1.5\\
722	-1.5\\
723	-1.5\\
724	-1.5\\
725	-1.5\\
726	-1.5\\
727	-1.5\\
728	-1.5\\
729	-1.5\\
730	-1.5\\
731	-1.5\\
732	-1.5\\
733	-1.5\\
734	-1.5\\
735	-1.5\\
736	-1.5\\
737	-1.5\\
738	-1.5\\
739	-1.5\\
740	-1.5\\
741	-1.5\\
742	-1.5\\
743	-1.5\\
744	-1.5\\
745	-1.5\\
746	-1.5\\
747	-1.5\\
748	-1.5\\
749	-1.5\\
750	-1.5\\
751	-1.5\\
752	-1.5\\
753	-1.5\\
754	-1.5\\
755	-1.5\\
756	-1.5\\
757	-1.5\\
758	-1.5\\
759	-1.5\\
760	-1.5\\
761	-1.5\\
762	-1.5\\
763	-1.5\\
764	-1.5\\
765	-1.5\\
766	-1.5\\
767	-1.5\\
768	-1.5\\
769	-1.5\\
770	-1.5\\
771	-1.5\\
772	-1.5\\
773	-1.5\\
774	-1.5\\
775	-1.5\\
776	-1.5\\
777	-1.5\\
778	-1.5\\
779	-1.5\\
780	-1.5\\
781	-1.5\\
782	-1.5\\
783	-1.5\\
784	-1.5\\
785	-1.5\\
786	-1.5\\
787	-1.5\\
788	-1.5\\
789	-1.5\\
790	-1.5\\
791	-1.5\\
792	-1.5\\
793	-1.5\\
794	-1.5\\
795	-1.5\\
796	-1.5\\
797	-1.5\\
798	-1.5\\
799	-1.5\\
800	-1.5\\
801	-1.5\\
802	-1.5\\
803	-1.5\\
804	-1.5\\
805	-1.5662\\
806	-1.7134\\
807	-1.9287\\
808	-2.1876\\
809	-2.4631\\
810	-2.7357\\
811	-2.9898\\
812	-3.2117\\
813	-3.3913\\
814	-3.5229\\
815	-3.6053\\
816	-3.6416\\
817	-3.638\\
818	-3.6029\\
819	-3.5462\\
820	-3.4777\\
821	-3.4065\\
822	-3.3405\\
823	-3.2855\\
824	-3.2457\\
825	-3.2231\\
826	-3.2181\\
827	-3.2295\\
828	-3.2549\\
829	-3.2913\\
830	-3.335\\
831	-3.3823\\
832	-3.4297\\
833	-3.4741\\
834	-3.5128\\
835	-3.5442\\
836	-3.567\\
837	-3.581\\
838	-3.5864\\
839	-3.5842\\
840	-3.5757\\
841	-3.5624\\
842	-3.5461\\
843	-3.5284\\
844	-3.5108\\
845	-3.4948\\
846	-3.4813\\
847	-3.4709\\
848	-3.4641\\
849	-3.4608\\
850	-3.4608\\
851	-3.4637\\
852	-3.4689\\
853	-3.4756\\
854	-3.4832\\
855	-3.491\\
856	-3.4983\\
857	-3.5048\\
858	-3.51\\
859	-3.5138\\
860	-3.516\\
861	-3.5168\\
862	-3.5162\\
863	-3.5145\\
864	-3.5119\\
865	-3.5089\\
866	-3.5056\\
867	-3.5023\\
868	-3.4994\\
869	-3.4968\\
870	-3.4949\\
871	-3.4936\\
872	-3.493\\
873	-3.493\\
874	-3.4935\\
875	-3.4944\\
876	-3.4956\\
877	-3.4969\\
878	-3.4983\\
879	-3.4997\\
880	-3.5008\\
881	-3.5018\\
882	-3.5025\\
883	-3.5029\\
884	-3.503\\
885	-3.5029\\
886	-3.5026\\
887	-3.5022\\
888	-3.5016\\
889	-3.501\\
890	-3.5004\\
891	-3.4999\\
892	-3.4994\\
893	-3.4991\\
894	-3.4988\\
895	-3.4987\\
896	-3.4987\\
897	-3.4988\\
898	-3.499\\
899	-3.4992\\
900	-3.4994\\
901	-3.4997\\
902	-3.4999\\
903	-3.5001\\
904	-3.5003\\
905	-3.5004\\
906	-3.5005\\
907	-3.5005\\
908	-3.5005\\
909	-3.5005\\
910	-3.5004\\
911	-3.5003\\
912	-3.5002\\
913	-3.5001\\
914	-3.5\\
915	-3.4999\\
916	-3.4998\\
917	-3.4998\\
918	-3.4998\\
919	-3.4998\\
920	-3.4998\\
921	-3.4998\\
922	-3.4999\\
923	-3.4999\\
924	-3.4999\\
925	-3.5\\
926	-3.5\\
927	-3.5001\\
928	-3.5001\\
929	-3.5001\\
930	-3.5001\\
931	-3.5001\\
932	-3.5001\\
933	-3.5001\\
934	-3.5001\\
935	-3.5\\
936	-3.5\\
937	-3.5\\
938	-3.5\\
939	-3.5\\
940	-3.5\\
941	-3.5\\
942	-3.5\\
943	-3.5\\
944	-3.5\\
945	-3.5\\
946	-3.5\\
947	-3.5\\
948	-3.5\\
949	-3.5\\
950	-3.5\\
951	-3.5\\
952	-3.5\\
953	-3.5\\
954	-3.5\\
955	-3.5\\
956	-3.5\\
957	-3.5\\
958	-3.5\\
959	-3.5\\
960	-3.5\\
961	-3.5\\
962	-3.5\\
963	-3.5\\
964	-3.5\\
965	-3.5\\
966	-3.5\\
967	-3.5\\
968	-3.5\\
969	-3.5\\
970	-3.5\\
971	-3.5\\
972	-3.5\\
973	-3.5\\
974	-3.5\\
975	-3.5\\
976	-3.5\\
977	-3.5\\
978	-3.5\\
979	-3.5\\
980	-3.5\\
981	-3.5\\
982	-3.5\\
983	-3.5\\
984	-3.5\\
985	-3.5\\
986	-3.5\\
987	-3.5\\
988	-3.5\\
989	-3.5\\
990	-3.5\\
991	-3.5\\
992	-3.5\\
993	-3.5\\
994	-3.5\\
995	-3.5\\
996	-3.5\\
997	-3.5\\
998	-3.5\\
999	-3.5\\
1000	-3.5\\
1001	-3.5\\
1002	-3.5\\
1003	-3.5\\
1004	-3.5\\
1005	-3.3771\\
1006	-3.1245\\
1007	-2.7436\\
1008	-2.3041\\
1009	-1.8633\\
1010	-1.4715\\
1011	-1.1551\\
1012	-0.92548\\
1013	-0.78294\\
1014	-0.72196\\
1015	-0.73262\\
1016	-0.80155\\
1017	-0.91287\\
1018	-1.0494\\
1019	-1.1942\\
1020	-1.3318\\
1021	-1.4493\\
1022	-1.5374\\
1023	-1.591\\
1024	-1.6087\\
1025	-1.5933\\
1026	-1.5505\\
1027	-1.4883\\
1028	-1.4154\\
1029	-1.3402\\
1030	-1.2701\\
1031	-1.2104\\
1032	-1.1648\\
1033	-1.1347\\
1034	-1.12\\
1035	-1.1193\\
1036	-1.13\\
1037	-1.1489\\
1038	-1.1727\\
1039	-1.1981\\
1040	-1.2221\\
1041	-1.2424\\
1042	-1.2574\\
1043	-1.2664\\
1044	-1.2692\\
1045	-1.2663\\
1046	-1.2589\\
1047	-1.2483\\
1048	-1.2358\\
1049	-1.2229\\
1050	-1.2109\\
1051	-1.2006\\
1052	-1.1928\\
1053	-1.1876\\
1054	-1.1851\\
1055	-1.1851\\
1056	-1.187\\
1057	-1.1904\\
1058	-1.1946\\
1059	-1.199\\
1060	-1.2032\\
1061	-1.2068\\
1062	-1.2094\\
1063	-1.2109\\
1064	-1.2114\\
1065	-1.2109\\
1066	-1.2096\\
1067	-1.2078\\
1068	-1.2057\\
1069	-1.2035\\
1070	-1.2015\\
1071	-1.1998\\
1072	-1.1984\\
1073	-1.1976\\
1074	-1.1972\\
1075	-1.1972\\
1076	-1.1976\\
1077	-1.1982\\
1078	-1.199\\
1079	-1.1997\\
1080	-1.2005\\
1081	-1.2011\\
1082	-1.2015\\
1083	-1.2018\\
1084	-1.2019\\
1085	-1.2018\\
1086	-1.2016\\
1087	-1.2013\\
1088	-1.2009\\
1089	-1.2005\\
1090	-1.2002\\
1091	-1.1999\\
1092	-1.1997\\
1093	-1.1995\\
1094	-1.1995\\
1095	-1.1995\\
1096	-1.1996\\
1097	-1.1997\\
1098	-1.1998\\
1099	-1.1999\\
1100	-1.2001\\
1101	-1.2002\\
1102	-1.2003\\
1103	-1.2003\\
1104	-1.2003\\
1105	-1.2003\\
1106	-1.2003\\
1107	-1.2002\\
1108	-1.2001\\
1109	-1.2001\\
1110	-1.2\\
1111	-1.2\\
1112	-1.1999\\
1113	-1.1999\\
1114	-1.1999\\
1115	-1.1999\\
1116	-1.1999\\
1117	-1.1999\\
1118	-1.2\\
1119	-1.2\\
1120	-1.2\\
1121	-1.2\\
1122	-1.2\\
1123	-1.2\\
1124	-1.2001\\
1125	-1.2\\
1126	-1.2\\
1127	-1.2\\
1128	-1.2\\
1129	-1.2\\
1130	-1.2\\
1131	-1.2\\
1132	-1.2\\
1133	-1.2\\
1134	-1.2\\
1135	-1.2\\
1136	-1.2\\
1137	-1.2\\
1138	-1.2\\
1139	-1.2\\
1140	-1.2\\
1141	-1.2\\
1142	-1.2\\
1143	-1.2\\
1144	-1.2\\
1145	-1.2\\
1146	-1.2\\
1147	-1.2\\
1148	-1.2\\
1149	-1.2\\
1150	-1.2\\
1151	-1.2\\
1152	-1.2\\
1153	-1.2\\
1154	-1.2\\
1155	-1.2\\
1156	-1.2\\
1157	-1.2\\
1158	-1.2\\
1159	-1.2\\
1160	-1.2\\
1161	-1.2\\
1162	-1.2\\
1163	-1.2\\
1164	-1.2\\
1165	-1.2\\
1166	-1.2\\
1167	-1.2\\
1168	-1.2\\
1169	-1.2\\
1170	-1.2\\
1171	-1.2\\
1172	-1.2\\
1173	-1.2\\
1174	-1.2\\
1175	-1.2\\
1176	-1.2\\
1177	-1.2\\
1178	-1.2\\
1179	-1.2\\
1180	-1.2\\
1181	-1.2\\
1182	-1.2\\
1183	-1.2\\
1184	-1.2\\
1185	-1.2\\
1186	-1.2\\
1187	-1.2\\
1188	-1.2\\
1189	-1.2\\
1190	-1.2\\
1191	-1.2\\
1192	-1.2\\
1193	-1.2\\
1194	-1.2\\
1195	-1.2\\
1196	-1.2\\
1197	-1.2\\
1198	-1.2\\
1199	-1.2\\
1200	-1.2\\
1201	-1.2\\
1202	-1.2\\
1203	-1.2\\
1204	-1.2\\
1205	-1.1602\\
1206	-1.0818\\
1207	-0.97257\\
1208	-0.85049\\
1209	-0.73004\\
1210	-0.62296\\
1211	-0.53612\\
1212	-0.47261\\
1213	-0.4324\\
1214	-0.4132\\
1215	-0.41116\\
1216	-0.42148\\
1217	-0.43905\\
1218	-0.45894\\
1219	-0.477\\
1220	-0.49013\\
1221	-0.49652\\
1222	-0.49566\\
1223	-0.48812\\
1224	-0.4753\\
1225	-0.45902\\
1226	-0.4412\\
1227	-0.42359\\
1228	-0.40756\\
1229	-0.39405\\
1230	-0.3835\\
1231	-0.37593\\
1232	-0.37106\\
1233	-0.36834\\
1234	-0.36713\\
1235	-0.36676\\
1236	-0.36663\\
1237	-0.36627\\
1238	-0.36537\\
1239	-0.36376\\
1240	-0.36144\\
1241	-0.3585\\
1242	-0.35513\\
1243	-0.35153\\
1244	-0.34792\\
1245	-0.34447\\
1246	-0.34132\\
1247	-0.33854\\
1248	-0.33617\\
1249	-0.33419\\
1250	-0.33255\\
1251	-0.33117\\
1252	-0.32998\\
1253	-0.3289\\
1254	-0.32787\\
1255	-0.32684\\
1256	-0.32578\\
1257	-0.32469\\
1258	-0.32356\\
1259	-0.32242\\
1260	-0.32128\\
1261	-0.32016\\
1262	-0.31909\\
1263	-0.31808\\
1264	-0.31713\\
1265	-0.31626\\
1266	-0.31546\\
1267	-0.31473\\
1268	-0.31405\\
1269	-0.31343\\
1270	-0.31284\\
1271	-0.31228\\
1272	-0.31174\\
1273	-0.31122\\
1274	-0.31072\\
1275	-0.31023\\
1276	-0.30975\\
1277	-0.30929\\
1278	-0.30885\\
1279	-0.30843\\
1280	-0.30802\\
1281	-0.30764\\
1282	-0.30728\\
1283	-0.30694\\
1284	-0.30661\\
1285	-0.30631\\
1286	-0.30602\\
1287	-0.30575\\
1288	-0.30549\\
1289	-0.30524\\
1290	-0.305\\
1291	-0.30477\\
1292	-0.30455\\
1293	-0.30434\\
1294	-0.30414\\
1295	-0.30395\\
1296	-0.30377\\
1297	-0.30359\\
1298	-0.30343\\
1299	-0.30327\\
1300	-0.30312\\
1301	-0.30297\\
1302	-0.30284\\
1303	-0.30271\\
1304	-0.30258\\
1305	-0.30247\\
1306	-0.30235\\
1307	-0.30225\\
1308	-0.30214\\
1309	-0.30204\\
1310	-0.30195\\
1311	-0.30186\\
1312	-0.30178\\
1313	-0.30169\\
1314	-0.30162\\
1315	-0.30154\\
1316	-0.30147\\
1317	-0.3014\\
1318	-0.30134\\
1319	-0.30128\\
1320	-0.30122\\
1321	-0.30116\\
1322	-0.30111\\
1323	-0.30106\\
1324	-0.30101\\
1325	-0.30097\\
1326	-0.30092\\
1327	-0.30088\\
1328	-0.30084\\
1329	-0.3008\\
1330	-0.30076\\
1331	-0.30073\\
1332	-0.3007\\
1333	-0.30066\\
1334	-0.30063\\
1335	-0.3006\\
1336	-0.30058\\
1337	-0.30055\\
1338	-0.30053\\
1339	-0.3005\\
1340	-0.30048\\
1341	-0.30046\\
1342	-0.30044\\
1343	-0.30042\\
1344	-0.3004\\
1345	-0.30038\\
1346	-0.30036\\
1347	-0.30034\\
1348	-0.30033\\
1349	-0.30031\\
1350	-0.3003\\
1351	-0.30029\\
1352	-0.30027\\
1353	-0.30026\\
1354	-0.30025\\
1355	-0.30024\\
1356	-0.30023\\
1357	-0.30022\\
1358	-0.30021\\
1359	-0.3002\\
1360	-0.30019\\
1361	-0.30018\\
1362	-0.30017\\
1363	-0.30016\\
1364	-0.30016\\
1365	-0.30015\\
1366	-0.30014\\
1367	-0.30014\\
1368	-0.30013\\
1369	-0.30012\\
1370	-0.30012\\
1371	-0.30011\\
1372	-0.30011\\
1373	-0.3001\\
1374	-0.3001\\
1375	-0.30009\\
1376	-0.30009\\
1377	-0.30008\\
1378	-0.30008\\
1379	-0.30008\\
1380	-0.30007\\
1381	-0.30007\\
1382	-0.30007\\
1383	-0.30006\\
1384	-0.30006\\
1385	-0.30006\\
1386	-0.30006\\
1387	-0.30005\\
1388	-0.30005\\
1389	-0.30005\\
1390	-0.30005\\
1391	-0.30004\\
1392	-0.30004\\
1393	-0.30004\\
1394	-0.30004\\
1395	-0.30004\\
1396	-0.30003\\
1397	-0.30003\\
1398	-0.30003\\
1399	-0.30003\\
1400	-0.30003\\
1401	-0.30003\\
1402	-0.30003\\
1403	-0.30003\\
1404	-0.30002\\
1405	-0.2918\\
1406	-0.27545\\
1407	-0.25355\\
1408	-0.22962\\
1409	-0.20662\\
1410	-0.18663\\
1411	-0.17075\\
1412	-0.15927\\
1413	-0.15182\\
1414	-0.14761\\
1415	-0.14564\\
1416	-0.14488\\
1417	-0.14442\\
1418	-0.14357\\
1419	-0.1419\\
1420	-0.13926\\
1421	-0.13569\\
1422	-0.1314\\
1423	-0.12666\\
1424	-0.12177\\
1425	-0.11698\\
1426	-0.11249\\
1427	-0.10841\\
1428	-0.10479\\
1429	-0.1016\\
1430	-0.09878\\
1431	-0.096248\\
1432	-0.093914\\
1433	-0.09147\\
1434	-0.088652\\
1435	-0.085401\\
1436	-0.081759\\
1437	-0.077631\\
1438	-0.072963\\
1439	-0.067853\\
1440	-0.062464\\
1441	-0.05682\\
1442	-0.050947\\
1443	-0.044846\\
1444	-0.038567\\
1445	-0.032162\\
1446	-0.025722\\
1447	-0.019337\\
1448	-0.013116\\
1449	-0.0072166\\
1450	-0.0017877\\
1451	0.0030547\\
1452	0.0072348\\
1453	0.010714\\
1454	0.013482\\
1455	0.015555\\
1456	0.016963\\
1457	0.017752\\
1458	0.017977\\
1459	0.017696\\
1460	0.016975\\
1461	0.015879\\
1462	0.014475\\
1463	0.012829\\
1464	0.011007\\
1465	0.0090701\\
1466	0.0070792\\
1467	0.0050904\\
1468	0.0031555\\
1469	0.0013209\\
1470	-0.0003727\\
1471	-0.0018915\\
1472	-0.0032087\\
1473	-0.0043051\\
1474	-0.0051693\\
1475	-0.0057974\\
1476	-0.0061928\\
1477	-0.0063656\\
1478	-0.0063317\\
1479	-0.0061122\\
1480	-0.0057319\\
1481	-0.0052182\\
1482	-0.0046001\\
1483	-0.003907\\
1484	-0.0031676\\
1485	-0.0024091\\
1486	-0.0016564\\
1487	-0.00093141\\
1488	-0.00025302\\
1489	0.00036351\\
1490	0.00090644\\
1491	0.0013675\\
1492	0.0017418\\
1493	0.0020275\\
1494	0.0022255\\
1495	0.0023392\\
1496	0.002374\\
1497	0.002337\\
1498	0.0022367\\
1499	0.0020822\\
1500	0.0018835\\
1501	0.0016508\\
1502	0.001394\\
1503	0.0011229\\
1504	0.00084666\\
1505	0.00057374\\
1506	0.00031162\\
1507	6.6744e-05\\
1508	-0.00015553\\
1509	-0.00035104\\
1510	-0.00051678\\
1511	-0.0006509\\
1512	-0.00075266\\
1513	-0.00082231\\
1514	-0.00086103\\
1515	-0.00087077\\
1516	-0.00085415\\
1517	-0.00081429\\
1518	-0.00075469\\
1519	-0.00067908\\
1520	-0.00059128\\
1521	-0.00049509\\
1522	-0.00039416\\
1523	-0.00029192\\
1524	-0.00019149\\
1525	-9.5611e-05\\
1526	-6.624e-06\\
1527	7.3579e-05\\
1528	0.00014355\\
1529	0.0002023\\
1530	0.00024924\\
1531	0.00028422\\
1532	0.00030742\\
1533	0.00031936\\
1534	0.00032083\\
1535	0.00031285\\
1536	0.00029661\\
1537	0.00027339\\
1538	0.00024459\\
1539	0.00021159\\
1540	0.00017578\\
1541	0.00013847\\
1542	0.0001009\\
1543	6.4181e-05\\
1544	2.9295e-05\\
1545	-2.9313e-06\\
1546	-3.183e-05\\
1547	-5.6897e-05\\
1548	-7.779e-05\\
1549	-9.4321e-05\\
1550	-0.00010645\\
1551	-0.00011426\\
1552	-0.00011797\\
1553	-0.00011788\\
1554	-0.00011437\\
1555	-0.00010789\\
1556	-9.8928e-05\\
1557	-8.7993e-05\\
1558	-7.5602e-05\\
1559	-6.2262e-05\\
1560	-4.8455e-05\\
1561	-3.4631e-05\\
1562	-2.1194e-05\\
1563	-8.4932e-06\\
1564	3.1749e-06\\
1565	1.3576e-05\\
1566	2.2535e-05\\
1567	2.9937e-05\\
1568	3.5724e-05\\
1569	3.9892e-05\\
1570	4.248e-05\\
1571	4.3575e-05\\
1572	4.3294e-05\\
1573	4.1784e-05\\
1574	3.9213e-05\\
1575	3.5762e-05\\
1576	3.162e-05\\
1577	2.6977e-05\\
1578	2.2018e-05\\
1579	1.6919e-05\\
1580	1.1842e-05\\
1581	6.9316e-06\\
1582	2.3147e-06\\
1583	-1.9047e-06\\
1584	-5.644e-06\\
1585	-8.8429e-06\\
1586	-1.1463e-05\\
1587	-1.3486e-05\\
1588	-1.4914e-05\\
1589	-1.5765e-05\\
1590	-1.6073e-05\\
1591	-1.5882e-05\\
1592	-1.525e-05\\
1593	-1.4238e-05\\
1594	-1.2914e-05\\
1595	-1.1349e-05\\
1596	-9.6117e-06\\
1597	-7.77e-06\\
1598	-5.8881e-06\\
1599	-4.0246e-06\\
1600	-2.2319e-06\\
1601	-5.5483e-07\\
1602	9.695e-07\\
1603	2.3122e-06\\
1604	3.4525e-06\\
1605	0.0010817\\
1606	0.0043126\\
1607	0.0096419\\
1608	0.016545\\
1609	0.024378\\
1610	0.03256\\
1611	0.04063\\
1612	0.04826\\
1613	0.055241\\
1614	0.061462\\
1615	0.06689\\
1616	0.071544\\
1617	0.075482\\
1618	0.078778\\
1619	0.081515\\
1620	0.083775\\
1621	0.085638\\
1622	0.087171\\
1623	0.088435\\
1624	0.089483\\
1625	0.090355\\
1626	0.091087\\
1627	0.091708\\
1628	0.09224\\
1629	0.0927\\
1630	0.093104\\
1631	0.093462\\
1632	0.093783\\
1633	0.094073\\
1634	0.094338\\
1635	0.094582\\
1636	0.094809\\
1637	0.09502\\
1638	0.095218\\
1639	0.095404\\
1640	0.095581\\
1641	0.095748\\
1642	0.095906\\
1643	0.096057\\
1644	0.096201\\
1645	0.096338\\
1646	0.096469\\
1647	0.096594\\
1648	0.096714\\
1649	0.096829\\
1650	0.096939\\
1651	0.097045\\
1652	0.097146\\
1653	0.097244\\
1654	0.097338\\
1655	0.097428\\
1656	0.097514\\
1657	0.097597\\
1658	0.097677\\
1659	0.097754\\
1660	0.097829\\
1661	0.0979\\
1662	0.097969\\
1663	0.098035\\
1664	0.098099\\
1665	0.098161\\
1666	0.09822\\
1667	0.098278\\
1668	0.098333\\
1669	0.098386\\
1670	0.098438\\
1671	0.098488\\
1672	0.098536\\
1673	0.098582\\
1674	0.098627\\
1675	0.09867\\
1676	0.098712\\
1677	0.098752\\
1678	0.098792\\
1679	0.098829\\
1680	0.098866\\
1681	0.098901\\
1682	0.098935\\
1683	0.098968\\
1684	0.099\\
1685	0.099031\\
1686	0.099061\\
1687	0.09909\\
1688	0.099118\\
1689	0.099145\\
1690	0.099171\\
1691	0.099197\\
1692	0.099221\\
1693	0.099245\\
1694	0.099268\\
1695	0.09929\\
1696	0.099312\\
1697	0.099333\\
1698	0.099353\\
1699	0.099373\\
1700	0.099392\\
1701	0.09941\\
1702	0.099428\\
1703	0.099445\\
1704	0.099462\\
1705	0.099478\\
1706	0.099494\\
1707	0.099509\\
1708	0.099524\\
1709	0.099538\\
1710	0.099552\\
1711	0.099565\\
1712	0.099579\\
1713	0.099591\\
1714	0.099603\\
1715	0.099615\\
1716	0.099627\\
1717	0.099638\\
1718	0.099649\\
1719	0.099659\\
1720	0.099669\\
1721	0.099679\\
1722	0.099689\\
1723	0.099698\\
1724	0.099707\\
1725	0.099716\\
1726	0.099724\\
1727	0.099732\\
1728	0.09974\\
1729	0.099748\\
1730	0.099756\\
1731	0.099763\\
1732	0.09977\\
1733	0.099777\\
1734	0.099783\\
1735	0.09979\\
1736	0.099796\\
1737	0.099802\\
1738	0.099808\\
1739	0.099814\\
1740	0.099819\\
1741	0.099824\\
1742	0.09983\\
1743	0.099835\\
1744	0.09984\\
1745	0.099844\\
1746	0.099849\\
1747	0.099853\\
1748	0.099858\\
1749	0.099862\\
1750	0.099866\\
1751	0.09987\\
1752	0.099874\\
1753	0.099877\\
1754	0.099881\\
1755	0.099885\\
1756	0.099888\\
1757	0.099891\\
1758	0.099894\\
1759	0.099898\\
1760	0.099901\\
1761	0.099904\\
1762	0.099906\\
1763	0.099909\\
1764	0.099912\\
1765	0.099914\\
1766	0.099917\\
1767	0.099919\\
1768	0.099922\\
1769	0.099924\\
1770	0.099926\\
1771	0.099928\\
1772	0.099931\\
1773	0.099933\\
1774	0.099935\\
1775	0.099936\\
1776	0.099938\\
1777	0.09994\\
1778	0.099942\\
1779	0.099944\\
1780	0.099945\\
1781	0.099947\\
1782	0.099948\\
1783	0.09995\\
1784	0.099951\\
1785	0.099953\\
1786	0.099954\\
1787	0.099956\\
1788	0.099957\\
1789	0.099958\\
1790	0.099959\\
1791	0.099961\\
1792	0.099962\\
1793	0.099963\\
1794	0.099964\\
1795	0.099965\\
1796	0.099966\\
1797	0.099967\\
1798	0.099968\\
1799	0.099969\\
1800	0.09997\\
};
\addlegendentry{Wyjście y}

\addplot [color=mycolor2, dashed]
  table[row sep=crcr]{%
1	0\\
2	0\\
3	0\\
4	0\\
5	0\\
6	0\\
7	0\\
8	0\\
9	0\\
10	0\\
11	0\\
12	0\\
13	0\\
14	0\\
15	0\\
16	0\\
17	0\\
18	0\\
19	0\\
20	-2\\
21	-2\\
22	-2\\
23	-2\\
24	-2\\
25	-2\\
26	-2\\
27	-2\\
28	-2\\
29	-2\\
30	-2\\
31	-2\\
32	-2\\
33	-2\\
34	-2\\
35	-2\\
36	-2\\
37	-2\\
38	-2\\
39	-2\\
40	-2\\
41	-2\\
42	-2\\
43	-2\\
44	-2\\
45	-2\\
46	-2\\
47	-2\\
48	-2\\
49	-2\\
50	-2\\
51	-2\\
52	-2\\
53	-2\\
54	-2\\
55	-2\\
56	-2\\
57	-2\\
58	-2\\
59	-2\\
60	-2\\
61	-2\\
62	-2\\
63	-2\\
64	-2\\
65	-2\\
66	-2\\
67	-2\\
68	-2\\
69	-2\\
70	-2\\
71	-2\\
72	-2\\
73	-2\\
74	-2\\
75	-2\\
76	-2\\
77	-2\\
78	-2\\
79	-2\\
80	-2\\
81	-2\\
82	-2\\
83	-2\\
84	-2\\
85	-2\\
86	-2\\
87	-2\\
88	-2\\
89	-2\\
90	-2\\
91	-2\\
92	-2\\
93	-2\\
94	-2\\
95	-2\\
96	-2\\
97	-2\\
98	-2\\
99	-2\\
100	-2\\
101	-2\\
102	-2\\
103	-2\\
104	-2\\
105	-2\\
106	-2\\
107	-2\\
108	-2\\
109	-2\\
110	-2\\
111	-2\\
112	-2\\
113	-2\\
114	-2\\
115	-2\\
116	-2\\
117	-2\\
118	-2\\
119	-2\\
120	-2\\
121	-2\\
122	-2\\
123	-2\\
124	-2\\
125	-2\\
126	-2\\
127	-2\\
128	-2\\
129	-2\\
130	-2\\
131	-2\\
132	-2\\
133	-2\\
134	-2\\
135	-2\\
136	-2\\
137	-2\\
138	-2\\
139	-2\\
140	-2\\
141	-2\\
142	-2\\
143	-2\\
144	-2\\
145	-2\\
146	-2\\
147	-2\\
148	-2\\
149	-2\\
150	-2\\
151	-2\\
152	-2\\
153	-2\\
154	-2\\
155	-2\\
156	-2\\
157	-2\\
158	-2\\
159	-2\\
160	-2\\
161	-2\\
162	-2\\
163	-2\\
164	-2\\
165	-2\\
166	-2\\
167	-2\\
168	-2\\
169	-2\\
170	-2\\
171	-2\\
172	-2\\
173	-2\\
174	-2\\
175	-2\\
176	-2\\
177	-2\\
178	-2\\
179	-2\\
180	-2\\
181	-2\\
182	-2\\
183	-2\\
184	-2\\
185	-2\\
186	-2\\
187	-2\\
188	-2\\
189	-2\\
190	-2\\
191	-2\\
192	-2\\
193	-2\\
194	-2\\
195	-2\\
196	-2\\
197	-2\\
198	-2\\
199	-2\\
200	-4.5\\
201	-4.5\\
202	-4.5\\
203	-4.5\\
204	-4.5\\
205	-4.5\\
206	-4.5\\
207	-4.5\\
208	-4.5\\
209	-4.5\\
210	-4.5\\
211	-4.5\\
212	-4.5\\
213	-4.5\\
214	-4.5\\
215	-4.5\\
216	-4.5\\
217	-4.5\\
218	-4.5\\
219	-4.5\\
220	-4.5\\
221	-4.5\\
222	-4.5\\
223	-4.5\\
224	-4.5\\
225	-4.5\\
226	-4.5\\
227	-4.5\\
228	-4.5\\
229	-4.5\\
230	-4.5\\
231	-4.5\\
232	-4.5\\
233	-4.5\\
234	-4.5\\
235	-4.5\\
236	-4.5\\
237	-4.5\\
238	-4.5\\
239	-4.5\\
240	-4.5\\
241	-4.5\\
242	-4.5\\
243	-4.5\\
244	-4.5\\
245	-4.5\\
246	-4.5\\
247	-4.5\\
248	-4.5\\
249	-4.5\\
250	-4.5\\
251	-4.5\\
252	-4.5\\
253	-4.5\\
254	-4.5\\
255	-4.5\\
256	-4.5\\
257	-4.5\\
258	-4.5\\
259	-4.5\\
260	-4.5\\
261	-4.5\\
262	-4.5\\
263	-4.5\\
264	-4.5\\
265	-4.5\\
266	-4.5\\
267	-4.5\\
268	-4.5\\
269	-4.5\\
270	-4.5\\
271	-4.5\\
272	-4.5\\
273	-4.5\\
274	-4.5\\
275	-4.5\\
276	-4.5\\
277	-4.5\\
278	-4.5\\
279	-4.5\\
280	-4.5\\
281	-4.5\\
282	-4.5\\
283	-4.5\\
284	-4.5\\
285	-4.5\\
286	-4.5\\
287	-4.5\\
288	-4.5\\
289	-4.5\\
290	-4.5\\
291	-4.5\\
292	-4.5\\
293	-4.5\\
294	-4.5\\
295	-4.5\\
296	-4.5\\
297	-4.5\\
298	-4.5\\
299	-4.5\\
300	-4.5\\
301	-4.5\\
302	-4.5\\
303	-4.5\\
304	-4.5\\
305	-4.5\\
306	-4.5\\
307	-4.5\\
308	-4.5\\
309	-4.5\\
310	-4.5\\
311	-4.5\\
312	-4.5\\
313	-4.5\\
314	-4.5\\
315	-4.5\\
316	-4.5\\
317	-4.5\\
318	-4.5\\
319	-4.5\\
320	-4.5\\
321	-4.5\\
322	-4.5\\
323	-4.5\\
324	-4.5\\
325	-4.5\\
326	-4.5\\
327	-4.5\\
328	-4.5\\
329	-4.5\\
330	-4.5\\
331	-4.5\\
332	-4.5\\
333	-4.5\\
334	-4.5\\
335	-4.5\\
336	-4.5\\
337	-4.5\\
338	-4.5\\
339	-4.5\\
340	-4.5\\
341	-4.5\\
342	-4.5\\
343	-4.5\\
344	-4.5\\
345	-4.5\\
346	-4.5\\
347	-4.5\\
348	-4.5\\
349	-4.5\\
350	-4.5\\
351	-4.5\\
352	-4.5\\
353	-4.5\\
354	-4.5\\
355	-4.5\\
356	-4.5\\
357	-4.5\\
358	-4.5\\
359	-4.5\\
360	-4.5\\
361	-4.5\\
362	-4.5\\
363	-4.5\\
364	-4.5\\
365	-4.5\\
366	-4.5\\
367	-4.5\\
368	-4.5\\
369	-4.5\\
370	-4.5\\
371	-4.5\\
372	-4.5\\
373	-4.5\\
374	-4.5\\
375	-4.5\\
376	-4.5\\
377	-4.5\\
378	-4.5\\
379	-4.5\\
380	-4.5\\
381	-4.5\\
382	-4.5\\
383	-4.5\\
384	-4.5\\
385	-4.5\\
386	-4.5\\
387	-4.5\\
388	-4.5\\
389	-4.5\\
390	-4.5\\
391	-4.5\\
392	-4.5\\
393	-4.5\\
394	-4.5\\
395	-4.5\\
396	-4.5\\
397	-4.5\\
398	-4.5\\
399	-4.5\\
400	-3\\
401	-3\\
402	-3\\
403	-3\\
404	-3\\
405	-3\\
406	-3\\
407	-3\\
408	-3\\
409	-3\\
410	-3\\
411	-3\\
412	-3\\
413	-3\\
414	-3\\
415	-3\\
416	-3\\
417	-3\\
418	-3\\
419	-3\\
420	-3\\
421	-3\\
422	-3\\
423	-3\\
424	-3\\
425	-3\\
426	-3\\
427	-3\\
428	-3\\
429	-3\\
430	-3\\
431	-3\\
432	-3\\
433	-3\\
434	-3\\
435	-3\\
436	-3\\
437	-3\\
438	-3\\
439	-3\\
440	-3\\
441	-3\\
442	-3\\
443	-3\\
444	-3\\
445	-3\\
446	-3\\
447	-3\\
448	-3\\
449	-3\\
450	-3\\
451	-3\\
452	-3\\
453	-3\\
454	-3\\
455	-3\\
456	-3\\
457	-3\\
458	-3\\
459	-3\\
460	-3\\
461	-3\\
462	-3\\
463	-3\\
464	-3\\
465	-3\\
466	-3\\
467	-3\\
468	-3\\
469	-3\\
470	-3\\
471	-3\\
472	-3\\
473	-3\\
474	-3\\
475	-3\\
476	-3\\
477	-3\\
478	-3\\
479	-3\\
480	-3\\
481	-3\\
482	-3\\
483	-3\\
484	-3\\
485	-3\\
486	-3\\
487	-3\\
488	-3\\
489	-3\\
490	-3\\
491	-3\\
492	-3\\
493	-3\\
494	-3\\
495	-3\\
496	-3\\
497	-3\\
498	-3\\
499	-3\\
500	-3\\
501	-3\\
502	-3\\
503	-3\\
504	-3\\
505	-3\\
506	-3\\
507	-3\\
508	-3\\
509	-3\\
510	-3\\
511	-3\\
512	-3\\
513	-3\\
514	-3\\
515	-3\\
516	-3\\
517	-3\\
518	-3\\
519	-3\\
520	-3\\
521	-3\\
522	-3\\
523	-3\\
524	-3\\
525	-3\\
526	-3\\
527	-3\\
528	-3\\
529	-3\\
530	-3\\
531	-3\\
532	-3\\
533	-3\\
534	-3\\
535	-3\\
536	-3\\
537	-3\\
538	-3\\
539	-3\\
540	-3\\
541	-3\\
542	-3\\
543	-3\\
544	-3\\
545	-3\\
546	-3\\
547	-3\\
548	-3\\
549	-3\\
550	-3\\
551	-3\\
552	-3\\
553	-3\\
554	-3\\
555	-3\\
556	-3\\
557	-3\\
558	-3\\
559	-3\\
560	-3\\
561	-3\\
562	-3\\
563	-3\\
564	-3\\
565	-3\\
566	-3\\
567	-3\\
568	-3\\
569	-3\\
570	-3\\
571	-3\\
572	-3\\
573	-3\\
574	-3\\
575	-3\\
576	-3\\
577	-3\\
578	-3\\
579	-3\\
580	-3\\
581	-3\\
582	-3\\
583	-3\\
584	-3\\
585	-3\\
586	-3\\
587	-3\\
588	-3\\
589	-3\\
590	-3\\
591	-3\\
592	-3\\
593	-3\\
594	-3\\
595	-3\\
596	-3\\
597	-3\\
598	-3\\
599	-3\\
600	-1.5\\
601	-1.5\\
602	-1.5\\
603	-1.5\\
604	-1.5\\
605	-1.5\\
606	-1.5\\
607	-1.5\\
608	-1.5\\
609	-1.5\\
610	-1.5\\
611	-1.5\\
612	-1.5\\
613	-1.5\\
614	-1.5\\
615	-1.5\\
616	-1.5\\
617	-1.5\\
618	-1.5\\
619	-1.5\\
620	-1.5\\
621	-1.5\\
622	-1.5\\
623	-1.5\\
624	-1.5\\
625	-1.5\\
626	-1.5\\
627	-1.5\\
628	-1.5\\
629	-1.5\\
630	-1.5\\
631	-1.5\\
632	-1.5\\
633	-1.5\\
634	-1.5\\
635	-1.5\\
636	-1.5\\
637	-1.5\\
638	-1.5\\
639	-1.5\\
640	-1.5\\
641	-1.5\\
642	-1.5\\
643	-1.5\\
644	-1.5\\
645	-1.5\\
646	-1.5\\
647	-1.5\\
648	-1.5\\
649	-1.5\\
650	-1.5\\
651	-1.5\\
652	-1.5\\
653	-1.5\\
654	-1.5\\
655	-1.5\\
656	-1.5\\
657	-1.5\\
658	-1.5\\
659	-1.5\\
660	-1.5\\
661	-1.5\\
662	-1.5\\
663	-1.5\\
664	-1.5\\
665	-1.5\\
666	-1.5\\
667	-1.5\\
668	-1.5\\
669	-1.5\\
670	-1.5\\
671	-1.5\\
672	-1.5\\
673	-1.5\\
674	-1.5\\
675	-1.5\\
676	-1.5\\
677	-1.5\\
678	-1.5\\
679	-1.5\\
680	-1.5\\
681	-1.5\\
682	-1.5\\
683	-1.5\\
684	-1.5\\
685	-1.5\\
686	-1.5\\
687	-1.5\\
688	-1.5\\
689	-1.5\\
690	-1.5\\
691	-1.5\\
692	-1.5\\
693	-1.5\\
694	-1.5\\
695	-1.5\\
696	-1.5\\
697	-1.5\\
698	-1.5\\
699	-1.5\\
700	-1.5\\
701	-1.5\\
702	-1.5\\
703	-1.5\\
704	-1.5\\
705	-1.5\\
706	-1.5\\
707	-1.5\\
708	-1.5\\
709	-1.5\\
710	-1.5\\
711	-1.5\\
712	-1.5\\
713	-1.5\\
714	-1.5\\
715	-1.5\\
716	-1.5\\
717	-1.5\\
718	-1.5\\
719	-1.5\\
720	-1.5\\
721	-1.5\\
722	-1.5\\
723	-1.5\\
724	-1.5\\
725	-1.5\\
726	-1.5\\
727	-1.5\\
728	-1.5\\
729	-1.5\\
730	-1.5\\
731	-1.5\\
732	-1.5\\
733	-1.5\\
734	-1.5\\
735	-1.5\\
736	-1.5\\
737	-1.5\\
738	-1.5\\
739	-1.5\\
740	-1.5\\
741	-1.5\\
742	-1.5\\
743	-1.5\\
744	-1.5\\
745	-1.5\\
746	-1.5\\
747	-1.5\\
748	-1.5\\
749	-1.5\\
750	-1.5\\
751	-1.5\\
752	-1.5\\
753	-1.5\\
754	-1.5\\
755	-1.5\\
756	-1.5\\
757	-1.5\\
758	-1.5\\
759	-1.5\\
760	-1.5\\
761	-1.5\\
762	-1.5\\
763	-1.5\\
764	-1.5\\
765	-1.5\\
766	-1.5\\
767	-1.5\\
768	-1.5\\
769	-1.5\\
770	-1.5\\
771	-1.5\\
772	-1.5\\
773	-1.5\\
774	-1.5\\
775	-1.5\\
776	-1.5\\
777	-1.5\\
778	-1.5\\
779	-1.5\\
780	-1.5\\
781	-1.5\\
782	-1.5\\
783	-1.5\\
784	-1.5\\
785	-1.5\\
786	-1.5\\
787	-1.5\\
788	-1.5\\
789	-1.5\\
790	-1.5\\
791	-1.5\\
792	-1.5\\
793	-1.5\\
794	-1.5\\
795	-1.5\\
796	-1.5\\
797	-1.5\\
798	-1.5\\
799	-1.5\\
800	-3.5\\
801	-3.5\\
802	-3.5\\
803	-3.5\\
804	-3.5\\
805	-3.5\\
806	-3.5\\
807	-3.5\\
808	-3.5\\
809	-3.5\\
810	-3.5\\
811	-3.5\\
812	-3.5\\
813	-3.5\\
814	-3.5\\
815	-3.5\\
816	-3.5\\
817	-3.5\\
818	-3.5\\
819	-3.5\\
820	-3.5\\
821	-3.5\\
822	-3.5\\
823	-3.5\\
824	-3.5\\
825	-3.5\\
826	-3.5\\
827	-3.5\\
828	-3.5\\
829	-3.5\\
830	-3.5\\
831	-3.5\\
832	-3.5\\
833	-3.5\\
834	-3.5\\
835	-3.5\\
836	-3.5\\
837	-3.5\\
838	-3.5\\
839	-3.5\\
840	-3.5\\
841	-3.5\\
842	-3.5\\
843	-3.5\\
844	-3.5\\
845	-3.5\\
846	-3.5\\
847	-3.5\\
848	-3.5\\
849	-3.5\\
850	-3.5\\
851	-3.5\\
852	-3.5\\
853	-3.5\\
854	-3.5\\
855	-3.5\\
856	-3.5\\
857	-3.5\\
858	-3.5\\
859	-3.5\\
860	-3.5\\
861	-3.5\\
862	-3.5\\
863	-3.5\\
864	-3.5\\
865	-3.5\\
866	-3.5\\
867	-3.5\\
868	-3.5\\
869	-3.5\\
870	-3.5\\
871	-3.5\\
872	-3.5\\
873	-3.5\\
874	-3.5\\
875	-3.5\\
876	-3.5\\
877	-3.5\\
878	-3.5\\
879	-3.5\\
880	-3.5\\
881	-3.5\\
882	-3.5\\
883	-3.5\\
884	-3.5\\
885	-3.5\\
886	-3.5\\
887	-3.5\\
888	-3.5\\
889	-3.5\\
890	-3.5\\
891	-3.5\\
892	-3.5\\
893	-3.5\\
894	-3.5\\
895	-3.5\\
896	-3.5\\
897	-3.5\\
898	-3.5\\
899	-3.5\\
900	-3.5\\
901	-3.5\\
902	-3.5\\
903	-3.5\\
904	-3.5\\
905	-3.5\\
906	-3.5\\
907	-3.5\\
908	-3.5\\
909	-3.5\\
910	-3.5\\
911	-3.5\\
912	-3.5\\
913	-3.5\\
914	-3.5\\
915	-3.5\\
916	-3.5\\
917	-3.5\\
918	-3.5\\
919	-3.5\\
920	-3.5\\
921	-3.5\\
922	-3.5\\
923	-3.5\\
924	-3.5\\
925	-3.5\\
926	-3.5\\
927	-3.5\\
928	-3.5\\
929	-3.5\\
930	-3.5\\
931	-3.5\\
932	-3.5\\
933	-3.5\\
934	-3.5\\
935	-3.5\\
936	-3.5\\
937	-3.5\\
938	-3.5\\
939	-3.5\\
940	-3.5\\
941	-3.5\\
942	-3.5\\
943	-3.5\\
944	-3.5\\
945	-3.5\\
946	-3.5\\
947	-3.5\\
948	-3.5\\
949	-3.5\\
950	-3.5\\
951	-3.5\\
952	-3.5\\
953	-3.5\\
954	-3.5\\
955	-3.5\\
956	-3.5\\
957	-3.5\\
958	-3.5\\
959	-3.5\\
960	-3.5\\
961	-3.5\\
962	-3.5\\
963	-3.5\\
964	-3.5\\
965	-3.5\\
966	-3.5\\
967	-3.5\\
968	-3.5\\
969	-3.5\\
970	-3.5\\
971	-3.5\\
972	-3.5\\
973	-3.5\\
974	-3.5\\
975	-3.5\\
976	-3.5\\
977	-3.5\\
978	-3.5\\
979	-3.5\\
980	-3.5\\
981	-3.5\\
982	-3.5\\
983	-3.5\\
984	-3.5\\
985	-3.5\\
986	-3.5\\
987	-3.5\\
988	-3.5\\
989	-3.5\\
990	-3.5\\
991	-3.5\\
992	-3.5\\
993	-3.5\\
994	-3.5\\
995	-3.5\\
996	-3.5\\
997	-3.5\\
998	-3.5\\
999	-3.5\\
1000	-1.2\\
1001	-1.2\\
1002	-1.2\\
1003	-1.2\\
1004	-1.2\\
1005	-1.2\\
1006	-1.2\\
1007	-1.2\\
1008	-1.2\\
1009	-1.2\\
1010	-1.2\\
1011	-1.2\\
1012	-1.2\\
1013	-1.2\\
1014	-1.2\\
1015	-1.2\\
1016	-1.2\\
1017	-1.2\\
1018	-1.2\\
1019	-1.2\\
1020	-1.2\\
1021	-1.2\\
1022	-1.2\\
1023	-1.2\\
1024	-1.2\\
1025	-1.2\\
1026	-1.2\\
1027	-1.2\\
1028	-1.2\\
1029	-1.2\\
1030	-1.2\\
1031	-1.2\\
1032	-1.2\\
1033	-1.2\\
1034	-1.2\\
1035	-1.2\\
1036	-1.2\\
1037	-1.2\\
1038	-1.2\\
1039	-1.2\\
1040	-1.2\\
1041	-1.2\\
1042	-1.2\\
1043	-1.2\\
1044	-1.2\\
1045	-1.2\\
1046	-1.2\\
1047	-1.2\\
1048	-1.2\\
1049	-1.2\\
1050	-1.2\\
1051	-1.2\\
1052	-1.2\\
1053	-1.2\\
1054	-1.2\\
1055	-1.2\\
1056	-1.2\\
1057	-1.2\\
1058	-1.2\\
1059	-1.2\\
1060	-1.2\\
1061	-1.2\\
1062	-1.2\\
1063	-1.2\\
1064	-1.2\\
1065	-1.2\\
1066	-1.2\\
1067	-1.2\\
1068	-1.2\\
1069	-1.2\\
1070	-1.2\\
1071	-1.2\\
1072	-1.2\\
1073	-1.2\\
1074	-1.2\\
1075	-1.2\\
1076	-1.2\\
1077	-1.2\\
1078	-1.2\\
1079	-1.2\\
1080	-1.2\\
1081	-1.2\\
1082	-1.2\\
1083	-1.2\\
1084	-1.2\\
1085	-1.2\\
1086	-1.2\\
1087	-1.2\\
1088	-1.2\\
1089	-1.2\\
1090	-1.2\\
1091	-1.2\\
1092	-1.2\\
1093	-1.2\\
1094	-1.2\\
1095	-1.2\\
1096	-1.2\\
1097	-1.2\\
1098	-1.2\\
1099	-1.2\\
1100	-1.2\\
1101	-1.2\\
1102	-1.2\\
1103	-1.2\\
1104	-1.2\\
1105	-1.2\\
1106	-1.2\\
1107	-1.2\\
1108	-1.2\\
1109	-1.2\\
1110	-1.2\\
1111	-1.2\\
1112	-1.2\\
1113	-1.2\\
1114	-1.2\\
1115	-1.2\\
1116	-1.2\\
1117	-1.2\\
1118	-1.2\\
1119	-1.2\\
1120	-1.2\\
1121	-1.2\\
1122	-1.2\\
1123	-1.2\\
1124	-1.2\\
1125	-1.2\\
1126	-1.2\\
1127	-1.2\\
1128	-1.2\\
1129	-1.2\\
1130	-1.2\\
1131	-1.2\\
1132	-1.2\\
1133	-1.2\\
1134	-1.2\\
1135	-1.2\\
1136	-1.2\\
1137	-1.2\\
1138	-1.2\\
1139	-1.2\\
1140	-1.2\\
1141	-1.2\\
1142	-1.2\\
1143	-1.2\\
1144	-1.2\\
1145	-1.2\\
1146	-1.2\\
1147	-1.2\\
1148	-1.2\\
1149	-1.2\\
1150	-1.2\\
1151	-1.2\\
1152	-1.2\\
1153	-1.2\\
1154	-1.2\\
1155	-1.2\\
1156	-1.2\\
1157	-1.2\\
1158	-1.2\\
1159	-1.2\\
1160	-1.2\\
1161	-1.2\\
1162	-1.2\\
1163	-1.2\\
1164	-1.2\\
1165	-1.2\\
1166	-1.2\\
1167	-1.2\\
1168	-1.2\\
1169	-1.2\\
1170	-1.2\\
1171	-1.2\\
1172	-1.2\\
1173	-1.2\\
1174	-1.2\\
1175	-1.2\\
1176	-1.2\\
1177	-1.2\\
1178	-1.2\\
1179	-1.2\\
1180	-1.2\\
1181	-1.2\\
1182	-1.2\\
1183	-1.2\\
1184	-1.2\\
1185	-1.2\\
1186	-1.2\\
1187	-1.2\\
1188	-1.2\\
1189	-1.2\\
1190	-1.2\\
1191	-1.2\\
1192	-1.2\\
1193	-1.2\\
1194	-1.2\\
1195	-1.2\\
1196	-1.2\\
1197	-1.2\\
1198	-1.2\\
1199	-1.2\\
1200	-0.3\\
1201	-0.3\\
1202	-0.3\\
1203	-0.3\\
1204	-0.3\\
1205	-0.3\\
1206	-0.3\\
1207	-0.3\\
1208	-0.3\\
1209	-0.3\\
1210	-0.3\\
1211	-0.3\\
1212	-0.3\\
1213	-0.3\\
1214	-0.3\\
1215	-0.3\\
1216	-0.3\\
1217	-0.3\\
1218	-0.3\\
1219	-0.3\\
1220	-0.3\\
1221	-0.3\\
1222	-0.3\\
1223	-0.3\\
1224	-0.3\\
1225	-0.3\\
1226	-0.3\\
1227	-0.3\\
1228	-0.3\\
1229	-0.3\\
1230	-0.3\\
1231	-0.3\\
1232	-0.3\\
1233	-0.3\\
1234	-0.3\\
1235	-0.3\\
1236	-0.3\\
1237	-0.3\\
1238	-0.3\\
1239	-0.3\\
1240	-0.3\\
1241	-0.3\\
1242	-0.3\\
1243	-0.3\\
1244	-0.3\\
1245	-0.3\\
1246	-0.3\\
1247	-0.3\\
1248	-0.3\\
1249	-0.3\\
1250	-0.3\\
1251	-0.3\\
1252	-0.3\\
1253	-0.3\\
1254	-0.3\\
1255	-0.3\\
1256	-0.3\\
1257	-0.3\\
1258	-0.3\\
1259	-0.3\\
1260	-0.3\\
1261	-0.3\\
1262	-0.3\\
1263	-0.3\\
1264	-0.3\\
1265	-0.3\\
1266	-0.3\\
1267	-0.3\\
1268	-0.3\\
1269	-0.3\\
1270	-0.3\\
1271	-0.3\\
1272	-0.3\\
1273	-0.3\\
1274	-0.3\\
1275	-0.3\\
1276	-0.3\\
1277	-0.3\\
1278	-0.3\\
1279	-0.3\\
1280	-0.3\\
1281	-0.3\\
1282	-0.3\\
1283	-0.3\\
1284	-0.3\\
1285	-0.3\\
1286	-0.3\\
1287	-0.3\\
1288	-0.3\\
1289	-0.3\\
1290	-0.3\\
1291	-0.3\\
1292	-0.3\\
1293	-0.3\\
1294	-0.3\\
1295	-0.3\\
1296	-0.3\\
1297	-0.3\\
1298	-0.3\\
1299	-0.3\\
1300	-0.3\\
1301	-0.3\\
1302	-0.3\\
1303	-0.3\\
1304	-0.3\\
1305	-0.3\\
1306	-0.3\\
1307	-0.3\\
1308	-0.3\\
1309	-0.3\\
1310	-0.3\\
1311	-0.3\\
1312	-0.3\\
1313	-0.3\\
1314	-0.3\\
1315	-0.3\\
1316	-0.3\\
1317	-0.3\\
1318	-0.3\\
1319	-0.3\\
1320	-0.3\\
1321	-0.3\\
1322	-0.3\\
1323	-0.3\\
1324	-0.3\\
1325	-0.3\\
1326	-0.3\\
1327	-0.3\\
1328	-0.3\\
1329	-0.3\\
1330	-0.3\\
1331	-0.3\\
1332	-0.3\\
1333	-0.3\\
1334	-0.3\\
1335	-0.3\\
1336	-0.3\\
1337	-0.3\\
1338	-0.3\\
1339	-0.3\\
1340	-0.3\\
1341	-0.3\\
1342	-0.3\\
1343	-0.3\\
1344	-0.3\\
1345	-0.3\\
1346	-0.3\\
1347	-0.3\\
1348	-0.3\\
1349	-0.3\\
1350	-0.3\\
1351	-0.3\\
1352	-0.3\\
1353	-0.3\\
1354	-0.3\\
1355	-0.3\\
1356	-0.3\\
1357	-0.3\\
1358	-0.3\\
1359	-0.3\\
1360	-0.3\\
1361	-0.3\\
1362	-0.3\\
1363	-0.3\\
1364	-0.3\\
1365	-0.3\\
1366	-0.3\\
1367	-0.3\\
1368	-0.3\\
1369	-0.3\\
1370	-0.3\\
1371	-0.3\\
1372	-0.3\\
1373	-0.3\\
1374	-0.3\\
1375	-0.3\\
1376	-0.3\\
1377	-0.3\\
1378	-0.3\\
1379	-0.3\\
1380	-0.3\\
1381	-0.3\\
1382	-0.3\\
1383	-0.3\\
1384	-0.3\\
1385	-0.3\\
1386	-0.3\\
1387	-0.3\\
1388	-0.3\\
1389	-0.3\\
1390	-0.3\\
1391	-0.3\\
1392	-0.3\\
1393	-0.3\\
1394	-0.3\\
1395	-0.3\\
1396	-0.3\\
1397	-0.3\\
1398	-0.3\\
1399	-0.3\\
1400	0\\
1401	0\\
1402	0\\
1403	0\\
1404	0\\
1405	0\\
1406	0\\
1407	0\\
1408	0\\
1409	0\\
1410	0\\
1411	0\\
1412	0\\
1413	0\\
1414	0\\
1415	0\\
1416	0\\
1417	0\\
1418	0\\
1419	0\\
1420	0\\
1421	0\\
1422	0\\
1423	0\\
1424	0\\
1425	0\\
1426	0\\
1427	0\\
1428	0\\
1429	0\\
1430	0\\
1431	0\\
1432	0\\
1433	0\\
1434	0\\
1435	0\\
1436	0\\
1437	0\\
1438	0\\
1439	0\\
1440	0\\
1441	0\\
1442	0\\
1443	0\\
1444	0\\
1445	0\\
1446	0\\
1447	0\\
1448	0\\
1449	0\\
1450	0\\
1451	0\\
1452	0\\
1453	0\\
1454	0\\
1455	0\\
1456	0\\
1457	0\\
1458	0\\
1459	0\\
1460	0\\
1461	0\\
1462	0\\
1463	0\\
1464	0\\
1465	0\\
1466	0\\
1467	0\\
1468	0\\
1469	0\\
1470	0\\
1471	0\\
1472	0\\
1473	0\\
1474	0\\
1475	0\\
1476	0\\
1477	0\\
1478	0\\
1479	0\\
1480	0\\
1481	0\\
1482	0\\
1483	0\\
1484	0\\
1485	0\\
1486	0\\
1487	0\\
1488	0\\
1489	0\\
1490	0\\
1491	0\\
1492	0\\
1493	0\\
1494	0\\
1495	0\\
1496	0\\
1497	0\\
1498	0\\
1499	0\\
1500	0\\
1501	0\\
1502	0\\
1503	0\\
1504	0\\
1505	0\\
1506	0\\
1507	0\\
1508	0\\
1509	0\\
1510	0\\
1511	0\\
1512	0\\
1513	0\\
1514	0\\
1515	0\\
1516	0\\
1517	0\\
1518	0\\
1519	0\\
1520	0\\
1521	0\\
1522	0\\
1523	0\\
1524	0\\
1525	0\\
1526	0\\
1527	0\\
1528	0\\
1529	0\\
1530	0\\
1531	0\\
1532	0\\
1533	0\\
1534	0\\
1535	0\\
1536	0\\
1537	0\\
1538	0\\
1539	0\\
1540	0\\
1541	0\\
1542	0\\
1543	0\\
1544	0\\
1545	0\\
1546	0\\
1547	0\\
1548	0\\
1549	0\\
1550	0\\
1551	0\\
1552	0\\
1553	0\\
1554	0\\
1555	0\\
1556	0\\
1557	0\\
1558	0\\
1559	0\\
1560	0\\
1561	0\\
1562	0\\
1563	0\\
1564	0\\
1565	0\\
1566	0\\
1567	0\\
1568	0\\
1569	0\\
1570	0\\
1571	0\\
1572	0\\
1573	0\\
1574	0\\
1575	0\\
1576	0\\
1577	0\\
1578	0\\
1579	0\\
1580	0\\
1581	0\\
1582	0\\
1583	0\\
1584	0\\
1585	0\\
1586	0\\
1587	0\\
1588	0\\
1589	0\\
1590	0\\
1591	0\\
1592	0\\
1593	0\\
1594	0\\
1595	0\\
1596	0\\
1597	0\\
1598	0\\
1599	0\\
1600	0.1\\
1601	0.1\\
1602	0.1\\
1603	0.1\\
1604	0.1\\
1605	0.1\\
1606	0.1\\
1607	0.1\\
1608	0.1\\
1609	0.1\\
1610	0.1\\
1611	0.1\\
1612	0.1\\
1613	0.1\\
1614	0.1\\
1615	0.1\\
1616	0.1\\
1617	0.1\\
1618	0.1\\
1619	0.1\\
1620	0.1\\
1621	0.1\\
1622	0.1\\
1623	0.1\\
1624	0.1\\
1625	0.1\\
1626	0.1\\
1627	0.1\\
1628	0.1\\
1629	0.1\\
1630	0.1\\
1631	0.1\\
1632	0.1\\
1633	0.1\\
1634	0.1\\
1635	0.1\\
1636	0.1\\
1637	0.1\\
1638	0.1\\
1639	0.1\\
1640	0.1\\
1641	0.1\\
1642	0.1\\
1643	0.1\\
1644	0.1\\
1645	0.1\\
1646	0.1\\
1647	0.1\\
1648	0.1\\
1649	0.1\\
1650	0.1\\
1651	0.1\\
1652	0.1\\
1653	0.1\\
1654	0.1\\
1655	0.1\\
1656	0.1\\
1657	0.1\\
1658	0.1\\
1659	0.1\\
1660	0.1\\
1661	0.1\\
1662	0.1\\
1663	0.1\\
1664	0.1\\
1665	0.1\\
1666	0.1\\
1667	0.1\\
1668	0.1\\
1669	0.1\\
1670	0.1\\
1671	0.1\\
1672	0.1\\
1673	0.1\\
1674	0.1\\
1675	0.1\\
1676	0.1\\
1677	0.1\\
1678	0.1\\
1679	0.1\\
1680	0.1\\
1681	0.1\\
1682	0.1\\
1683	0.1\\
1684	0.1\\
1685	0.1\\
1686	0.1\\
1687	0.1\\
1688	0.1\\
1689	0.1\\
1690	0.1\\
1691	0.1\\
1692	0.1\\
1693	0.1\\
1694	0.1\\
1695	0.1\\
1696	0.1\\
1697	0.1\\
1698	0.1\\
1699	0.1\\
1700	0.1\\
1701	0.1\\
1702	0.1\\
1703	0.1\\
1704	0.1\\
1705	0.1\\
1706	0.1\\
1707	0.1\\
1708	0.1\\
1709	0.1\\
1710	0.1\\
1711	0.1\\
1712	0.1\\
1713	0.1\\
1714	0.1\\
1715	0.1\\
1716	0.1\\
1717	0.1\\
1718	0.1\\
1719	0.1\\
1720	0.1\\
1721	0.1\\
1722	0.1\\
1723	0.1\\
1724	0.1\\
1725	0.1\\
1726	0.1\\
1727	0.1\\
1728	0.1\\
1729	0.1\\
1730	0.1\\
1731	0.1\\
1732	0.1\\
1733	0.1\\
1734	0.1\\
1735	0.1\\
1736	0.1\\
1737	0.1\\
1738	0.1\\
1739	0.1\\
1740	0.1\\
1741	0.1\\
1742	0.1\\
1743	0.1\\
1744	0.1\\
1745	0.1\\
1746	0.1\\
1747	0.1\\
1748	0.1\\
1749	0.1\\
1750	0.1\\
1751	0.1\\
1752	0.1\\
1753	0.1\\
1754	0.1\\
1755	0.1\\
1756	0.1\\
1757	0.1\\
1758	0.1\\
1759	0.1\\
1760	0.1\\
1761	0.1\\
1762	0.1\\
1763	0.1\\
1764	0.1\\
1765	0.1\\
1766	0.1\\
1767	0.1\\
1768	0.1\\
1769	0.1\\
1770	0.1\\
1771	0.1\\
1772	0.1\\
1773	0.1\\
1774	0.1\\
1775	0.1\\
1776	0.1\\
1777	0.1\\
1778	0.1\\
1779	0.1\\
1780	0.1\\
1781	0.1\\
1782	0.1\\
1783	0.1\\
1784	0.1\\
1785	0.1\\
1786	0.1\\
1787	0.1\\
1788	0.1\\
1789	0.1\\
1790	0.1\\
1791	0.1\\
1792	0.1\\
1793	0.1\\
1794	0.1\\
1795	0.1\\
1796	0.1\\
1797	0.1\\
1798	0.1\\
1799	0.1\\
1800	0.1\\
};
\addlegendentry{$\text{Wartość zadana y}_{\text{zad}}$}

\end{axis}

\begin{axis}[%
width=4.521in,
height=1.493in,
at={(0.758in,0.481in)},
scale only axis,
xmin=1,
xmax=1800,
xlabel style={font=\color{white!15!black}},
xlabel={k},
ymin=-1,
ymax=0.5,
ylabel style={font=\color{white!15!black}},
ylabel={u},
axis background/.style={fill=white},
xmajorgrids,
ymajorgrids,
legend style={legend cell align=left, align=left, draw=white!15!black}
]
\addplot [color=mycolor1]
  table[row sep=crcr]{%
1	0\\
2	0\\
3	0\\
4	0\\
5	0\\
6	0\\
7	0\\
8	0\\
9	0\\
10	0\\
11	0\\
12	0\\
13	0\\
14	0\\
15	0\\
16	0\\
17	0\\
18	0\\
19	0\\
20	-0.3399\\
21	-0.6634\\
22	-0.97925\\
23	-1\\
24	-1\\
25	-1\\
26	-0.98614\\
27	-0.95007\\
28	-0.89951\\
29	-0.83738\\
30	-0.77061\\
31	-0.70354\\
32	-0.64083\\
33	-0.58575\\
34	-0.54098\\
35	-0.50822\\
36	-0.4884\\
37	-0.48149\\
38	-0.48663\\
39	-0.502\\
40	-0.52505\\
41	-0.55267\\
42	-0.58158\\
43	-0.60877\\
44	-0.63178\\
45	-0.64897\\
46	-0.65956\\
47	-0.66354\\
48	-0.66155\\
49	-0.65464\\
50	-0.64415\\
51	-0.63146\\
52	-0.61792\\
53	-0.60474\\
54	-0.59291\\
55	-0.58319\\
56	-0.57604\\
57	-0.57167\\
58	-0.57004\\
59	-0.5709\\
60	-0.5738\\
61	-0.57821\\
62	-0.58351\\
63	-0.5891\\
64	-0.59445\\
65	-0.59911\\
66	-0.60277\\
67	-0.60525\\
68	-0.6065\\
69	-0.6066\\
70	-0.6057\\
71	-0.60402\\
72	-0.60182\\
73	-0.59937\\
74	-0.5969\\
75	-0.59462\\
76	-0.59271\\
77	-0.59125\\
78	-0.59032\\
79	-0.58991\\
80	-0.58998\\
81	-0.59046\\
82	-0.59124\\
83	-0.5922\\
84	-0.59325\\
85	-0.59427\\
86	-0.59519\\
87	-0.59593\\
88	-0.59646\\
89	-0.59677\\
90	-0.59686\\
91	-0.59675\\
92	-0.59648\\
93	-0.5961\\
94	-0.59566\\
95	-0.5952\\
96	-0.59477\\
97	-0.5944\\
98	-0.59411\\
99	-0.59391\\
100	-0.59381\\
101	-0.5938\\
102	-0.59387\\
103	-0.594\\
104	-0.59417\\
105	-0.59437\\
106	-0.59456\\
107	-0.59474\\
108	-0.59489\\
109	-0.595\\
110	-0.59506\\
111	-0.59509\\
112	-0.59508\\
113	-0.59504\\
114	-0.59498\\
115	-0.5949\\
116	-0.59481\\
117	-0.59473\\
118	-0.59466\\
119	-0.5946\\
120	-0.59456\\
121	-0.59454\\
122	-0.59453\\
123	-0.59454\\
124	-0.59456\\
125	-0.59459\\
126	-0.59463\\
127	-0.59466\\
128	-0.5947\\
129	-0.59472\\
130	-0.59475\\
131	-0.59476\\
132	-0.59477\\
133	-0.59477\\
134	-0.59476\\
135	-0.59475\\
136	-0.59474\\
137	-0.59472\\
138	-0.59471\\
139	-0.59469\\
140	-0.59468\\
141	-0.59467\\
142	-0.59467\\
143	-0.59467\\
144	-0.59467\\
145	-0.59467\\
146	-0.59468\\
147	-0.59468\\
148	-0.59469\\
149	-0.5947\\
150	-0.5947\\
151	-0.59471\\
152	-0.59471\\
153	-0.59471\\
154	-0.59471\\
155	-0.59471\\
156	-0.59471\\
157	-0.59471\\
158	-0.5947\\
159	-0.5947\\
160	-0.5947\\
161	-0.5947\\
162	-0.59469\\
163	-0.59469\\
164	-0.59469\\
165	-0.59469\\
166	-0.59469\\
167	-0.59469\\
168	-0.59469\\
169	-0.5947\\
170	-0.5947\\
171	-0.5947\\
172	-0.5947\\
173	-0.5947\\
174	-0.5947\\
175	-0.5947\\
176	-0.5947\\
177	-0.5947\\
178	-0.5947\\
179	-0.5947\\
180	-0.5947\\
181	-0.5947\\
182	-0.5947\\
183	-0.5947\\
184	-0.5947\\
185	-0.5947\\
186	-0.5947\\
187	-0.5947\\
188	-0.5947\\
189	-0.5947\\
190	-0.5947\\
191	-0.5947\\
192	-0.5947\\
193	-0.5947\\
194	-0.5947\\
195	-0.5947\\
196	-0.5947\\
197	-0.5947\\
198	-0.5947\\
199	-0.5947\\
200	-1\\
201	-1\\
202	-1\\
203	-1\\
204	-1\\
205	-0.98856\\
206	-0.96673\\
207	-0.93239\\
208	-0.901\\
209	-0.87435\\
210	-0.85682\\
211	-0.84801\\
212	-0.84802\\
213	-0.8555\\
214	-0.86914\\
215	-0.88716\\
216	-0.90782\\
217	-0.92933\\
218	-0.9501\\
219	-0.96879\\
220	-0.9844\\
221	-0.99629\\
222	-1\\
223	-1\\
224	-1\\
225	-0.9972\\
226	-0.99195\\
227	-0.98509\\
228	-0.97755\\
229	-0.97004\\
230	-0.96309\\
231	-0.95713\\
232	-0.95243\\
233	-0.94913\\
234	-0.94727\\
235	-0.94679\\
236	-0.94753\\
237	-0.94927\\
238	-0.95178\\
239	-0.95476\\
240	-0.95796\\
241	-0.96111\\
242	-0.96402\\
243	-0.96651\\
244	-0.96846\\
245	-0.96982\\
246	-0.97058\\
247	-0.97075\\
248	-0.97042\\
249	-0.96967\\
250	-0.96862\\
251	-0.96737\\
252	-0.96605\\
253	-0.96474\\
254	-0.96354\\
255	-0.96251\\
256	-0.96171\\
257	-0.96115\\
258	-0.96084\\
259	-0.96078\\
260	-0.96092\\
261	-0.96123\\
262	-0.96167\\
263	-0.96219\\
264	-0.96274\\
265	-0.96329\\
266	-0.96378\\
267	-0.96421\\
268	-0.96454\\
269	-0.96477\\
270	-0.96489\\
271	-0.96492\\
272	-0.96486\\
273	-0.96472\\
274	-0.96454\\
275	-0.96432\\
276	-0.96409\\
277	-0.96386\\
278	-0.96366\\
279	-0.96348\\
280	-0.96335\\
281	-0.96325\\
282	-0.9632\\
283	-0.96319\\
284	-0.96322\\
285	-0.96328\\
286	-0.96335\\
287	-0.96344\\
288	-0.96354\\
289	-0.96363\\
290	-0.96372\\
291	-0.96379\\
292	-0.96385\\
293	-0.96389\\
294	-0.96391\\
295	-0.96391\\
296	-0.9639\\
297	-0.96387\\
298	-0.96384\\
299	-0.9638\\
300	-0.96376\\
301	-0.96373\\
302	-0.96369\\
303	-0.96366\\
304	-0.96364\\
305	-0.96362\\
306	-0.96361\\
307	-0.96361\\
308	-0.96362\\
309	-0.96363\\
310	-0.96364\\
311	-0.96366\\
312	-0.96367\\
313	-0.96369\\
314	-0.9637\\
315	-0.96372\\
316	-0.96373\\
317	-0.96373\\
318	-0.96374\\
319	-0.96374\\
320	-0.96373\\
321	-0.96373\\
322	-0.96372\\
323	-0.96372\\
324	-0.96371\\
325	-0.9637\\
326	-0.9637\\
327	-0.96369\\
328	-0.96369\\
329	-0.96369\\
330	-0.96368\\
331	-0.96368\\
332	-0.96368\\
333	-0.96369\\
334	-0.96369\\
335	-0.96369\\
336	-0.96369\\
337	-0.9637\\
338	-0.9637\\
339	-0.9637\\
340	-0.9637\\
341	-0.9637\\
342	-0.96371\\
343	-0.96371\\
344	-0.96371\\
345	-0.9637\\
346	-0.9637\\
347	-0.9637\\
348	-0.9637\\
349	-0.9637\\
350	-0.9637\\
351	-0.9637\\
352	-0.9637\\
353	-0.9637\\
354	-0.9637\\
355	-0.9637\\
356	-0.9637\\
357	-0.9637\\
358	-0.9637\\
359	-0.9637\\
360	-0.9637\\
361	-0.9637\\
362	-0.9637\\
363	-0.9637\\
364	-0.9637\\
365	-0.9637\\
366	-0.9637\\
367	-0.9637\\
368	-0.9637\\
369	-0.9637\\
370	-0.9637\\
371	-0.9637\\
372	-0.9637\\
373	-0.9637\\
374	-0.9637\\
375	-0.9637\\
376	-0.9637\\
377	-0.9637\\
378	-0.9637\\
379	-0.9637\\
380	-0.9637\\
381	-0.9637\\
382	-0.9637\\
383	-0.9637\\
384	-0.9637\\
385	-0.9637\\
386	-0.9637\\
387	-0.9637\\
388	-0.9637\\
389	-0.9637\\
390	-0.9637\\
391	-0.9637\\
392	-0.9637\\
393	-0.9637\\
394	-0.9637\\
395	-0.9637\\
396	-0.9637\\
397	-0.9637\\
398	-0.9637\\
399	-0.9637\\
400	-0.71541\\
401	-0.6661\\
402	-0.57616\\
403	-0.55368\\
404	-0.53245\\
405	-0.54245\\
406	-0.56341\\
407	-0.60146\\
408	-0.64697\\
409	-0.69763\\
410	-0.74685\\
411	-0.79101\\
412	-0.82643\\
413	-0.85144\\
414	-0.86538\\
415	-0.86888\\
416	-0.86327\\
417	-0.85044\\
418	-0.83247\\
419	-0.81148\\
420	-0.78944\\
421	-0.76808\\
422	-0.74883\\
423	-0.73277\\
424	-0.7206\\
425	-0.71269\\
426	-0.70906\\
427	-0.70941\\
428	-0.7132\\
429	-0.71967\\
430	-0.72797\\
431	-0.73719\\
432	-0.74648\\
433	-0.75507\\
434	-0.76239\\
435	-0.76802\\
436	-0.77176\\
437	-0.77358\\
438	-0.77361\\
439	-0.77211\\
440	-0.76941\\
441	-0.76588\\
442	-0.76192\\
443	-0.75788\\
444	-0.75408\\
445	-0.75078\\
446	-0.74816\\
447	-0.74631\\
448	-0.74529\\
449	-0.74504\\
450	-0.74548\\
451	-0.74648\\
452	-0.74789\\
453	-0.74954\\
454	-0.75127\\
455	-0.75293\\
456	-0.7544\\
457	-0.7556\\
458	-0.75648\\
459	-0.757\\
460	-0.75718\\
461	-0.75705\\
462	-0.75667\\
463	-0.75609\\
464	-0.7554\\
465	-0.75466\\
466	-0.75394\\
467	-0.75328\\
468	-0.75273\\
469	-0.75232\\
470	-0.75206\\
471	-0.75194\\
472	-0.75196\\
473	-0.7521\\
474	-0.75232\\
475	-0.75261\\
476	-0.75292\\
477	-0.75324\\
478	-0.75353\\
479	-0.75378\\
480	-0.75397\\
481	-0.7541\\
482	-0.75416\\
483	-0.75417\\
484	-0.75412\\
485	-0.75403\\
486	-0.75391\\
487	-0.75378\\
488	-0.75365\\
489	-0.75352\\
490	-0.75341\\
491	-0.75332\\
492	-0.75326\\
493	-0.75322\\
494	-0.75321\\
495	-0.75323\\
496	-0.75326\\
497	-0.75331\\
498	-0.75336\\
499	-0.75342\\
500	-0.75348\\
501	-0.75353\\
502	-0.75357\\
503	-0.7536\\
504	-0.75362\\
505	-0.75362\\
506	-0.75362\\
507	-0.75361\\
508	-0.75359\\
509	-0.75356\\
510	-0.75354\\
511	-0.75351\\
512	-0.75349\\
513	-0.75347\\
514	-0.75346\\
515	-0.75345\\
516	-0.75345\\
517	-0.75345\\
518	-0.75345\\
519	-0.75346\\
520	-0.75347\\
521	-0.75348\\
522	-0.75349\\
523	-0.7535\\
524	-0.75351\\
525	-0.75352\\
526	-0.75352\\
527	-0.75352\\
528	-0.75352\\
529	-0.75352\\
530	-0.75352\\
531	-0.75351\\
532	-0.75351\\
533	-0.7535\\
534	-0.7535\\
535	-0.7535\\
536	-0.75349\\
537	-0.75349\\
538	-0.75349\\
539	-0.75349\\
540	-0.75349\\
541	-0.75349\\
542	-0.75349\\
543	-0.75349\\
544	-0.7535\\
545	-0.7535\\
546	-0.7535\\
547	-0.7535\\
548	-0.7535\\
549	-0.7535\\
550	-0.7535\\
551	-0.7535\\
552	-0.7535\\
553	-0.7535\\
554	-0.7535\\
555	-0.7535\\
556	-0.7535\\
557	-0.7535\\
558	-0.7535\\
559	-0.7535\\
560	-0.7535\\
561	-0.7535\\
562	-0.7535\\
563	-0.7535\\
564	-0.7535\\
565	-0.7535\\
566	-0.7535\\
567	-0.7535\\
568	-0.7535\\
569	-0.7535\\
570	-0.7535\\
571	-0.7535\\
572	-0.7535\\
573	-0.7535\\
574	-0.7535\\
575	-0.7535\\
576	-0.7535\\
577	-0.7535\\
578	-0.7535\\
579	-0.7535\\
580	-0.7535\\
581	-0.7535\\
582	-0.7535\\
583	-0.7535\\
584	-0.7535\\
585	-0.7535\\
586	-0.7535\\
587	-0.7535\\
588	-0.7535\\
589	-0.7535\\
590	-0.7535\\
591	-0.7535\\
592	-0.7535\\
593	-0.7535\\
594	-0.7535\\
595	-0.7535\\
596	-0.7535\\
597	-0.7535\\
598	-0.7535\\
599	-0.7535\\
600	-0.50521\\
601	-0.4559\\
602	-0.36596\\
603	-0.34348\\
604	-0.32225\\
605	-0.33316\\
606	-0.3554\\
607	-0.39495\\
608	-0.44081\\
609	-0.48977\\
610	-0.53446\\
611	-0.57127\\
612	-0.59706\\
613	-0.61116\\
614	-0.61403\\
615	-0.60737\\
616	-0.59337\\
617	-0.5745\\
618	-0.55317\\
619	-0.53156\\
620	-0.5115\\
621	-0.49441\\
622	-0.48126\\
623	-0.47257\\
624	-0.46844\\
625	-0.46857\\
626	-0.47232\\
627	-0.4788\\
628	-0.48699\\
629	-0.49585\\
630	-0.50441\\
631	-0.51188\\
632	-0.5177\\
633	-0.52156\\
634	-0.5234\\
635	-0.52334\\
636	-0.52169\\
637	-0.51884\\
638	-0.51523\\
639	-0.51129\\
640	-0.50743\\
641	-0.50399\\
642	-0.50119\\
643	-0.4992\\
644	-0.49807\\
645	-0.49776\\
646	-0.49817\\
647	-0.49916\\
648	-0.50055\\
649	-0.50213\\
650	-0.50373\\
651	-0.5052\\
652	-0.50642\\
653	-0.50732\\
654	-0.50785\\
655	-0.50803\\
656	-0.50788\\
657	-0.50748\\
658	-0.5069\\
659	-0.50622\\
660	-0.50552\\
661	-0.50485\\
662	-0.50429\\
663	-0.50386\\
664	-0.50358\\
665	-0.50345\\
666	-0.50347\\
667	-0.5036\\
668	-0.50383\\
669	-0.5041\\
670	-0.50439\\
671	-0.50467\\
672	-0.50492\\
673	-0.50511\\
674	-0.50524\\
675	-0.50531\\
676	-0.50531\\
677	-0.50526\\
678	-0.50517\\
679	-0.50506\\
680	-0.50493\\
681	-0.50481\\
682	-0.5047\\
683	-0.50461\\
684	-0.50455\\
685	-0.50451\\
686	-0.5045\\
687	-0.50452\\
688	-0.50455\\
689	-0.50459\\
690	-0.50465\\
691	-0.5047\\
692	-0.50475\\
693	-0.50479\\
694	-0.50482\\
695	-0.50483\\
696	-0.50484\\
697	-0.50484\\
698	-0.50482\\
699	-0.5048\\
700	-0.50478\\
701	-0.50476\\
702	-0.50474\\
703	-0.50472\\
704	-0.50471\\
705	-0.5047\\
706	-0.50469\\
707	-0.50469\\
708	-0.5047\\
709	-0.50471\\
710	-0.50472\\
711	-0.50472\\
712	-0.50473\\
713	-0.50474\\
714	-0.50475\\
715	-0.50475\\
716	-0.50475\\
717	-0.50476\\
718	-0.50475\\
719	-0.50475\\
720	-0.50475\\
721	-0.50474\\
722	-0.50474\\
723	-0.50474\\
724	-0.50473\\
725	-0.50473\\
726	-0.50473\\
727	-0.50473\\
728	-0.50473\\
729	-0.50473\\
730	-0.50473\\
731	-0.50473\\
732	-0.50474\\
733	-0.50474\\
734	-0.50474\\
735	-0.50474\\
736	-0.50474\\
737	-0.50474\\
738	-0.50474\\
739	-0.50474\\
740	-0.50474\\
741	-0.50474\\
742	-0.50474\\
743	-0.50474\\
744	-0.50474\\
745	-0.50474\\
746	-0.50474\\
747	-0.50474\\
748	-0.50474\\
749	-0.50474\\
750	-0.50474\\
751	-0.50474\\
752	-0.50474\\
753	-0.50474\\
754	-0.50474\\
755	-0.50474\\
756	-0.50474\\
757	-0.50474\\
758	-0.50474\\
759	-0.50474\\
760	-0.50474\\
761	-0.50474\\
762	-0.50474\\
763	-0.50474\\
764	-0.50474\\
765	-0.50474\\
766	-0.50474\\
767	-0.50474\\
768	-0.50474\\
769	-0.50474\\
770	-0.50474\\
771	-0.50474\\
772	-0.50474\\
773	-0.50474\\
774	-0.50474\\
775	-0.50474\\
776	-0.50474\\
777	-0.50474\\
778	-0.50474\\
779	-0.50474\\
780	-0.50474\\
781	-0.50474\\
782	-0.50474\\
783	-0.50474\\
784	-0.50474\\
785	-0.50474\\
786	-0.50474\\
787	-0.50474\\
788	-0.50474\\
789	-0.50474\\
790	-0.50474\\
791	-0.50474\\
792	-0.50474\\
793	-0.50474\\
794	-0.50474\\
795	-0.50474\\
796	-0.50474\\
797	-0.50474\\
798	-0.50474\\
799	-0.50474\\
800	-0.83578\\
801	-0.90154\\
802	-1\\
803	-1\\
804	-1\\
805	-0.9905\\
806	-0.97061\\
807	-0.93525\\
808	-0.89611\\
809	-0.8556\\
810	-0.81884\\
811	-0.78737\\
812	-0.76302\\
813	-0.7463\\
814	-0.73746\\
815	-0.73606\\
816	-0.74133\\
817	-0.75206\\
818	-0.76681\\
819	-0.78396\\
820	-0.80191\\
821	-0.81919\\
822	-0.83456\\
823	-0.84712\\
824	-0.85631\\
825	-0.86193\\
826	-0.86408\\
827	-0.86312\\
828	-0.85957\\
829	-0.85407\\
830	-0.84732\\
831	-0.83999\\
832	-0.83268\\
833	-0.82593\\
834	-0.82015\\
835	-0.81561\\
836	-0.81249\\
837	-0.81081\\
838	-0.81051\\
839	-0.81142\\
840	-0.81331\\
841	-0.81591\\
842	-0.81893\\
843	-0.82208\\
844	-0.82512\\
845	-0.82784\\
846	-0.83007\\
847	-0.83171\\
848	-0.83273\\
849	-0.83314\\
850	-0.83299\\
851	-0.83237\\
852	-0.8314\\
853	-0.83018\\
854	-0.82885\\
855	-0.82752\\
856	-0.82629\\
857	-0.82523\\
858	-0.8244\\
859	-0.82383\\
860	-0.82352\\
861	-0.82346\\
862	-0.82363\\
863	-0.82397\\
864	-0.82444\\
865	-0.82498\\
866	-0.82555\\
867	-0.82611\\
868	-0.8266\\
869	-0.827\\
870	-0.8273\\
871	-0.82749\\
872	-0.82757\\
873	-0.82754\\
874	-0.82743\\
875	-0.82726\\
876	-0.82704\\
877	-0.8268\\
878	-0.82656\\
879	-0.82634\\
880	-0.82614\\
881	-0.82599\\
882	-0.82589\\
883	-0.82583\\
884	-0.82582\\
885	-0.82585\\
886	-0.82591\\
887	-0.82599\\
888	-0.82609\\
889	-0.8262\\
890	-0.8263\\
891	-0.82638\\
892	-0.82646\\
893	-0.82651\\
894	-0.82655\\
895	-0.82656\\
896	-0.82656\\
897	-0.82654\\
898	-0.82651\\
899	-0.82647\\
900	-0.82642\\
901	-0.82638\\
902	-0.82634\\
903	-0.82631\\
904	-0.82628\\
905	-0.82626\\
906	-0.82625\\
907	-0.82625\\
908	-0.82625\\
909	-0.82626\\
910	-0.82628\\
911	-0.82629\\
912	-0.82631\\
913	-0.82633\\
914	-0.82635\\
915	-0.82636\\
916	-0.82637\\
917	-0.82638\\
918	-0.82638\\
919	-0.82638\\
920	-0.82638\\
921	-0.82637\\
922	-0.82636\\
923	-0.82636\\
924	-0.82635\\
925	-0.82634\\
926	-0.82633\\
927	-0.82633\\
928	-0.82633\\
929	-0.82632\\
930	-0.82632\\
931	-0.82632\\
932	-0.82633\\
933	-0.82633\\
934	-0.82633\\
935	-0.82634\\
936	-0.82634\\
937	-0.82634\\
938	-0.82634\\
939	-0.82635\\
940	-0.82635\\
941	-0.82635\\
942	-0.82635\\
943	-0.82635\\
944	-0.82635\\
945	-0.82634\\
946	-0.82634\\
947	-0.82634\\
948	-0.82634\\
949	-0.82634\\
950	-0.82634\\
951	-0.82634\\
952	-0.82634\\
953	-0.82634\\
954	-0.82634\\
955	-0.82634\\
956	-0.82634\\
957	-0.82634\\
958	-0.82634\\
959	-0.82634\\
960	-0.82634\\
961	-0.82634\\
962	-0.82634\\
963	-0.82634\\
964	-0.82634\\
965	-0.82634\\
966	-0.82634\\
967	-0.82634\\
968	-0.82634\\
969	-0.82634\\
970	-0.82634\\
971	-0.82634\\
972	-0.82634\\
973	-0.82634\\
974	-0.82634\\
975	-0.82634\\
976	-0.82634\\
977	-0.82634\\
978	-0.82634\\
979	-0.82634\\
980	-0.82634\\
981	-0.82634\\
982	-0.82634\\
983	-0.82634\\
984	-0.82634\\
985	-0.82634\\
986	-0.82634\\
987	-0.82634\\
988	-0.82634\\
989	-0.82634\\
990	-0.82634\\
991	-0.82634\\
992	-0.82634\\
993	-0.82634\\
994	-0.82634\\
995	-0.82634\\
996	-0.82634\\
997	-0.82634\\
998	-0.82634\\
999	-0.82634\\
1000	-0.44564\\
1001	-0.37002\\
1002	-0.23212\\
1003	-0.19764\\
1004	-0.1651\\
1005	-0.18377\\
1006	-0.22197\\
1007	-0.29025\\
1008	-0.3688\\
1009	-0.45047\\
1010	-0.52149\\
1011	-0.57591\\
1012	-0.60963\\
1013	-0.62321\\
1014	-0.61898\\
1015	-0.6009\\
1016	-0.57321\\
1017	-0.54014\\
1018	-0.50544\\
1019	-0.47226\\
1020	-0.44307\\
1021	-0.41966\\
1022	-0.4031\\
1023	-0.39383\\
1024	-0.39161\\
1025	-0.39562\\
1026	-0.40456\\
1027	-0.41678\\
1028	-0.43053\\
1029	-0.4441\\
1030	-0.4561\\
1031	-0.46551\\
1032	-0.47176\\
1033	-0.47472\\
1034	-0.47462\\
1035	-0.47195\\
1036	-0.46738\\
1037	-0.46165\\
1038	-0.45546\\
1039	-0.44946\\
1040	-0.44417\\
1041	-0.43996\\
1042	-0.43705\\
1043	-0.43549\\
1044	-0.43522\\
1045	-0.43604\\
1046	-0.43769\\
1047	-0.43987\\
1048	-0.44227\\
1049	-0.44462\\
1050	-0.44669\\
1051	-0.44831\\
1052	-0.44939\\
1053	-0.44991\\
1054	-0.4499\\
1055	-0.44945\\
1056	-0.44866\\
1057	-0.44767\\
1058	-0.44661\\
1059	-0.44557\\
1060	-0.44467\\
1061	-0.44395\\
1062	-0.44346\\
1063	-0.44321\\
1064	-0.44317\\
1065	-0.44333\\
1066	-0.44363\\
1067	-0.44401\\
1068	-0.44443\\
1069	-0.44483\\
1070	-0.44519\\
1071	-0.44546\\
1072	-0.44565\\
1073	-0.44574\\
1074	-0.44573\\
1075	-0.44565\\
1076	-0.44552\\
1077	-0.44535\\
1078	-0.44516\\
1079	-0.44499\\
1080	-0.44483\\
1081	-0.44471\\
1082	-0.44463\\
1083	-0.44459\\
1084	-0.44459\\
1085	-0.44461\\
1086	-0.44467\\
1087	-0.44473\\
1088	-0.44481\\
1089	-0.44488\\
1090	-0.44494\\
1091	-0.44499\\
1092	-0.44502\\
1093	-0.44503\\
1094	-0.44503\\
1095	-0.44502\\
1096	-0.44499\\
1097	-0.44496\\
1098	-0.44493\\
1099	-0.4449\\
1100	-0.44487\\
1101	-0.44485\\
1102	-0.44484\\
1103	-0.44483\\
1104	-0.44483\\
1105	-0.44484\\
1106	-0.44485\\
1107	-0.44486\\
1108	-0.44487\\
1109	-0.44489\\
1110	-0.4449\\
1111	-0.4449\\
1112	-0.44491\\
1113	-0.44491\\
1114	-0.44491\\
1115	-0.44491\\
1116	-0.4449\\
1117	-0.4449\\
1118	-0.44489\\
1119	-0.44489\\
1120	-0.44488\\
1121	-0.44488\\
1122	-0.44488\\
1123	-0.44488\\
1124	-0.44488\\
1125	-0.44488\\
1126	-0.44488\\
1127	-0.44488\\
1128	-0.44488\\
1129	-0.44489\\
1130	-0.44489\\
1131	-0.44489\\
1132	-0.44489\\
1133	-0.44489\\
1134	-0.44489\\
1135	-0.44489\\
1136	-0.44489\\
1137	-0.44489\\
1138	-0.44489\\
1139	-0.44489\\
1140	-0.44489\\
1141	-0.44489\\
1142	-0.44489\\
1143	-0.44488\\
1144	-0.44489\\
1145	-0.44489\\
1146	-0.44489\\
1147	-0.44489\\
1148	-0.44489\\
1149	-0.44489\\
1150	-0.44489\\
1151	-0.44489\\
1152	-0.44489\\
1153	-0.44489\\
1154	-0.44489\\
1155	-0.44489\\
1156	-0.44489\\
1157	-0.44489\\
1158	-0.44489\\
1159	-0.44489\\
1160	-0.44489\\
1161	-0.44489\\
1162	-0.44489\\
1163	-0.44489\\
1164	-0.44489\\
1165	-0.44489\\
1166	-0.44489\\
1167	-0.44489\\
1168	-0.44489\\
1169	-0.44489\\
1170	-0.44489\\
1171	-0.44489\\
1172	-0.44489\\
1173	-0.44489\\
1174	-0.44489\\
1175	-0.44489\\
1176	-0.44489\\
1177	-0.44489\\
1178	-0.44489\\
1179	-0.44489\\
1180	-0.44489\\
1181	-0.44489\\
1182	-0.44489\\
1183	-0.44489\\
1184	-0.44489\\
1185	-0.44489\\
1186	-0.44489\\
1187	-0.44489\\
1188	-0.44489\\
1189	-0.44489\\
1190	-0.44489\\
1191	-0.44489\\
1192	-0.44489\\
1193	-0.44489\\
1194	-0.44489\\
1195	-0.44489\\
1196	-0.44489\\
1197	-0.44489\\
1198	-0.44489\\
1199	-0.44489\\
1200	-0.29592\\
1201	-0.26633\\
1202	-0.21237\\
1203	-0.19887\\
1204	-0.18614\\
1205	-0.19208\\
1206	-0.20337\\
1207	-0.22232\\
1208	-0.24214\\
1209	-0.26103\\
1210	-0.27558\\
1211	-0.28476\\
1212	-0.28799\\
1213	-0.28587\\
1214	-0.27942\\
1215	-0.27006\\
1216	-0.25921\\
1217	-0.24818\\
1218	-0.23804\\
1219	-0.22956\\
1220	-0.22319\\
1221	-0.21904\\
1222	-0.21696\\
1223	-0.21658\\
1224	-0.21742\\
1225	-0.21894\\
1226	-0.22062\\
1227	-0.22205\\
1228	-0.22295\\
1229	-0.22315\\
1230	-0.22265\\
1231	-0.22151\\
1232	-0.21989\\
1233	-0.21797\\
1234	-0.21593\\
1235	-0.21396\\
1236	-0.21216\\
1237	-0.21064\\
1238	-0.20941\\
1239	-0.20849\\
1240	-0.20783\\
1241	-0.20738\\
1242	-0.20706\\
1243	-0.20682\\
1244	-0.20658\\
1245	-0.20632\\
1246	-0.20601\\
1247	-0.20562\\
1248	-0.20518\\
1249	-0.20468\\
1250	-0.20416\\
1251	-0.20362\\
1252	-0.2031\\
1253	-0.20261\\
1254	-0.20216\\
1255	-0.20175\\
1256	-0.20139\\
1257	-0.20108\\
1258	-0.2008\\
1259	-0.20055\\
1260	-0.20032\\
1261	-0.2001\\
1262	-0.19989\\
1263	-0.19969\\
1264	-0.19948\\
1265	-0.19927\\
1266	-0.19907\\
1267	-0.19887\\
1268	-0.19867\\
1269	-0.19848\\
1270	-0.1983\\
1271	-0.19813\\
1272	-0.19797\\
1273	-0.19782\\
1274	-0.19768\\
1275	-0.19755\\
1276	-0.19742\\
1277	-0.19731\\
1278	-0.1972\\
1279	-0.19709\\
1280	-0.19699\\
1281	-0.19689\\
1282	-0.1968\\
1283	-0.19671\\
1284	-0.19662\\
1285	-0.19654\\
1286	-0.19646\\
1287	-0.19638\\
1288	-0.19631\\
1289	-0.19624\\
1290	-0.19617\\
1291	-0.19611\\
1292	-0.19605\\
1293	-0.19599\\
1294	-0.19594\\
1295	-0.19589\\
1296	-0.19584\\
1297	-0.19579\\
1298	-0.19575\\
1299	-0.1957\\
1300	-0.19566\\
1301	-0.19562\\
1302	-0.19559\\
1303	-0.19555\\
1304	-0.19552\\
1305	-0.19548\\
1306	-0.19545\\
1307	-0.19542\\
1308	-0.1954\\
1309	-0.19537\\
1310	-0.19534\\
1311	-0.19532\\
1312	-0.1953\\
1313	-0.19527\\
1314	-0.19525\\
1315	-0.19523\\
1316	-0.19521\\
1317	-0.19519\\
1318	-0.19518\\
1319	-0.19516\\
1320	-0.19514\\
1321	-0.19513\\
1322	-0.19511\\
1323	-0.1951\\
1324	-0.19509\\
1325	-0.19507\\
1326	-0.19506\\
1327	-0.19505\\
1328	-0.19504\\
1329	-0.19503\\
1330	-0.19502\\
1331	-0.19501\\
1332	-0.195\\
1333	-0.19499\\
1334	-0.19498\\
1335	-0.19498\\
1336	-0.19497\\
1337	-0.19496\\
1338	-0.19495\\
1339	-0.19495\\
1340	-0.19494\\
1341	-0.19494\\
1342	-0.19493\\
1343	-0.19492\\
1344	-0.19492\\
1345	-0.19491\\
1346	-0.19491\\
1347	-0.1949\\
1348	-0.1949\\
1349	-0.1949\\
1350	-0.19489\\
1351	-0.19489\\
1352	-0.19489\\
1353	-0.19488\\
1354	-0.19488\\
1355	-0.19488\\
1356	-0.19487\\
1357	-0.19487\\
1358	-0.19487\\
1359	-0.19486\\
1360	-0.19486\\
1361	-0.19486\\
1362	-0.19486\\
1363	-0.19486\\
1364	-0.19485\\
1365	-0.19485\\
1366	-0.19485\\
1367	-0.19485\\
1368	-0.19485\\
1369	-0.19484\\
1370	-0.19484\\
1371	-0.19484\\
1372	-0.19484\\
1373	-0.19484\\
1374	-0.19484\\
1375	-0.19484\\
1376	-0.19483\\
1377	-0.19483\\
1378	-0.19483\\
1379	-0.19483\\
1380	-0.19483\\
1381	-0.19483\\
1382	-0.19483\\
1383	-0.19483\\
1384	-0.19483\\
1385	-0.19483\\
1386	-0.19483\\
1387	-0.19482\\
1388	-0.19482\\
1389	-0.19482\\
1390	-0.19482\\
1391	-0.19482\\
1392	-0.19482\\
1393	-0.19482\\
1394	-0.19482\\
1395	-0.19482\\
1396	-0.19482\\
1397	-0.19482\\
1398	-0.19482\\
1399	-0.19482\\
1400	-0.14516\\
1401	-0.1353\\
1402	-0.11731\\
1403	-0.11281\\
1404	-0.10857\\
1405	-0.10971\\
1406	-0.11169\\
1407	-0.11498\\
1408	-0.11767\\
1409	-0.11949\\
1410	-0.11981\\
1411	-0.11868\\
1412	-0.11622\\
1413	-0.11279\\
1414	-0.10878\\
1415	-0.10458\\
1416	-0.10052\\
1417	-0.096829\\
1418	-0.093649\\
1419	-0.091007\\
1420	-0.08886\\
1421	-0.087108\\
1422	-0.085627\\
1423	-0.084294\\
1424	-0.083\\
1425	-0.081668\\
1426	-0.080252\\
1427	-0.078739\\
1428	-0.074968\\
1429	-0.07122\\
1430	-0.067513\\
1431	-0.063869\\
1432	-0.058386\\
1433	-0.053081\\
1434	-0.048011\\
1435	-0.043219\\
1436	-0.036994\\
1437	-0.031174\\
1438	-0.024226\\
1439	-0.01785\\
1440	-0.010691\\
1441	-0.0042532\\
1442	0.002567\\
1443	0.0084896\\
1444	0.013494\\
1445	0.017565\\
1446	0.020703\\
1447	0.022922\\
1448	0.024252\\
1449	0.024749\\
1450	0.024491\\
1451	0.023571\\
1452	0.022095\\
1453	0.020169\\
1454	0.017903\\
1455	0.015397\\
1456	0.012747\\
1457	0.010043\\
1458	0.0073606\\
1459	0.0047698\\
1460	0.0023289\\
1461	8.6335e-05\\
1462	-0.0019191\\
1463	-0.0036586\\
1464	-0.0051123\\
1465	-0.0062696\\
1466	-0.0071283\\
1467	-0.0076938\\
1468	-0.007979\\
1469	-0.0080029\\
1470	-0.0077898\\
1471	-0.0073686\\
1472	-0.0067713\\
1473	-0.0060322\\
1474	-0.0051867\\
1475	-0.0042698\\
1476	-0.0033158\\
1477	-0.0023566\\
1478	-0.0014214\\
1479	-0.00053577\\
1480	0.00027867\\
1481	0.0010046\\
1482	0.0016292\\
1483	0.0021441\\
1484	0.002545\\
1485	0.0028316\\
1486	0.003007\\
1487	0.0030774\\
1488	0.0030516\\
1489	0.0029402\\
1490	0.0027551\\
1491	0.0025093\\
1492	0.0022162\\
1493	0.0018891\\
1494	0.0015408\\
1495	0.0011837\\
1496	0.00082878\\
1497	0.0004861\\
1498	0.00016424\\
1499	-0.00012967\\
1500	-0.00039001\\
1501	-0.0006127\\
1502	-0.00079512\\
1503	-0.00093607\\
1504	-0.0010356\\
1505	-0.0010951\\
1506	-0.0011168\\
1507	-0.0011038\\
1508	-0.00106\\
1509	-0.00098981\\
1510	-0.0008978\\
1511	-0.00078888\\
1512	-0.0006679\\
1513	-0.00053959\\
1514	-0.00040843\\
1515	-0.00027852\\
1516	-0.00015353\\
1517	-3.6613e-05\\
1518	6.9638e-05\\
1519	0.00016319\\
1520	0.00024261\\
1521	0.00030698\\
1522	0.00035596\\
1523	0.00038967\\
1524	0.00040867\\
1525	0.00041388\\
1526	0.00040656\\
1527	0.00038817\\
1528	0.00036037\\
1529	0.00032491\\
1530	0.00028361\\
1531	0.00023824\\
1532	0.00019053\\
1533	0.00014211\\
1534	9.4444e-05\\
1535	4.8845e-05\\
1536	6.4257e-06\\
1537	-3.1905e-05\\
1538	-6.5446e-05\\
1539	-9.3708e-05\\
1540	-0.0001164\\
1541	-0.00013343\\
1542	-0.00014487\\
1543	-0.00015095\\
1544	-0.00015203\\
1545	-0.00014859\\
1546	-0.00014118\\
1547	-0.00013041\\
1548	-0.00011694\\
1549	-0.00010142\\
1550	-8.4519e-05\\
1551	-6.6863e-05\\
1552	-4.9041e-05\\
1553	-3.1587e-05\\
1554	-1.4972e-05\\
1555	4.0567e-07\\
1556	1.4224e-05\\
1557	2.6239e-05\\
1558	3.6282e-05\\
1559	4.426e-05\\
1560	5.015e-05\\
1561	5.399e-05\\
1562	5.5875e-05\\
1563	5.5947e-05\\
1564	5.4388e-05\\
1565	5.1406e-05\\
1566	4.7231e-05\\
1567	4.2106e-05\\
1568	3.6273e-05\\
1569	2.9975e-05\\
1570	2.344e-05\\
1571	1.6883e-05\\
1572	1.0496e-05\\
1573	4.447e-06\\
1574	-1.1219e-06\\
1575	-6.0974e-06\\
1576	-1.0395e-05\\
1577	-1.3958e-05\\
1578	-1.6756e-05\\
1579	-1.8787e-05\\
1580	-2.0067e-05\\
1581	-2.0635e-05\\
1582	-2.0547e-05\\
1583	-1.9871e-05\\
1584	-1.8686e-05\\
1585	-1.7078e-05\\
1586	-1.5135e-05\\
1587	-1.2948e-05\\
1588	-1.0606e-05\\
1589	-8.1909e-06\\
1590	-5.7813e-06\\
1591	-3.4463e-06\\
1592	-1.2463e-06\\
1593	7.6843e-07\\
1594	2.558e-06\\
1595	4.0931e-06\\
1596	5.3548e-06\\
1597	6.3339e-06\\
1598	7.0305e-06\\
1599	7.4527e-06\\
1600	0.017003\\
1601	0.033177\\
1602	0.04897\\
1603	0.064374\\
1604	0.079399\\
1605	0.09393\\
1606	0.10771\\
1607	0.1205\\
1608	0.13212\\
1609	0.14246\\
1610	0.15154\\
1611	0.15939\\
1612	0.16613\\
1613	0.17188\\
1614	0.17677\\
1615	0.18093\\
1616	0.18448\\
1617	0.18754\\
1618	0.19018\\
1619	0.19249\\
1620	0.19452\\
1621	0.19634\\
1622	0.19798\\
1623	0.19947\\
1624	0.20084\\
1625	0.20211\\
1626	0.2033\\
1627	0.20441\\
1628	0.20545\\
1629	0.20644\\
1630	0.20738\\
1631	0.20828\\
1632	0.20913\\
1633	0.20995\\
1634	0.21073\\
1635	0.21148\\
1636	0.2122\\
1637	0.21289\\
1638	0.21355\\
1639	0.21419\\
1640	0.21481\\
1641	0.2154\\
1642	0.21597\\
1643	0.21652\\
1644	0.21705\\
1645	0.21756\\
1646	0.21805\\
1647	0.21853\\
1648	0.21898\\
1649	0.21943\\
1650	0.21985\\
1651	0.22027\\
1652	0.22067\\
1653	0.22105\\
1654	0.22142\\
1655	0.22178\\
1656	0.22213\\
1657	0.22246\\
1658	0.22279\\
1659	0.2231\\
1660	0.22341\\
1661	0.2237\\
1662	0.22398\\
1663	0.22426\\
1664	0.22453\\
1665	0.22478\\
1666	0.22503\\
1667	0.22527\\
1668	0.22551\\
1669	0.22573\\
1670	0.22595\\
1671	0.22616\\
1672	0.22637\\
1673	0.22656\\
1674	0.22676\\
1675	0.22694\\
1676	0.22712\\
1677	0.2273\\
1678	0.22747\\
1679	0.22763\\
1680	0.22779\\
1681	0.22794\\
1682	0.22809\\
1683	0.22824\\
1684	0.22838\\
1685	0.22851\\
1686	0.22864\\
1687	0.22877\\
1688	0.2289\\
1689	0.22902\\
1690	0.22913\\
1691	0.22924\\
1692	0.22935\\
1693	0.22946\\
1694	0.22956\\
1695	0.22966\\
1696	0.22976\\
1697	0.22985\\
1698	0.22994\\
1699	0.23003\\
1700	0.23011\\
1701	0.2302\\
1702	0.23028\\
1703	0.23035\\
1704	0.23043\\
1705	0.2305\\
1706	0.23057\\
1707	0.23064\\
1708	0.23071\\
1709	0.23077\\
1710	0.23084\\
1711	0.2309\\
1712	0.23096\\
1713	0.23101\\
1714	0.23107\\
1715	0.23112\\
1716	0.23118\\
1717	0.23123\\
1718	0.23128\\
1719	0.23132\\
1720	0.23137\\
1721	0.23142\\
1722	0.23146\\
1723	0.2315\\
1724	0.23154\\
1725	0.23158\\
1726	0.23162\\
1727	0.23166\\
1728	0.23169\\
1729	0.23173\\
1730	0.23176\\
1731	0.2318\\
1732	0.23183\\
1733	0.23186\\
1734	0.23189\\
1735	0.23192\\
1736	0.23195\\
1737	0.23198\\
1738	0.232\\
1739	0.23203\\
1740	0.23206\\
1741	0.23208\\
1742	0.2321\\
1743	0.23213\\
1744	0.23215\\
1745	0.23217\\
1746	0.23219\\
1747	0.23221\\
1748	0.23223\\
1749	0.23225\\
1750	0.23227\\
1751	0.23229\\
1752	0.23231\\
1753	0.23232\\
1754	0.23234\\
1755	0.23236\\
1756	0.23237\\
1757	0.23239\\
1758	0.2324\\
1759	0.23242\\
1760	0.23243\\
1761	0.23245\\
1762	0.23246\\
1763	0.23247\\
1764	0.23248\\
1765	0.2325\\
1766	0.23251\\
1767	0.23252\\
1768	0.23253\\
1769	0.23254\\
1770	0.23255\\
1771	0.23256\\
1772	0.23257\\
1773	0.23258\\
1774	0.23259\\
1775	0.2326\\
1776	0.23261\\
1777	0.23261\\
1778	0.23262\\
1779	0.23263\\
1780	0.23264\\
1781	0.23265\\
1782	0.23265\\
1783	0.23266\\
1784	0.23267\\
1785	0.23267\\
1786	0.23268\\
1787	0.23269\\
1788	0.23269\\
1789	0.2327\\
1790	0.2327\\
1791	0.23271\\
1792	0.23271\\
1793	0.23272\\
1794	0.23272\\
1795	0.23273\\
1796	0.23273\\
1797	0.23274\\
1798	0.23274\\
1799	0.23275\\
1800	0.23275\\
};
\addlegendentry{Sterowanie u}

\end{axis}
\end{tikzpicture}%
   \caption{Dwa regulatory lokalne DMC}
   \label{projekt:zad7:DMC:2:figure}
\end{figure}

\begin{figure}[H] 
   \centering
   % This file was created by matlab2tikz.
%
\definecolor{mycolor1}{rgb}{0.00000,0.44700,0.74100}%
\definecolor{mycolor2}{rgb}{0.85000,0.32500,0.09800}%
%
\begin{tikzpicture}

\begin{axis}[%
width=4.521in,
height=1.493in,
at={(0.758in,2.554in)},
scale only axis,
xmin=1,
xmax=1800,
xlabel style={font=\color{white!15!black}},
xlabel={k},
ymin=-4.5436,
ymax=0.1,
ylabel style={font=\color{white!15!black}},
ylabel={y},
axis background/.style={fill=white},
title style={font=\bfseries, align=center},
title={E=220.4705\\[1ex]N= [70         70         20]\\[1ex]$\text{N}_\text{u}\text{= [6          8          5]}$\\[1ex]lambda= [25         27          8]},
xmajorgrids,
ymajorgrids,
legend style={legend cell align=left, align=left, draw=white!15!black}
]
\addplot [color=mycolor1]
  table[row sep=crcr]{%
1	0\\
2	0\\
3	0\\
4	0\\
5	0\\
6	0\\
7	0\\
8	0\\
9	0\\
10	0\\
11	0\\
12	0\\
13	0\\
14	0\\
15	0\\
16	0\\
17	0\\
18	0\\
19	0\\
20	0\\
21	0\\
22	0\\
23	0\\
24	0\\
25	-0.040206\\
26	-0.18838\\
27	-0.46056\\
28	-0.81841\\
29	-1.2156\\
30	-1.6203\\
31	-2.0088\\
32	-2.3595\\
33	-2.6542\\
34	-2.8793\\
35	-3.0266\\
36	-3.0937\\
37	-3.0842\\
38	-3.0076\\
39	-2.8781\\
40	-2.713\\
41	-2.5312\\
42	-2.35\\
43	-2.1843\\
44	-2.0447\\
45	-1.9379\\
46	-1.8665\\
47	-1.8298\\
48	-1.8243\\
49	-1.8448\\
50	-1.8845\\
51	-1.9364\\
52	-1.9935\\
53	-2.0494\\
54	-2.0986\\
55	-2.1373\\
56	-2.1631\\
57	-2.175\\
58	-2.1735\\
59	-2.1603\\
60	-2.138\\
61	-2.1095\\
62	-2.0781\\
63	-2.0466\\
64	-2.0177\\
65	-1.9933\\
66	-1.9746\\
67	-1.9624\\
68	-1.9564\\
69	-1.9563\\
70	-1.9609\\
71	-1.9691\\
72	-1.9794\\
73	-1.9907\\
74	-2.0017\\
75	-2.0113\\
76	-2.019\\
77	-2.0243\\
78	-2.0269\\
79	-2.0271\\
80	-2.0252\\
81	-2.0215\\
82	-2.0168\\
83	-2.0114\\
84	-2.006\\
85	-2.0011\\
86	-1.9969\\
87	-1.9937\\
88	-1.9917\\
89	-1.9908\\
90	-1.9909\\
91	-1.9919\\
92	-1.9935\\
93	-1.9955\\
94	-1.9976\\
95	-1.9997\\
96	-2.0015\\
97	-2.0029\\
98	-2.0039\\
99	-2.0045\\
100	-2.0045\\
101	-2.0042\\
102	-2.0036\\
103	-2.0028\\
104	-2.0018\\
105	-2.0009\\
106	-2\\
107	-1.9993\\
108	-1.9987\\
109	-1.9984\\
110	-1.9982\\
111	-1.9983\\
112	-1.9985\\
113	-1.9988\\
114	-1.9991\\
115	-1.9995\\
116	-1.9999\\
117	-2.0002\\
118	-2.0005\\
119	-2.0007\\
120	-2.0008\\
121	-2.0008\\
122	-2.0007\\
123	-2.0006\\
124	-2.0005\\
125	-2.0003\\
126	-2.0001\\
127	-2\\
128	-1.9999\\
129	-1.9998\\
130	-1.9997\\
131	-1.9997\\
132	-1.9997\\
133	-1.9997\\
134	-1.9998\\
135	-1.9998\\
136	-1.9999\\
137	-2\\
138	-2\\
139	-2.0001\\
140	-2.0001\\
141	-2.0001\\
142	-2.0001\\
143	-2.0001\\
144	-2.0001\\
145	-2.0001\\
146	-2.0001\\
147	-2\\
148	-2\\
149	-2\\
150	-2\\
151	-1.9999\\
152	-1.9999\\
153	-1.9999\\
154	-1.9999\\
155	-2\\
156	-2\\
157	-2\\
158	-2\\
159	-2\\
160	-2\\
161	-2\\
162	-2\\
163	-2\\
164	-2\\
165	-2\\
166	-2\\
167	-2\\
168	-2\\
169	-2\\
170	-2\\
171	-2\\
172	-2\\
173	-2\\
174	-2\\
175	-2\\
176	-2\\
177	-2\\
178	-2\\
179	-2\\
180	-2\\
181	-2\\
182	-2\\
183	-2\\
184	-2\\
185	-2\\
186	-2\\
187	-2\\
188	-2\\
189	-2\\
190	-2\\
191	-2\\
192	-2\\
193	-2\\
194	-2\\
195	-2\\
196	-2\\
197	-2\\
198	-2\\
199	-2\\
200	-2\\
201	-2\\
202	-2\\
203	-2\\
204	-2\\
205	-2.0801\\
206	-2.2463\\
207	-2.4603\\
208	-2.6956\\
209	-2.9345\\
210	-3.1611\\
211	-3.3635\\
212	-3.5312\\
213	-3.659\\
214	-3.7469\\
215	-3.7996\\
216	-3.8249\\
217	-3.8319\\
218	-3.83\\
219	-3.8274\\
220	-3.8307\\
221	-3.8446\\
222	-3.8717\\
223	-3.9126\\
224	-3.9664\\
225	-4.0309\\
226	-4.1028\\
227	-4.1774\\
228	-4.2505\\
229	-4.3192\\
230	-4.381\\
231	-4.4339\\
232	-4.4767\\
233	-4.5085\\
234	-4.5297\\
235	-4.541\\
236	-4.5436\\
237	-4.5393\\
238	-4.5298\\
239	-4.5171\\
240	-4.5029\\
241	-4.4888\\
242	-4.476\\
243	-4.4655\\
244	-4.458\\
245	-4.4537\\
246	-4.4527\\
247	-4.4546\\
248	-4.4589\\
249	-4.4653\\
250	-4.4728\\
251	-4.481\\
252	-4.4892\\
253	-4.4969\\
254	-4.5035\\
255	-4.5089\\
256	-4.5128\\
257	-4.5152\\
258	-4.5162\\
259	-4.5158\\
260	-4.5143\\
261	-4.5121\\
262	-4.5092\\
263	-4.5061\\
264	-4.503\\
265	-4.5001\\
266	-4.4977\\
267	-4.4957\\
268	-4.4943\\
269	-4.4935\\
270	-4.4933\\
271	-4.4935\\
272	-4.4942\\
273	-4.4952\\
274	-4.4964\\
275	-4.4977\\
276	-4.499\\
277	-4.5001\\
278	-4.5011\\
279	-4.5019\\
280	-4.5024\\
281	-4.5027\\
282	-4.5027\\
283	-4.5026\\
284	-4.5023\\
285	-4.5019\\
286	-4.5014\\
287	-4.5008\\
288	-4.5003\\
289	-4.4999\\
290	-4.4995\\
291	-4.4992\\
292	-4.499\\
293	-4.4989\\
294	-4.4989\\
295	-4.499\\
296	-4.4991\\
297	-4.4993\\
298	-4.4995\\
299	-4.4997\\
300	-4.4999\\
301	-4.5001\\
302	-4.5002\\
303	-4.5003\\
304	-4.5004\\
305	-4.5005\\
306	-4.5004\\
307	-4.5004\\
308	-4.5004\\
309	-4.5003\\
310	-4.5002\\
311	-4.5001\\
312	-4.5\\
313	-4.5\\
314	-4.4999\\
315	-4.4999\\
316	-4.4998\\
317	-4.4998\\
318	-4.4998\\
319	-4.4998\\
320	-4.4999\\
321	-4.4999\\
322	-4.4999\\
323	-4.5\\
324	-4.5\\
325	-4.5\\
326	-4.5\\
327	-4.5001\\
328	-4.5001\\
329	-4.5001\\
330	-4.5001\\
331	-4.5001\\
332	-4.5001\\
333	-4.5\\
334	-4.5\\
335	-4.5\\
336	-4.5\\
337	-4.5\\
338	-4.5\\
339	-4.5\\
340	-4.5\\
341	-4.5\\
342	-4.5\\
343	-4.5\\
344	-4.5\\
345	-4.5\\
346	-4.5\\
347	-4.5\\
348	-4.5\\
349	-4.5\\
350	-4.5\\
351	-4.5\\
352	-4.5\\
353	-4.5\\
354	-4.5\\
355	-4.5\\
356	-4.5\\
357	-4.5\\
358	-4.5\\
359	-4.5\\
360	-4.5\\
361	-4.5\\
362	-4.5\\
363	-4.5\\
364	-4.5\\
365	-4.5\\
366	-4.5\\
367	-4.5\\
368	-4.5\\
369	-4.5\\
370	-4.5\\
371	-4.5\\
372	-4.5\\
373	-4.5\\
374	-4.5\\
375	-4.5\\
376	-4.5\\
377	-4.5\\
378	-4.5\\
379	-4.5\\
380	-4.5\\
381	-4.5\\
382	-4.5\\
383	-4.5\\
384	-4.5\\
385	-4.5\\
386	-4.5\\
387	-4.5\\
388	-4.5\\
389	-4.5\\
390	-4.5\\
391	-4.5\\
392	-4.5\\
393	-4.5\\
394	-4.5\\
395	-4.5\\
396	-4.5\\
397	-4.5\\
398	-4.5\\
399	-4.5\\
400	-4.5\\
401	-4.5\\
402	-4.5\\
403	-4.5\\
404	-4.5\\
405	-4.4271\\
406	-4.2765\\
407	-4.0491\\
408	-3.7754\\
409	-3.4815\\
410	-3.197\\
411	-2.9435\\
412	-2.7374\\
413	-2.5874\\
414	-2.4964\\
415	-2.4622\\
416	-2.4789\\
417	-2.5374\\
418	-2.6273\\
419	-2.7368\\
420	-2.8548\\
421	-2.9707\\
422	-3.0757\\
423	-3.163\\
424	-3.2279\\
425	-3.2684\\
426	-3.2842\\
427	-3.2775\\
428	-3.2517\\
429	-3.2113\\
430	-3.1616\\
431	-3.1077\\
432	-3.0545\\
433	-3.0058\\
434	-2.9649\\
435	-2.9336\\
436	-2.9131\\
437	-2.9032\\
438	-2.9031\\
439	-2.9112\\
440	-2.9258\\
441	-2.9446\\
442	-2.9655\\
443	-2.9865\\
444	-3.0059\\
445	-3.0223\\
446	-3.0349\\
447	-3.0432\\
448	-3.047\\
449	-3.0468\\
450	-3.043\\
451	-3.0365\\
452	-3.0282\\
453	-3.019\\
454	-3.0096\\
455	-3.001\\
456	-2.9936\\
457	-2.9878\\
458	-2.9839\\
459	-2.9819\\
460	-2.9816\\
461	-2.9829\\
462	-2.9854\\
463	-2.9887\\
464	-2.9924\\
465	-2.9963\\
466	-2.9998\\
467	-3.0029\\
468	-3.0054\\
469	-3.0071\\
470	-3.0079\\
471	-3.0081\\
472	-3.0076\\
473	-3.0065\\
474	-3.0051\\
475	-3.0035\\
476	-3.0019\\
477	-3.0003\\
478	-2.999\\
479	-2.9979\\
480	-2.9972\\
481	-2.9968\\
482	-2.9967\\
483	-2.9968\\
484	-2.9972\\
485	-2.9978\\
486	-2.9985\\
487	-2.9992\\
488	-2.9998\\
489	-3.0004\\
490	-3.0009\\
491	-3.0012\\
492	-3.0014\\
493	-3.0014\\
494	-3.0014\\
495	-3.0012\\
496	-3.001\\
497	-3.0007\\
498	-3.0004\\
499	-3.0001\\
500	-2.9999\\
501	-2.9997\\
502	-2.9995\\
503	-2.9994\\
504	-2.9994\\
505	-2.9994\\
506	-2.9995\\
507	-2.9996\\
508	-2.9997\\
509	-2.9998\\
510	-2.9999\\
511	-3\\
512	-3.0001\\
513	-3.0002\\
514	-3.0002\\
515	-3.0003\\
516	-3.0002\\
517	-3.0002\\
518	-3.0002\\
519	-3.0001\\
520	-3.0001\\
521	-3\\
522	-3\\
523	-2.9999\\
524	-2.9999\\
525	-2.9999\\
526	-2.9999\\
527	-2.9999\\
528	-2.9999\\
529	-2.9999\\
530	-2.9999\\
531	-3\\
532	-3\\
533	-3\\
534	-3\\
535	-3\\
536	-3\\
537	-3\\
538	-3\\
539	-3\\
540	-3\\
541	-3\\
542	-3\\
543	-3\\
544	-3\\
545	-3\\
546	-3\\
547	-3\\
548	-3\\
549	-3\\
550	-3\\
551	-3\\
552	-3\\
553	-3\\
554	-3\\
555	-3\\
556	-3\\
557	-3\\
558	-3\\
559	-3\\
560	-3\\
561	-3\\
562	-3\\
563	-3\\
564	-3\\
565	-3\\
566	-3\\
567	-3\\
568	-3\\
569	-3\\
570	-3\\
571	-3\\
572	-3\\
573	-3\\
574	-3\\
575	-3\\
576	-3\\
577	-3\\
578	-3\\
579	-3\\
580	-3\\
581	-3\\
582	-3\\
583	-3\\
584	-3\\
585	-3\\
586	-3\\
587	-3\\
588	-3\\
589	-3\\
590	-3\\
591	-3\\
592	-3\\
593	-3\\
594	-3\\
595	-3\\
596	-3\\
597	-3\\
598	-3\\
599	-3\\
600	-3\\
601	-3\\
602	-3\\
603	-3\\
604	-3\\
605	-2.9208\\
606	-2.7632\\
607	-2.53\\
608	-2.259\\
609	-1.9795\\
610	-1.7218\\
611	-1.5048\\
612	-1.3404\\
613	-1.2326\\
614	-1.1802\\
615	-1.1776\\
616	-1.2161\\
617	-1.2851\\
618	-1.3731\\
619	-1.4685\\
620	-1.561\\
621	-1.6417\\
622	-1.7042\\
623	-1.7447\\
624	-1.7621\\
625	-1.7575\\
626	-1.7342\\
627	-1.6969\\
628	-1.651\\
629	-1.6019\\
630	-1.5545\\
631	-1.5126\\
632	-1.4791\\
633	-1.4555\\
634	-1.4422\\
635	-1.4385\\
636	-1.4431\\
637	-1.454\\
638	-1.4691\\
639	-1.4862\\
640	-1.5031\\
641	-1.5183\\
642	-1.5303\\
643	-1.5386\\
644	-1.5428\\
645	-1.543\\
646	-1.5397\\
647	-1.5339\\
648	-1.5263\\
649	-1.5179\\
650	-1.5095\\
651	-1.502\\
652	-1.4959\\
653	-1.4914\\
654	-1.4887\\
655	-1.4878\\
656	-1.4884\\
657	-1.4902\\
658	-1.4929\\
659	-1.4959\\
660	-1.4991\\
661	-1.5019\\
662	-1.5042\\
663	-1.5059\\
664	-1.5069\\
665	-1.5071\\
666	-1.5067\\
667	-1.5058\\
668	-1.5046\\
669	-1.5031\\
670	-1.5017\\
671	-1.5004\\
672	-1.4992\\
673	-1.4984\\
674	-1.4979\\
675	-1.4977\\
676	-1.4977\\
677	-1.498\\
678	-1.4985\\
679	-1.499\\
680	-1.4996\\
681	-1.5001\\
682	-1.5006\\
683	-1.5009\\
684	-1.5011\\
685	-1.5012\\
686	-1.5011\\
687	-1.501\\
688	-1.5008\\
689	-1.5006\\
690	-1.5003\\
691	-1.5001\\
692	-1.4999\\
693	-1.4997\\
694	-1.4996\\
695	-1.4996\\
696	-1.4996\\
697	-1.4996\\
698	-1.4997\\
699	-1.4998\\
700	-1.4999\\
701	-1.5\\
702	-1.5001\\
703	-1.5001\\
704	-1.5002\\
705	-1.5002\\
706	-1.5002\\
707	-1.5002\\
708	-1.5001\\
709	-1.5001\\
710	-1.5001\\
711	-1.5\\
712	-1.5\\
713	-1.5\\
714	-1.4999\\
715	-1.4999\\
716	-1.4999\\
717	-1.4999\\
718	-1.4999\\
719	-1.5\\
720	-1.5\\
721	-1.5\\
722	-1.5\\
723	-1.5\\
724	-1.5\\
725	-1.5\\
726	-1.5\\
727	-1.5\\
728	-1.5\\
729	-1.5\\
730	-1.5\\
731	-1.5\\
732	-1.5\\
733	-1.5\\
734	-1.5\\
735	-1.5\\
736	-1.5\\
737	-1.5\\
738	-1.5\\
739	-1.5\\
740	-1.5\\
741	-1.5\\
742	-1.5\\
743	-1.5\\
744	-1.5\\
745	-1.5\\
746	-1.5\\
747	-1.5\\
748	-1.5\\
749	-1.5\\
750	-1.5\\
751	-1.5\\
752	-1.5\\
753	-1.5\\
754	-1.5\\
755	-1.5\\
756	-1.5\\
757	-1.5\\
758	-1.5\\
759	-1.5\\
760	-1.5\\
761	-1.5\\
762	-1.5\\
763	-1.5\\
764	-1.5\\
765	-1.5\\
766	-1.5\\
767	-1.5\\
768	-1.5\\
769	-1.5\\
770	-1.5\\
771	-1.5\\
772	-1.5\\
773	-1.5\\
774	-1.5\\
775	-1.5\\
776	-1.5\\
777	-1.5\\
778	-1.5\\
779	-1.5\\
780	-1.5\\
781	-1.5\\
782	-1.5\\
783	-1.5\\
784	-1.5\\
785	-1.5\\
786	-1.5\\
787	-1.5\\
788	-1.5\\
789	-1.5\\
790	-1.5\\
791	-1.5\\
792	-1.5\\
793	-1.5\\
794	-1.5\\
795	-1.5\\
796	-1.5\\
797	-1.5\\
798	-1.5\\
799	-1.5\\
800	-1.5\\
801	-1.5\\
802	-1.5\\
803	-1.5\\
804	-1.5\\
805	-1.5704\\
806	-1.7238\\
807	-1.9434\\
808	-2.2044\\
809	-2.4806\\
810	-2.7512\\
811	-3.0005\\
812	-3.2148\\
813	-3.3849\\
814	-3.5058\\
815	-3.5777\\
816	-3.6046\\
817	-3.5937\\
818	-3.5544\\
819	-3.4966\\
820	-3.4303\\
821	-3.3642\\
822	-3.3056\\
823	-3.2597\\
824	-3.2297\\
825	-3.217\\
826	-3.2211\\
827	-3.2403\\
828	-3.2719\\
829	-3.3124\\
830	-3.358\\
831	-3.4052\\
832	-3.4507\\
833	-3.4915\\
834	-3.5257\\
835	-3.5519\\
836	-3.5693\\
837	-3.5782\\
838	-3.5792\\
839	-3.5734\\
840	-3.5625\\
841	-3.548\\
842	-3.5316\\
843	-3.5149\\
844	-3.4993\\
845	-3.4858\\
846	-3.4752\\
847	-3.4679\\
848	-3.464\\
849	-3.4634\\
850	-3.4655\\
851	-3.47\\
852	-3.476\\
853	-3.483\\
854	-3.4903\\
855	-3.4972\\
856	-3.5034\\
857	-3.5084\\
858	-3.5121\\
859	-3.5143\\
860	-3.5152\\
861	-3.5147\\
862	-3.5132\\
863	-3.5109\\
864	-3.5081\\
865	-3.5051\\
866	-3.5022\\
867	-3.4995\\
868	-3.4972\\
869	-3.4955\\
870	-3.4943\\
871	-3.4938\\
872	-3.4938\\
873	-3.4943\\
874	-3.4951\\
875	-3.4962\\
876	-3.4974\\
877	-3.4987\\
878	-3.4999\\
879	-3.5009\\
880	-3.5017\\
881	-3.5022\\
882	-3.5026\\
883	-3.5026\\
884	-3.5025\\
885	-3.5022\\
886	-3.5018\\
887	-3.5013\\
888	-3.5007\\
889	-3.5002\\
890	-3.4998\\
891	-3.4994\\
892	-3.4991\\
893	-3.499\\
894	-3.4989\\
895	-3.4989\\
896	-3.499\\
897	-3.4992\\
898	-3.4994\\
899	-3.4996\\
900	-3.4998\\
901	-3.5\\
902	-3.5002\\
903	-3.5003\\
904	-3.5004\\
905	-3.5005\\
906	-3.5005\\
907	-3.5004\\
908	-3.5004\\
909	-3.5003\\
910	-3.5002\\
911	-3.5001\\
912	-3.5\\
913	-3.4999\\
914	-3.4999\\
915	-3.4998\\
916	-3.4998\\
917	-3.4998\\
918	-3.4998\\
919	-3.4998\\
920	-3.4999\\
921	-3.4999\\
922	-3.4999\\
923	-3.5\\
924	-3.5\\
925	-3.5\\
926	-3.5001\\
927	-3.5001\\
928	-3.5001\\
929	-3.5001\\
930	-3.5001\\
931	-3.5001\\
932	-3.5\\
933	-3.5\\
934	-3.5\\
935	-3.5\\
936	-3.5\\
937	-3.5\\
938	-3.5\\
939	-3.5\\
940	-3.5\\
941	-3.5\\
942	-3.5\\
943	-3.5\\
944	-3.5\\
945	-3.5\\
946	-3.5\\
947	-3.5\\
948	-3.5\\
949	-3.5\\
950	-3.5\\
951	-3.5\\
952	-3.5\\
953	-3.5\\
954	-3.5\\
955	-3.5\\
956	-3.5\\
957	-3.5\\
958	-3.5\\
959	-3.5\\
960	-3.5\\
961	-3.5\\
962	-3.5\\
963	-3.5\\
964	-3.5\\
965	-3.5\\
966	-3.5\\
967	-3.5\\
968	-3.5\\
969	-3.5\\
970	-3.5\\
971	-3.5\\
972	-3.5\\
973	-3.5\\
974	-3.5\\
975	-3.5\\
976	-3.5\\
977	-3.5\\
978	-3.5\\
979	-3.5\\
980	-3.5\\
981	-3.5\\
982	-3.5\\
983	-3.5\\
984	-3.5\\
985	-3.5\\
986	-3.5\\
987	-3.5\\
988	-3.5\\
989	-3.5\\
990	-3.5\\
991	-3.5\\
992	-3.5\\
993	-3.5\\
994	-3.5\\
995	-3.5\\
996	-3.5\\
997	-3.5\\
998	-3.5\\
999	-3.5\\
1000	-3.5\\
1001	-3.5\\
1002	-3.5\\
1003	-3.5\\
1004	-3.5\\
1005	-3.3638\\
1006	-3.0977\\
1007	-2.7001\\
1008	-2.252\\
1009	-1.8094\\
1010	-1.423\\
1011	-1.1164\\
1012	-0.8995\\
1013	-0.77129\\
1014	-0.72519\\
1015	-0.75012\\
1016	-0.83147\\
1017	-0.95223\\
1018	-1.0943\\
1019	-1.2404\\
1020	-1.3748\\
1021	-1.4854\\
1022	-1.5637\\
1023	-1.6056\\
1024	-1.6113\\
1025	-1.5849\\
1026	-1.5333\\
1027	-1.4653\\
1028	-1.3898\\
1029	-1.3153\\
1030	-1.2484\\
1031	-1.1941\\
1032	-1.1549\\
1033	-1.1317\\
1034	-1.1236\\
1035	-1.1285\\
1036	-1.1434\\
1037	-1.1649\\
1038	-1.1896\\
1039	-1.2142\\
1040	-1.236\\
1041	-1.2531\\
1042	-1.2644\\
1043	-1.2693\\
1044	-1.2684\\
1045	-1.2624\\
1046	-1.2527\\
1047	-1.2407\\
1048	-1.2278\\
1049	-1.2155\\
1050	-1.2046\\
1051	-1.196\\
1052	-1.1901\\
1053	-1.1869\\
1054	-1.1863\\
1055	-1.1877\\
1056	-1.1907\\
1057	-1.1946\\
1058	-1.1989\\
1059	-1.203\\
1060	-1.2066\\
1061	-1.2092\\
1062	-1.2108\\
1063	-1.2113\\
1064	-1.2109\\
1065	-1.2097\\
1066	-1.2079\\
1067	-1.2059\\
1068	-1.2037\\
1069	-1.2017\\
1070	-1.2\\
1071	-1.1987\\
1072	-1.1979\\
1073	-1.1975\\
1074	-1.1975\\
1075	-1.1979\\
1076	-1.1985\\
1077	-1.1992\\
1078	-1.1999\\
1079	-1.2006\\
1080	-1.2012\\
1081	-1.2016\\
1082	-1.2018\\
1083	-1.2018\\
1084	-1.2017\\
1085	-1.2015\\
1086	-1.2012\\
1087	-1.2008\\
1088	-1.2005\\
1089	-1.2001\\
1090	-1.1999\\
1091	-1.1997\\
1092	-1.1996\\
1093	-1.1995\\
1094	-1.1996\\
1095	-1.1997\\
1096	-1.1998\\
1097	-1.1999\\
1098	-1.2\\
1099	-1.2001\\
1100	-1.2002\\
1101	-1.2003\\
1102	-1.2003\\
1103	-1.2003\\
1104	-1.2003\\
1105	-1.2002\\
1106	-1.2002\\
1107	-1.2001\\
1108	-1.2001\\
1109	-1.2\\
1110	-1.2\\
1111	-1.1999\\
1112	-1.1999\\
1113	-1.1999\\
1114	-1.1999\\
1115	-1.1999\\
1116	-1.2\\
1117	-1.2\\
1118	-1.2\\
1119	-1.2\\
1120	-1.2\\
1121	-1.2\\
1122	-1.2\\
1123	-1.2\\
1124	-1.2\\
1125	-1.2\\
1126	-1.2\\
1127	-1.2\\
1128	-1.2\\
1129	-1.2\\
1130	-1.2\\
1131	-1.2\\
1132	-1.2\\
1133	-1.2\\
1134	-1.2\\
1135	-1.2\\
1136	-1.2\\
1137	-1.2\\
1138	-1.2\\
1139	-1.2\\
1140	-1.2\\
1141	-1.2\\
1142	-1.2\\
1143	-1.2\\
1144	-1.2\\
1145	-1.2\\
1146	-1.2\\
1147	-1.2\\
1148	-1.2\\
1149	-1.2\\
1150	-1.2\\
1151	-1.2\\
1152	-1.2\\
1153	-1.2\\
1154	-1.2\\
1155	-1.2\\
1156	-1.2\\
1157	-1.2\\
1158	-1.2\\
1159	-1.2\\
1160	-1.2\\
1161	-1.2\\
1162	-1.2\\
1163	-1.2\\
1164	-1.2\\
1165	-1.2\\
1166	-1.2\\
1167	-1.2\\
1168	-1.2\\
1169	-1.2\\
1170	-1.2\\
1171	-1.2\\
1172	-1.2\\
1173	-1.2\\
1174	-1.2\\
1175	-1.2\\
1176	-1.2\\
1177	-1.2\\
1178	-1.2\\
1179	-1.2\\
1180	-1.2\\
1181	-1.2\\
1182	-1.2\\
1183	-1.2\\
1184	-1.2\\
1185	-1.2\\
1186	-1.2\\
1187	-1.2\\
1188	-1.2\\
1189	-1.2\\
1190	-1.2\\
1191	-1.2\\
1192	-1.2\\
1193	-1.2\\
1194	-1.2\\
1195	-1.2\\
1196	-1.2\\
1197	-1.2\\
1198	-1.2\\
1199	-1.2\\
1200	-1.2\\
1201	-1.2\\
1202	-1.2\\
1203	-1.2\\
1204	-1.2\\
1205	-1.1568\\
1206	-1.075\\
1207	-0.96218\\
1208	-0.83809\\
1209	-0.71711\\
1210	-0.61113\\
1211	-0.52659\\
1212	-0.46625\\
1213	-0.42959\\
1214	-0.41393\\
1215	-0.41499\\
1216	-0.42766\\
1217	-0.44657\\
1218	-0.46672\\
1219	-0.48401\\
1220	-0.49555\\
1221	-0.49985\\
1222	-0.49673\\
1223	-0.48712\\
1224	-0.47264\\
1225	-0.45528\\
1226	-0.43702\\
1227	-0.41956\\
1228	-0.40415\\
1229	-0.39157\\
1230	-0.38209\\
1231	-0.37559\\
1232	-0.37164\\
1233	-0.36962\\
1234	-0.36881\\
1235	-0.36857\\
1236	-0.36831\\
1237	-0.36763\\
1238	-0.36628\\
1239	-0.36419\\
1240	-0.36141\\
1241	-0.35811\\
1242	-0.35448\\
1243	-0.35075\\
1244	-0.34713\\
1245	-0.34377\\
1246	-0.34077\\
1247	-0.33819\\
1248	-0.33602\\
1249	-0.33421\\
1250	-0.3327\\
1251	-0.33141\\
1252	-0.33025\\
1253	-0.32916\\
1254	-0.32808\\
1255	-0.32698\\
1256	-0.32585\\
1257	-0.32468\\
1258	-0.32349\\
1259	-0.32231\\
1260	-0.32114\\
1261	-0.32002\\
1262	-0.31896\\
1263	-0.31797\\
1264	-0.31705\\
1265	-0.31621\\
1266	-0.31544\\
1267	-0.31473\\
1268	-0.31407\\
1269	-0.31345\\
1270	-0.31286\\
1271	-0.31229\\
1272	-0.31175\\
1273	-0.31122\\
1274	-0.3107\\
1275	-0.3102\\
1276	-0.30972\\
1277	-0.30926\\
1278	-0.30881\\
1279	-0.30839\\
1280	-0.30799\\
1281	-0.30761\\
1282	-0.30726\\
1283	-0.30692\\
1284	-0.3066\\
1285	-0.3063\\
1286	-0.30601\\
1287	-0.30574\\
1288	-0.30548\\
1289	-0.30523\\
1290	-0.30499\\
1291	-0.30476\\
1292	-0.30454\\
1293	-0.30433\\
1294	-0.30413\\
1295	-0.30394\\
1296	-0.30375\\
1297	-0.30358\\
1298	-0.30342\\
1299	-0.30326\\
1300	-0.30311\\
1301	-0.30296\\
1302	-0.30283\\
1303	-0.3027\\
1304	-0.30258\\
1305	-0.30246\\
1306	-0.30235\\
1307	-0.30224\\
1308	-0.30214\\
1309	-0.30204\\
1310	-0.30194\\
1311	-0.30185\\
1312	-0.30177\\
1313	-0.30169\\
1314	-0.30161\\
1315	-0.30154\\
1316	-0.30147\\
1317	-0.3014\\
1318	-0.30133\\
1319	-0.30127\\
1320	-0.30122\\
1321	-0.30116\\
1322	-0.30111\\
1323	-0.30106\\
1324	-0.30101\\
1325	-0.30096\\
1326	-0.30092\\
1327	-0.30088\\
1328	-0.30084\\
1329	-0.3008\\
1330	-0.30076\\
1331	-0.30073\\
1332	-0.30069\\
1333	-0.30066\\
1334	-0.30063\\
1335	-0.3006\\
1336	-0.30057\\
1337	-0.30055\\
1338	-0.30052\\
1339	-0.3005\\
1340	-0.30048\\
1341	-0.30045\\
1342	-0.30043\\
1343	-0.30041\\
1344	-0.30039\\
1345	-0.30038\\
1346	-0.30036\\
1347	-0.30034\\
1348	-0.30033\\
1349	-0.30031\\
1350	-0.3003\\
1351	-0.30028\\
1352	-0.30027\\
1353	-0.30026\\
1354	-0.30025\\
1355	-0.30024\\
1356	-0.30022\\
1357	-0.30021\\
1358	-0.3002\\
1359	-0.3002\\
1360	-0.30019\\
1361	-0.30018\\
1362	-0.30017\\
1363	-0.30016\\
1364	-0.30015\\
1365	-0.30015\\
1366	-0.30014\\
1367	-0.30013\\
1368	-0.30013\\
1369	-0.30012\\
1370	-0.30012\\
1371	-0.30011\\
1372	-0.30011\\
1373	-0.3001\\
1374	-0.3001\\
1375	-0.30009\\
1376	-0.30009\\
1377	-0.30008\\
1378	-0.30008\\
1379	-0.30008\\
1380	-0.30007\\
1381	-0.30007\\
1382	-0.30007\\
1383	-0.30006\\
1384	-0.30006\\
1385	-0.30006\\
1386	-0.30006\\
1387	-0.30005\\
1388	-0.30005\\
1389	-0.30005\\
1390	-0.30005\\
1391	-0.30004\\
1392	-0.30004\\
1393	-0.30004\\
1394	-0.30004\\
1395	-0.30004\\
1396	-0.30003\\
1397	-0.30003\\
1398	-0.30003\\
1399	-0.30003\\
1400	-0.30003\\
1401	-0.30003\\
1402	-0.30003\\
1403	-0.30002\\
1404	-0.30002\\
1405	-0.29119\\
1406	-0.27417\\
1407	-0.2516\\
1408	-0.2273\\
1409	-0.20422\\
1410	-0.18445\\
1411	-0.16901\\
1412	-0.15807\\
1413	-0.15117\\
1414	-0.14744\\
1415	-0.14586\\
1416	-0.14538\\
1417	-0.1451\\
1418	-0.14435\\
1419	-0.14272\\
1420	-0.14006\\
1421	-0.13647\\
1422	-0.13214\\
1423	-0.12738\\
1424	-0.12247\\
1425	-0.11769\\
1426	-0.11321\\
1427	-0.10915\\
1428	-0.10554\\
1429	-0.10237\\
1430	-0.099556\\
1431	-0.097026\\
1432	-0.094688\\
1433	-0.092462\\
1434	-0.090285\\
1435	-0.088114\\
1436	-0.08593\\
1437	-0.083497\\
1438	-0.080612\\
1439	-0.077267\\
1440	-0.073548\\
1441	-0.069573\\
1442	-0.065266\\
1443	-0.060567\\
1444	-0.055561\\
1445	-0.050235\\
1446	-0.0446\\
1447	-0.038794\\
1448	-0.032873\\
1449	-0.026899\\
1450	-0.02093\\
1451	-0.014973\\
1452	-0.0091148\\
1453	-0.003523\\
1454	0.0016335\\
1455	0.0062194\\
1456	0.010142\\
1457	0.013349\\
1458	0.015818\\
1459	0.017555\\
1460	0.018583\\
1461	0.018944\\
1462	0.018691\\
1463	0.017887\\
1464	0.016603\\
1465	0.014913\\
1466	0.012897\\
1467	0.010636\\
1468	0.008214\\
1469	0.0057123\\
1470	0.0032114\\
1471	0.00078806\\
1472	-0.0014869\\
1473	-0.0035501\\
1474	-0.0053475\\
1475	-0.0068361\\
1476	-0.0079856\\
1477	-0.0087786\\
1478	-0.0092114\\
1479	-0.0092937\\
1480	-0.0090472\\
1481	-0.0085047\\
1482	-0.0077078\\
1483	-0.0067046\\
1484	-0.0055471\\
1485	-0.0042891\\
1486	-0.0029832\\
1487	-0.0016795\\
1488	-0.00042325\\
1489	0.00074616\\
1490	0.0017961\\
1491	0.0027011\\
1492	0.0034431\\
1493	0.0040111\\
1494	0.0044012\\
1495	0.0046151\\
1496	0.0046605\\
1497	0.0045495\\
1498	0.0042982\\
1499	0.0039258\\
1500	0.0034537\\
1501	0.0029049\\
1502	0.0023028\\
1503	0.0016711\\
1504	0.0010325\\
1505	0.00040834\\
1506	-0.00018188\\
1507	-0.0007212\\
1508	-0.0011955\\
1509	-0.0015937\\
1510	-0.0019082\\
1511	-0.0021346\\
1512	-0.0022719\\
1513	-0.002322\\
1514	-0.0022899\\
1515	-0.0021829\\
1516	-0.0020104\\
1517	-0.0017833\\
1518	-0.0015136\\
1519	-0.0012138\\
1520	-0.00089648\\
1521	-0.00057369\\
1522	-0.00025684\\
1523	4.3863e-05\\
1524	0.00031956\\
1525	0.00056298\\
1526	0.00076851\\
1527	0.00093231\\
1528	0.0010522\\
1529	0.0011276\\
1530	0.0011596\\
1531	0.0011505\\
1532	0.0011039\\
1533	0.0010241\\
1534	0.00091644\\
1535	0.00078662\\
1536	0.00064066\\
1537	0.00048465\\
1538	0.00032453\\
1539	0.00016595\\
1540	1.4063e-05\\
1541	-0.00012658\\
1542	-0.00025215\\
1543	-0.0003596\\
1544	-0.00044674\\
1545	-0.00051219\\
1546	-0.00055544\\
1547	-0.00057674\\
1548	-0.00057705\\
1549	-0.000558\\
1550	-0.00052171\\
1551	-0.00047073\\
1552	-0.00040793\\
1553	-0.00033631\\
1554	-0.00025899\\
1555	-0.00017898\\
1556	-9.9193e-05\\
1557	-2.2279e-05\\
1558	4.9406e-05\\
1559	0.00011387\\
1560	0.00016952\\
1561	0.00021517\\
1562	0.00025009\\
1563	0.00027394\\
1564	0.00028678\\
1565	0.00028905\\
1566	0.00028148\\
1567	0.0002651\\
1568	0.00024113\\
1569	0.00021097\\
1570	0.0001761\\
1571	0.00013804\\
1572	9.832e-05\\
1573	5.8388e-05\\
1574	1.9595e-05\\
1575	-1.685e-05\\
1576	-4.9912e-05\\
1577	-7.8745e-05\\
1578	-0.00010271\\
1579	-0.00012139\\
1580	-0.00013456\\
1581	-0.00014221\\
1582	-0.00014451\\
1583	-0.00014179\\
1584	-0.00013454\\
1585	-0.00012335\\
1586	-0.00010889\\
1587	-9.1916e-05\\
1588	-7.3179e-05\\
1589	-5.3446e-05\\
1590	-3.3454e-05\\
1591	-1.3893e-05\\
1592	4.6157e-06\\
1593	2.1535e-05\\
1594	3.6423e-05\\
1595	4.8937e-05\\
1596	5.8845e-05\\
1597	6.6017e-05\\
1598	7.0426e-05\\
1599	7.2138e-05\\
1600	7.1305e-05\\
1601	6.8153e-05\\
1602	6.2967e-05\\
1603	5.6077e-05\\
1604	4.7846e-05\\
1605	0.0013494\\
1606	0.0052296\\
1607	0.011556\\
1608	0.019637\\
1609	0.02866\\
1610	0.037915\\
1611	0.046864\\
1612	0.055142\\
1613	0.062542\\
1614	0.068974\\
1615	0.074441\\
1616	0.079002\\
1617	0.08275\\
1618	0.085791\\
1619	0.088236\\
1620	0.090187\\
1621	0.091736\\
1622	0.092961\\
1623	0.09393\\
1624	0.094697\\
1625	0.095307\\
1626	0.095795\\
1627	0.096189\\
1628	0.09651\\
1629	0.096776\\
1630	0.096998\\
1631	0.097188\\
1632	0.097352\\
1633	0.097496\\
1634	0.097625\\
1635	0.097741\\
1636	0.097846\\
1637	0.097944\\
1638	0.098034\\
1639	0.098119\\
1640	0.098198\\
1641	0.098273\\
1642	0.098344\\
1643	0.098412\\
1644	0.098476\\
1645	0.098537\\
1646	0.098595\\
1647	0.098651\\
1648	0.098704\\
1649	0.098755\\
1650	0.098803\\
1651	0.09885\\
1652	0.098895\\
1653	0.098938\\
1654	0.098979\\
1655	0.099018\\
1656	0.099056\\
1657	0.099092\\
1658	0.099127\\
1659	0.09916\\
1660	0.099192\\
1661	0.099223\\
1662	0.099252\\
1663	0.099281\\
1664	0.099308\\
1665	0.099334\\
1666	0.099359\\
1667	0.099383\\
1668	0.099407\\
1669	0.099429\\
1670	0.09945\\
1671	0.099471\\
1672	0.099491\\
1673	0.09951\\
1674	0.099528\\
1675	0.099546\\
1676	0.099563\\
1677	0.099579\\
1678	0.099595\\
1679	0.09961\\
1680	0.099624\\
1681	0.099638\\
1682	0.099651\\
1683	0.099664\\
1684	0.099677\\
1685	0.099689\\
1686	0.0997\\
1687	0.099711\\
1688	0.099722\\
1689	0.099732\\
1690	0.099742\\
1691	0.099752\\
1692	0.099761\\
1693	0.09977\\
1694	0.099778\\
1695	0.099786\\
1696	0.099794\\
1697	0.099802\\
1698	0.099809\\
1699	0.099816\\
1700	0.099823\\
1701	0.099829\\
1702	0.099835\\
1703	0.099841\\
1704	0.099847\\
1705	0.099853\\
1706	0.099858\\
1707	0.099863\\
1708	0.099868\\
1709	0.099873\\
1710	0.099878\\
1711	0.099882\\
1712	0.099887\\
1713	0.099891\\
1714	0.099895\\
1715	0.099899\\
1716	0.099902\\
1717	0.099906\\
1718	0.099909\\
1719	0.099913\\
1720	0.099916\\
1721	0.099919\\
1722	0.099922\\
1723	0.099925\\
1724	0.099927\\
1725	0.09993\\
1726	0.099933\\
1727	0.099935\\
1728	0.099937\\
1729	0.09994\\
1730	0.099942\\
1731	0.099944\\
1732	0.099946\\
1733	0.099948\\
1734	0.09995\\
1735	0.099952\\
1736	0.099953\\
1737	0.099955\\
1738	0.099957\\
1739	0.099958\\
1740	0.09996\\
1741	0.099961\\
1742	0.099963\\
1743	0.099964\\
1744	0.099965\\
1745	0.099967\\
1746	0.099968\\
1747	0.099969\\
1748	0.09997\\
1749	0.099971\\
1750	0.099972\\
1751	0.099973\\
1752	0.099974\\
1753	0.099975\\
1754	0.099976\\
1755	0.099977\\
1756	0.099978\\
1757	0.099979\\
1758	0.099979\\
1759	0.09998\\
1760	0.099981\\
1761	0.099982\\
1762	0.099982\\
1763	0.099983\\
1764	0.099984\\
1765	0.099984\\
1766	0.099985\\
1767	0.099985\\
1768	0.099986\\
1769	0.099986\\
1770	0.099987\\
1771	0.099987\\
1772	0.099988\\
1773	0.099988\\
1774	0.099989\\
1775	0.099989\\
1776	0.099989\\
1777	0.09999\\
1778	0.09999\\
1779	0.099991\\
1780	0.099991\\
1781	0.099991\\
1782	0.099992\\
1783	0.099992\\
1784	0.099992\\
1785	0.099992\\
1786	0.099993\\
1787	0.099993\\
1788	0.099993\\
1789	0.099993\\
1790	0.099994\\
1791	0.099994\\
1792	0.099994\\
1793	0.099994\\
1794	0.099995\\
1795	0.099995\\
1796	0.099995\\
1797	0.099995\\
1798	0.099995\\
1799	0.099995\\
1800	0.099996\\
};
\addlegendentry{Wyjście y}

\addplot [color=mycolor2, dashed]
  table[row sep=crcr]{%
1	0\\
2	0\\
3	0\\
4	0\\
5	0\\
6	0\\
7	0\\
8	0\\
9	0\\
10	0\\
11	0\\
12	0\\
13	0\\
14	0\\
15	0\\
16	0\\
17	0\\
18	0\\
19	0\\
20	-2\\
21	-2\\
22	-2\\
23	-2\\
24	-2\\
25	-2\\
26	-2\\
27	-2\\
28	-2\\
29	-2\\
30	-2\\
31	-2\\
32	-2\\
33	-2\\
34	-2\\
35	-2\\
36	-2\\
37	-2\\
38	-2\\
39	-2\\
40	-2\\
41	-2\\
42	-2\\
43	-2\\
44	-2\\
45	-2\\
46	-2\\
47	-2\\
48	-2\\
49	-2\\
50	-2\\
51	-2\\
52	-2\\
53	-2\\
54	-2\\
55	-2\\
56	-2\\
57	-2\\
58	-2\\
59	-2\\
60	-2\\
61	-2\\
62	-2\\
63	-2\\
64	-2\\
65	-2\\
66	-2\\
67	-2\\
68	-2\\
69	-2\\
70	-2\\
71	-2\\
72	-2\\
73	-2\\
74	-2\\
75	-2\\
76	-2\\
77	-2\\
78	-2\\
79	-2\\
80	-2\\
81	-2\\
82	-2\\
83	-2\\
84	-2\\
85	-2\\
86	-2\\
87	-2\\
88	-2\\
89	-2\\
90	-2\\
91	-2\\
92	-2\\
93	-2\\
94	-2\\
95	-2\\
96	-2\\
97	-2\\
98	-2\\
99	-2\\
100	-2\\
101	-2\\
102	-2\\
103	-2\\
104	-2\\
105	-2\\
106	-2\\
107	-2\\
108	-2\\
109	-2\\
110	-2\\
111	-2\\
112	-2\\
113	-2\\
114	-2\\
115	-2\\
116	-2\\
117	-2\\
118	-2\\
119	-2\\
120	-2\\
121	-2\\
122	-2\\
123	-2\\
124	-2\\
125	-2\\
126	-2\\
127	-2\\
128	-2\\
129	-2\\
130	-2\\
131	-2\\
132	-2\\
133	-2\\
134	-2\\
135	-2\\
136	-2\\
137	-2\\
138	-2\\
139	-2\\
140	-2\\
141	-2\\
142	-2\\
143	-2\\
144	-2\\
145	-2\\
146	-2\\
147	-2\\
148	-2\\
149	-2\\
150	-2\\
151	-2\\
152	-2\\
153	-2\\
154	-2\\
155	-2\\
156	-2\\
157	-2\\
158	-2\\
159	-2\\
160	-2\\
161	-2\\
162	-2\\
163	-2\\
164	-2\\
165	-2\\
166	-2\\
167	-2\\
168	-2\\
169	-2\\
170	-2\\
171	-2\\
172	-2\\
173	-2\\
174	-2\\
175	-2\\
176	-2\\
177	-2\\
178	-2\\
179	-2\\
180	-2\\
181	-2\\
182	-2\\
183	-2\\
184	-2\\
185	-2\\
186	-2\\
187	-2\\
188	-2\\
189	-2\\
190	-2\\
191	-2\\
192	-2\\
193	-2\\
194	-2\\
195	-2\\
196	-2\\
197	-2\\
198	-2\\
199	-2\\
200	-4.5\\
201	-4.5\\
202	-4.5\\
203	-4.5\\
204	-4.5\\
205	-4.5\\
206	-4.5\\
207	-4.5\\
208	-4.5\\
209	-4.5\\
210	-4.5\\
211	-4.5\\
212	-4.5\\
213	-4.5\\
214	-4.5\\
215	-4.5\\
216	-4.5\\
217	-4.5\\
218	-4.5\\
219	-4.5\\
220	-4.5\\
221	-4.5\\
222	-4.5\\
223	-4.5\\
224	-4.5\\
225	-4.5\\
226	-4.5\\
227	-4.5\\
228	-4.5\\
229	-4.5\\
230	-4.5\\
231	-4.5\\
232	-4.5\\
233	-4.5\\
234	-4.5\\
235	-4.5\\
236	-4.5\\
237	-4.5\\
238	-4.5\\
239	-4.5\\
240	-4.5\\
241	-4.5\\
242	-4.5\\
243	-4.5\\
244	-4.5\\
245	-4.5\\
246	-4.5\\
247	-4.5\\
248	-4.5\\
249	-4.5\\
250	-4.5\\
251	-4.5\\
252	-4.5\\
253	-4.5\\
254	-4.5\\
255	-4.5\\
256	-4.5\\
257	-4.5\\
258	-4.5\\
259	-4.5\\
260	-4.5\\
261	-4.5\\
262	-4.5\\
263	-4.5\\
264	-4.5\\
265	-4.5\\
266	-4.5\\
267	-4.5\\
268	-4.5\\
269	-4.5\\
270	-4.5\\
271	-4.5\\
272	-4.5\\
273	-4.5\\
274	-4.5\\
275	-4.5\\
276	-4.5\\
277	-4.5\\
278	-4.5\\
279	-4.5\\
280	-4.5\\
281	-4.5\\
282	-4.5\\
283	-4.5\\
284	-4.5\\
285	-4.5\\
286	-4.5\\
287	-4.5\\
288	-4.5\\
289	-4.5\\
290	-4.5\\
291	-4.5\\
292	-4.5\\
293	-4.5\\
294	-4.5\\
295	-4.5\\
296	-4.5\\
297	-4.5\\
298	-4.5\\
299	-4.5\\
300	-4.5\\
301	-4.5\\
302	-4.5\\
303	-4.5\\
304	-4.5\\
305	-4.5\\
306	-4.5\\
307	-4.5\\
308	-4.5\\
309	-4.5\\
310	-4.5\\
311	-4.5\\
312	-4.5\\
313	-4.5\\
314	-4.5\\
315	-4.5\\
316	-4.5\\
317	-4.5\\
318	-4.5\\
319	-4.5\\
320	-4.5\\
321	-4.5\\
322	-4.5\\
323	-4.5\\
324	-4.5\\
325	-4.5\\
326	-4.5\\
327	-4.5\\
328	-4.5\\
329	-4.5\\
330	-4.5\\
331	-4.5\\
332	-4.5\\
333	-4.5\\
334	-4.5\\
335	-4.5\\
336	-4.5\\
337	-4.5\\
338	-4.5\\
339	-4.5\\
340	-4.5\\
341	-4.5\\
342	-4.5\\
343	-4.5\\
344	-4.5\\
345	-4.5\\
346	-4.5\\
347	-4.5\\
348	-4.5\\
349	-4.5\\
350	-4.5\\
351	-4.5\\
352	-4.5\\
353	-4.5\\
354	-4.5\\
355	-4.5\\
356	-4.5\\
357	-4.5\\
358	-4.5\\
359	-4.5\\
360	-4.5\\
361	-4.5\\
362	-4.5\\
363	-4.5\\
364	-4.5\\
365	-4.5\\
366	-4.5\\
367	-4.5\\
368	-4.5\\
369	-4.5\\
370	-4.5\\
371	-4.5\\
372	-4.5\\
373	-4.5\\
374	-4.5\\
375	-4.5\\
376	-4.5\\
377	-4.5\\
378	-4.5\\
379	-4.5\\
380	-4.5\\
381	-4.5\\
382	-4.5\\
383	-4.5\\
384	-4.5\\
385	-4.5\\
386	-4.5\\
387	-4.5\\
388	-4.5\\
389	-4.5\\
390	-4.5\\
391	-4.5\\
392	-4.5\\
393	-4.5\\
394	-4.5\\
395	-4.5\\
396	-4.5\\
397	-4.5\\
398	-4.5\\
399	-4.5\\
400	-3\\
401	-3\\
402	-3\\
403	-3\\
404	-3\\
405	-3\\
406	-3\\
407	-3\\
408	-3\\
409	-3\\
410	-3\\
411	-3\\
412	-3\\
413	-3\\
414	-3\\
415	-3\\
416	-3\\
417	-3\\
418	-3\\
419	-3\\
420	-3\\
421	-3\\
422	-3\\
423	-3\\
424	-3\\
425	-3\\
426	-3\\
427	-3\\
428	-3\\
429	-3\\
430	-3\\
431	-3\\
432	-3\\
433	-3\\
434	-3\\
435	-3\\
436	-3\\
437	-3\\
438	-3\\
439	-3\\
440	-3\\
441	-3\\
442	-3\\
443	-3\\
444	-3\\
445	-3\\
446	-3\\
447	-3\\
448	-3\\
449	-3\\
450	-3\\
451	-3\\
452	-3\\
453	-3\\
454	-3\\
455	-3\\
456	-3\\
457	-3\\
458	-3\\
459	-3\\
460	-3\\
461	-3\\
462	-3\\
463	-3\\
464	-3\\
465	-3\\
466	-3\\
467	-3\\
468	-3\\
469	-3\\
470	-3\\
471	-3\\
472	-3\\
473	-3\\
474	-3\\
475	-3\\
476	-3\\
477	-3\\
478	-3\\
479	-3\\
480	-3\\
481	-3\\
482	-3\\
483	-3\\
484	-3\\
485	-3\\
486	-3\\
487	-3\\
488	-3\\
489	-3\\
490	-3\\
491	-3\\
492	-3\\
493	-3\\
494	-3\\
495	-3\\
496	-3\\
497	-3\\
498	-3\\
499	-3\\
500	-3\\
501	-3\\
502	-3\\
503	-3\\
504	-3\\
505	-3\\
506	-3\\
507	-3\\
508	-3\\
509	-3\\
510	-3\\
511	-3\\
512	-3\\
513	-3\\
514	-3\\
515	-3\\
516	-3\\
517	-3\\
518	-3\\
519	-3\\
520	-3\\
521	-3\\
522	-3\\
523	-3\\
524	-3\\
525	-3\\
526	-3\\
527	-3\\
528	-3\\
529	-3\\
530	-3\\
531	-3\\
532	-3\\
533	-3\\
534	-3\\
535	-3\\
536	-3\\
537	-3\\
538	-3\\
539	-3\\
540	-3\\
541	-3\\
542	-3\\
543	-3\\
544	-3\\
545	-3\\
546	-3\\
547	-3\\
548	-3\\
549	-3\\
550	-3\\
551	-3\\
552	-3\\
553	-3\\
554	-3\\
555	-3\\
556	-3\\
557	-3\\
558	-3\\
559	-3\\
560	-3\\
561	-3\\
562	-3\\
563	-3\\
564	-3\\
565	-3\\
566	-3\\
567	-3\\
568	-3\\
569	-3\\
570	-3\\
571	-3\\
572	-3\\
573	-3\\
574	-3\\
575	-3\\
576	-3\\
577	-3\\
578	-3\\
579	-3\\
580	-3\\
581	-3\\
582	-3\\
583	-3\\
584	-3\\
585	-3\\
586	-3\\
587	-3\\
588	-3\\
589	-3\\
590	-3\\
591	-3\\
592	-3\\
593	-3\\
594	-3\\
595	-3\\
596	-3\\
597	-3\\
598	-3\\
599	-3\\
600	-1.5\\
601	-1.5\\
602	-1.5\\
603	-1.5\\
604	-1.5\\
605	-1.5\\
606	-1.5\\
607	-1.5\\
608	-1.5\\
609	-1.5\\
610	-1.5\\
611	-1.5\\
612	-1.5\\
613	-1.5\\
614	-1.5\\
615	-1.5\\
616	-1.5\\
617	-1.5\\
618	-1.5\\
619	-1.5\\
620	-1.5\\
621	-1.5\\
622	-1.5\\
623	-1.5\\
624	-1.5\\
625	-1.5\\
626	-1.5\\
627	-1.5\\
628	-1.5\\
629	-1.5\\
630	-1.5\\
631	-1.5\\
632	-1.5\\
633	-1.5\\
634	-1.5\\
635	-1.5\\
636	-1.5\\
637	-1.5\\
638	-1.5\\
639	-1.5\\
640	-1.5\\
641	-1.5\\
642	-1.5\\
643	-1.5\\
644	-1.5\\
645	-1.5\\
646	-1.5\\
647	-1.5\\
648	-1.5\\
649	-1.5\\
650	-1.5\\
651	-1.5\\
652	-1.5\\
653	-1.5\\
654	-1.5\\
655	-1.5\\
656	-1.5\\
657	-1.5\\
658	-1.5\\
659	-1.5\\
660	-1.5\\
661	-1.5\\
662	-1.5\\
663	-1.5\\
664	-1.5\\
665	-1.5\\
666	-1.5\\
667	-1.5\\
668	-1.5\\
669	-1.5\\
670	-1.5\\
671	-1.5\\
672	-1.5\\
673	-1.5\\
674	-1.5\\
675	-1.5\\
676	-1.5\\
677	-1.5\\
678	-1.5\\
679	-1.5\\
680	-1.5\\
681	-1.5\\
682	-1.5\\
683	-1.5\\
684	-1.5\\
685	-1.5\\
686	-1.5\\
687	-1.5\\
688	-1.5\\
689	-1.5\\
690	-1.5\\
691	-1.5\\
692	-1.5\\
693	-1.5\\
694	-1.5\\
695	-1.5\\
696	-1.5\\
697	-1.5\\
698	-1.5\\
699	-1.5\\
700	-1.5\\
701	-1.5\\
702	-1.5\\
703	-1.5\\
704	-1.5\\
705	-1.5\\
706	-1.5\\
707	-1.5\\
708	-1.5\\
709	-1.5\\
710	-1.5\\
711	-1.5\\
712	-1.5\\
713	-1.5\\
714	-1.5\\
715	-1.5\\
716	-1.5\\
717	-1.5\\
718	-1.5\\
719	-1.5\\
720	-1.5\\
721	-1.5\\
722	-1.5\\
723	-1.5\\
724	-1.5\\
725	-1.5\\
726	-1.5\\
727	-1.5\\
728	-1.5\\
729	-1.5\\
730	-1.5\\
731	-1.5\\
732	-1.5\\
733	-1.5\\
734	-1.5\\
735	-1.5\\
736	-1.5\\
737	-1.5\\
738	-1.5\\
739	-1.5\\
740	-1.5\\
741	-1.5\\
742	-1.5\\
743	-1.5\\
744	-1.5\\
745	-1.5\\
746	-1.5\\
747	-1.5\\
748	-1.5\\
749	-1.5\\
750	-1.5\\
751	-1.5\\
752	-1.5\\
753	-1.5\\
754	-1.5\\
755	-1.5\\
756	-1.5\\
757	-1.5\\
758	-1.5\\
759	-1.5\\
760	-1.5\\
761	-1.5\\
762	-1.5\\
763	-1.5\\
764	-1.5\\
765	-1.5\\
766	-1.5\\
767	-1.5\\
768	-1.5\\
769	-1.5\\
770	-1.5\\
771	-1.5\\
772	-1.5\\
773	-1.5\\
774	-1.5\\
775	-1.5\\
776	-1.5\\
777	-1.5\\
778	-1.5\\
779	-1.5\\
780	-1.5\\
781	-1.5\\
782	-1.5\\
783	-1.5\\
784	-1.5\\
785	-1.5\\
786	-1.5\\
787	-1.5\\
788	-1.5\\
789	-1.5\\
790	-1.5\\
791	-1.5\\
792	-1.5\\
793	-1.5\\
794	-1.5\\
795	-1.5\\
796	-1.5\\
797	-1.5\\
798	-1.5\\
799	-1.5\\
800	-3.5\\
801	-3.5\\
802	-3.5\\
803	-3.5\\
804	-3.5\\
805	-3.5\\
806	-3.5\\
807	-3.5\\
808	-3.5\\
809	-3.5\\
810	-3.5\\
811	-3.5\\
812	-3.5\\
813	-3.5\\
814	-3.5\\
815	-3.5\\
816	-3.5\\
817	-3.5\\
818	-3.5\\
819	-3.5\\
820	-3.5\\
821	-3.5\\
822	-3.5\\
823	-3.5\\
824	-3.5\\
825	-3.5\\
826	-3.5\\
827	-3.5\\
828	-3.5\\
829	-3.5\\
830	-3.5\\
831	-3.5\\
832	-3.5\\
833	-3.5\\
834	-3.5\\
835	-3.5\\
836	-3.5\\
837	-3.5\\
838	-3.5\\
839	-3.5\\
840	-3.5\\
841	-3.5\\
842	-3.5\\
843	-3.5\\
844	-3.5\\
845	-3.5\\
846	-3.5\\
847	-3.5\\
848	-3.5\\
849	-3.5\\
850	-3.5\\
851	-3.5\\
852	-3.5\\
853	-3.5\\
854	-3.5\\
855	-3.5\\
856	-3.5\\
857	-3.5\\
858	-3.5\\
859	-3.5\\
860	-3.5\\
861	-3.5\\
862	-3.5\\
863	-3.5\\
864	-3.5\\
865	-3.5\\
866	-3.5\\
867	-3.5\\
868	-3.5\\
869	-3.5\\
870	-3.5\\
871	-3.5\\
872	-3.5\\
873	-3.5\\
874	-3.5\\
875	-3.5\\
876	-3.5\\
877	-3.5\\
878	-3.5\\
879	-3.5\\
880	-3.5\\
881	-3.5\\
882	-3.5\\
883	-3.5\\
884	-3.5\\
885	-3.5\\
886	-3.5\\
887	-3.5\\
888	-3.5\\
889	-3.5\\
890	-3.5\\
891	-3.5\\
892	-3.5\\
893	-3.5\\
894	-3.5\\
895	-3.5\\
896	-3.5\\
897	-3.5\\
898	-3.5\\
899	-3.5\\
900	-3.5\\
901	-3.5\\
902	-3.5\\
903	-3.5\\
904	-3.5\\
905	-3.5\\
906	-3.5\\
907	-3.5\\
908	-3.5\\
909	-3.5\\
910	-3.5\\
911	-3.5\\
912	-3.5\\
913	-3.5\\
914	-3.5\\
915	-3.5\\
916	-3.5\\
917	-3.5\\
918	-3.5\\
919	-3.5\\
920	-3.5\\
921	-3.5\\
922	-3.5\\
923	-3.5\\
924	-3.5\\
925	-3.5\\
926	-3.5\\
927	-3.5\\
928	-3.5\\
929	-3.5\\
930	-3.5\\
931	-3.5\\
932	-3.5\\
933	-3.5\\
934	-3.5\\
935	-3.5\\
936	-3.5\\
937	-3.5\\
938	-3.5\\
939	-3.5\\
940	-3.5\\
941	-3.5\\
942	-3.5\\
943	-3.5\\
944	-3.5\\
945	-3.5\\
946	-3.5\\
947	-3.5\\
948	-3.5\\
949	-3.5\\
950	-3.5\\
951	-3.5\\
952	-3.5\\
953	-3.5\\
954	-3.5\\
955	-3.5\\
956	-3.5\\
957	-3.5\\
958	-3.5\\
959	-3.5\\
960	-3.5\\
961	-3.5\\
962	-3.5\\
963	-3.5\\
964	-3.5\\
965	-3.5\\
966	-3.5\\
967	-3.5\\
968	-3.5\\
969	-3.5\\
970	-3.5\\
971	-3.5\\
972	-3.5\\
973	-3.5\\
974	-3.5\\
975	-3.5\\
976	-3.5\\
977	-3.5\\
978	-3.5\\
979	-3.5\\
980	-3.5\\
981	-3.5\\
982	-3.5\\
983	-3.5\\
984	-3.5\\
985	-3.5\\
986	-3.5\\
987	-3.5\\
988	-3.5\\
989	-3.5\\
990	-3.5\\
991	-3.5\\
992	-3.5\\
993	-3.5\\
994	-3.5\\
995	-3.5\\
996	-3.5\\
997	-3.5\\
998	-3.5\\
999	-3.5\\
1000	-1.2\\
1001	-1.2\\
1002	-1.2\\
1003	-1.2\\
1004	-1.2\\
1005	-1.2\\
1006	-1.2\\
1007	-1.2\\
1008	-1.2\\
1009	-1.2\\
1010	-1.2\\
1011	-1.2\\
1012	-1.2\\
1013	-1.2\\
1014	-1.2\\
1015	-1.2\\
1016	-1.2\\
1017	-1.2\\
1018	-1.2\\
1019	-1.2\\
1020	-1.2\\
1021	-1.2\\
1022	-1.2\\
1023	-1.2\\
1024	-1.2\\
1025	-1.2\\
1026	-1.2\\
1027	-1.2\\
1028	-1.2\\
1029	-1.2\\
1030	-1.2\\
1031	-1.2\\
1032	-1.2\\
1033	-1.2\\
1034	-1.2\\
1035	-1.2\\
1036	-1.2\\
1037	-1.2\\
1038	-1.2\\
1039	-1.2\\
1040	-1.2\\
1041	-1.2\\
1042	-1.2\\
1043	-1.2\\
1044	-1.2\\
1045	-1.2\\
1046	-1.2\\
1047	-1.2\\
1048	-1.2\\
1049	-1.2\\
1050	-1.2\\
1051	-1.2\\
1052	-1.2\\
1053	-1.2\\
1054	-1.2\\
1055	-1.2\\
1056	-1.2\\
1057	-1.2\\
1058	-1.2\\
1059	-1.2\\
1060	-1.2\\
1061	-1.2\\
1062	-1.2\\
1063	-1.2\\
1064	-1.2\\
1065	-1.2\\
1066	-1.2\\
1067	-1.2\\
1068	-1.2\\
1069	-1.2\\
1070	-1.2\\
1071	-1.2\\
1072	-1.2\\
1073	-1.2\\
1074	-1.2\\
1075	-1.2\\
1076	-1.2\\
1077	-1.2\\
1078	-1.2\\
1079	-1.2\\
1080	-1.2\\
1081	-1.2\\
1082	-1.2\\
1083	-1.2\\
1084	-1.2\\
1085	-1.2\\
1086	-1.2\\
1087	-1.2\\
1088	-1.2\\
1089	-1.2\\
1090	-1.2\\
1091	-1.2\\
1092	-1.2\\
1093	-1.2\\
1094	-1.2\\
1095	-1.2\\
1096	-1.2\\
1097	-1.2\\
1098	-1.2\\
1099	-1.2\\
1100	-1.2\\
1101	-1.2\\
1102	-1.2\\
1103	-1.2\\
1104	-1.2\\
1105	-1.2\\
1106	-1.2\\
1107	-1.2\\
1108	-1.2\\
1109	-1.2\\
1110	-1.2\\
1111	-1.2\\
1112	-1.2\\
1113	-1.2\\
1114	-1.2\\
1115	-1.2\\
1116	-1.2\\
1117	-1.2\\
1118	-1.2\\
1119	-1.2\\
1120	-1.2\\
1121	-1.2\\
1122	-1.2\\
1123	-1.2\\
1124	-1.2\\
1125	-1.2\\
1126	-1.2\\
1127	-1.2\\
1128	-1.2\\
1129	-1.2\\
1130	-1.2\\
1131	-1.2\\
1132	-1.2\\
1133	-1.2\\
1134	-1.2\\
1135	-1.2\\
1136	-1.2\\
1137	-1.2\\
1138	-1.2\\
1139	-1.2\\
1140	-1.2\\
1141	-1.2\\
1142	-1.2\\
1143	-1.2\\
1144	-1.2\\
1145	-1.2\\
1146	-1.2\\
1147	-1.2\\
1148	-1.2\\
1149	-1.2\\
1150	-1.2\\
1151	-1.2\\
1152	-1.2\\
1153	-1.2\\
1154	-1.2\\
1155	-1.2\\
1156	-1.2\\
1157	-1.2\\
1158	-1.2\\
1159	-1.2\\
1160	-1.2\\
1161	-1.2\\
1162	-1.2\\
1163	-1.2\\
1164	-1.2\\
1165	-1.2\\
1166	-1.2\\
1167	-1.2\\
1168	-1.2\\
1169	-1.2\\
1170	-1.2\\
1171	-1.2\\
1172	-1.2\\
1173	-1.2\\
1174	-1.2\\
1175	-1.2\\
1176	-1.2\\
1177	-1.2\\
1178	-1.2\\
1179	-1.2\\
1180	-1.2\\
1181	-1.2\\
1182	-1.2\\
1183	-1.2\\
1184	-1.2\\
1185	-1.2\\
1186	-1.2\\
1187	-1.2\\
1188	-1.2\\
1189	-1.2\\
1190	-1.2\\
1191	-1.2\\
1192	-1.2\\
1193	-1.2\\
1194	-1.2\\
1195	-1.2\\
1196	-1.2\\
1197	-1.2\\
1198	-1.2\\
1199	-1.2\\
1200	-0.3\\
1201	-0.3\\
1202	-0.3\\
1203	-0.3\\
1204	-0.3\\
1205	-0.3\\
1206	-0.3\\
1207	-0.3\\
1208	-0.3\\
1209	-0.3\\
1210	-0.3\\
1211	-0.3\\
1212	-0.3\\
1213	-0.3\\
1214	-0.3\\
1215	-0.3\\
1216	-0.3\\
1217	-0.3\\
1218	-0.3\\
1219	-0.3\\
1220	-0.3\\
1221	-0.3\\
1222	-0.3\\
1223	-0.3\\
1224	-0.3\\
1225	-0.3\\
1226	-0.3\\
1227	-0.3\\
1228	-0.3\\
1229	-0.3\\
1230	-0.3\\
1231	-0.3\\
1232	-0.3\\
1233	-0.3\\
1234	-0.3\\
1235	-0.3\\
1236	-0.3\\
1237	-0.3\\
1238	-0.3\\
1239	-0.3\\
1240	-0.3\\
1241	-0.3\\
1242	-0.3\\
1243	-0.3\\
1244	-0.3\\
1245	-0.3\\
1246	-0.3\\
1247	-0.3\\
1248	-0.3\\
1249	-0.3\\
1250	-0.3\\
1251	-0.3\\
1252	-0.3\\
1253	-0.3\\
1254	-0.3\\
1255	-0.3\\
1256	-0.3\\
1257	-0.3\\
1258	-0.3\\
1259	-0.3\\
1260	-0.3\\
1261	-0.3\\
1262	-0.3\\
1263	-0.3\\
1264	-0.3\\
1265	-0.3\\
1266	-0.3\\
1267	-0.3\\
1268	-0.3\\
1269	-0.3\\
1270	-0.3\\
1271	-0.3\\
1272	-0.3\\
1273	-0.3\\
1274	-0.3\\
1275	-0.3\\
1276	-0.3\\
1277	-0.3\\
1278	-0.3\\
1279	-0.3\\
1280	-0.3\\
1281	-0.3\\
1282	-0.3\\
1283	-0.3\\
1284	-0.3\\
1285	-0.3\\
1286	-0.3\\
1287	-0.3\\
1288	-0.3\\
1289	-0.3\\
1290	-0.3\\
1291	-0.3\\
1292	-0.3\\
1293	-0.3\\
1294	-0.3\\
1295	-0.3\\
1296	-0.3\\
1297	-0.3\\
1298	-0.3\\
1299	-0.3\\
1300	-0.3\\
1301	-0.3\\
1302	-0.3\\
1303	-0.3\\
1304	-0.3\\
1305	-0.3\\
1306	-0.3\\
1307	-0.3\\
1308	-0.3\\
1309	-0.3\\
1310	-0.3\\
1311	-0.3\\
1312	-0.3\\
1313	-0.3\\
1314	-0.3\\
1315	-0.3\\
1316	-0.3\\
1317	-0.3\\
1318	-0.3\\
1319	-0.3\\
1320	-0.3\\
1321	-0.3\\
1322	-0.3\\
1323	-0.3\\
1324	-0.3\\
1325	-0.3\\
1326	-0.3\\
1327	-0.3\\
1328	-0.3\\
1329	-0.3\\
1330	-0.3\\
1331	-0.3\\
1332	-0.3\\
1333	-0.3\\
1334	-0.3\\
1335	-0.3\\
1336	-0.3\\
1337	-0.3\\
1338	-0.3\\
1339	-0.3\\
1340	-0.3\\
1341	-0.3\\
1342	-0.3\\
1343	-0.3\\
1344	-0.3\\
1345	-0.3\\
1346	-0.3\\
1347	-0.3\\
1348	-0.3\\
1349	-0.3\\
1350	-0.3\\
1351	-0.3\\
1352	-0.3\\
1353	-0.3\\
1354	-0.3\\
1355	-0.3\\
1356	-0.3\\
1357	-0.3\\
1358	-0.3\\
1359	-0.3\\
1360	-0.3\\
1361	-0.3\\
1362	-0.3\\
1363	-0.3\\
1364	-0.3\\
1365	-0.3\\
1366	-0.3\\
1367	-0.3\\
1368	-0.3\\
1369	-0.3\\
1370	-0.3\\
1371	-0.3\\
1372	-0.3\\
1373	-0.3\\
1374	-0.3\\
1375	-0.3\\
1376	-0.3\\
1377	-0.3\\
1378	-0.3\\
1379	-0.3\\
1380	-0.3\\
1381	-0.3\\
1382	-0.3\\
1383	-0.3\\
1384	-0.3\\
1385	-0.3\\
1386	-0.3\\
1387	-0.3\\
1388	-0.3\\
1389	-0.3\\
1390	-0.3\\
1391	-0.3\\
1392	-0.3\\
1393	-0.3\\
1394	-0.3\\
1395	-0.3\\
1396	-0.3\\
1397	-0.3\\
1398	-0.3\\
1399	-0.3\\
1400	0\\
1401	0\\
1402	0\\
1403	0\\
1404	0\\
1405	0\\
1406	0\\
1407	0\\
1408	0\\
1409	0\\
1410	0\\
1411	0\\
1412	0\\
1413	0\\
1414	0\\
1415	0\\
1416	0\\
1417	0\\
1418	0\\
1419	0\\
1420	0\\
1421	0\\
1422	0\\
1423	0\\
1424	0\\
1425	0\\
1426	0\\
1427	0\\
1428	0\\
1429	0\\
1430	0\\
1431	0\\
1432	0\\
1433	0\\
1434	0\\
1435	0\\
1436	0\\
1437	0\\
1438	0\\
1439	0\\
1440	0\\
1441	0\\
1442	0\\
1443	0\\
1444	0\\
1445	0\\
1446	0\\
1447	0\\
1448	0\\
1449	0\\
1450	0\\
1451	0\\
1452	0\\
1453	0\\
1454	0\\
1455	0\\
1456	0\\
1457	0\\
1458	0\\
1459	0\\
1460	0\\
1461	0\\
1462	0\\
1463	0\\
1464	0\\
1465	0\\
1466	0\\
1467	0\\
1468	0\\
1469	0\\
1470	0\\
1471	0\\
1472	0\\
1473	0\\
1474	0\\
1475	0\\
1476	0\\
1477	0\\
1478	0\\
1479	0\\
1480	0\\
1481	0\\
1482	0\\
1483	0\\
1484	0\\
1485	0\\
1486	0\\
1487	0\\
1488	0\\
1489	0\\
1490	0\\
1491	0\\
1492	0\\
1493	0\\
1494	0\\
1495	0\\
1496	0\\
1497	0\\
1498	0\\
1499	0\\
1500	0\\
1501	0\\
1502	0\\
1503	0\\
1504	0\\
1505	0\\
1506	0\\
1507	0\\
1508	0\\
1509	0\\
1510	0\\
1511	0\\
1512	0\\
1513	0\\
1514	0\\
1515	0\\
1516	0\\
1517	0\\
1518	0\\
1519	0\\
1520	0\\
1521	0\\
1522	0\\
1523	0\\
1524	0\\
1525	0\\
1526	0\\
1527	0\\
1528	0\\
1529	0\\
1530	0\\
1531	0\\
1532	0\\
1533	0\\
1534	0\\
1535	0\\
1536	0\\
1537	0\\
1538	0\\
1539	0\\
1540	0\\
1541	0\\
1542	0\\
1543	0\\
1544	0\\
1545	0\\
1546	0\\
1547	0\\
1548	0\\
1549	0\\
1550	0\\
1551	0\\
1552	0\\
1553	0\\
1554	0\\
1555	0\\
1556	0\\
1557	0\\
1558	0\\
1559	0\\
1560	0\\
1561	0\\
1562	0\\
1563	0\\
1564	0\\
1565	0\\
1566	0\\
1567	0\\
1568	0\\
1569	0\\
1570	0\\
1571	0\\
1572	0\\
1573	0\\
1574	0\\
1575	0\\
1576	0\\
1577	0\\
1578	0\\
1579	0\\
1580	0\\
1581	0\\
1582	0\\
1583	0\\
1584	0\\
1585	0\\
1586	0\\
1587	0\\
1588	0\\
1589	0\\
1590	0\\
1591	0\\
1592	0\\
1593	0\\
1594	0\\
1595	0\\
1596	0\\
1597	0\\
1598	0\\
1599	0\\
1600	0.1\\
1601	0.1\\
1602	0.1\\
1603	0.1\\
1604	0.1\\
1605	0.1\\
1606	0.1\\
1607	0.1\\
1608	0.1\\
1609	0.1\\
1610	0.1\\
1611	0.1\\
1612	0.1\\
1613	0.1\\
1614	0.1\\
1615	0.1\\
1616	0.1\\
1617	0.1\\
1618	0.1\\
1619	0.1\\
1620	0.1\\
1621	0.1\\
1622	0.1\\
1623	0.1\\
1624	0.1\\
1625	0.1\\
1626	0.1\\
1627	0.1\\
1628	0.1\\
1629	0.1\\
1630	0.1\\
1631	0.1\\
1632	0.1\\
1633	0.1\\
1634	0.1\\
1635	0.1\\
1636	0.1\\
1637	0.1\\
1638	0.1\\
1639	0.1\\
1640	0.1\\
1641	0.1\\
1642	0.1\\
1643	0.1\\
1644	0.1\\
1645	0.1\\
1646	0.1\\
1647	0.1\\
1648	0.1\\
1649	0.1\\
1650	0.1\\
1651	0.1\\
1652	0.1\\
1653	0.1\\
1654	0.1\\
1655	0.1\\
1656	0.1\\
1657	0.1\\
1658	0.1\\
1659	0.1\\
1660	0.1\\
1661	0.1\\
1662	0.1\\
1663	0.1\\
1664	0.1\\
1665	0.1\\
1666	0.1\\
1667	0.1\\
1668	0.1\\
1669	0.1\\
1670	0.1\\
1671	0.1\\
1672	0.1\\
1673	0.1\\
1674	0.1\\
1675	0.1\\
1676	0.1\\
1677	0.1\\
1678	0.1\\
1679	0.1\\
1680	0.1\\
1681	0.1\\
1682	0.1\\
1683	0.1\\
1684	0.1\\
1685	0.1\\
1686	0.1\\
1687	0.1\\
1688	0.1\\
1689	0.1\\
1690	0.1\\
1691	0.1\\
1692	0.1\\
1693	0.1\\
1694	0.1\\
1695	0.1\\
1696	0.1\\
1697	0.1\\
1698	0.1\\
1699	0.1\\
1700	0.1\\
1701	0.1\\
1702	0.1\\
1703	0.1\\
1704	0.1\\
1705	0.1\\
1706	0.1\\
1707	0.1\\
1708	0.1\\
1709	0.1\\
1710	0.1\\
1711	0.1\\
1712	0.1\\
1713	0.1\\
1714	0.1\\
1715	0.1\\
1716	0.1\\
1717	0.1\\
1718	0.1\\
1719	0.1\\
1720	0.1\\
1721	0.1\\
1722	0.1\\
1723	0.1\\
1724	0.1\\
1725	0.1\\
1726	0.1\\
1727	0.1\\
1728	0.1\\
1729	0.1\\
1730	0.1\\
1731	0.1\\
1732	0.1\\
1733	0.1\\
1734	0.1\\
1735	0.1\\
1736	0.1\\
1737	0.1\\
1738	0.1\\
1739	0.1\\
1740	0.1\\
1741	0.1\\
1742	0.1\\
1743	0.1\\
1744	0.1\\
1745	0.1\\
1746	0.1\\
1747	0.1\\
1748	0.1\\
1749	0.1\\
1750	0.1\\
1751	0.1\\
1752	0.1\\
1753	0.1\\
1754	0.1\\
1755	0.1\\
1756	0.1\\
1757	0.1\\
1758	0.1\\
1759	0.1\\
1760	0.1\\
1761	0.1\\
1762	0.1\\
1763	0.1\\
1764	0.1\\
1765	0.1\\
1766	0.1\\
1767	0.1\\
1768	0.1\\
1769	0.1\\
1770	0.1\\
1771	0.1\\
1772	0.1\\
1773	0.1\\
1774	0.1\\
1775	0.1\\
1776	0.1\\
1777	0.1\\
1778	0.1\\
1779	0.1\\
1780	0.1\\
1781	0.1\\
1782	0.1\\
1783	0.1\\
1784	0.1\\
1785	0.1\\
1786	0.1\\
1787	0.1\\
1788	0.1\\
1789	0.1\\
1790	0.1\\
1791	0.1\\
1792	0.1\\
1793	0.1\\
1794	0.1\\
1795	0.1\\
1796	0.1\\
1797	0.1\\
1798	0.1\\
1799	0.1\\
1800	0.1\\
};
\addlegendentry{$\text{Wartość zadana y}_{\text{zad}}$}

\end{axis}

\begin{axis}[%
width=4.521in,
height=1.493in,
at={(0.758in,0.481in)},
scale only axis,
xmin=1,
xmax=1800,
xlabel style={font=\color{white!15!black}},
xlabel={k},
ymin=-1,
ymax=0.5,
ylabel style={font=\color{white!15!black}},
ylabel={u},
axis background/.style={fill=white},
xmajorgrids,
ymajorgrids,
legend style={legend cell align=left, align=left, draw=white!15!black}
]
\addplot [color=mycolor1]
  table[row sep=crcr]{%
1	0\\
2	0\\
3	0\\
4	0\\
5	0\\
6	0\\
7	0\\
8	0\\
9	0\\
10	0\\
11	0\\
12	0\\
13	0\\
14	0\\
15	0\\
16	0\\
17	0\\
18	0\\
19	0\\
20	-0.41608\\
21	-0.80759\\
22	-1\\
23	-1\\
24	-1\\
25	-0.99682\\
26	-0.97862\\
27	-0.932\\
28	-0.87452\\
29	-0.8069\\
30	-0.73821\\
31	-0.67143\\
32	-0.61156\\
33	-0.56104\\
34	-0.52235\\
35	-0.49655\\
36	-0.48412\\
37	-0.48442\\
38	-0.496\\
39	-0.51648\\
40	-0.54283\\
41	-0.57169\\
42	-0.59982\\
43	-0.62444\\
44	-0.64359\\
45	-0.65619\\
46	-0.66202\\
47	-0.66156\\
48	-0.65583\\
49	-0.64612\\
50	-0.63388\\
51	-0.62051\\
52	-0.60728\\
53	-0.59524\\
54	-0.58521\\
55	-0.57771\\
56	-0.57299\\
57	-0.57102\\
58	-0.57158\\
59	-0.57423\\
60	-0.57842\\
61	-0.58356\\
62	-0.58902\\
63	-0.59426\\
64	-0.59883\\
65	-0.60241\\
66	-0.60483\\
67	-0.60604\\
68	-0.6061\\
69	-0.6052\\
70	-0.60354\\
71	-0.60138\\
72	-0.599\\
73	-0.59662\\
74	-0.59445\\
75	-0.59265\\
76	-0.59132\\
77	-0.5905\\
78	-0.59018\\
79	-0.59032\\
80	-0.59084\\
81	-0.59162\\
82	-0.59257\\
83	-0.59357\\
84	-0.59453\\
85	-0.59536\\
86	-0.59601\\
87	-0.59646\\
88	-0.59668\\
89	-0.5967\\
90	-0.59654\\
91	-0.59625\\
92	-0.59586\\
93	-0.59544\\
94	-0.59501\\
95	-0.59462\\
96	-0.5943\\
97	-0.59407\\
98	-0.59392\\
99	-0.59387\\
100	-0.5939\\
101	-0.59399\\
102	-0.59414\\
103	-0.59431\\
104	-0.59449\\
105	-0.59466\\
106	-0.59481\\
107	-0.59493\\
108	-0.59501\\
109	-0.59505\\
110	-0.59505\\
111	-0.59503\\
112	-0.59497\\
113	-0.5949\\
114	-0.59483\\
115	-0.59475\\
116	-0.59468\\
117	-0.59462\\
118	-0.59458\\
119	-0.59456\\
120	-0.59455\\
121	-0.59455\\
122	-0.59457\\
123	-0.5946\\
124	-0.59463\\
125	-0.59466\\
126	-0.59469\\
127	-0.59472\\
128	-0.59474\\
129	-0.59475\\
130	-0.59476\\
131	-0.59476\\
132	-0.59476\\
133	-0.59475\\
134	-0.59473\\
135	-0.59472\\
136	-0.59471\\
137	-0.59469\\
138	-0.59468\\
139	-0.59468\\
140	-0.59467\\
141	-0.59467\\
142	-0.59467\\
143	-0.59467\\
144	-0.59468\\
145	-0.59469\\
146	-0.59469\\
147	-0.5947\\
148	-0.5947\\
149	-0.59471\\
150	-0.59471\\
151	-0.59471\\
152	-0.59471\\
153	-0.59471\\
154	-0.59471\\
155	-0.5947\\
156	-0.5947\\
157	-0.5947\\
158	-0.5947\\
159	-0.5947\\
160	-0.59469\\
161	-0.59469\\
162	-0.59469\\
163	-0.59469\\
164	-0.59469\\
165	-0.59469\\
166	-0.5947\\
167	-0.5947\\
168	-0.5947\\
169	-0.5947\\
170	-0.5947\\
171	-0.5947\\
172	-0.5947\\
173	-0.5947\\
174	-0.5947\\
175	-0.5947\\
176	-0.5947\\
177	-0.5947\\
178	-0.5947\\
179	-0.5947\\
180	-0.5947\\
181	-0.5947\\
182	-0.5947\\
183	-0.5947\\
184	-0.5947\\
185	-0.5947\\
186	-0.5947\\
187	-0.5947\\
188	-0.5947\\
189	-0.5947\\
190	-0.5947\\
191	-0.5947\\
192	-0.5947\\
193	-0.5947\\
194	-0.5947\\
195	-0.5947\\
196	-0.5947\\
197	-0.5947\\
198	-0.5947\\
199	-0.5947\\
200	-1\\
201	-1\\
202	-1\\
203	-1\\
204	-1\\
205	-0.97929\\
206	-0.95666\\
207	-0.91937\\
208	-0.88931\\
209	-0.86373\\
210	-0.84907\\
211	-0.84308\\
212	-0.84638\\
213	-0.85684\\
214	-0.8732\\
215	-0.89331\\
216	-0.91539\\
217	-0.93754\\
218	-0.95822\\
219	-0.97614\\
220	-0.99045\\
221	-1\\
222	-1\\
223	-1\\
224	-0.99858\\
225	-0.9943\\
226	-0.98793\\
227	-0.98047\\
228	-0.97277\\
229	-0.96549\\
230	-0.95911\\
231	-0.95394\\
232	-0.95018\\
233	-0.9479\\
234	-0.94704\\
235	-0.94747\\
236	-0.94898\\
237	-0.95132\\
238	-0.95421\\
239	-0.95736\\
240	-0.96051\\
241	-0.96344\\
242	-0.96598\\
243	-0.968\\
244	-0.96943\\
245	-0.97025\\
246	-0.97049\\
247	-0.97021\\
248	-0.96952\\
249	-0.96852\\
250	-0.96732\\
251	-0.96603\\
252	-0.96477\\
253	-0.9636\\
254	-0.96261\\
255	-0.96184\\
256	-0.9613\\
257	-0.96101\\
258	-0.96095\\
259	-0.9611\\
260	-0.9614\\
261	-0.96183\\
262	-0.96233\\
263	-0.96285\\
264	-0.96336\\
265	-0.96382\\
266	-0.96421\\
267	-0.96451\\
268	-0.9647\\
269	-0.9648\\
270	-0.96481\\
271	-0.96474\\
272	-0.9646\\
273	-0.96442\\
274	-0.96422\\
275	-0.96401\\
276	-0.9638\\
277	-0.96362\\
278	-0.96347\\
279	-0.96335\\
280	-0.96328\\
281	-0.96325\\
282	-0.96325\\
283	-0.96329\\
284	-0.96334\\
285	-0.96342\\
286	-0.9635\\
287	-0.96359\\
288	-0.96367\\
289	-0.96374\\
290	-0.9638\\
291	-0.96385\\
292	-0.96387\\
293	-0.96388\\
294	-0.96388\\
295	-0.96386\\
296	-0.96384\\
297	-0.96381\\
298	-0.96377\\
299	-0.96374\\
300	-0.9637\\
301	-0.96368\\
302	-0.96365\\
303	-0.96364\\
304	-0.96363\\
305	-0.96362\\
306	-0.96363\\
307	-0.96363\\
308	-0.96364\\
309	-0.96366\\
310	-0.96367\\
311	-0.96369\\
312	-0.9637\\
313	-0.96371\\
314	-0.96372\\
315	-0.96373\\
316	-0.96373\\
317	-0.96373\\
318	-0.96373\\
319	-0.96372\\
320	-0.96372\\
321	-0.96371\\
322	-0.96371\\
323	-0.9637\\
324	-0.9637\\
325	-0.96369\\
326	-0.96369\\
327	-0.96369\\
328	-0.96369\\
329	-0.96369\\
330	-0.96369\\
331	-0.96369\\
332	-0.96369\\
333	-0.96369\\
334	-0.9637\\
335	-0.9637\\
336	-0.9637\\
337	-0.9637\\
338	-0.9637\\
339	-0.9637\\
340	-0.9637\\
341	-0.9637\\
342	-0.9637\\
343	-0.9637\\
344	-0.9637\\
345	-0.9637\\
346	-0.9637\\
347	-0.9637\\
348	-0.9637\\
349	-0.9637\\
350	-0.9637\\
351	-0.9637\\
352	-0.9637\\
353	-0.9637\\
354	-0.9637\\
355	-0.9637\\
356	-0.9637\\
357	-0.9637\\
358	-0.9637\\
359	-0.9637\\
360	-0.9637\\
361	-0.9637\\
362	-0.9637\\
363	-0.9637\\
364	-0.9637\\
365	-0.9637\\
366	-0.9637\\
367	-0.9637\\
368	-0.9637\\
369	-0.9637\\
370	-0.9637\\
371	-0.9637\\
372	-0.9637\\
373	-0.9637\\
374	-0.9637\\
375	-0.9637\\
376	-0.9637\\
377	-0.9637\\
378	-0.9637\\
379	-0.9637\\
380	-0.9637\\
381	-0.9637\\
382	-0.9637\\
383	-0.9637\\
384	-0.9637\\
385	-0.9637\\
386	-0.9637\\
387	-0.9637\\
388	-0.9637\\
389	-0.9637\\
390	-0.9637\\
391	-0.9637\\
392	-0.9637\\
393	-0.9637\\
394	-0.9637\\
395	-0.9637\\
396	-0.9637\\
397	-0.9637\\
398	-0.9637\\
399	-0.9637\\
400	-0.69699\\
401	-0.65983\\
402	-0.56282\\
403	-0.54972\\
404	-0.52799\\
405	-0.54484\\
406	-0.56808\\
407	-0.61092\\
408	-0.65847\\
409	-0.71108\\
410	-0.76013\\
411	-0.80317\\
412	-0.83608\\
413	-0.85797\\
414	-0.86836\\
415	-0.86838\\
416	-0.85964\\
417	-0.84428\\
418	-0.82453\\
419	-0.80258\\
420	-0.78037\\
421	-0.75959\\
422	-0.74153\\
423	-0.72713\\
424	-0.71694\\
425	-0.71114\\
426	-0.70957\\
427	-0.71178\\
428	-0.71708\\
429	-0.72462\\
430	-0.73349\\
431	-0.74278\\
432	-0.75165\\
433	-0.75945\\
434	-0.76569\\
435	-0.77008\\
436	-0.77254\\
437	-0.77315\\
438	-0.77213\\
439	-0.76981\\
440	-0.76656\\
441	-0.76276\\
442	-0.7588\\
443	-0.75501\\
444	-0.75166\\
445	-0.74895\\
446	-0.74699\\
447	-0.74583\\
448	-0.74545\\
449	-0.74577\\
450	-0.74665\\
451	-0.74796\\
452	-0.74952\\
453	-0.75118\\
454	-0.75278\\
455	-0.75421\\
456	-0.75539\\
457	-0.75624\\
458	-0.75675\\
459	-0.75694\\
460	-0.75682\\
461	-0.75646\\
462	-0.75592\\
463	-0.75526\\
464	-0.75457\\
465	-0.75388\\
466	-0.75327\\
467	-0.75276\\
468	-0.75239\\
469	-0.75216\\
470	-0.75206\\
471	-0.7521\\
472	-0.75224\\
473	-0.75246\\
474	-0.75273\\
475	-0.75302\\
476	-0.75331\\
477	-0.75357\\
478	-0.75379\\
479	-0.75396\\
480	-0.75406\\
481	-0.7541\\
482	-0.75409\\
483	-0.75404\\
484	-0.75395\\
485	-0.75384\\
486	-0.75371\\
487	-0.75359\\
488	-0.75348\\
489	-0.75339\\
490	-0.75331\\
491	-0.75327\\
492	-0.75325\\
493	-0.75325\\
494	-0.75327\\
495	-0.75331\\
496	-0.75335\\
497	-0.7534\\
498	-0.75346\\
499	-0.7535\\
500	-0.75354\\
501	-0.75358\\
502	-0.7536\\
503	-0.75361\\
504	-0.75361\\
505	-0.7536\\
506	-0.75358\\
507	-0.75356\\
508	-0.75354\\
509	-0.75352\\
510	-0.7535\\
511	-0.75348\\
512	-0.75347\\
513	-0.75346\\
514	-0.75345\\
515	-0.75345\\
516	-0.75346\\
517	-0.75346\\
518	-0.75347\\
519	-0.75348\\
520	-0.75349\\
521	-0.7535\\
522	-0.75351\\
523	-0.75351\\
524	-0.75352\\
525	-0.75352\\
526	-0.75352\\
527	-0.75352\\
528	-0.75351\\
529	-0.75351\\
530	-0.75351\\
531	-0.7535\\
532	-0.7535\\
533	-0.7535\\
534	-0.75349\\
535	-0.75349\\
536	-0.75349\\
537	-0.75349\\
538	-0.75349\\
539	-0.75349\\
540	-0.75349\\
541	-0.7535\\
542	-0.7535\\
543	-0.7535\\
544	-0.7535\\
545	-0.7535\\
546	-0.7535\\
547	-0.7535\\
548	-0.7535\\
549	-0.7535\\
550	-0.7535\\
551	-0.7535\\
552	-0.7535\\
553	-0.7535\\
554	-0.7535\\
555	-0.7535\\
556	-0.7535\\
557	-0.7535\\
558	-0.7535\\
559	-0.7535\\
560	-0.7535\\
561	-0.7535\\
562	-0.7535\\
563	-0.7535\\
564	-0.7535\\
565	-0.7535\\
566	-0.7535\\
567	-0.7535\\
568	-0.7535\\
569	-0.7535\\
570	-0.7535\\
571	-0.7535\\
572	-0.7535\\
573	-0.7535\\
574	-0.7535\\
575	-0.7535\\
576	-0.7535\\
577	-0.7535\\
578	-0.7535\\
579	-0.7535\\
580	-0.7535\\
581	-0.7535\\
582	-0.7535\\
583	-0.7535\\
584	-0.7535\\
585	-0.7535\\
586	-0.7535\\
587	-0.7535\\
588	-0.7535\\
589	-0.7535\\
590	-0.7535\\
591	-0.7535\\
592	-0.7535\\
593	-0.7535\\
594	-0.7535\\
595	-0.7535\\
596	-0.7535\\
597	-0.7535\\
598	-0.7535\\
599	-0.7535\\
600	-0.48679\\
601	-0.44963\\
602	-0.35262\\
603	-0.33953\\
604	-0.31779\\
605	-0.33576\\
606	-0.3604\\
607	-0.40485\\
608	-0.45257\\
609	-0.50309\\
610	-0.54698\\
611	-0.58201\\
612	-0.60473\\
613	-0.6154\\
614	-0.6147\\
615	-0.60485\\
616	-0.58826\\
617	-0.5676\\
618	-0.54534\\
619	-0.52364\\
620	-0.50424\\
621	-0.48843\\
622	-0.47701\\
623	-0.47031\\
624	-0.46822\\
625	-0.47024\\
626	-0.47555\\
627	-0.48314\\
628	-0.49192\\
629	-0.50082\\
630	-0.50894\\
631	-0.51557\\
632	-0.52029\\
633	-0.52294\\
634	-0.52357\\
635	-0.52244\\
636	-0.51993\\
637	-0.51649\\
638	-0.51258\\
639	-0.50864\\
640	-0.50502\\
641	-0.50202\\
642	-0.49979\\
643	-0.49844\\
644	-0.49793\\
645	-0.49819\\
646	-0.49907\\
647	-0.50038\\
648	-0.50193\\
649	-0.50353\\
650	-0.50502\\
651	-0.50626\\
652	-0.50719\\
653	-0.50775\\
654	-0.50796\\
655	-0.50784\\
656	-0.50746\\
657	-0.50689\\
658	-0.50622\\
659	-0.50552\\
660	-0.50487\\
661	-0.50431\\
662	-0.50389\\
663	-0.50362\\
664	-0.50351\\
665	-0.50353\\
666	-0.50367\\
667	-0.50389\\
668	-0.50416\\
669	-0.50444\\
670	-0.50471\\
671	-0.50495\\
672	-0.50512\\
673	-0.50524\\
674	-0.50529\\
675	-0.50528\\
676	-0.50523\\
677	-0.50513\\
678	-0.50502\\
679	-0.5049\\
680	-0.50478\\
681	-0.50468\\
682	-0.5046\\
683	-0.50454\\
684	-0.50452\\
685	-0.50452\\
686	-0.50454\\
687	-0.50458\\
688	-0.50462\\
689	-0.50467\\
690	-0.50472\\
691	-0.50476\\
692	-0.5048\\
693	-0.50482\\
694	-0.50483\\
695	-0.50483\\
696	-0.50482\\
697	-0.50481\\
698	-0.50479\\
699	-0.50477\\
700	-0.50475\\
701	-0.50473\\
702	-0.50471\\
703	-0.5047\\
704	-0.5047\\
705	-0.5047\\
706	-0.5047\\
707	-0.50471\\
708	-0.50471\\
709	-0.50472\\
710	-0.50473\\
711	-0.50474\\
712	-0.50475\\
713	-0.50475\\
714	-0.50475\\
715	-0.50475\\
716	-0.50475\\
717	-0.50475\\
718	-0.50475\\
719	-0.50474\\
720	-0.50474\\
721	-0.50474\\
722	-0.50473\\
723	-0.50473\\
724	-0.50473\\
725	-0.50473\\
726	-0.50473\\
727	-0.50473\\
728	-0.50473\\
729	-0.50473\\
730	-0.50474\\
731	-0.50474\\
732	-0.50474\\
733	-0.50474\\
734	-0.50474\\
735	-0.50474\\
736	-0.50474\\
737	-0.50474\\
738	-0.50474\\
739	-0.50474\\
740	-0.50474\\
741	-0.50474\\
742	-0.50474\\
743	-0.50474\\
744	-0.50474\\
745	-0.50474\\
746	-0.50474\\
747	-0.50474\\
748	-0.50474\\
749	-0.50474\\
750	-0.50474\\
751	-0.50474\\
752	-0.50474\\
753	-0.50474\\
754	-0.50474\\
755	-0.50474\\
756	-0.50474\\
757	-0.50474\\
758	-0.50474\\
759	-0.50474\\
760	-0.50474\\
761	-0.50474\\
762	-0.50474\\
763	-0.50474\\
764	-0.50474\\
765	-0.50474\\
766	-0.50474\\
767	-0.50474\\
768	-0.50474\\
769	-0.50474\\
770	-0.50474\\
771	-0.50474\\
772	-0.50474\\
773	-0.50474\\
774	-0.50474\\
775	-0.50474\\
776	-0.50474\\
777	-0.50474\\
778	-0.50474\\
779	-0.50474\\
780	-0.50474\\
781	-0.50474\\
782	-0.50474\\
783	-0.50474\\
784	-0.50474\\
785	-0.50474\\
786	-0.50474\\
787	-0.50474\\
788	-0.50474\\
789	-0.50474\\
790	-0.50474\\
791	-0.50474\\
792	-0.50474\\
793	-0.50474\\
794	-0.50474\\
795	-0.50474\\
796	-0.50474\\
797	-0.50474\\
798	-0.50474\\
799	-0.50474\\
800	-0.86035\\
801	-0.9099\\
802	-1\\
803	-1\\
804	-1\\
805	-0.98232\\
806	-0.96043\\
807	-0.92107\\
808	-0.88152\\
809	-0.84072\\
810	-0.80559\\
811	-0.77623\\
812	-0.75491\\
813	-0.74144\\
814	-0.73603\\
815	-0.73787\\
816	-0.746\\
817	-0.75898\\
818	-0.77522\\
819	-0.79303\\
820	-0.81083\\
821	-0.82722\\
822	-0.84111\\
823	-0.85179\\
824	-0.8589\\
825	-0.86243\\
826	-0.86267\\
827	-0.86009\\
828	-0.85532\\
829	-0.84906\\
830	-0.842\\
831	-0.83478\\
832	-0.82796\\
833	-0.82201\\
834	-0.81722\\
835	-0.8138\\
836	-0.81181\\
837	-0.81121\\
838	-0.81183\\
839	-0.81347\\
840	-0.81586\\
841	-0.81871\\
842	-0.82173\\
843	-0.82468\\
844	-0.82734\\
845	-0.82954\\
846	-0.83119\\
847	-0.83223\\
848	-0.83268\\
849	-0.83258\\
850	-0.83202\\
851	-0.83111\\
852	-0.82997\\
853	-0.82872\\
854	-0.82747\\
855	-0.82631\\
856	-0.82532\\
857	-0.82455\\
858	-0.82403\\
859	-0.82375\\
860	-0.82371\\
861	-0.82388\\
862	-0.82421\\
863	-0.82465\\
864	-0.82516\\
865	-0.82569\\
866	-0.82619\\
867	-0.82663\\
868	-0.82699\\
869	-0.82724\\
870	-0.82739\\
871	-0.82744\\
872	-0.8274\\
873	-0.82729\\
874	-0.82712\\
875	-0.82691\\
876	-0.82669\\
877	-0.82648\\
878	-0.82628\\
879	-0.82612\\
880	-0.826\\
881	-0.82592\\
882	-0.82588\\
883	-0.82589\\
884	-0.82592\\
885	-0.82599\\
886	-0.82607\\
887	-0.82616\\
888	-0.82625\\
889	-0.82634\\
890	-0.82641\\
891	-0.82647\\
892	-0.82651\\
893	-0.82653\\
894	-0.82653\\
895	-0.82652\\
896	-0.8265\\
897	-0.82647\\
898	-0.82643\\
899	-0.82639\\
900	-0.82635\\
901	-0.82632\\
902	-0.8263\\
903	-0.82628\\
904	-0.82626\\
905	-0.82626\\
906	-0.82626\\
907	-0.82627\\
908	-0.82628\\
909	-0.8263\\
910	-0.82631\\
911	-0.82633\\
912	-0.82634\\
913	-0.82636\\
914	-0.82636\\
915	-0.82637\\
916	-0.82637\\
917	-0.82637\\
918	-0.82637\\
919	-0.82637\\
920	-0.82636\\
921	-0.82635\\
922	-0.82635\\
923	-0.82634\\
924	-0.82634\\
925	-0.82633\\
926	-0.82633\\
927	-0.82633\\
928	-0.82633\\
929	-0.82633\\
930	-0.82633\\
931	-0.82633\\
932	-0.82633\\
933	-0.82634\\
934	-0.82634\\
935	-0.82634\\
936	-0.82634\\
937	-0.82634\\
938	-0.82635\\
939	-0.82635\\
940	-0.82635\\
941	-0.82635\\
942	-0.82634\\
943	-0.82634\\
944	-0.82634\\
945	-0.82634\\
946	-0.82634\\
947	-0.82634\\
948	-0.82634\\
949	-0.82634\\
950	-0.82634\\
951	-0.82634\\
952	-0.82634\\
953	-0.82634\\
954	-0.82634\\
955	-0.82634\\
956	-0.82634\\
957	-0.82634\\
958	-0.82634\\
959	-0.82634\\
960	-0.82634\\
961	-0.82634\\
962	-0.82634\\
963	-0.82634\\
964	-0.82634\\
965	-0.82634\\
966	-0.82634\\
967	-0.82634\\
968	-0.82634\\
969	-0.82634\\
970	-0.82634\\
971	-0.82634\\
972	-0.82634\\
973	-0.82634\\
974	-0.82634\\
975	-0.82634\\
976	-0.82634\\
977	-0.82634\\
978	-0.82634\\
979	-0.82634\\
980	-0.82634\\
981	-0.82634\\
982	-0.82634\\
983	-0.82634\\
984	-0.82634\\
985	-0.82634\\
986	-0.82634\\
987	-0.82634\\
988	-0.82634\\
989	-0.82634\\
990	-0.82634\\
991	-0.82634\\
992	-0.82634\\
993	-0.82634\\
994	-0.82634\\
995	-0.82634\\
996	-0.82634\\
997	-0.82634\\
998	-0.82634\\
999	-0.82634\\
1000	-0.41739\\
1001	-0.3604\\
1002	-0.21165\\
1003	-0.19158\\
1004	-0.15826\\
1005	-0.18842\\
1006	-0.23091\\
1007	-0.30776\\
1008	-0.38943\\
1009	-0.4732\\
1010	-0.54206\\
1011	-0.59265\\
1012	-0.62067\\
1013	-0.62834\\
1014	-0.61838\\
1015	-0.59554\\
1016	-0.56428\\
1017	-0.52904\\
1018	-0.49351\\
1019	-0.46076\\
1020	-0.43307\\
1021	-0.41198\\
1022	-0.39831\\
1023	-0.39216\\
1024	-0.393\\
1025	-0.39971\\
1026	-0.41071\\
1027	-0.42418\\
1028	-0.43829\\
1029	-0.45139\\
1030	-0.46222\\
1031	-0.46997\\
1032	-0.47433\\
1033	-0.47539\\
1034	-0.47358\\
1035	-0.46954\\
1036	-0.46403\\
1037	-0.4578\\
1038	-0.45157\\
1039	-0.44593\\
1040	-0.44129\\
1041	-0.43795\\
1042	-0.436\\
1043	-0.43539\\
1044	-0.43596\\
1045	-0.43744\\
1046	-0.43954\\
1047	-0.44193\\
1048	-0.44431\\
1049	-0.44644\\
1050	-0.44814\\
1051	-0.44929\\
1052	-0.44988\\
1053	-0.44992\\
1054	-0.4495\\
1055	-0.44873\\
1056	-0.44775\\
1057	-0.44668\\
1058	-0.44565\\
1059	-0.44473\\
1060	-0.44401\\
1061	-0.44352\\
1062	-0.44327\\
1063	-0.44324\\
1064	-0.4434\\
1065	-0.4437\\
1066	-0.44408\\
1067	-0.44449\\
1068	-0.44489\\
1069	-0.44523\\
1070	-0.44549\\
1071	-0.44566\\
1072	-0.44573\\
1073	-0.44571\\
1074	-0.44562\\
1075	-0.44548\\
1076	-0.4453\\
1077	-0.44512\\
1078	-0.44495\\
1079	-0.44481\\
1080	-0.4447\\
1081	-0.44463\\
1082	-0.4446\\
1083	-0.44461\\
1084	-0.44464\\
1085	-0.4447\\
1086	-0.44477\\
1087	-0.44484\\
1088	-0.4449\\
1089	-0.44496\\
1090	-0.445\\
1091	-0.44502\\
1092	-0.44503\\
1093	-0.44502\\
1094	-0.445\\
1095	-0.44498\\
1096	-0.44495\\
1097	-0.44492\\
1098	-0.44489\\
1099	-0.44487\\
1100	-0.44485\\
1101	-0.44484\\
1102	-0.44484\\
1103	-0.44484\\
1104	-0.44485\\
1105	-0.44486\\
1106	-0.44487\\
1107	-0.44488\\
1108	-0.44489\\
1109	-0.4449\\
1110	-0.44491\\
1111	-0.44491\\
1112	-0.44491\\
1113	-0.44491\\
1114	-0.4449\\
1115	-0.4449\\
1116	-0.44489\\
1117	-0.44489\\
1118	-0.44489\\
1119	-0.44488\\
1120	-0.44488\\
1121	-0.44488\\
1122	-0.44488\\
1123	-0.44488\\
1124	-0.44488\\
1125	-0.44488\\
1126	-0.44488\\
1127	-0.44489\\
1128	-0.44489\\
1129	-0.44489\\
1130	-0.44489\\
1131	-0.44489\\
1132	-0.44489\\
1133	-0.44489\\
1134	-0.44489\\
1135	-0.44489\\
1136	-0.44489\\
1137	-0.44489\\
1138	-0.44489\\
1139	-0.44489\\
1140	-0.44489\\
1141	-0.44489\\
1142	-0.44489\\
1143	-0.44489\\
1144	-0.44489\\
1145	-0.44489\\
1146	-0.44489\\
1147	-0.44489\\
1148	-0.44489\\
1149	-0.44489\\
1150	-0.44489\\
1151	-0.44489\\
1152	-0.44489\\
1153	-0.44489\\
1154	-0.44489\\
1155	-0.44489\\
1156	-0.44489\\
1157	-0.44489\\
1158	-0.44489\\
1159	-0.44489\\
1160	-0.44489\\
1161	-0.44489\\
1162	-0.44489\\
1163	-0.44489\\
1164	-0.44489\\
1165	-0.44489\\
1166	-0.44489\\
1167	-0.44489\\
1168	-0.44489\\
1169	-0.44489\\
1170	-0.44489\\
1171	-0.44489\\
1172	-0.44489\\
1173	-0.44489\\
1174	-0.44489\\
1175	-0.44489\\
1176	-0.44489\\
1177	-0.44489\\
1178	-0.44489\\
1179	-0.44489\\
1180	-0.44489\\
1181	-0.44489\\
1182	-0.44489\\
1183	-0.44489\\
1184	-0.44489\\
1185	-0.44489\\
1186	-0.44489\\
1187	-0.44489\\
1188	-0.44489\\
1189	-0.44489\\
1190	-0.44489\\
1191	-0.44489\\
1192	-0.44489\\
1193	-0.44489\\
1194	-0.44489\\
1195	-0.44489\\
1196	-0.44489\\
1197	-0.44489\\
1198	-0.44489\\
1199	-0.44489\\
1200	-0.28486\\
1201	-0.26256\\
1202	-0.20436\\
1203	-0.1965\\
1204	-0.18346\\
1205	-0.19347\\
1206	-0.20587\\
1207	-0.22713\\
1208	-0.24739\\
1209	-0.26651\\
1210	-0.28014\\
1211	-0.2881\\
1212	-0.28969\\
1213	-0.28601\\
1214	-0.27817\\
1215	-0.2678\\
1216	-0.25635\\
1217	-0.24516\\
1218	-0.23526\\
1219	-0.22734\\
1220	-0.22171\\
1221	-0.21838\\
1222	-0.21708\\
1223	-0.21736\\
1224	-0.21865\\
1225	-0.22038\\
1226	-0.22206\\
1227	-0.2233\\
1228	-0.22387\\
1229	-0.22369\\
1230	-0.2228\\
1231	-0.22132\\
1232	-0.21944\\
1233	-0.21736\\
1234	-0.21528\\
1235	-0.21334\\
1236	-0.21165\\
1237	-0.21028\\
1238	-0.20923\\
1239	-0.20846\\
1240	-0.20794\\
1241	-0.20758\\
1242	-0.20731\\
1243	-0.20707\\
1244	-0.20681\\
1245	-0.20649\\
1246	-0.20611\\
1247	-0.20565\\
1248	-0.20515\\
1249	-0.20461\\
1250	-0.20405\\
1251	-0.20351\\
1252	-0.20299\\
1253	-0.20252\\
1254	-0.20209\\
1255	-0.20171\\
1256	-0.20138\\
1257	-0.20108\\
1258	-0.20082\\
1259	-0.20058\\
1260	-0.20035\\
1261	-0.20013\\
1262	-0.19991\\
1263	-0.19969\\
1264	-0.19948\\
1265	-0.19926\\
1266	-0.19905\\
1267	-0.19885\\
1268	-0.19865\\
1269	-0.19846\\
1270	-0.19828\\
1271	-0.19811\\
1272	-0.19796\\
1273	-0.19781\\
1274	-0.19767\\
1275	-0.19754\\
1276	-0.19742\\
1277	-0.19731\\
1278	-0.1972\\
1279	-0.19709\\
1280	-0.19699\\
1281	-0.19689\\
1282	-0.19679\\
1283	-0.1967\\
1284	-0.19661\\
1285	-0.19653\\
1286	-0.19645\\
1287	-0.19637\\
1288	-0.1963\\
1289	-0.19623\\
1290	-0.19617\\
1291	-0.1961\\
1292	-0.19604\\
1293	-0.19599\\
1294	-0.19593\\
1295	-0.19588\\
1296	-0.19583\\
1297	-0.19579\\
1298	-0.19574\\
1299	-0.1957\\
1300	-0.19566\\
1301	-0.19562\\
1302	-0.19558\\
1303	-0.19555\\
1304	-0.19551\\
1305	-0.19548\\
1306	-0.19545\\
1307	-0.19542\\
1308	-0.19539\\
1309	-0.19537\\
1310	-0.19534\\
1311	-0.19532\\
1312	-0.19529\\
1313	-0.19527\\
1314	-0.19525\\
1315	-0.19523\\
1316	-0.19521\\
1317	-0.19519\\
1318	-0.19518\\
1319	-0.19516\\
1320	-0.19514\\
1321	-0.19513\\
1322	-0.19511\\
1323	-0.1951\\
1324	-0.19509\\
1325	-0.19507\\
1326	-0.19506\\
1327	-0.19505\\
1328	-0.19504\\
1329	-0.19503\\
1330	-0.19502\\
1331	-0.19501\\
1332	-0.195\\
1333	-0.19499\\
1334	-0.19498\\
1335	-0.19498\\
1336	-0.19497\\
1337	-0.19496\\
1338	-0.19495\\
1339	-0.19495\\
1340	-0.19494\\
1341	-0.19493\\
1342	-0.19493\\
1343	-0.19492\\
1344	-0.19492\\
1345	-0.19491\\
1346	-0.19491\\
1347	-0.1949\\
1348	-0.1949\\
1349	-0.1949\\
1350	-0.19489\\
1351	-0.19489\\
1352	-0.19488\\
1353	-0.19488\\
1354	-0.19488\\
1355	-0.19487\\
1356	-0.19487\\
1357	-0.19487\\
1358	-0.19487\\
1359	-0.19486\\
1360	-0.19486\\
1361	-0.19486\\
1362	-0.19486\\
1363	-0.19485\\
1364	-0.19485\\
1365	-0.19485\\
1366	-0.19485\\
1367	-0.19485\\
1368	-0.19485\\
1369	-0.19484\\
1370	-0.19484\\
1371	-0.19484\\
1372	-0.19484\\
1373	-0.19484\\
1374	-0.19484\\
1375	-0.19484\\
1376	-0.19483\\
1377	-0.19483\\
1378	-0.19483\\
1379	-0.19483\\
1380	-0.19483\\
1381	-0.19483\\
1382	-0.19483\\
1383	-0.19483\\
1384	-0.19483\\
1385	-0.19483\\
1386	-0.19483\\
1387	-0.19482\\
1388	-0.19482\\
1389	-0.19482\\
1390	-0.19482\\
1391	-0.19482\\
1392	-0.19482\\
1393	-0.19482\\
1394	-0.19482\\
1395	-0.19482\\
1396	-0.19482\\
1397	-0.19482\\
1398	-0.19482\\
1399	-0.19482\\
1400	-0.14148\\
1401	-0.13404\\
1402	-0.11464\\
1403	-0.11202\\
1404	-0.10768\\
1405	-0.11002\\
1406	-0.11216\\
1407	-0.11575\\
1408	-0.11841\\
1409	-0.12022\\
1410	-0.12047\\
1411	-0.11927\\
1412	-0.11672\\
1413	-0.11322\\
1414	-0.10916\\
1415	-0.10495\\
1416	-0.1009\\
1417	-0.097245\\
1418	-0.094106\\
1419	-0.091508\\
1420	-0.089396\\
1421	-0.08767\\
1422	-0.086202\\
1423	-0.084869\\
1424	-0.083568\\
1425	-0.082223\\
1426	-0.080794\\
1427	-0.079268\\
1428	-0.077659\\
1429	-0.075949\\
1430	-0.07422\\
1431	-0.072507\\
1432	-0.068497\\
1433	-0.064598\\
1434	-0.060819\\
1435	-0.057163\\
1436	-0.053627\\
1437	-0.048148\\
1438	-0.042928\\
1439	-0.038008\\
1440	-0.031534\\
1441	-0.025481\\
1442	-0.019881\\
1443	-0.013219\\
1444	-0.0071939\\
1445	-0.00055192\\
1446	0.0063696\\
1447	0.012255\\
1448	0.017133\\
1449	0.020982\\
1450	0.023811\\
1451	0.025627\\
1452	0.026453\\
1453	0.026346\\
1454	0.025393\\
1455	0.023707\\
1456	0.021414\\
1457	0.018647\\
1458	0.01554\\
1459	0.012222\\
1460	0.0088155\\
1461	0.0054328\\
1462	0.0021744\\
1463	-0.00087143\\
1464	-0.0036299\\
1465	-0.0060401\\
1466	-0.0080555\\
1467	-0.0096439\\
1468	-0.010787\\
1469	-0.011483\\
1470	-0.011739\\
1471	-0.01158\\
1472	-0.011039\\
1473	-0.010162\\
1474	-0.0090009\\
1475	-0.0076149\\
1476	-0.0060672\\
1477	-0.0044227\\
1478	-0.0027451\\
1479	-0.0010951\\
1480	0.00047179\\
1481	0.0019072\\
1482	0.003171\\
1483	0.0042325\\
1484	0.0050705\\
1485	0.0056738\\
1486	0.0060404\\
1487	0.0061767\\
1488	0.0060967\\
1489	0.0058208\\
1490	0.0053742\\
1491	0.0047857\\
1492	0.0040865\\
1493	0.0033087\\
1494	0.0024843\\
1495	0.001644\\
1496	0.00081683\\
1497	2.8638e-05\\
1498	-0.00069783\\
1499	-0.0013436\\
1500	-0.001894\\
1501	-0.0023381\\
1502	-0.0026697\\
1503	-0.0028863\\
1504	-0.0029896\\
1505	-0.0029848\\
1506	-0.0028802\\
1507	-0.0026868\\
1508	-0.0024178\\
1509	-0.002088\\
1510	-0.0017129\\
1511	-0.0013087\\
1512	-0.00089121\\
1513	-0.00047553\\
1514	-7.5563e-05\\
1515	0.00029636\\
1516	0.00062983\\
1517	0.00091655\\
1518	0.0011505\\
1519	0.0013278\\
1520	0.0014469\\
1521	0.0015084\\
1522	0.0015147\\
1523	0.00147\\
1524	0.0013797\\
1525	0.0012504\\
1526	0.0010895\\
1527	0.00090493\\
1528	0.0007046\\
1529	0.00049645\\
1530	0.00028801\\
1531	8.6253e-05\\
1532	-0.00010264\\
1533	-0.00027339\\
1534	-0.00042171\\
1535	-0.00054442\\
1536	-0.00063941\\
1537	-0.00070566\\
1538	-0.00074318\\
1539	-0.00075293\\
1540	-0.00073672\\
1541	-0.00069711\\
1542	-0.0006372\\
1543	-0.00056058\\
1544	-0.00047109\\
1545	-0.00037271\\
1546	-0.00026945\\
1547	-0.00016512\\
1548	-6.331e-05\\
1549	3.2756e-05\\
1550	0.0001203\\
1551	0.00019706\\
1552	0.00026128\\
1553	0.00031179\\
1554	0.00034795\\
1555	0.00036968\\
1556	0.00037735\\
1557	0.00037182\\
1558	0.00035429\\
1559	0.00032627\\
1560	0.00028953\\
1561	0.00024596\\
1562	0.00019756\\
1563	0.00014632\\
1564	9.4163e-05\\
1565	4.2912e-05\\
1566	-5.7946e-06\\
1567	-5.0527e-05\\
1568	-9.01e-05\\
1569	-0.00012359\\
1570	-0.00015035\\
1571	-0.00017001\\
1572	-0.00018246\\
1573	-0.00018785\\
1574	-0.00018653\\
1575	-0.00017908\\
1576	-0.00016621\\
1577	-0.00014877\\
1578	-0.00012771\\
1579	-0.00010401\\
1580	-7.8657e-05\\
1581	-5.2638e-05\\
1582	-2.6871e-05\\
1583	-2.2007e-06\\
1584	2.0633e-05\\
1585	4.1009e-05\\
1586	5.8435e-05\\
1587	7.2556e-05\\
1588	8.3157e-05\\
1589	9.0154e-05\\
1590	9.3595e-05\\
1591	9.3638e-05\\
1592	9.0547e-05\\
1593	8.4667e-05\\
1594	7.6413e-05\\
1595	6.6243e-05\\
1596	5.4646e-05\\
1597	4.2121e-05\\
1598	2.9159e-05\\
1599	1.6227e-05\\
1600	0.020808\\
1601	0.040372\\
1602	0.059377\\
1603	0.077816\\
1604	0.095703\\
1605	0.11287\\
1606	0.12896\\
1607	0.14362\\
1608	0.15662\\
1609	0.16788\\
1610	0.17743\\
1611	0.18539\\
1612	0.19193\\
1613	0.19725\\
1614	0.20156\\
1615	0.20503\\
1616	0.20784\\
1617	0.21013\\
1618	0.212\\
1619	0.21355\\
1620	0.21485\\
1621	0.21596\\
1622	0.21692\\
1623	0.21777\\
1624	0.21852\\
1625	0.2192\\
1626	0.21982\\
1627	0.22039\\
1628	0.22092\\
1629	0.22142\\
1630	0.22189\\
1631	0.22234\\
1632	0.22276\\
1633	0.22316\\
1634	0.22354\\
1635	0.22391\\
1636	0.22426\\
1637	0.22459\\
1638	0.22491\\
1639	0.22522\\
1640	0.22551\\
1641	0.22579\\
1642	0.22606\\
1643	0.22632\\
1644	0.22657\\
1645	0.22681\\
1646	0.22704\\
1647	0.22726\\
1648	0.22747\\
1649	0.22768\\
1650	0.22787\\
1651	0.22806\\
1652	0.22824\\
1653	0.22842\\
1654	0.22858\\
1655	0.22874\\
1656	0.2289\\
1657	0.22905\\
1658	0.22919\\
1659	0.22933\\
1660	0.22946\\
1661	0.22959\\
1662	0.22971\\
1663	0.22983\\
1664	0.22994\\
1665	0.23005\\
1666	0.23016\\
1667	0.23026\\
1668	0.23035\\
1669	0.23045\\
1670	0.23054\\
1671	0.23062\\
1672	0.23071\\
1673	0.23079\\
1674	0.23087\\
1675	0.23094\\
1676	0.23101\\
1677	0.23108\\
1678	0.23115\\
1679	0.23121\\
1680	0.23127\\
1681	0.23133\\
1682	0.23139\\
1683	0.23144\\
1684	0.2315\\
1685	0.23155\\
1686	0.2316\\
1687	0.23165\\
1688	0.23169\\
1689	0.23173\\
1690	0.23178\\
1691	0.23182\\
1692	0.23186\\
1693	0.23189\\
1694	0.23193\\
1695	0.23197\\
1696	0.232\\
1697	0.23203\\
1698	0.23206\\
1699	0.23209\\
1700	0.23212\\
1701	0.23215\\
1702	0.23218\\
1703	0.2322\\
1704	0.23223\\
1705	0.23225\\
1706	0.23228\\
1707	0.2323\\
1708	0.23232\\
1709	0.23234\\
1710	0.23236\\
1711	0.23238\\
1712	0.2324\\
1713	0.23242\\
1714	0.23243\\
1715	0.23245\\
1716	0.23247\\
1717	0.23248\\
1718	0.2325\\
1719	0.23251\\
1720	0.23253\\
1721	0.23254\\
1722	0.23255\\
1723	0.23256\\
1724	0.23258\\
1725	0.23259\\
1726	0.2326\\
1727	0.23261\\
1728	0.23262\\
1729	0.23263\\
1730	0.23264\\
1731	0.23265\\
1732	0.23266\\
1733	0.23267\\
1734	0.23267\\
1735	0.23268\\
1736	0.23269\\
1737	0.2327\\
1738	0.2327\\
1739	0.23271\\
1740	0.23272\\
1741	0.23272\\
1742	0.23273\\
1743	0.23274\\
1744	0.23274\\
1745	0.23275\\
1746	0.23275\\
1747	0.23276\\
1748	0.23276\\
1749	0.23277\\
1750	0.23277\\
1751	0.23278\\
1752	0.23278\\
1753	0.23278\\
1754	0.23279\\
1755	0.23279\\
1756	0.2328\\
1757	0.2328\\
1758	0.2328\\
1759	0.23281\\
1760	0.23281\\
1761	0.23281\\
1762	0.23281\\
1763	0.23282\\
1764	0.23282\\
1765	0.23282\\
1766	0.23283\\
1767	0.23283\\
1768	0.23283\\
1769	0.23283\\
1770	0.23283\\
1771	0.23284\\
1772	0.23284\\
1773	0.23284\\
1774	0.23284\\
1775	0.23284\\
1776	0.23285\\
1777	0.23285\\
1778	0.23285\\
1779	0.23285\\
1780	0.23285\\
1781	0.23285\\
1782	0.23286\\
1783	0.23286\\
1784	0.23286\\
1785	0.23286\\
1786	0.23286\\
1787	0.23286\\
1788	0.23286\\
1789	0.23286\\
1790	0.23286\\
1791	0.23287\\
1792	0.23287\\
1793	0.23287\\
1794	0.23287\\
1795	0.23287\\
1796	0.23287\\
1797	0.23287\\
1798	0.23287\\
1799	0.23287\\
1800	0.23287\\
};
\addlegendentry{Sterowanie u}

\end{axis}
\end{tikzpicture}%
   \caption{Trzy regulatory lokalne DMC}
   \label{projekt:zad7:DMC:3:figure}
\end{figure}

\begin{figure}[H] 
   \centering
   % This file was created by matlab2tikz.
%
\definecolor{mycolor1}{rgb}{0.00000,0.44700,0.74100}%
\definecolor{mycolor2}{rgb}{0.85000,0.32500,0.09800}%
%
\begin{tikzpicture}

\begin{axis}[%
width=4.521in,
height=1.493in,
at={(0.758in,2.554in)},
scale only axis,
xmin=1,
xmax=1800,
xlabel style={font=\color{white!15!black}},
xlabel={k},
ymin=-4.5167,
ymax=0.10028,
ylabel style={font=\color{white!15!black}},
ylabel={y},
axis background/.style={fill=white},
title style={font=\bfseries, align=center},
title={E=219.8135\\[1ex]N= [70         50         25         25]\\[1ex]$\text{N}_\text{u}\text{= [5          7         10         10]}$\\[1ex]lambda= [18         10         10          1]},
xmajorgrids,
ymajorgrids,
legend style={legend cell align=left, align=left, draw=white!15!black}
]
\addplot [color=mycolor1]
  table[row sep=crcr]{%
1	0\\
2	0\\
3	0\\
4	0\\
5	0\\
6	0\\
7	0\\
8	0\\
9	0\\
10	0\\
11	0\\
12	0\\
13	0\\
14	0\\
15	0\\
16	0\\
17	0\\
18	0\\
19	0\\
20	0\\
21	0\\
22	0\\
23	0\\
24	0\\
25	-0.041775\\
26	-0.19537\\
27	-0.4731\\
28	-0.83396\\
29	-1.2324\\
30	-1.6377\\
31	-2.0301\\
32	-2.386\\
33	-2.685\\
34	-2.9122\\
35	-3.0597\\
36	-3.1252\\
37	-3.1132\\
38	-3.0333\\
39	-2.9005\\
40	-2.733\\
41	-2.5499\\
42	-2.3692\\
43	-2.2056\\
44	-2.0698\\
45	-1.968\\
46	-1.9024\\
47	-1.8716\\
48	-1.8715\\
49	-1.8961\\
50	-1.9382\\
51	-1.9902\\
52	-2.0448\\
53	-2.0957\\
54	-2.1376\\
55	-2.1672\\
56	-2.1828\\
57	-2.1842\\
58	-2.1727\\
59	-2.1509\\
60	-2.1218\\
61	-2.0889\\
62	-2.0555\\
63	-2.0245\\
64	-1.9982\\
65	-1.9781\\
66	-1.9649\\
67	-1.9586\\
68	-1.9585\\
69	-1.9636\\
70	-1.9724\\
71	-1.9835\\
72	-1.9954\\
73	-2.0066\\
74	-2.0162\\
75	-2.0235\\
76	-2.028\\
77	-2.0296\\
78	-2.0286\\
79	-2.0254\\
80	-2.0207\\
81	-2.0149\\
82	-2.0089\\
83	-2.0032\\
84	-1.9982\\
85	-1.9943\\
86	-1.9917\\
87	-1.9904\\
88	-1.9903\\
89	-1.9912\\
90	-1.993\\
91	-1.9952\\
92	-1.9976\\
93	-1.9999\\
94	-2.0019\\
95	-2.0035\\
96	-2.0046\\
97	-2.0051\\
98	-2.0051\\
99	-2.0046\\
100	-2.0038\\
101	-2.0028\\
102	-2.0017\\
103	-2.0006\\
104	-1.9996\\
105	-1.9989\\
106	-1.9983\\
107	-1.998\\
108	-1.998\\
109	-1.9981\\
110	-1.9985\\
111	-1.9989\\
112	-1.9993\\
113	-1.9998\\
114	-2.0002\\
115	-2.0006\\
116	-2.0008\\
117	-2.0009\\
118	-2.001\\
119	-2.0009\\
120	-2.0007\\
121	-2.0006\\
122	-2.0004\\
123	-2.0001\\
124	-2\\
125	-1.9998\\
126	-1.9997\\
127	-1.9996\\
128	-1.9996\\
129	-1.9996\\
130	-1.9997\\
131	-1.9998\\
132	-1.9998\\
133	-1.9999\\
134	-2\\
135	-2.0001\\
136	-2.0001\\
137	-2.0002\\
138	-2.0002\\
139	-2.0002\\
140	-2.0001\\
141	-2.0001\\
142	-2.0001\\
143	-2\\
144	-2\\
145	-2\\
146	-1.9999\\
147	-1.9999\\
148	-1.9999\\
149	-1.9999\\
150	-1.9999\\
151	-1.9999\\
152	-2\\
153	-2\\
154	-2\\
155	-2\\
156	-2\\
157	-2\\
158	-2\\
159	-2\\
160	-2\\
161	-2\\
162	-2\\
163	-2\\
164	-2\\
165	-2\\
166	-2\\
167	-2\\
168	-2\\
169	-2\\
170	-2\\
171	-2\\
172	-2\\
173	-2\\
174	-2\\
175	-2\\
176	-2\\
177	-2\\
178	-2\\
179	-2\\
180	-2\\
181	-2\\
182	-2\\
183	-2\\
184	-2\\
185	-2\\
186	-2\\
187	-2\\
188	-2\\
189	-2\\
190	-2\\
191	-2\\
192	-2\\
193	-2\\
194	-2\\
195	-2\\
196	-2\\
197	-2\\
198	-2\\
199	-2\\
200	-2\\
201	-2\\
202	-2\\
203	-2\\
204	-2\\
205	-2.0801\\
206	-2.2463\\
207	-2.4603\\
208	-2.6922\\
209	-2.9273\\
210	-3.1462\\
211	-3.3363\\
212	-3.4867\\
213	-3.5946\\
214	-3.6624\\
215	-3.6978\\
216	-3.7109\\
217	-3.7123\\
218	-3.712\\
219	-3.7181\\
220	-3.7361\\
221	-3.7694\\
222	-3.8188\\
223	-3.8833\\
224	-3.9601\\
225	-4.0437\\
226	-4.1281\\
227	-4.2096\\
228	-4.285\\
229	-4.3517\\
230	-4.4075\\
231	-4.4512\\
232	-4.4825\\
233	-4.5021\\
234	-4.5113\\
235	-4.5119\\
236	-4.5061\\
237	-4.496\\
238	-4.4838\\
239	-4.4714\\
240	-4.4601\\
241	-4.4513\\
242	-4.4456\\
243	-4.4433\\
244	-4.4443\\
245	-4.4484\\
246	-4.455\\
247	-4.4633\\
248	-4.4727\\
249	-4.4823\\
250	-4.4915\\
251	-4.4998\\
252	-4.5066\\
253	-4.5117\\
254	-4.5151\\
255	-4.5167\\
256	-4.5167\\
257	-4.5154\\
258	-4.513\\
259	-4.5099\\
260	-4.5065\\
261	-4.503\\
262	-4.4998\\
263	-4.4971\\
264	-4.4949\\
265	-4.4934\\
266	-4.4927\\
267	-4.4926\\
268	-4.493\\
269	-4.494\\
270	-4.4952\\
271	-4.4967\\
272	-4.4981\\
273	-4.4995\\
274	-4.5008\\
275	-4.5018\\
276	-4.5025\\
277	-4.5029\\
278	-4.5031\\
279	-4.503\\
280	-4.5026\\
281	-4.5022\\
282	-4.5016\\
283	-4.501\\
284	-4.5004\\
285	-4.4998\\
286	-4.4994\\
287	-4.4991\\
288	-4.4988\\
289	-4.4987\\
290	-4.4987\\
291	-4.4988\\
292	-4.499\\
293	-4.4993\\
294	-4.4995\\
295	-4.4998\\
296	-4.5\\
297	-4.5002\\
298	-4.5003\\
299	-4.5005\\
300	-4.5005\\
301	-4.5005\\
302	-4.5005\\
303	-4.5004\\
304	-4.5003\\
305	-4.5002\\
306	-4.5001\\
307	-4.5\\
308	-4.4999\\
309	-4.4999\\
310	-4.4998\\
311	-4.4998\\
312	-4.4998\\
313	-4.4998\\
314	-4.4998\\
315	-4.4998\\
316	-4.4999\\
317	-4.4999\\
318	-4.5\\
319	-4.5\\
320	-4.5\\
321	-4.5001\\
322	-4.5001\\
323	-4.5001\\
324	-4.5001\\
325	-4.5001\\
326	-4.5001\\
327	-4.5001\\
328	-4.5\\
329	-4.5\\
330	-4.5\\
331	-4.5\\
332	-4.5\\
333	-4.5\\
334	-4.5\\
335	-4.5\\
336	-4.5\\
337	-4.5\\
338	-4.5\\
339	-4.5\\
340	-4.5\\
341	-4.5\\
342	-4.5\\
343	-4.5\\
344	-4.5\\
345	-4.5\\
346	-4.5\\
347	-4.5\\
348	-4.5\\
349	-4.5\\
350	-4.5\\
351	-4.5\\
352	-4.5\\
353	-4.5\\
354	-4.5\\
355	-4.5\\
356	-4.5\\
357	-4.5\\
358	-4.5\\
359	-4.5\\
360	-4.5\\
361	-4.5\\
362	-4.5\\
363	-4.5\\
364	-4.5\\
365	-4.5\\
366	-4.5\\
367	-4.5\\
368	-4.5\\
369	-4.5\\
370	-4.5\\
371	-4.5\\
372	-4.5\\
373	-4.5\\
374	-4.5\\
375	-4.5\\
376	-4.5\\
377	-4.5\\
378	-4.5\\
379	-4.5\\
380	-4.5\\
381	-4.5\\
382	-4.5\\
383	-4.5\\
384	-4.5\\
385	-4.5\\
386	-4.5\\
387	-4.5\\
388	-4.5\\
389	-4.5\\
390	-4.5\\
391	-4.5\\
392	-4.5\\
393	-4.5\\
394	-4.5\\
395	-4.5\\
396	-4.5\\
397	-4.5\\
398	-4.5\\
399	-4.5\\
400	-4.5\\
401	-4.5\\
402	-4.5\\
403	-4.5\\
404	-4.5\\
405	-4.4111\\
406	-4.2403\\
407	-3.9844\\
408	-3.687\\
409	-3.3743\\
410	-3.0818\\
411	-2.8304\\
412	-2.6366\\
413	-2.507\\
414	-2.4427\\
415	-2.4387\\
416	-2.4865\\
417	-2.5744\\
418	-2.6895\\
419	-2.8183\\
420	-2.9481\\
421	-3.0679\\
422	-3.1688\\
423	-3.2448\\
424	-3.2926\\
425	-3.3119\\
426	-3.3048\\
427	-3.2753\\
428	-3.2289\\
429	-3.1718\\
430	-3.1103\\
431	-3.0502\\
432	-2.9963\\
433	-2.952\\
434	-2.9196\\
435	-2.9\\
436	-2.8928\\
437	-2.8966\\
438	-2.9095\\
439	-2.9289\\
440	-2.9522\\
441	-2.9768\\
442	-3.0003\\
443	-3.0209\\
444	-3.0371\\
445	-3.0482\\
446	-3.0539\\
447	-3.0545\\
448	-3.0506\\
449	-3.0432\\
450	-3.0334\\
451	-3.0223\\
452	-3.0111\\
453	-3.0007\\
454	-2.9918\\
455	-2.9851\\
456	-2.9807\\
457	-2.9786\\
458	-2.9787\\
459	-2.9807\\
460	-2.9841\\
461	-2.9884\\
462	-2.993\\
463	-2.9977\\
464	-3.0019\\
465	-3.0053\\
466	-3.0078\\
467	-3.0093\\
468	-3.0098\\
469	-3.0095\\
470	-3.0083\\
471	-3.0067\\
472	-3.0047\\
473	-3.0026\\
474	-3.0006\\
475	-2.9989\\
476	-2.9975\\
477	-2.9965\\
478	-2.996\\
479	-2.9959\\
480	-2.9961\\
481	-2.9967\\
482	-2.9974\\
483	-2.9983\\
484	-2.9992\\
485	-3\\
486	-3.0007\\
487	-3.0013\\
488	-3.0016\\
489	-3.0018\\
490	-3.0018\\
491	-3.0016\\
492	-3.0014\\
493	-3.001\\
494	-3.0006\\
495	-3.0002\\
496	-2.9999\\
497	-2.9996\\
498	-2.9994\\
499	-2.9993\\
500	-2.9992\\
501	-2.9992\\
502	-2.9993\\
503	-2.9995\\
504	-2.9996\\
505	-2.9998\\
506	-2.9999\\
507	-3.0001\\
508	-3.0002\\
509	-3.0003\\
510	-3.0003\\
511	-3.0003\\
512	-3.0003\\
513	-3.0003\\
514	-3.0002\\
515	-3.0001\\
516	-3.0001\\
517	-3\\
518	-2.9999\\
519	-2.9999\\
520	-2.9999\\
521	-2.9999\\
522	-2.9999\\
523	-2.9999\\
524	-2.9999\\
525	-2.9999\\
526	-2.9999\\
527	-3\\
528	-3\\
529	-3\\
530	-3\\
531	-3.0001\\
532	-3.0001\\
533	-3.0001\\
534	-3.0001\\
535	-3\\
536	-3\\
537	-3\\
538	-3\\
539	-3\\
540	-3\\
541	-3\\
542	-3\\
543	-3\\
544	-3\\
545	-3\\
546	-3\\
547	-3\\
548	-3\\
549	-3\\
550	-3\\
551	-3\\
552	-3\\
553	-3\\
554	-3\\
555	-3\\
556	-3\\
557	-3\\
558	-3\\
559	-3\\
560	-3\\
561	-3\\
562	-3\\
563	-3\\
564	-3\\
565	-3\\
566	-3\\
567	-3\\
568	-3\\
569	-3\\
570	-3\\
571	-3\\
572	-3\\
573	-3\\
574	-3\\
575	-3\\
576	-3\\
577	-3\\
578	-3\\
579	-3\\
580	-3\\
581	-3\\
582	-3\\
583	-3\\
584	-3\\
585	-3\\
586	-3\\
587	-3\\
588	-3\\
589	-3\\
590	-3\\
591	-3\\
592	-3\\
593	-3\\
594	-3\\
595	-3\\
596	-3\\
597	-3\\
598	-3\\
599	-3\\
600	-3\\
601	-3\\
602	-3\\
603	-3\\
604	-3\\
605	-2.9023\\
606	-2.7233\\
607	-2.4613\\
608	-2.1692\\
609	-1.8758\\
610	-1.6157\\
611	-1.4059\\
612	-1.2574\\
613	-1.1716\\
614	-1.1453\\
615	-1.1702\\
616	-1.2353\\
617	-1.3275\\
618	-1.4333\\
619	-1.5397\\
620	-1.6355\\
621	-1.712\\
622	-1.7637\\
623	-1.7882\\
624	-1.7865\\
625	-1.762\\
626	-1.7204\\
627	-1.6679\\
628	-1.6115\\
629	-1.5569\\
630	-1.5092\\
631	-1.4716\\
632	-1.4461\\
633	-1.4328\\
634	-1.431\\
635	-1.4386\\
636	-1.4534\\
637	-1.4725\\
638	-1.493\\
639	-1.5127\\
640	-1.5294\\
641	-1.5419\\
642	-1.5493\\
643	-1.5516\\
644	-1.5493\\
645	-1.5432\\
646	-1.5344\\
647	-1.5242\\
648	-1.5138\\
649	-1.5042\\
650	-1.4961\\
651	-1.4902\\
652	-1.4866\\
653	-1.4852\\
654	-1.4859\\
655	-1.4882\\
656	-1.4916\\
657	-1.4955\\
658	-1.4995\\
659	-1.503\\
660	-1.5059\\
661	-1.5079\\
662	-1.5089\\
663	-1.509\\
664	-1.5082\\
665	-1.5069\\
666	-1.5051\\
667	-1.5032\\
668	-1.5013\\
669	-1.4996\\
670	-1.4983\\
671	-1.4975\\
672	-1.497\\
673	-1.497\\
674	-1.4973\\
675	-1.4979\\
676	-1.4986\\
677	-1.4994\\
678	-1.5001\\
679	-1.5007\\
680	-1.5012\\
681	-1.5015\\
682	-1.5016\\
683	-1.5016\\
684	-1.5014\\
685	-1.5011\\
686	-1.5007\\
687	-1.5004\\
688	-1.5\\
689	-1.4998\\
690	-1.4996\\
691	-1.4994\\
692	-1.4994\\
693	-1.4994\\
694	-1.4995\\
695	-1.4996\\
696	-1.4998\\
697	-1.4999\\
698	-1.5001\\
699	-1.5002\\
700	-1.5002\\
701	-1.5003\\
702	-1.5003\\
703	-1.5003\\
704	-1.5002\\
705	-1.5002\\
706	-1.5001\\
707	-1.5\\
708	-1.5\\
709	-1.4999\\
710	-1.4999\\
711	-1.4999\\
712	-1.4999\\
713	-1.4999\\
714	-1.4999\\
715	-1.4999\\
716	-1.5\\
717	-1.5\\
718	-1.5\\
719	-1.5\\
720	-1.5001\\
721	-1.5001\\
722	-1.5001\\
723	-1.5\\
724	-1.5\\
725	-1.5\\
726	-1.5\\
727	-1.5\\
728	-1.5\\
729	-1.5\\
730	-1.5\\
731	-1.5\\
732	-1.5\\
733	-1.5\\
734	-1.5\\
735	-1.5\\
736	-1.5\\
737	-1.5\\
738	-1.5\\
739	-1.5\\
740	-1.5\\
741	-1.5\\
742	-1.5\\
743	-1.5\\
744	-1.5\\
745	-1.5\\
746	-1.5\\
747	-1.5\\
748	-1.5\\
749	-1.5\\
750	-1.5\\
751	-1.5\\
752	-1.5\\
753	-1.5\\
754	-1.5\\
755	-1.5\\
756	-1.5\\
757	-1.5\\
758	-1.5\\
759	-1.5\\
760	-1.5\\
761	-1.5\\
762	-1.5\\
763	-1.5\\
764	-1.5\\
765	-1.5\\
766	-1.5\\
767	-1.5\\
768	-1.5\\
769	-1.5\\
770	-1.5\\
771	-1.5\\
772	-1.5\\
773	-1.5\\
774	-1.5\\
775	-1.5\\
776	-1.5\\
777	-1.5\\
778	-1.5\\
779	-1.5\\
780	-1.5\\
781	-1.5\\
782	-1.5\\
783	-1.5\\
784	-1.5\\
785	-1.5\\
786	-1.5\\
787	-1.5\\
788	-1.5\\
789	-1.5\\
790	-1.5\\
791	-1.5\\
792	-1.5\\
793	-1.5\\
794	-1.5\\
795	-1.5\\
796	-1.5\\
797	-1.5\\
798	-1.5\\
799	-1.5\\
800	-1.5\\
801	-1.5\\
802	-1.5\\
803	-1.5\\
804	-1.5\\
805	-1.5807\\
806	-1.7509\\
807	-1.9826\\
808	-2.247\\
809	-2.5222\\
810	-2.7856\\
811	-3.021\\
812	-3.2141\\
813	-3.3573\\
814	-3.4481\\
815	-3.4899\\
816	-3.4898\\
817	-3.4578\\
818	-3.4052\\
819	-3.3432\\
820	-3.2817\\
821	-3.2287\\
822	-3.1899\\
823	-3.1687\\
824	-3.1662\\
825	-3.1815\\
826	-3.2123\\
827	-3.255\\
828	-3.3056\\
829	-3.3598\\
830	-3.4134\\
831	-3.4629\\
832	-3.5055\\
833	-3.5391\\
834	-3.5627\\
835	-3.5762\\
836	-3.5802\\
837	-3.5759\\
838	-3.5651\\
839	-3.5498\\
840	-3.5321\\
841	-3.5137\\
842	-3.4966\\
843	-3.4818\\
844	-3.4704\\
845	-3.4629\\
846	-3.4593\\
847	-3.4594\\
848	-3.4627\\
849	-3.4685\\
850	-3.4759\\
851	-3.4842\\
852	-3.4926\\
853	-3.5003\\
854	-3.5068\\
855	-3.5119\\
856	-3.5152\\
857	-3.5168\\
858	-3.5168\\
859	-3.5154\\
860	-3.513\\
861	-3.5098\\
862	-3.5063\\
863	-3.5028\\
864	-3.4996\\
865	-3.4968\\
866	-3.4947\\
867	-3.4934\\
868	-3.4928\\
869	-3.4928\\
870	-3.4935\\
871	-3.4946\\
872	-3.4959\\
873	-3.4974\\
874	-3.4989\\
875	-3.5003\\
876	-3.5014\\
877	-3.5023\\
878	-3.5029\\
879	-3.5031\\
880	-3.5031\\
881	-3.5028\\
882	-3.5023\\
883	-3.5017\\
884	-3.5011\\
885	-3.5004\\
886	-3.4998\\
887	-3.4994\\
888	-3.499\\
889	-3.4988\\
890	-3.4987\\
891	-3.4987\\
892	-3.4988\\
893	-3.499\\
894	-3.4993\\
895	-3.4996\\
896	-3.4998\\
897	-3.5001\\
898	-3.5003\\
899	-3.5004\\
900	-3.5005\\
901	-3.5006\\
902	-3.5006\\
903	-3.5005\\
904	-3.5004\\
905	-3.5003\\
906	-3.5002\\
907	-3.5001\\
908	-3.5\\
909	-3.4999\\
910	-3.4998\\
911	-3.4998\\
912	-3.4998\\
913	-3.4998\\
914	-3.4998\\
915	-3.4998\\
916	-3.4999\\
917	-3.4999\\
918	-3.5\\
919	-3.5\\
920	-3.5001\\
921	-3.5001\\
922	-3.5001\\
923	-3.5001\\
924	-3.5001\\
925	-3.5001\\
926	-3.5001\\
927	-3.5001\\
928	-3.5\\
929	-3.5\\
930	-3.5\\
931	-3.5\\
932	-3.5\\
933	-3.5\\
934	-3.5\\
935	-3.5\\
936	-3.5\\
937	-3.5\\
938	-3.5\\
939	-3.5\\
940	-3.5\\
941	-3.5\\
942	-3.5\\
943	-3.5\\
944	-3.5\\
945	-3.5\\
946	-3.5\\
947	-3.5\\
948	-3.5\\
949	-3.5\\
950	-3.5\\
951	-3.5\\
952	-3.5\\
953	-3.5\\
954	-3.5\\
955	-3.5\\
956	-3.5\\
957	-3.5\\
958	-3.5\\
959	-3.5\\
960	-3.5\\
961	-3.5\\
962	-3.5\\
963	-3.5\\
964	-3.5\\
965	-3.5\\
966	-3.5\\
967	-3.5\\
968	-3.5\\
969	-3.5\\
970	-3.5\\
971	-3.5\\
972	-3.5\\
973	-3.5\\
974	-3.5\\
975	-3.5\\
976	-3.5\\
977	-3.5\\
978	-3.5\\
979	-3.5\\
980	-3.5\\
981	-3.5\\
982	-3.5\\
983	-3.5\\
984	-3.5\\
985	-3.5\\
986	-3.5\\
987	-3.5\\
988	-3.5\\
989	-3.5\\
990	-3.5\\
991	-3.5\\
992	-3.5\\
993	-3.5\\
994	-3.5\\
995	-3.5\\
996	-3.5\\
997	-3.5\\
998	-3.5\\
999	-3.5\\
1000	-3.5\\
1001	-3.5\\
1002	-3.5\\
1003	-3.5\\
1004	-3.5\\
1005	-3.3259\\
1006	-3.0198\\
1007	-2.5701\\
1008	-2.0918\\
1009	-1.6373\\
1010	-1.2592\\
1011	-0.97404\\
1012	-0.78809\\
1013	-0.69716\\
1014	-0.69216\\
1015	-0.75873\\
1016	-0.87874\\
1017	-1.0316\\
1018	-1.1963\\
1019	-1.3535\\
1020	-1.4871\\
1021	-1.5854\\
1022	-1.642\\
1023	-1.6556\\
1024	-1.6302\\
1025	-1.5737\\
1026	-1.4966\\
1027	-1.41\\
1028	-1.3243\\
1029	-1.2478\\
1030	-1.1864\\
1031	-1.1433\\
1032	-1.1194\\
1033	-1.1133\\
1034	-1.1224\\
1035	-1.1427\\
1036	-1.1701\\
1037	-1.2\\
1038	-1.2287\\
1039	-1.253\\
1040	-1.2707\\
1041	-1.2807\\
1042	-1.283\\
1043	-1.2783\\
1044	-1.2681\\
1045	-1.2541\\
1046	-1.2383\\
1047	-1.2226\\
1048	-1.2085\\
1049	-1.197\\
1050	-1.1889\\
1051	-1.1843\\
1052	-1.1831\\
1053	-1.1847\\
1054	-1.1885\\
1055	-1.1935\\
1056	-1.199\\
1057	-1.2043\\
1058	-1.2088\\
1059	-1.2121\\
1060	-1.214\\
1061	-1.2145\\
1062	-1.2137\\
1063	-1.2118\\
1064	-1.2093\\
1065	-1.2065\\
1066	-1.2037\\
1067	-1.2012\\
1068	-1.1991\\
1069	-1.1977\\
1070	-1.1969\\
1071	-1.1967\\
1072	-1.197\\
1073	-1.1977\\
1074	-1.1986\\
1075	-1.1996\\
1076	-1.2006\\
1077	-1.2015\\
1078	-1.2021\\
1079	-1.2024\\
1080	-1.2025\\
1081	-1.2024\\
1082	-1.2021\\
1083	-1.2016\\
1084	-1.2011\\
1085	-1.2006\\
1086	-1.2001\\
1087	-1.1998\\
1088	-1.1995\\
1089	-1.1994\\
1090	-1.1993\\
1091	-1.1994\\
1092	-1.1995\\
1093	-1.1997\\
1094	-1.1999\\
1095	-1.2001\\
1096	-1.2002\\
1097	-1.2004\\
1098	-1.2004\\
1099	-1.2004\\
1100	-1.2004\\
1101	-1.2004\\
1102	-1.2003\\
1103	-1.2002\\
1104	-1.2001\\
1105	-1.2\\
1106	-1.2\\
1107	-1.1999\\
1108	-1.1999\\
1109	-1.1999\\
1110	-1.1999\\
1111	-1.1999\\
1112	-1.1999\\
1113	-1.2\\
1114	-1.2\\
1115	-1.2\\
1116	-1.2001\\
1117	-1.2001\\
1118	-1.2001\\
1119	-1.2001\\
1120	-1.2001\\
1121	-1.2\\
1122	-1.2\\
1123	-1.2\\
1124	-1.2\\
1125	-1.2\\
1126	-1.2\\
1127	-1.2\\
1128	-1.2\\
1129	-1.2\\
1130	-1.2\\
1131	-1.2\\
1132	-1.2\\
1133	-1.2\\
1134	-1.2\\
1135	-1.2\\
1136	-1.2\\
1137	-1.2\\
1138	-1.2\\
1139	-1.2\\
1140	-1.2\\
1141	-1.2\\
1142	-1.2\\
1143	-1.2\\
1144	-1.2\\
1145	-1.2\\
1146	-1.2\\
1147	-1.2\\
1148	-1.2\\
1149	-1.2\\
1150	-1.2\\
1151	-1.2\\
1152	-1.2\\
1153	-1.2\\
1154	-1.2\\
1155	-1.2\\
1156	-1.2\\
1157	-1.2\\
1158	-1.2\\
1159	-1.2\\
1160	-1.2\\
1161	-1.2\\
1162	-1.2\\
1163	-1.2\\
1164	-1.2\\
1165	-1.2\\
1166	-1.2\\
1167	-1.2\\
1168	-1.2\\
1169	-1.2\\
1170	-1.2\\
1171	-1.2\\
1172	-1.2\\
1173	-1.2\\
1174	-1.2\\
1175	-1.2\\
1176	-1.2\\
1177	-1.2\\
1178	-1.2\\
1179	-1.2\\
1180	-1.2\\
1181	-1.2\\
1182	-1.2\\
1183	-1.2\\
1184	-1.2\\
1185	-1.2\\
1186	-1.2\\
1187	-1.2\\
1188	-1.2\\
1189	-1.2\\
1190	-1.2\\
1191	-1.2\\
1192	-1.2\\
1193	-1.2\\
1194	-1.2\\
1195	-1.2\\
1196	-1.2\\
1197	-1.2\\
1198	-1.2\\
1199	-1.2\\
1200	-1.2\\
1201	-1.2\\
1202	-1.2\\
1203	-1.2\\
1204	-1.2\\
1205	-1.1481\\
1206	-1.0567\\
1207	-0.93281\\
1208	-0.80149\\
1209	-0.67678\\
1210	-0.57151\\
1211	-0.49105\\
1212	-0.43758\\
1213	-0.40945\\
1214	-0.403\\
1215	-0.41277\\
1216	-0.4327\\
1217	-0.45658\\
1218	-0.47902\\
1219	-0.49587\\
1220	-0.50465\\
1221	-0.50456\\
1222	-0.4963\\
1223	-0.48164\\
1224	-0.46293\\
1225	-0.44263\\
1226	-0.42291\\
1227	-0.40544\\
1228	-0.39125\\
1229	-0.38076\\
1230	-0.37386\\
1231	-0.37003\\
1232	-0.36848\\
1233	-0.36834\\
1234	-0.36878\\
1235	-0.36908\\
1236	-0.36875\\
1237	-0.3675\\
1238	-0.36528\\
1239	-0.36219\\
1240	-0.35846\\
1241	-0.35439\\
1242	-0.35026\\
1243	-0.34632\\
1244	-0.34277\\
1245	-0.33971\\
1246	-0.33717\\
1247	-0.33511\\
1248	-0.33346\\
1249	-0.33211\\
1250	-0.33094\\
1251	-0.32986\\
1252	-0.32879\\
1253	-0.32768\\
1254	-0.3265\\
1255	-0.32525\\
1256	-0.32397\\
1257	-0.32267\\
1258	-0.32139\\
1259	-0.32017\\
1260	-0.31903\\
1261	-0.31797\\
1262	-0.31701\\
1263	-0.31615\\
1264	-0.31536\\
1265	-0.31464\\
1266	-0.31397\\
1267	-0.31334\\
1268	-0.31274\\
1269	-0.31215\\
1270	-0.31157\\
1271	-0.31099\\
1272	-0.31042\\
1273	-0.30984\\
1274	-0.30928\\
1275	-0.30873\\
1276	-0.3082\\
1277	-0.30769\\
1278	-0.3072\\
1279	-0.30675\\
1280	-0.30632\\
1281	-0.30592\\
1282	-0.30554\\
1283	-0.30518\\
1284	-0.30484\\
1285	-0.30452\\
1286	-0.30422\\
1287	-0.30393\\
1288	-0.30365\\
1289	-0.30338\\
1290	-0.30315\\
1291	-0.30293\\
1292	-0.30275\\
1293	-0.30258\\
1294	-0.30244\\
1295	-0.30231\\
1296	-0.3022\\
1297	-0.3021\\
1298	-0.302\\
1299	-0.30191\\
1300	-0.30183\\
1301	-0.30174\\
1302	-0.30166\\
1303	-0.30158\\
1304	-0.3015\\
1305	-0.30143\\
1306	-0.30135\\
1307	-0.30128\\
1308	-0.30122\\
1309	-0.30116\\
1310	-0.3011\\
1311	-0.30105\\
1312	-0.301\\
1313	-0.30095\\
1314	-0.30091\\
1315	-0.30086\\
1316	-0.30082\\
1317	-0.30078\\
1318	-0.30075\\
1319	-0.30071\\
1320	-0.30067\\
1321	-0.30064\\
1322	-0.30061\\
1323	-0.30058\\
1324	-0.30055\\
1325	-0.30052\\
1326	-0.3005\\
1327	-0.30048\\
1328	-0.30045\\
1329	-0.30043\\
1330	-0.30041\\
1331	-0.30039\\
1332	-0.30037\\
1333	-0.30035\\
1334	-0.30034\\
1335	-0.30032\\
1336	-0.3003\\
1337	-0.30029\\
1338	-0.30028\\
1339	-0.30026\\
1340	-0.30025\\
1341	-0.30024\\
1342	-0.30023\\
1343	-0.30022\\
1344	-0.3002\\
1345	-0.30019\\
1346	-0.30019\\
1347	-0.30018\\
1348	-0.30017\\
1349	-0.30016\\
1350	-0.30015\\
1351	-0.30014\\
1352	-0.30014\\
1353	-0.30013\\
1354	-0.30012\\
1355	-0.30012\\
1356	-0.30011\\
1357	-0.30011\\
1358	-0.3001\\
1359	-0.3001\\
1360	-0.30009\\
1361	-0.30009\\
1362	-0.30008\\
1363	-0.30008\\
1364	-0.30008\\
1365	-0.30007\\
1366	-0.30007\\
1367	-0.30007\\
1368	-0.30006\\
1369	-0.30006\\
1370	-0.30006\\
1371	-0.30005\\
1372	-0.30005\\
1373	-0.30005\\
1374	-0.30005\\
1375	-0.30004\\
1376	-0.30004\\
1377	-0.30004\\
1378	-0.30004\\
1379	-0.30004\\
1380	-0.30003\\
1381	-0.30003\\
1382	-0.30003\\
1383	-0.30003\\
1384	-0.30003\\
1385	-0.30003\\
1386	-0.30003\\
1387	-0.30002\\
1388	-0.30002\\
1389	-0.30002\\
1390	-0.30002\\
1391	-0.30002\\
1392	-0.30002\\
1393	-0.30002\\
1394	-0.30002\\
1395	-0.30002\\
1396	-0.30002\\
1397	-0.30001\\
1398	-0.30001\\
1399	-0.30001\\
1400	-0.30001\\
1401	-0.30001\\
1402	-0.30001\\
1403	-0.30001\\
1404	-0.30001\\
1405	-0.28737\\
1406	-0.26616\\
1407	-0.23894\\
1408	-0.21166\\
1409	-0.18703\\
1410	-0.16767\\
1411	-0.15396\\
1412	-0.14585\\
1413	-0.14209\\
1414	-0.14067\\
1415	-0.13861\\
1416	-0.13466\\
1417	-0.12883\\
1418	-0.1197\\
1419	-0.10638\\
1420	-0.089368\\
1421	-0.069874\\
1422	-0.049319\\
1423	-0.029022\\
1424	-0.010043\\
1425	0.0068796\\
1426	0.021326\\
1427	0.033147\\
1428	0.04239\\
1429	0.049222\\
1430	0.053871\\
1431	0.056578\\
1432	0.057576\\
1433	0.057076\\
1434	0.055147\\
1435	0.05173\\
1436	0.046759\\
1437	0.040141\\
1438	0.031975\\
1439	0.022468\\
1440	0.012058\\
1441	0.0012793\\
1442	-0.0093269\\
1443	-0.019264\\
1444	-0.02816\\
1445	-0.035699\\
1446	-0.041612\\
1447	-0.045688\\
1448	-0.047795\\
1449	-0.047902\\
1450	-0.046084\\
1451	-0.042525\\
1452	-0.037496\\
1453	-0.031336\\
1454	-0.024421\\
1455	-0.01713\\
1456	-0.0098206\\
1457	-0.002803\\
1458	0.0036702\\
1459	0.0094106\\
1460	0.014292\\
1461	0.018243\\
1462	0.021239\\
1463	0.023291\\
1464	0.024437\\
1465	0.024734\\
1466	0.024255\\
1467	0.023081\\
1468	0.021302\\
1469	0.019013\\
1470	0.016313\\
1471	0.013305\\
1472	0.010094\\
1473	0.0067841\\
1474	0.0034804\\
1475	0.00028388\\
1476	-0.0027104\\
1477	-0.0054161\\
1478	-0.0077587\\
1479	-0.0096782\\
1480	-0.011131\\
1481	-0.012093\\
1482	-0.01256\\
1483	-0.012544\\
1484	-0.012078\\
1485	-0.011211\\
1486	-0.010004\\
1487	-0.0085265\\
1488	-0.0068549\\
1489	-0.0050663\\
1490	-0.003236\\
1491	-0.0014341\\
1492	0.00027702\\
1493	0.0018443\\
1494	0.0032249\\
1495	0.0043872\\
1496	0.0053097\\
1497	0.0059815\\
1498	0.0064008\\
1499	0.0065742\\
1500	0.0065158\\
1501	0.0062454\\
1502	0.0057882\\
1503	0.0051729\\
1504	0.0044312\\
1505	0.0035963\\
1506	0.0027022\\
1507	0.0017823\\
1508	0.00086875\\
1509	-8.4538e-06\\
1510	-0.00082246\\
1511	-0.0015501\\
1512	-0.0021727\\
1513	-0.002676\\
1514	-0.0030511\\
1515	-0.0032938\\
1516	-0.0034051\\
1517	-0.0033903\\
1518	-0.0032591\\
1519	-0.0030244\\
1520	-0.0027022\\
1521	-0.00231\\
1522	-0.0018671\\
1523	-0.0013925\\
1524	-0.00090536\\
1525	-0.0004235\\
1526	3.6754e-05\\
1527	0.00046117\\
1528	0.0008379\\
1529	0.0011577\\
1530	0.001414\\
1531	0.0016029\\
1532	0.0017231\\
1533	0.0017759\\
1534	0.0017644\\
1535	0.0016939\\
1536	0.0015713\\
1537	0.0014044\\
1538	0.0012022\\
1539	0.00097411\\
1540	0.00072973\\
1541	0.00047851\\
1542	0.00022946\\
1543	-9.1469e-06\\
1544	-0.00022994\\
1545	-0.0004267\\
1546	-0.00059445\\
1547	-0.00072955\\
1548	-0.00082976\\
1549	-0.00089417\\
1550	-0.00092319\\
1551	-0.0009184\\
1552	-0.00088244\\
1553	-0.00081886\\
1554	-0.00073187\\
1555	-0.0006262\\
1556	-0.00050687\\
1557	-0.000379\\
1558	-0.00024762\\
1559	-0.00011751\\
1560	6.9673e-06\\
1561	0.00012196\\
1562	0.00022425\\
1563	0.00031127\\
1564	0.0003812\\
1565	0.00043292\\
1566	0.00046603\\
1567	0.00048078\\
1568	0.00047803\\
1569	0.00045919\\
1570	0.00042607\\
1571	0.00038087\\
1572	0.00032602\\
1573	0.0002641\\
1574	0.00019774\\
1575	0.00012953\\
1576	6.1936e-05\\
1577	-2.7901e-06\\
1578	-6.2646e-05\\
1579	-0.00011595\\
1580	-0.00016135\\
1581	-0.00019789\\
1582	-0.00022497\\
1583	-0.00024235\\
1584	-0.00025016\\
1585	-0.00024883\\
1586	-0.00023908\\
1587	-0.00022188\\
1588	-0.00019835\\
1589	-0.00016978\\
1590	-0.00013751\\
1591	-0.00010293\\
1592	-6.7378e-05\\
1593	-3.2157e-05\\
1594	1.5573e-06\\
1595	3.2723e-05\\
1596	6.0462e-05\\
1597	8.4082e-05\\
1598	0.00010308\\
1599	0.00011715\\
1600	0.00012617\\
1601	0.00013022\\
1602	0.00012952\\
1603	0.00012445\\
1604	0.0001155\\
1605	0.0014549\\
1606	0.0054434\\
1607	0.011933\\
1608	0.0202\\
1609	0.029405\\
1610	0.038817\\
1611	0.047887\\
1612	0.056247\\
1613	0.06369\\
1614	0.070135\\
1615	0.075589\\
1616	0.080123\\
1617	0.083882\\
1618	0.087005\\
1619	0.089579\\
1620	0.091694\\
1621	0.093433\\
1622	0.094858\\
1623	0.096017\\
1624	0.096962\\
1625	0.09773\\
1626	0.098351\\
1627	0.098848\\
1628	0.099242\\
1629	0.09955\\
1630	0.099786\\
1631	0.099963\\
1632	0.10009\\
1633	0.10018\\
1634	0.10023\\
1635	0.10027\\
1636	0.10028\\
1637	0.10027\\
1638	0.10026\\
1639	0.10023\\
1640	0.10021\\
1641	0.10018\\
1642	0.10015\\
1643	0.10011\\
1644	0.10008\\
1645	0.10006\\
1646	0.10003\\
1647	0.10001\\
1648	0.099984\\
1649	0.099966\\
1650	0.09995\\
1651	0.099936\\
1652	0.099925\\
1653	0.099916\\
1654	0.099908\\
1655	0.099903\\
1656	0.099898\\
1657	0.099895\\
1658	0.099892\\
1659	0.099891\\
1660	0.09989\\
1661	0.09989\\
1662	0.09989\\
1663	0.09989\\
1664	0.099891\\
1665	0.099892\\
1666	0.099892\\
1667	0.099893\\
1668	0.099894\\
1669	0.099895\\
1670	0.099896\\
1671	0.099896\\
1672	0.099897\\
1673	0.099898\\
1674	0.099898\\
1675	0.099898\\
1676	0.099899\\
1677	0.099899\\
1678	0.099899\\
1679	0.0999\\
1680	0.0999\\
1681	0.0999\\
1682	0.0999\\
1683	0.0999\\
1684	0.0999\\
1685	0.0999\\
1686	0.0999\\
1687	0.0999\\
1688	0.0999\\
1689	0.0999\\
1690	0.0999\\
1691	0.0999\\
1692	0.0999\\
1693	0.0999\\
1694	0.0999\\
1695	0.0999\\
1696	0.0999\\
1697	0.0999\\
1698	0.0999\\
1699	0.0999\\
1700	0.0999\\
1701	0.0999\\
1702	0.0999\\
1703	0.0999\\
1704	0.0999\\
1705	0.0999\\
1706	0.0999\\
1707	0.0999\\
1708	0.0999\\
1709	0.0999\\
1710	0.0999\\
1711	0.0999\\
1712	0.0999\\
1713	0.0999\\
1714	0.0999\\
1715	0.0999\\
1716	0.0999\\
1717	0.0999\\
1718	0.0999\\
1719	0.0999\\
1720	0.0999\\
1721	0.0999\\
1722	0.0999\\
1723	0.0999\\
1724	0.0999\\
1725	0.0999\\
1726	0.0999\\
1727	0.0999\\
1728	0.0999\\
1729	0.0999\\
1730	0.0999\\
1731	0.0999\\
1732	0.0999\\
1733	0.0999\\
1734	0.0999\\
1735	0.0999\\
1736	0.0999\\
1737	0.0999\\
1738	0.0999\\
1739	0.0999\\
1740	0.0999\\
1741	0.0999\\
1742	0.0999\\
1743	0.0999\\
1744	0.0999\\
1745	0.0999\\
1746	0.0999\\
1747	0.0999\\
1748	0.0999\\
1749	0.0999\\
1750	0.0999\\
1751	0.0999\\
1752	0.0999\\
1753	0.0999\\
1754	0.0999\\
1755	0.0999\\
1756	0.0999\\
1757	0.0999\\
1758	0.0999\\
1759	0.0999\\
1760	0.0999\\
1761	0.0999\\
1762	0.0999\\
1763	0.0999\\
1764	0.0999\\
1765	0.0999\\
1766	0.0999\\
1767	0.0999\\
1768	0.0999\\
1769	0.0999\\
1770	0.0999\\
1771	0.0999\\
1772	0.0999\\
1773	0.0999\\
1774	0.0999\\
1775	0.0999\\
1776	0.0999\\
1777	0.0999\\
1778	0.0999\\
1779	0.0999\\
1780	0.0999\\
1781	0.0999\\
1782	0.0999\\
1783	0.0999\\
1784	0.0999\\
1785	0.0999\\
1786	0.0999\\
1787	0.0999\\
1788	0.0999\\
1789	0.0999\\
1790	0.0999\\
1791	0.0999\\
1792	0.0999\\
1793	0.0999\\
1794	0.0999\\
1795	0.0999\\
1796	0.0999\\
1797	0.0999\\
1798	0.0999\\
1799	0.0999\\
1800	0.0999\\
};
\addlegendentry{Wyjście y}

\addplot [color=mycolor2, dashed]
  table[row sep=crcr]{%
1	0\\
2	0\\
3	0\\
4	0\\
5	0\\
6	0\\
7	0\\
8	0\\
9	0\\
10	0\\
11	0\\
12	0\\
13	0\\
14	0\\
15	0\\
16	0\\
17	0\\
18	0\\
19	0\\
20	-2\\
21	-2\\
22	-2\\
23	-2\\
24	-2\\
25	-2\\
26	-2\\
27	-2\\
28	-2\\
29	-2\\
30	-2\\
31	-2\\
32	-2\\
33	-2\\
34	-2\\
35	-2\\
36	-2\\
37	-2\\
38	-2\\
39	-2\\
40	-2\\
41	-2\\
42	-2\\
43	-2\\
44	-2\\
45	-2\\
46	-2\\
47	-2\\
48	-2\\
49	-2\\
50	-2\\
51	-2\\
52	-2\\
53	-2\\
54	-2\\
55	-2\\
56	-2\\
57	-2\\
58	-2\\
59	-2\\
60	-2\\
61	-2\\
62	-2\\
63	-2\\
64	-2\\
65	-2\\
66	-2\\
67	-2\\
68	-2\\
69	-2\\
70	-2\\
71	-2\\
72	-2\\
73	-2\\
74	-2\\
75	-2\\
76	-2\\
77	-2\\
78	-2\\
79	-2\\
80	-2\\
81	-2\\
82	-2\\
83	-2\\
84	-2\\
85	-2\\
86	-2\\
87	-2\\
88	-2\\
89	-2\\
90	-2\\
91	-2\\
92	-2\\
93	-2\\
94	-2\\
95	-2\\
96	-2\\
97	-2\\
98	-2\\
99	-2\\
100	-2\\
101	-2\\
102	-2\\
103	-2\\
104	-2\\
105	-2\\
106	-2\\
107	-2\\
108	-2\\
109	-2\\
110	-2\\
111	-2\\
112	-2\\
113	-2\\
114	-2\\
115	-2\\
116	-2\\
117	-2\\
118	-2\\
119	-2\\
120	-2\\
121	-2\\
122	-2\\
123	-2\\
124	-2\\
125	-2\\
126	-2\\
127	-2\\
128	-2\\
129	-2\\
130	-2\\
131	-2\\
132	-2\\
133	-2\\
134	-2\\
135	-2\\
136	-2\\
137	-2\\
138	-2\\
139	-2\\
140	-2\\
141	-2\\
142	-2\\
143	-2\\
144	-2\\
145	-2\\
146	-2\\
147	-2\\
148	-2\\
149	-2\\
150	-2\\
151	-2\\
152	-2\\
153	-2\\
154	-2\\
155	-2\\
156	-2\\
157	-2\\
158	-2\\
159	-2\\
160	-2\\
161	-2\\
162	-2\\
163	-2\\
164	-2\\
165	-2\\
166	-2\\
167	-2\\
168	-2\\
169	-2\\
170	-2\\
171	-2\\
172	-2\\
173	-2\\
174	-2\\
175	-2\\
176	-2\\
177	-2\\
178	-2\\
179	-2\\
180	-2\\
181	-2\\
182	-2\\
183	-2\\
184	-2\\
185	-2\\
186	-2\\
187	-2\\
188	-2\\
189	-2\\
190	-2\\
191	-2\\
192	-2\\
193	-2\\
194	-2\\
195	-2\\
196	-2\\
197	-2\\
198	-2\\
199	-2\\
200	-4.5\\
201	-4.5\\
202	-4.5\\
203	-4.5\\
204	-4.5\\
205	-4.5\\
206	-4.5\\
207	-4.5\\
208	-4.5\\
209	-4.5\\
210	-4.5\\
211	-4.5\\
212	-4.5\\
213	-4.5\\
214	-4.5\\
215	-4.5\\
216	-4.5\\
217	-4.5\\
218	-4.5\\
219	-4.5\\
220	-4.5\\
221	-4.5\\
222	-4.5\\
223	-4.5\\
224	-4.5\\
225	-4.5\\
226	-4.5\\
227	-4.5\\
228	-4.5\\
229	-4.5\\
230	-4.5\\
231	-4.5\\
232	-4.5\\
233	-4.5\\
234	-4.5\\
235	-4.5\\
236	-4.5\\
237	-4.5\\
238	-4.5\\
239	-4.5\\
240	-4.5\\
241	-4.5\\
242	-4.5\\
243	-4.5\\
244	-4.5\\
245	-4.5\\
246	-4.5\\
247	-4.5\\
248	-4.5\\
249	-4.5\\
250	-4.5\\
251	-4.5\\
252	-4.5\\
253	-4.5\\
254	-4.5\\
255	-4.5\\
256	-4.5\\
257	-4.5\\
258	-4.5\\
259	-4.5\\
260	-4.5\\
261	-4.5\\
262	-4.5\\
263	-4.5\\
264	-4.5\\
265	-4.5\\
266	-4.5\\
267	-4.5\\
268	-4.5\\
269	-4.5\\
270	-4.5\\
271	-4.5\\
272	-4.5\\
273	-4.5\\
274	-4.5\\
275	-4.5\\
276	-4.5\\
277	-4.5\\
278	-4.5\\
279	-4.5\\
280	-4.5\\
281	-4.5\\
282	-4.5\\
283	-4.5\\
284	-4.5\\
285	-4.5\\
286	-4.5\\
287	-4.5\\
288	-4.5\\
289	-4.5\\
290	-4.5\\
291	-4.5\\
292	-4.5\\
293	-4.5\\
294	-4.5\\
295	-4.5\\
296	-4.5\\
297	-4.5\\
298	-4.5\\
299	-4.5\\
300	-4.5\\
301	-4.5\\
302	-4.5\\
303	-4.5\\
304	-4.5\\
305	-4.5\\
306	-4.5\\
307	-4.5\\
308	-4.5\\
309	-4.5\\
310	-4.5\\
311	-4.5\\
312	-4.5\\
313	-4.5\\
314	-4.5\\
315	-4.5\\
316	-4.5\\
317	-4.5\\
318	-4.5\\
319	-4.5\\
320	-4.5\\
321	-4.5\\
322	-4.5\\
323	-4.5\\
324	-4.5\\
325	-4.5\\
326	-4.5\\
327	-4.5\\
328	-4.5\\
329	-4.5\\
330	-4.5\\
331	-4.5\\
332	-4.5\\
333	-4.5\\
334	-4.5\\
335	-4.5\\
336	-4.5\\
337	-4.5\\
338	-4.5\\
339	-4.5\\
340	-4.5\\
341	-4.5\\
342	-4.5\\
343	-4.5\\
344	-4.5\\
345	-4.5\\
346	-4.5\\
347	-4.5\\
348	-4.5\\
349	-4.5\\
350	-4.5\\
351	-4.5\\
352	-4.5\\
353	-4.5\\
354	-4.5\\
355	-4.5\\
356	-4.5\\
357	-4.5\\
358	-4.5\\
359	-4.5\\
360	-4.5\\
361	-4.5\\
362	-4.5\\
363	-4.5\\
364	-4.5\\
365	-4.5\\
366	-4.5\\
367	-4.5\\
368	-4.5\\
369	-4.5\\
370	-4.5\\
371	-4.5\\
372	-4.5\\
373	-4.5\\
374	-4.5\\
375	-4.5\\
376	-4.5\\
377	-4.5\\
378	-4.5\\
379	-4.5\\
380	-4.5\\
381	-4.5\\
382	-4.5\\
383	-4.5\\
384	-4.5\\
385	-4.5\\
386	-4.5\\
387	-4.5\\
388	-4.5\\
389	-4.5\\
390	-4.5\\
391	-4.5\\
392	-4.5\\
393	-4.5\\
394	-4.5\\
395	-4.5\\
396	-4.5\\
397	-4.5\\
398	-4.5\\
399	-4.5\\
400	-3\\
401	-3\\
402	-3\\
403	-3\\
404	-3\\
405	-3\\
406	-3\\
407	-3\\
408	-3\\
409	-3\\
410	-3\\
411	-3\\
412	-3\\
413	-3\\
414	-3\\
415	-3\\
416	-3\\
417	-3\\
418	-3\\
419	-3\\
420	-3\\
421	-3\\
422	-3\\
423	-3\\
424	-3\\
425	-3\\
426	-3\\
427	-3\\
428	-3\\
429	-3\\
430	-3\\
431	-3\\
432	-3\\
433	-3\\
434	-3\\
435	-3\\
436	-3\\
437	-3\\
438	-3\\
439	-3\\
440	-3\\
441	-3\\
442	-3\\
443	-3\\
444	-3\\
445	-3\\
446	-3\\
447	-3\\
448	-3\\
449	-3\\
450	-3\\
451	-3\\
452	-3\\
453	-3\\
454	-3\\
455	-3\\
456	-3\\
457	-3\\
458	-3\\
459	-3\\
460	-3\\
461	-3\\
462	-3\\
463	-3\\
464	-3\\
465	-3\\
466	-3\\
467	-3\\
468	-3\\
469	-3\\
470	-3\\
471	-3\\
472	-3\\
473	-3\\
474	-3\\
475	-3\\
476	-3\\
477	-3\\
478	-3\\
479	-3\\
480	-3\\
481	-3\\
482	-3\\
483	-3\\
484	-3\\
485	-3\\
486	-3\\
487	-3\\
488	-3\\
489	-3\\
490	-3\\
491	-3\\
492	-3\\
493	-3\\
494	-3\\
495	-3\\
496	-3\\
497	-3\\
498	-3\\
499	-3\\
500	-3\\
501	-3\\
502	-3\\
503	-3\\
504	-3\\
505	-3\\
506	-3\\
507	-3\\
508	-3\\
509	-3\\
510	-3\\
511	-3\\
512	-3\\
513	-3\\
514	-3\\
515	-3\\
516	-3\\
517	-3\\
518	-3\\
519	-3\\
520	-3\\
521	-3\\
522	-3\\
523	-3\\
524	-3\\
525	-3\\
526	-3\\
527	-3\\
528	-3\\
529	-3\\
530	-3\\
531	-3\\
532	-3\\
533	-3\\
534	-3\\
535	-3\\
536	-3\\
537	-3\\
538	-3\\
539	-3\\
540	-3\\
541	-3\\
542	-3\\
543	-3\\
544	-3\\
545	-3\\
546	-3\\
547	-3\\
548	-3\\
549	-3\\
550	-3\\
551	-3\\
552	-3\\
553	-3\\
554	-3\\
555	-3\\
556	-3\\
557	-3\\
558	-3\\
559	-3\\
560	-3\\
561	-3\\
562	-3\\
563	-3\\
564	-3\\
565	-3\\
566	-3\\
567	-3\\
568	-3\\
569	-3\\
570	-3\\
571	-3\\
572	-3\\
573	-3\\
574	-3\\
575	-3\\
576	-3\\
577	-3\\
578	-3\\
579	-3\\
580	-3\\
581	-3\\
582	-3\\
583	-3\\
584	-3\\
585	-3\\
586	-3\\
587	-3\\
588	-3\\
589	-3\\
590	-3\\
591	-3\\
592	-3\\
593	-3\\
594	-3\\
595	-3\\
596	-3\\
597	-3\\
598	-3\\
599	-3\\
600	-1.5\\
601	-1.5\\
602	-1.5\\
603	-1.5\\
604	-1.5\\
605	-1.5\\
606	-1.5\\
607	-1.5\\
608	-1.5\\
609	-1.5\\
610	-1.5\\
611	-1.5\\
612	-1.5\\
613	-1.5\\
614	-1.5\\
615	-1.5\\
616	-1.5\\
617	-1.5\\
618	-1.5\\
619	-1.5\\
620	-1.5\\
621	-1.5\\
622	-1.5\\
623	-1.5\\
624	-1.5\\
625	-1.5\\
626	-1.5\\
627	-1.5\\
628	-1.5\\
629	-1.5\\
630	-1.5\\
631	-1.5\\
632	-1.5\\
633	-1.5\\
634	-1.5\\
635	-1.5\\
636	-1.5\\
637	-1.5\\
638	-1.5\\
639	-1.5\\
640	-1.5\\
641	-1.5\\
642	-1.5\\
643	-1.5\\
644	-1.5\\
645	-1.5\\
646	-1.5\\
647	-1.5\\
648	-1.5\\
649	-1.5\\
650	-1.5\\
651	-1.5\\
652	-1.5\\
653	-1.5\\
654	-1.5\\
655	-1.5\\
656	-1.5\\
657	-1.5\\
658	-1.5\\
659	-1.5\\
660	-1.5\\
661	-1.5\\
662	-1.5\\
663	-1.5\\
664	-1.5\\
665	-1.5\\
666	-1.5\\
667	-1.5\\
668	-1.5\\
669	-1.5\\
670	-1.5\\
671	-1.5\\
672	-1.5\\
673	-1.5\\
674	-1.5\\
675	-1.5\\
676	-1.5\\
677	-1.5\\
678	-1.5\\
679	-1.5\\
680	-1.5\\
681	-1.5\\
682	-1.5\\
683	-1.5\\
684	-1.5\\
685	-1.5\\
686	-1.5\\
687	-1.5\\
688	-1.5\\
689	-1.5\\
690	-1.5\\
691	-1.5\\
692	-1.5\\
693	-1.5\\
694	-1.5\\
695	-1.5\\
696	-1.5\\
697	-1.5\\
698	-1.5\\
699	-1.5\\
700	-1.5\\
701	-1.5\\
702	-1.5\\
703	-1.5\\
704	-1.5\\
705	-1.5\\
706	-1.5\\
707	-1.5\\
708	-1.5\\
709	-1.5\\
710	-1.5\\
711	-1.5\\
712	-1.5\\
713	-1.5\\
714	-1.5\\
715	-1.5\\
716	-1.5\\
717	-1.5\\
718	-1.5\\
719	-1.5\\
720	-1.5\\
721	-1.5\\
722	-1.5\\
723	-1.5\\
724	-1.5\\
725	-1.5\\
726	-1.5\\
727	-1.5\\
728	-1.5\\
729	-1.5\\
730	-1.5\\
731	-1.5\\
732	-1.5\\
733	-1.5\\
734	-1.5\\
735	-1.5\\
736	-1.5\\
737	-1.5\\
738	-1.5\\
739	-1.5\\
740	-1.5\\
741	-1.5\\
742	-1.5\\
743	-1.5\\
744	-1.5\\
745	-1.5\\
746	-1.5\\
747	-1.5\\
748	-1.5\\
749	-1.5\\
750	-1.5\\
751	-1.5\\
752	-1.5\\
753	-1.5\\
754	-1.5\\
755	-1.5\\
756	-1.5\\
757	-1.5\\
758	-1.5\\
759	-1.5\\
760	-1.5\\
761	-1.5\\
762	-1.5\\
763	-1.5\\
764	-1.5\\
765	-1.5\\
766	-1.5\\
767	-1.5\\
768	-1.5\\
769	-1.5\\
770	-1.5\\
771	-1.5\\
772	-1.5\\
773	-1.5\\
774	-1.5\\
775	-1.5\\
776	-1.5\\
777	-1.5\\
778	-1.5\\
779	-1.5\\
780	-1.5\\
781	-1.5\\
782	-1.5\\
783	-1.5\\
784	-1.5\\
785	-1.5\\
786	-1.5\\
787	-1.5\\
788	-1.5\\
789	-1.5\\
790	-1.5\\
791	-1.5\\
792	-1.5\\
793	-1.5\\
794	-1.5\\
795	-1.5\\
796	-1.5\\
797	-1.5\\
798	-1.5\\
799	-1.5\\
800	-3.5\\
801	-3.5\\
802	-3.5\\
803	-3.5\\
804	-3.5\\
805	-3.5\\
806	-3.5\\
807	-3.5\\
808	-3.5\\
809	-3.5\\
810	-3.5\\
811	-3.5\\
812	-3.5\\
813	-3.5\\
814	-3.5\\
815	-3.5\\
816	-3.5\\
817	-3.5\\
818	-3.5\\
819	-3.5\\
820	-3.5\\
821	-3.5\\
822	-3.5\\
823	-3.5\\
824	-3.5\\
825	-3.5\\
826	-3.5\\
827	-3.5\\
828	-3.5\\
829	-3.5\\
830	-3.5\\
831	-3.5\\
832	-3.5\\
833	-3.5\\
834	-3.5\\
835	-3.5\\
836	-3.5\\
837	-3.5\\
838	-3.5\\
839	-3.5\\
840	-3.5\\
841	-3.5\\
842	-3.5\\
843	-3.5\\
844	-3.5\\
845	-3.5\\
846	-3.5\\
847	-3.5\\
848	-3.5\\
849	-3.5\\
850	-3.5\\
851	-3.5\\
852	-3.5\\
853	-3.5\\
854	-3.5\\
855	-3.5\\
856	-3.5\\
857	-3.5\\
858	-3.5\\
859	-3.5\\
860	-3.5\\
861	-3.5\\
862	-3.5\\
863	-3.5\\
864	-3.5\\
865	-3.5\\
866	-3.5\\
867	-3.5\\
868	-3.5\\
869	-3.5\\
870	-3.5\\
871	-3.5\\
872	-3.5\\
873	-3.5\\
874	-3.5\\
875	-3.5\\
876	-3.5\\
877	-3.5\\
878	-3.5\\
879	-3.5\\
880	-3.5\\
881	-3.5\\
882	-3.5\\
883	-3.5\\
884	-3.5\\
885	-3.5\\
886	-3.5\\
887	-3.5\\
888	-3.5\\
889	-3.5\\
890	-3.5\\
891	-3.5\\
892	-3.5\\
893	-3.5\\
894	-3.5\\
895	-3.5\\
896	-3.5\\
897	-3.5\\
898	-3.5\\
899	-3.5\\
900	-3.5\\
901	-3.5\\
902	-3.5\\
903	-3.5\\
904	-3.5\\
905	-3.5\\
906	-3.5\\
907	-3.5\\
908	-3.5\\
909	-3.5\\
910	-3.5\\
911	-3.5\\
912	-3.5\\
913	-3.5\\
914	-3.5\\
915	-3.5\\
916	-3.5\\
917	-3.5\\
918	-3.5\\
919	-3.5\\
920	-3.5\\
921	-3.5\\
922	-3.5\\
923	-3.5\\
924	-3.5\\
925	-3.5\\
926	-3.5\\
927	-3.5\\
928	-3.5\\
929	-3.5\\
930	-3.5\\
931	-3.5\\
932	-3.5\\
933	-3.5\\
934	-3.5\\
935	-3.5\\
936	-3.5\\
937	-3.5\\
938	-3.5\\
939	-3.5\\
940	-3.5\\
941	-3.5\\
942	-3.5\\
943	-3.5\\
944	-3.5\\
945	-3.5\\
946	-3.5\\
947	-3.5\\
948	-3.5\\
949	-3.5\\
950	-3.5\\
951	-3.5\\
952	-3.5\\
953	-3.5\\
954	-3.5\\
955	-3.5\\
956	-3.5\\
957	-3.5\\
958	-3.5\\
959	-3.5\\
960	-3.5\\
961	-3.5\\
962	-3.5\\
963	-3.5\\
964	-3.5\\
965	-3.5\\
966	-3.5\\
967	-3.5\\
968	-3.5\\
969	-3.5\\
970	-3.5\\
971	-3.5\\
972	-3.5\\
973	-3.5\\
974	-3.5\\
975	-3.5\\
976	-3.5\\
977	-3.5\\
978	-3.5\\
979	-3.5\\
980	-3.5\\
981	-3.5\\
982	-3.5\\
983	-3.5\\
984	-3.5\\
985	-3.5\\
986	-3.5\\
987	-3.5\\
988	-3.5\\
989	-3.5\\
990	-3.5\\
991	-3.5\\
992	-3.5\\
993	-3.5\\
994	-3.5\\
995	-3.5\\
996	-3.5\\
997	-3.5\\
998	-3.5\\
999	-3.5\\
1000	-1.2\\
1001	-1.2\\
1002	-1.2\\
1003	-1.2\\
1004	-1.2\\
1005	-1.2\\
1006	-1.2\\
1007	-1.2\\
1008	-1.2\\
1009	-1.2\\
1010	-1.2\\
1011	-1.2\\
1012	-1.2\\
1013	-1.2\\
1014	-1.2\\
1015	-1.2\\
1016	-1.2\\
1017	-1.2\\
1018	-1.2\\
1019	-1.2\\
1020	-1.2\\
1021	-1.2\\
1022	-1.2\\
1023	-1.2\\
1024	-1.2\\
1025	-1.2\\
1026	-1.2\\
1027	-1.2\\
1028	-1.2\\
1029	-1.2\\
1030	-1.2\\
1031	-1.2\\
1032	-1.2\\
1033	-1.2\\
1034	-1.2\\
1035	-1.2\\
1036	-1.2\\
1037	-1.2\\
1038	-1.2\\
1039	-1.2\\
1040	-1.2\\
1041	-1.2\\
1042	-1.2\\
1043	-1.2\\
1044	-1.2\\
1045	-1.2\\
1046	-1.2\\
1047	-1.2\\
1048	-1.2\\
1049	-1.2\\
1050	-1.2\\
1051	-1.2\\
1052	-1.2\\
1053	-1.2\\
1054	-1.2\\
1055	-1.2\\
1056	-1.2\\
1057	-1.2\\
1058	-1.2\\
1059	-1.2\\
1060	-1.2\\
1061	-1.2\\
1062	-1.2\\
1063	-1.2\\
1064	-1.2\\
1065	-1.2\\
1066	-1.2\\
1067	-1.2\\
1068	-1.2\\
1069	-1.2\\
1070	-1.2\\
1071	-1.2\\
1072	-1.2\\
1073	-1.2\\
1074	-1.2\\
1075	-1.2\\
1076	-1.2\\
1077	-1.2\\
1078	-1.2\\
1079	-1.2\\
1080	-1.2\\
1081	-1.2\\
1082	-1.2\\
1083	-1.2\\
1084	-1.2\\
1085	-1.2\\
1086	-1.2\\
1087	-1.2\\
1088	-1.2\\
1089	-1.2\\
1090	-1.2\\
1091	-1.2\\
1092	-1.2\\
1093	-1.2\\
1094	-1.2\\
1095	-1.2\\
1096	-1.2\\
1097	-1.2\\
1098	-1.2\\
1099	-1.2\\
1100	-1.2\\
1101	-1.2\\
1102	-1.2\\
1103	-1.2\\
1104	-1.2\\
1105	-1.2\\
1106	-1.2\\
1107	-1.2\\
1108	-1.2\\
1109	-1.2\\
1110	-1.2\\
1111	-1.2\\
1112	-1.2\\
1113	-1.2\\
1114	-1.2\\
1115	-1.2\\
1116	-1.2\\
1117	-1.2\\
1118	-1.2\\
1119	-1.2\\
1120	-1.2\\
1121	-1.2\\
1122	-1.2\\
1123	-1.2\\
1124	-1.2\\
1125	-1.2\\
1126	-1.2\\
1127	-1.2\\
1128	-1.2\\
1129	-1.2\\
1130	-1.2\\
1131	-1.2\\
1132	-1.2\\
1133	-1.2\\
1134	-1.2\\
1135	-1.2\\
1136	-1.2\\
1137	-1.2\\
1138	-1.2\\
1139	-1.2\\
1140	-1.2\\
1141	-1.2\\
1142	-1.2\\
1143	-1.2\\
1144	-1.2\\
1145	-1.2\\
1146	-1.2\\
1147	-1.2\\
1148	-1.2\\
1149	-1.2\\
1150	-1.2\\
1151	-1.2\\
1152	-1.2\\
1153	-1.2\\
1154	-1.2\\
1155	-1.2\\
1156	-1.2\\
1157	-1.2\\
1158	-1.2\\
1159	-1.2\\
1160	-1.2\\
1161	-1.2\\
1162	-1.2\\
1163	-1.2\\
1164	-1.2\\
1165	-1.2\\
1166	-1.2\\
1167	-1.2\\
1168	-1.2\\
1169	-1.2\\
1170	-1.2\\
1171	-1.2\\
1172	-1.2\\
1173	-1.2\\
1174	-1.2\\
1175	-1.2\\
1176	-1.2\\
1177	-1.2\\
1178	-1.2\\
1179	-1.2\\
1180	-1.2\\
1181	-1.2\\
1182	-1.2\\
1183	-1.2\\
1184	-1.2\\
1185	-1.2\\
1186	-1.2\\
1187	-1.2\\
1188	-1.2\\
1189	-1.2\\
1190	-1.2\\
1191	-1.2\\
1192	-1.2\\
1193	-1.2\\
1194	-1.2\\
1195	-1.2\\
1196	-1.2\\
1197	-1.2\\
1198	-1.2\\
1199	-1.2\\
1200	-0.3\\
1201	-0.3\\
1202	-0.3\\
1203	-0.3\\
1204	-0.3\\
1205	-0.3\\
1206	-0.3\\
1207	-0.3\\
1208	-0.3\\
1209	-0.3\\
1210	-0.3\\
1211	-0.3\\
1212	-0.3\\
1213	-0.3\\
1214	-0.3\\
1215	-0.3\\
1216	-0.3\\
1217	-0.3\\
1218	-0.3\\
1219	-0.3\\
1220	-0.3\\
1221	-0.3\\
1222	-0.3\\
1223	-0.3\\
1224	-0.3\\
1225	-0.3\\
1226	-0.3\\
1227	-0.3\\
1228	-0.3\\
1229	-0.3\\
1230	-0.3\\
1231	-0.3\\
1232	-0.3\\
1233	-0.3\\
1234	-0.3\\
1235	-0.3\\
1236	-0.3\\
1237	-0.3\\
1238	-0.3\\
1239	-0.3\\
1240	-0.3\\
1241	-0.3\\
1242	-0.3\\
1243	-0.3\\
1244	-0.3\\
1245	-0.3\\
1246	-0.3\\
1247	-0.3\\
1248	-0.3\\
1249	-0.3\\
1250	-0.3\\
1251	-0.3\\
1252	-0.3\\
1253	-0.3\\
1254	-0.3\\
1255	-0.3\\
1256	-0.3\\
1257	-0.3\\
1258	-0.3\\
1259	-0.3\\
1260	-0.3\\
1261	-0.3\\
1262	-0.3\\
1263	-0.3\\
1264	-0.3\\
1265	-0.3\\
1266	-0.3\\
1267	-0.3\\
1268	-0.3\\
1269	-0.3\\
1270	-0.3\\
1271	-0.3\\
1272	-0.3\\
1273	-0.3\\
1274	-0.3\\
1275	-0.3\\
1276	-0.3\\
1277	-0.3\\
1278	-0.3\\
1279	-0.3\\
1280	-0.3\\
1281	-0.3\\
1282	-0.3\\
1283	-0.3\\
1284	-0.3\\
1285	-0.3\\
1286	-0.3\\
1287	-0.3\\
1288	-0.3\\
1289	-0.3\\
1290	-0.3\\
1291	-0.3\\
1292	-0.3\\
1293	-0.3\\
1294	-0.3\\
1295	-0.3\\
1296	-0.3\\
1297	-0.3\\
1298	-0.3\\
1299	-0.3\\
1300	-0.3\\
1301	-0.3\\
1302	-0.3\\
1303	-0.3\\
1304	-0.3\\
1305	-0.3\\
1306	-0.3\\
1307	-0.3\\
1308	-0.3\\
1309	-0.3\\
1310	-0.3\\
1311	-0.3\\
1312	-0.3\\
1313	-0.3\\
1314	-0.3\\
1315	-0.3\\
1316	-0.3\\
1317	-0.3\\
1318	-0.3\\
1319	-0.3\\
1320	-0.3\\
1321	-0.3\\
1322	-0.3\\
1323	-0.3\\
1324	-0.3\\
1325	-0.3\\
1326	-0.3\\
1327	-0.3\\
1328	-0.3\\
1329	-0.3\\
1330	-0.3\\
1331	-0.3\\
1332	-0.3\\
1333	-0.3\\
1334	-0.3\\
1335	-0.3\\
1336	-0.3\\
1337	-0.3\\
1338	-0.3\\
1339	-0.3\\
1340	-0.3\\
1341	-0.3\\
1342	-0.3\\
1343	-0.3\\
1344	-0.3\\
1345	-0.3\\
1346	-0.3\\
1347	-0.3\\
1348	-0.3\\
1349	-0.3\\
1350	-0.3\\
1351	-0.3\\
1352	-0.3\\
1353	-0.3\\
1354	-0.3\\
1355	-0.3\\
1356	-0.3\\
1357	-0.3\\
1358	-0.3\\
1359	-0.3\\
1360	-0.3\\
1361	-0.3\\
1362	-0.3\\
1363	-0.3\\
1364	-0.3\\
1365	-0.3\\
1366	-0.3\\
1367	-0.3\\
1368	-0.3\\
1369	-0.3\\
1370	-0.3\\
1371	-0.3\\
1372	-0.3\\
1373	-0.3\\
1374	-0.3\\
1375	-0.3\\
1376	-0.3\\
1377	-0.3\\
1378	-0.3\\
1379	-0.3\\
1380	-0.3\\
1381	-0.3\\
1382	-0.3\\
1383	-0.3\\
1384	-0.3\\
1385	-0.3\\
1386	-0.3\\
1387	-0.3\\
1388	-0.3\\
1389	-0.3\\
1390	-0.3\\
1391	-0.3\\
1392	-0.3\\
1393	-0.3\\
1394	-0.3\\
1395	-0.3\\
1396	-0.3\\
1397	-0.3\\
1398	-0.3\\
1399	-0.3\\
1400	0\\
1401	0\\
1402	0\\
1403	0\\
1404	0\\
1405	0\\
1406	0\\
1407	0\\
1408	0\\
1409	0\\
1410	0\\
1411	0\\
1412	0\\
1413	0\\
1414	0\\
1415	0\\
1416	0\\
1417	0\\
1418	0\\
1419	0\\
1420	0\\
1421	0\\
1422	0\\
1423	0\\
1424	0\\
1425	0\\
1426	0\\
1427	0\\
1428	0\\
1429	0\\
1430	0\\
1431	0\\
1432	0\\
1433	0\\
1434	0\\
1435	0\\
1436	0\\
1437	0\\
1438	0\\
1439	0\\
1440	0\\
1441	0\\
1442	0\\
1443	0\\
1444	0\\
1445	0\\
1446	0\\
1447	0\\
1448	0\\
1449	0\\
1450	0\\
1451	0\\
1452	0\\
1453	0\\
1454	0\\
1455	0\\
1456	0\\
1457	0\\
1458	0\\
1459	0\\
1460	0\\
1461	0\\
1462	0\\
1463	0\\
1464	0\\
1465	0\\
1466	0\\
1467	0\\
1468	0\\
1469	0\\
1470	0\\
1471	0\\
1472	0\\
1473	0\\
1474	0\\
1475	0\\
1476	0\\
1477	0\\
1478	0\\
1479	0\\
1480	0\\
1481	0\\
1482	0\\
1483	0\\
1484	0\\
1485	0\\
1486	0\\
1487	0\\
1488	0\\
1489	0\\
1490	0\\
1491	0\\
1492	0\\
1493	0\\
1494	0\\
1495	0\\
1496	0\\
1497	0\\
1498	0\\
1499	0\\
1500	0\\
1501	0\\
1502	0\\
1503	0\\
1504	0\\
1505	0\\
1506	0\\
1507	0\\
1508	0\\
1509	0\\
1510	0\\
1511	0\\
1512	0\\
1513	0\\
1514	0\\
1515	0\\
1516	0\\
1517	0\\
1518	0\\
1519	0\\
1520	0\\
1521	0\\
1522	0\\
1523	0\\
1524	0\\
1525	0\\
1526	0\\
1527	0\\
1528	0\\
1529	0\\
1530	0\\
1531	0\\
1532	0\\
1533	0\\
1534	0\\
1535	0\\
1536	0\\
1537	0\\
1538	0\\
1539	0\\
1540	0\\
1541	0\\
1542	0\\
1543	0\\
1544	0\\
1545	0\\
1546	0\\
1547	0\\
1548	0\\
1549	0\\
1550	0\\
1551	0\\
1552	0\\
1553	0\\
1554	0\\
1555	0\\
1556	0\\
1557	0\\
1558	0\\
1559	0\\
1560	0\\
1561	0\\
1562	0\\
1563	0\\
1564	0\\
1565	0\\
1566	0\\
1567	0\\
1568	0\\
1569	0\\
1570	0\\
1571	0\\
1572	0\\
1573	0\\
1574	0\\
1575	0\\
1576	0\\
1577	0\\
1578	0\\
1579	0\\
1580	0\\
1581	0\\
1582	0\\
1583	0\\
1584	0\\
1585	0\\
1586	0\\
1587	0\\
1588	0\\
1589	0\\
1590	0\\
1591	0\\
1592	0\\
1593	0\\
1594	0\\
1595	0\\
1596	0\\
1597	0\\
1598	0\\
1599	0\\
1600	0.1\\
1601	0.1\\
1602	0.1\\
1603	0.1\\
1604	0.1\\
1605	0.1\\
1606	0.1\\
1607	0.1\\
1608	0.1\\
1609	0.1\\
1610	0.1\\
1611	0.1\\
1612	0.1\\
1613	0.1\\
1614	0.1\\
1615	0.1\\
1616	0.1\\
1617	0.1\\
1618	0.1\\
1619	0.1\\
1620	0.1\\
1621	0.1\\
1622	0.1\\
1623	0.1\\
1624	0.1\\
1625	0.1\\
1626	0.1\\
1627	0.1\\
1628	0.1\\
1629	0.1\\
1630	0.1\\
1631	0.1\\
1632	0.1\\
1633	0.1\\
1634	0.1\\
1635	0.1\\
1636	0.1\\
1637	0.1\\
1638	0.1\\
1639	0.1\\
1640	0.1\\
1641	0.1\\
1642	0.1\\
1643	0.1\\
1644	0.1\\
1645	0.1\\
1646	0.1\\
1647	0.1\\
1648	0.1\\
1649	0.1\\
1650	0.1\\
1651	0.1\\
1652	0.1\\
1653	0.1\\
1654	0.1\\
1655	0.1\\
1656	0.1\\
1657	0.1\\
1658	0.1\\
1659	0.1\\
1660	0.1\\
1661	0.1\\
1662	0.1\\
1663	0.1\\
1664	0.1\\
1665	0.1\\
1666	0.1\\
1667	0.1\\
1668	0.1\\
1669	0.1\\
1670	0.1\\
1671	0.1\\
1672	0.1\\
1673	0.1\\
1674	0.1\\
1675	0.1\\
1676	0.1\\
1677	0.1\\
1678	0.1\\
1679	0.1\\
1680	0.1\\
1681	0.1\\
1682	0.1\\
1683	0.1\\
1684	0.1\\
1685	0.1\\
1686	0.1\\
1687	0.1\\
1688	0.1\\
1689	0.1\\
1690	0.1\\
1691	0.1\\
1692	0.1\\
1693	0.1\\
1694	0.1\\
1695	0.1\\
1696	0.1\\
1697	0.1\\
1698	0.1\\
1699	0.1\\
1700	0.1\\
1701	0.1\\
1702	0.1\\
1703	0.1\\
1704	0.1\\
1705	0.1\\
1706	0.1\\
1707	0.1\\
1708	0.1\\
1709	0.1\\
1710	0.1\\
1711	0.1\\
1712	0.1\\
1713	0.1\\
1714	0.1\\
1715	0.1\\
1716	0.1\\
1717	0.1\\
1718	0.1\\
1719	0.1\\
1720	0.1\\
1721	0.1\\
1722	0.1\\
1723	0.1\\
1724	0.1\\
1725	0.1\\
1726	0.1\\
1727	0.1\\
1728	0.1\\
1729	0.1\\
1730	0.1\\
1731	0.1\\
1732	0.1\\
1733	0.1\\
1734	0.1\\
1735	0.1\\
1736	0.1\\
1737	0.1\\
1738	0.1\\
1739	0.1\\
1740	0.1\\
1741	0.1\\
1742	0.1\\
1743	0.1\\
1744	0.1\\
1745	0.1\\
1746	0.1\\
1747	0.1\\
1748	0.1\\
1749	0.1\\
1750	0.1\\
1751	0.1\\
1752	0.1\\
1753	0.1\\
1754	0.1\\
1755	0.1\\
1756	0.1\\
1757	0.1\\
1758	0.1\\
1759	0.1\\
1760	0.1\\
1761	0.1\\
1762	0.1\\
1763	0.1\\
1764	0.1\\
1765	0.1\\
1766	0.1\\
1767	0.1\\
1768	0.1\\
1769	0.1\\
1770	0.1\\
1771	0.1\\
1772	0.1\\
1773	0.1\\
1774	0.1\\
1775	0.1\\
1776	0.1\\
1777	0.1\\
1778	0.1\\
1779	0.1\\
1780	0.1\\
1781	0.1\\
1782	0.1\\
1783	0.1\\
1784	0.1\\
1785	0.1\\
1786	0.1\\
1787	0.1\\
1788	0.1\\
1789	0.1\\
1790	0.1\\
1791	0.1\\
1792	0.1\\
1793	0.1\\
1794	0.1\\
1795	0.1\\
1796	0.1\\
1797	0.1\\
1798	0.1\\
1799	0.1\\
1800	0.1\\
};
\addlegendentry{$\text{Wartość zadana y}_{\text{zad}}$}

\end{axis}

\begin{axis}[%
width=4.521in,
height=1.493in,
at={(0.758in,0.481in)},
scale only axis,
xmin=1,
xmax=1800,
xlabel style={font=\color{white!15!black}},
xlabel={k},
ymin=-1,
ymax=0.5,
ylabel style={font=\color{white!15!black}},
ylabel={u},
axis background/.style={fill=white},
xmajorgrids,
ymajorgrids,
legend style={legend cell align=left, align=left, draw=white!15!black}
]
\addplot [color=mycolor1]
  table[row sep=crcr]{%
1	0\\
2	0\\
3	0\\
4	0\\
5	0\\
6	0\\
7	0\\
8	0\\
9	0\\
10	0\\
11	0\\
12	0\\
13	0\\
14	0\\
15	0\\
16	0\\
17	0\\
18	0\\
19	0\\
20	-0.42971\\
21	-0.83288\\
22	-1\\
23	-1\\
24	-1\\
25	-1\\
26	-1\\
27	-0.94049\\
28	-0.88125\\
29	-0.80814\\
30	-0.73973\\
31	-0.67124\\
32	-0.61185\\
33	-0.56107\\
34	-0.52342\\
35	-0.49875\\
36	-0.48836\\
37	-0.49092\\
38	-0.50509\\
39	-0.52788\\
40	-0.55606\\
41	-0.58583\\
42	-0.6138\\
43	-0.6371\\
44	-0.65392\\
45	-0.66343\\
46	-0.66574\\
47	-0.6617\\
48	-0.65266\\
49	-0.6402\\
50	-0.62596\\
51	-0.61145\\
52	-0.59798\\
53	-0.58656\\
54	-0.57789\\
55	-0.57231\\
56	-0.56987\\
57	-0.57032\\
58	-0.57319\\
59	-0.57784\\
60	-0.58356\\
61	-0.58963\\
62	-0.59539\\
63	-0.60033\\
64	-0.60409\\
65	-0.60646\\
66	-0.60743\\
67	-0.60711\\
68	-0.60572\\
69	-0.60356\\
70	-0.60093\\
71	-0.59815\\
72	-0.59549\\
73	-0.59319\\
74	-0.5914\\
75	-0.59021\\
76	-0.58964\\
77	-0.58965\\
78	-0.59015\\
79	-0.59101\\
80	-0.59211\\
81	-0.59329\\
82	-0.59444\\
83	-0.59545\\
84	-0.59624\\
85	-0.59678\\
86	-0.59704\\
87	-0.59705\\
88	-0.59683\\
89	-0.59646\\
90	-0.59597\\
91	-0.59544\\
92	-0.59493\\
93	-0.59447\\
94	-0.5941\\
95	-0.59385\\
96	-0.59372\\
97	-0.5937\\
98	-0.59378\\
99	-0.59393\\
100	-0.59413\\
101	-0.59436\\
102	-0.59459\\
103	-0.59479\\
104	-0.59496\\
105	-0.59507\\
106	-0.59513\\
107	-0.59515\\
108	-0.59512\\
109	-0.59505\\
110	-0.59496\\
111	-0.59486\\
112	-0.59476\\
113	-0.59467\\
114	-0.5946\\
115	-0.59454\\
116	-0.59451\\
117	-0.5945\\
118	-0.59451\\
119	-0.59454\\
120	-0.59458\\
121	-0.59462\\
122	-0.59467\\
123	-0.59471\\
124	-0.59474\\
125	-0.59476\\
126	-0.59478\\
127	-0.59478\\
128	-0.59478\\
129	-0.59477\\
130	-0.59475\\
131	-0.59473\\
132	-0.59471\\
133	-0.5947\\
134	-0.59468\\
135	-0.59467\\
136	-0.59466\\
137	-0.59466\\
138	-0.59466\\
139	-0.59467\\
140	-0.59467\\
141	-0.59468\\
142	-0.59469\\
143	-0.5947\\
144	-0.5947\\
145	-0.59471\\
146	-0.59471\\
147	-0.59471\\
148	-0.59471\\
149	-0.59471\\
150	-0.59471\\
151	-0.59471\\
152	-0.5947\\
153	-0.5947\\
154	-0.5947\\
155	-0.59469\\
156	-0.59469\\
157	-0.59469\\
158	-0.59469\\
159	-0.59469\\
160	-0.59469\\
161	-0.59469\\
162	-0.5947\\
163	-0.5947\\
164	-0.5947\\
165	-0.5947\\
166	-0.5947\\
167	-0.5947\\
168	-0.5947\\
169	-0.5947\\
170	-0.5947\\
171	-0.5947\\
172	-0.5947\\
173	-0.5947\\
174	-0.5947\\
175	-0.5947\\
176	-0.5947\\
177	-0.5947\\
178	-0.5947\\
179	-0.5947\\
180	-0.5947\\
181	-0.5947\\
182	-0.5947\\
183	-0.5947\\
184	-0.5947\\
185	-0.5947\\
186	-0.5947\\
187	-0.5947\\
188	-0.5947\\
189	-0.5947\\
190	-0.5947\\
191	-0.5947\\
192	-0.5947\\
193	-0.5947\\
194	-0.5947\\
195	-0.5947\\
196	-0.5947\\
197	-0.5947\\
198	-0.5947\\
199	-0.5947\\
200	-1\\
201	-1\\
202	-1\\
203	-0.98196\\
204	-1\\
205	-0.95238\\
206	-0.93105\\
207	-0.88249\\
208	-0.85749\\
209	-0.83254\\
210	-0.82629\\
211	-0.82671\\
212	-0.83949\\
213	-0.8578\\
214	-0.88226\\
215	-0.90843\\
216	-0.93522\\
217	-0.95976\\
218	-0.98095\\
219	-0.99738\\
220	-1\\
221	-1\\
222	-1\\
223	-0.99706\\
224	-0.99085\\
225	-0.98272\\
226	-0.97399\\
227	-0.96562\\
228	-0.95827\\
229	-0.95237\\
230	-0.94816\\
231	-0.94572\\
232	-0.94497\\
233	-0.94575\\
234	-0.94777\\
235	-0.9507\\
236	-0.95421\\
237	-0.95793\\
238	-0.96156\\
239	-0.96482\\
240	-0.96753\\
241	-0.96956\\
242	-0.97085\\
243	-0.9714\\
244	-0.97129\\
245	-0.97062\\
246	-0.96952\\
247	-0.96813\\
248	-0.96661\\
249	-0.96509\\
250	-0.96368\\
251	-0.96248\\
252	-0.96154\\
253	-0.9609\\
254	-0.96056\\
255	-0.96051\\
256	-0.96071\\
257	-0.96111\\
258	-0.96165\\
259	-0.96227\\
260	-0.9629\\
261	-0.96351\\
262	-0.96405\\
263	-0.96448\\
264	-0.96479\\
265	-0.96497\\
266	-0.96503\\
267	-0.96498\\
268	-0.96484\\
269	-0.96463\\
270	-0.96439\\
271	-0.96412\\
272	-0.96386\\
273	-0.96363\\
274	-0.96343\\
275	-0.96329\\
276	-0.96319\\
277	-0.96315\\
278	-0.96316\\
279	-0.9632\\
280	-0.96328\\
281	-0.96338\\
282	-0.96349\\
283	-0.96359\\
284	-0.9637\\
285	-0.96378\\
286	-0.96385\\
287	-0.9639\\
288	-0.96392\\
289	-0.96393\\
290	-0.96391\\
291	-0.96389\\
292	-0.96385\\
293	-0.9638\\
294	-0.96376\\
295	-0.96371\\
296	-0.96368\\
297	-0.96365\\
298	-0.96362\\
299	-0.96361\\
300	-0.9636\\
301	-0.96361\\
302	-0.96362\\
303	-0.96363\\
304	-0.96365\\
305	-0.96367\\
306	-0.96369\\
307	-0.9637\\
308	-0.96372\\
309	-0.96373\\
310	-0.96374\\
311	-0.96374\\
312	-0.96374\\
313	-0.96373\\
314	-0.96373\\
315	-0.96372\\
316	-0.96371\\
317	-0.96371\\
318	-0.9637\\
319	-0.96369\\
320	-0.96369\\
321	-0.96369\\
322	-0.96368\\
323	-0.96368\\
324	-0.96368\\
325	-0.96369\\
326	-0.96369\\
327	-0.96369\\
328	-0.96369\\
329	-0.9637\\
330	-0.9637\\
331	-0.9637\\
332	-0.9637\\
333	-0.96371\\
334	-0.96371\\
335	-0.96371\\
336	-0.96371\\
337	-0.9637\\
338	-0.9637\\
339	-0.9637\\
340	-0.9637\\
341	-0.9637\\
342	-0.9637\\
343	-0.9637\\
344	-0.9637\\
345	-0.9637\\
346	-0.9637\\
347	-0.9637\\
348	-0.9637\\
349	-0.9637\\
350	-0.9637\\
351	-0.9637\\
352	-0.9637\\
353	-0.9637\\
354	-0.9637\\
355	-0.9637\\
356	-0.9637\\
357	-0.9637\\
358	-0.9637\\
359	-0.9637\\
360	-0.9637\\
361	-0.9637\\
362	-0.9637\\
363	-0.9637\\
364	-0.9637\\
365	-0.9637\\
366	-0.9637\\
367	-0.9637\\
368	-0.9637\\
369	-0.9637\\
370	-0.9637\\
371	-0.9637\\
372	-0.9637\\
373	-0.9637\\
374	-0.9637\\
375	-0.9637\\
376	-0.9637\\
377	-0.9637\\
378	-0.9637\\
379	-0.9637\\
380	-0.9637\\
381	-0.9637\\
382	-0.9637\\
383	-0.9637\\
384	-0.9637\\
385	-0.9637\\
386	-0.9637\\
387	-0.9637\\
388	-0.9637\\
389	-0.9637\\
390	-0.9637\\
391	-0.9637\\
392	-0.9637\\
393	-0.9637\\
394	-0.9637\\
395	-0.9637\\
396	-0.9637\\
397	-0.9637\\
398	-0.9637\\
399	-0.9637\\
400	-0.65005\\
401	-0.63908\\
402	-0.52044\\
403	-0.53126\\
404	-0.50445\\
405	-0.54156\\
406	-0.56952\\
407	-0.62906\\
408	-0.68341\\
409	-0.74573\\
410	-0.79679\\
411	-0.8405\\
412	-0.86872\\
413	-0.8843\\
414	-0.88587\\
415	-0.87689\\
416	-0.85907\\
417	-0.83586\\
418	-0.80968\\
419	-0.78326\\
420	-0.75858\\
421	-0.73738\\
422	-0.72074\\
423	-0.70937\\
424	-0.70343\\
425	-0.70267\\
426	-0.70645\\
427	-0.71385\\
428	-0.72372\\
429	-0.73488\\
430	-0.74615\\
431	-0.75652\\
432	-0.7652\\
433	-0.77166\\
434	-0.77566\\
435	-0.77722\\
436	-0.77656\\
437	-0.77406\\
438	-0.77022\\
439	-0.76555\\
440	-0.76057\\
441	-0.75574\\
442	-0.75144\\
443	-0.74795\\
444	-0.74544\\
445	-0.74399\\
446	-0.74356\\
447	-0.74405\\
448	-0.74527\\
449	-0.74703\\
450	-0.74909\\
451	-0.75123\\
452	-0.75326\\
453	-0.75502\\
454	-0.7564\\
455	-0.75733\\
456	-0.7578\\
457	-0.75784\\
458	-0.75751\\
459	-0.75688\\
460	-0.75606\\
461	-0.75515\\
462	-0.75422\\
463	-0.75337\\
464	-0.75266\\
465	-0.75212\\
466	-0.75178\\
467	-0.75163\\
468	-0.75166\\
469	-0.75185\\
470	-0.75215\\
471	-0.75252\\
472	-0.75292\\
473	-0.75331\\
474	-0.75366\\
475	-0.75395\\
476	-0.75416\\
477	-0.75428\\
478	-0.75432\\
479	-0.75428\\
480	-0.75418\\
481	-0.75404\\
482	-0.75387\\
483	-0.7537\\
484	-0.75353\\
485	-0.75339\\
486	-0.75327\\
487	-0.7532\\
488	-0.75315\\
489	-0.75315\\
490	-0.75317\\
491	-0.75322\\
492	-0.75329\\
493	-0.75336\\
494	-0.75344\\
495	-0.75351\\
496	-0.75357\\
497	-0.75361\\
498	-0.75364\\
499	-0.75365\\
500	-0.75365\\
501	-0.75363\\
502	-0.75361\\
503	-0.75358\\
504	-0.75355\\
505	-0.75352\\
506	-0.75349\\
507	-0.75346\\
508	-0.75345\\
509	-0.75344\\
510	-0.75343\\
511	-0.75344\\
512	-0.75344\\
513	-0.75346\\
514	-0.75347\\
515	-0.75348\\
516	-0.7535\\
517	-0.75351\\
518	-0.75352\\
519	-0.75352\\
520	-0.75353\\
521	-0.75353\\
522	-0.75353\\
523	-0.75352\\
524	-0.75352\\
525	-0.75351\\
526	-0.7535\\
527	-0.7535\\
528	-0.75349\\
529	-0.75349\\
530	-0.75349\\
531	-0.75349\\
532	-0.75349\\
533	-0.75349\\
534	-0.75349\\
535	-0.75349\\
536	-0.7535\\
537	-0.7535\\
538	-0.7535\\
539	-0.7535\\
540	-0.7535\\
541	-0.7535\\
542	-0.7535\\
543	-0.7535\\
544	-0.7535\\
545	-0.7535\\
546	-0.7535\\
547	-0.7535\\
548	-0.7535\\
549	-0.7535\\
550	-0.7535\\
551	-0.7535\\
552	-0.7535\\
553	-0.7535\\
554	-0.7535\\
555	-0.7535\\
556	-0.7535\\
557	-0.7535\\
558	-0.7535\\
559	-0.7535\\
560	-0.7535\\
561	-0.7535\\
562	-0.7535\\
563	-0.7535\\
564	-0.7535\\
565	-0.7535\\
566	-0.7535\\
567	-0.7535\\
568	-0.7535\\
569	-0.7535\\
570	-0.7535\\
571	-0.7535\\
572	-0.7535\\
573	-0.7535\\
574	-0.7535\\
575	-0.7535\\
576	-0.7535\\
577	-0.7535\\
578	-0.7535\\
579	-0.7535\\
580	-0.7535\\
581	-0.7535\\
582	-0.7535\\
583	-0.7535\\
584	-0.7535\\
585	-0.7535\\
586	-0.7535\\
587	-0.7535\\
588	-0.7535\\
589	-0.7535\\
590	-0.7535\\
591	-0.7535\\
592	-0.7535\\
593	-0.7535\\
594	-0.7535\\
595	-0.7535\\
596	-0.7535\\
597	-0.7535\\
598	-0.7535\\
599	-0.7535\\
600	-0.43985\\
601	-0.42888\\
602	-0.31024\\
603	-0.32106\\
604	-0.29425\\
605	-0.33319\\
606	-0.36292\\
607	-0.42451\\
608	-0.47838\\
609	-0.53722\\
610	-0.58093\\
611	-0.61427\\
612	-0.63018\\
613	-0.63311\\
614	-0.62304\\
615	-0.60459\\
616	-0.58017\\
617	-0.55362\\
618	-0.52735\\
619	-0.50389\\
620	-0.48476\\
621	-0.4711\\
622	-0.46331\\
623	-0.46129\\
624	-0.46436\\
625	-0.47149\\
626	-0.48133\\
627	-0.49243\\
628	-0.50341\\
629	-0.5131\\
630	-0.52067\\
631	-0.52563\\
632	-0.52787\\
633	-0.52758\\
634	-0.52518\\
635	-0.52125\\
636	-0.51642\\
637	-0.51129\\
638	-0.50641\\
639	-0.50223\\
640	-0.49903\\
641	-0.49697\\
642	-0.49608\\
643	-0.49626\\
644	-0.49732\\
645	-0.49901\\
646	-0.50106\\
647	-0.5032\\
648	-0.5052\\
649	-0.50686\\
650	-0.50808\\
651	-0.50879\\
652	-0.509\\
653	-0.50876\\
654	-0.50817\\
655	-0.50735\\
656	-0.5064\\
657	-0.50545\\
658	-0.50458\\
659	-0.50388\\
660	-0.50338\\
661	-0.50311\\
662	-0.50305\\
663	-0.50317\\
664	-0.50344\\
665	-0.50381\\
666	-0.50421\\
667	-0.50461\\
668	-0.50496\\
669	-0.50524\\
670	-0.50543\\
671	-0.50552\\
672	-0.50551\\
673	-0.50544\\
674	-0.5053\\
675	-0.50513\\
676	-0.50495\\
677	-0.50478\\
678	-0.50463\\
679	-0.50452\\
680	-0.50444\\
681	-0.50441\\
682	-0.50442\\
683	-0.50446\\
684	-0.50452\\
685	-0.50459\\
686	-0.50467\\
687	-0.50474\\
688	-0.5048\\
689	-0.50485\\
690	-0.50488\\
691	-0.50488\\
692	-0.50488\\
693	-0.50486\\
694	-0.50483\\
695	-0.50479\\
696	-0.50476\\
697	-0.50473\\
698	-0.5047\\
699	-0.50469\\
700	-0.50468\\
701	-0.50467\\
702	-0.50468\\
703	-0.50469\\
704	-0.5047\\
705	-0.50472\\
706	-0.50473\\
707	-0.50474\\
708	-0.50475\\
709	-0.50476\\
710	-0.50476\\
711	-0.50476\\
712	-0.50476\\
713	-0.50476\\
714	-0.50475\\
715	-0.50474\\
716	-0.50474\\
717	-0.50473\\
718	-0.50473\\
719	-0.50473\\
720	-0.50473\\
721	-0.50473\\
722	-0.50473\\
723	-0.50473\\
724	-0.50473\\
725	-0.50473\\
726	-0.50474\\
727	-0.50474\\
728	-0.50474\\
729	-0.50474\\
730	-0.50474\\
731	-0.50474\\
732	-0.50474\\
733	-0.50474\\
734	-0.50474\\
735	-0.50474\\
736	-0.50474\\
737	-0.50474\\
738	-0.50473\\
739	-0.50473\\
740	-0.50473\\
741	-0.50473\\
742	-0.50473\\
743	-0.50474\\
744	-0.50474\\
745	-0.50474\\
746	-0.50474\\
747	-0.50474\\
748	-0.50474\\
749	-0.50474\\
750	-0.50474\\
751	-0.50474\\
752	-0.50474\\
753	-0.50474\\
754	-0.50474\\
755	-0.50474\\
756	-0.50474\\
757	-0.50474\\
758	-0.50474\\
759	-0.50474\\
760	-0.50474\\
761	-0.50474\\
762	-0.50474\\
763	-0.50474\\
764	-0.50474\\
765	-0.50474\\
766	-0.50474\\
767	-0.50474\\
768	-0.50474\\
769	-0.50474\\
770	-0.50474\\
771	-0.50474\\
772	-0.50474\\
773	-0.50474\\
774	-0.50474\\
775	-0.50474\\
776	-0.50474\\
777	-0.50474\\
778	-0.50474\\
779	-0.50474\\
780	-0.50474\\
781	-0.50474\\
782	-0.50474\\
783	-0.50474\\
784	-0.50474\\
785	-0.50474\\
786	-0.50474\\
787	-0.50474\\
788	-0.50474\\
789	-0.50474\\
790	-0.50474\\
791	-0.50474\\
792	-0.50474\\
793	-0.50474\\
794	-0.50474\\
795	-0.50474\\
796	-0.50474\\
797	-0.50474\\
798	-0.50474\\
799	-0.50474\\
800	-0.92293\\
801	-0.93756\\
802	-1\\
803	-0.98557\\
804	-1\\
805	-0.95843\\
806	-0.93347\\
807	-0.88038\\
808	-0.84061\\
809	-0.79704\\
810	-0.76668\\
811	-0.74192\\
812	-0.72907\\
813	-0.7241\\
814	-0.72856\\
815	-0.73949\\
816	-0.75613\\
817	-0.77579\\
818	-0.79684\\
819	-0.81708\\
820	-0.83508\\
821	-0.84955\\
822	-0.85988\\
823	-0.8658\\
824	-0.86753\\
825	-0.86557\\
826	-0.86068\\
827	-0.8537\\
828	-0.84553\\
829	-0.837\\
830	-0.82885\\
831	-0.82167\\
832	-0.81589\\
833	-0.81179\\
834	-0.80945\\
835	-0.80881\\
836	-0.80968\\
837	-0.8118\\
838	-0.81479\\
839	-0.81831\\
840	-0.82198\\
841	-0.82549\\
842	-0.82856\\
843	-0.83101\\
844	-0.83272\\
845	-0.83366\\
846	-0.83387\\
847	-0.83344\\
848	-0.83249\\
849	-0.83118\\
850	-0.82966\\
851	-0.82809\\
852	-0.82661\\
853	-0.82532\\
854	-0.8243\\
855	-0.82359\\
856	-0.82321\\
857	-0.82314\\
858	-0.82334\\
859	-0.82375\\
860	-0.82432\\
861	-0.82497\\
862	-0.82564\\
863	-0.82627\\
864	-0.82682\\
865	-0.82725\\
866	-0.82754\\
867	-0.82769\\
868	-0.82771\\
869	-0.82762\\
870	-0.82743\\
871	-0.82718\\
872	-0.8269\\
873	-0.82662\\
874	-0.82635\\
875	-0.82612\\
876	-0.82594\\
877	-0.82582\\
878	-0.82576\\
879	-0.82576\\
880	-0.8258\\
881	-0.82588\\
882	-0.82599\\
883	-0.82611\\
884	-0.82623\\
885	-0.82634\\
886	-0.82644\\
887	-0.82652\\
888	-0.82657\\
889	-0.82659\\
890	-0.82659\\
891	-0.82657\\
892	-0.82653\\
893	-0.82649\\
894	-0.82644\\
895	-0.82638\\
896	-0.82634\\
897	-0.82629\\
898	-0.82626\\
899	-0.82624\\
900	-0.82623\\
901	-0.82623\\
902	-0.82624\\
903	-0.82626\\
904	-0.82628\\
905	-0.8263\\
906	-0.82632\\
907	-0.82634\\
908	-0.82636\\
909	-0.82637\\
910	-0.82638\\
911	-0.82639\\
912	-0.82639\\
913	-0.82638\\
914	-0.82637\\
915	-0.82637\\
916	-0.82636\\
917	-0.82635\\
918	-0.82634\\
919	-0.82633\\
920	-0.82633\\
921	-0.82632\\
922	-0.82632\\
923	-0.82632\\
924	-0.82632\\
925	-0.82633\\
926	-0.82633\\
927	-0.82633\\
928	-0.82634\\
929	-0.82634\\
930	-0.82634\\
931	-0.82635\\
932	-0.82635\\
933	-0.82635\\
934	-0.82635\\
935	-0.82635\\
936	-0.82635\\
937	-0.82634\\
938	-0.82634\\
939	-0.82634\\
940	-0.82634\\
941	-0.82634\\
942	-0.82634\\
943	-0.82634\\
944	-0.82634\\
945	-0.82634\\
946	-0.82634\\
947	-0.82634\\
948	-0.82634\\
949	-0.82634\\
950	-0.82634\\
951	-0.82634\\
952	-0.82634\\
953	-0.82634\\
954	-0.82634\\
955	-0.82634\\
956	-0.82634\\
957	-0.82634\\
958	-0.82634\\
959	-0.82634\\
960	-0.82634\\
961	-0.82634\\
962	-0.82634\\
963	-0.82634\\
964	-0.82634\\
965	-0.82634\\
966	-0.82634\\
967	-0.82634\\
968	-0.82634\\
969	-0.82634\\
970	-0.82634\\
971	-0.82634\\
972	-0.82634\\
973	-0.82634\\
974	-0.82634\\
975	-0.82634\\
976	-0.82634\\
977	-0.82634\\
978	-0.82634\\
979	-0.82634\\
980	-0.82634\\
981	-0.82634\\
982	-0.82634\\
983	-0.82634\\
984	-0.82634\\
985	-0.82634\\
986	-0.82634\\
987	-0.82634\\
988	-0.82634\\
989	-0.82634\\
990	-0.82634\\
991	-0.82634\\
992	-0.82634\\
993	-0.82634\\
994	-0.82634\\
995	-0.82634\\
996	-0.82634\\
997	-0.82634\\
998	-0.82634\\
999	-0.82634\\
1000	-0.34541\\
1001	-0.32859\\
1002	-0.14667\\
1003	-0.16327\\
1004	-0.12215\\
1005	-0.18695\\
1006	-0.23934\\
1007	-0.34594\\
1008	-0.43759\\
1009	-0.53298\\
1010	-0.59806\\
1011	-0.64213\\
1012	-0.65602\\
1013	-0.64937\\
1014	-0.6239\\
1015	-0.58803\\
1016	-0.54588\\
1017	-0.50333\\
1018	-0.46369\\
1019	-0.43029\\
1020	-0.40487\\
1021	-0.38867\\
1022	-0.38178\\
1023	-0.38361\\
1024	-0.39269\\
1025	-0.407\\
1026	-0.42413\\
1027	-0.44168\\
1028	-0.45751\\
1029	-0.47006\\
1030	-0.4784\\
1031	-0.48228\\
1032	-0.48198\\
1033	-0.47823\\
1034	-0.47199\\
1035	-0.46429\\
1036	-0.45615\\
1037	-0.44847\\
1038	-0.44192\\
1039	-0.43699\\
1040	-0.4339\\
1041	-0.43267\\
1042	-0.4331\\
1043	-0.43488\\
1044	-0.43759\\
1045	-0.44077\\
1046	-0.444\\
1047	-0.4469\\
1048	-0.44921\\
1049	-0.45077\\
1050	-0.45152\\
1051	-0.45152\\
1052	-0.45087\\
1053	-0.44976\\
1054	-0.44836\\
1055	-0.44688\\
1056	-0.44547\\
1057	-0.44427\\
1058	-0.44337\\
1059	-0.44281\\
1060	-0.44259\\
1061	-0.44268\\
1062	-0.44301\\
1063	-0.44351\\
1064	-0.4441\\
1065	-0.44469\\
1066	-0.44522\\
1067	-0.44564\\
1068	-0.44593\\
1069	-0.44607\\
1070	-0.44607\\
1071	-0.44596\\
1072	-0.44575\\
1073	-0.4455\\
1074	-0.44523\\
1075	-0.44498\\
1076	-0.44476\\
1077	-0.4446\\
1078	-0.4445\\
1079	-0.44446\\
1080	-0.44448\\
1081	-0.44454\\
1082	-0.44463\\
1083	-0.44474\\
1084	-0.44485\\
1085	-0.44494\\
1086	-0.44502\\
1087	-0.44507\\
1088	-0.4451\\
1089	-0.4451\\
1090	-0.44508\\
1091	-0.44504\\
1092	-0.445\\
1093	-0.44495\\
1094	-0.4449\\
1095	-0.44486\\
1096	-0.44483\\
1097	-0.44481\\
1098	-0.44481\\
1099	-0.44481\\
1100	-0.44482\\
1101	-0.44484\\
1102	-0.44486\\
1103	-0.44488\\
1104	-0.4449\\
1105	-0.44491\\
1106	-0.44492\\
1107	-0.44492\\
1108	-0.44492\\
1109	-0.44492\\
1110	-0.44491\\
1111	-0.44491\\
1112	-0.4449\\
1113	-0.44489\\
1114	-0.44488\\
1115	-0.44488\\
1116	-0.44487\\
1117	-0.44487\\
1118	-0.44487\\
1119	-0.44487\\
1120	-0.44488\\
1121	-0.44488\\
1122	-0.44488\\
1123	-0.44489\\
1124	-0.44489\\
1125	-0.44489\\
1126	-0.44489\\
1127	-0.44489\\
1128	-0.44489\\
1129	-0.44489\\
1130	-0.44489\\
1131	-0.44489\\
1132	-0.44489\\
1133	-0.44489\\
1134	-0.44488\\
1135	-0.44488\\
1136	-0.44488\\
1137	-0.44488\\
1138	-0.44488\\
1139	-0.44488\\
1140	-0.44489\\
1141	-0.44489\\
1142	-0.44489\\
1143	-0.44489\\
1144	-0.44489\\
1145	-0.44489\\
1146	-0.44489\\
1147	-0.44489\\
1148	-0.44489\\
1149	-0.44489\\
1150	-0.44489\\
1151	-0.44489\\
1152	-0.44489\\
1153	-0.44489\\
1154	-0.44489\\
1155	-0.44489\\
1156	-0.44489\\
1157	-0.44489\\
1158	-0.44489\\
1159	-0.44489\\
1160	-0.44489\\
1161	-0.44489\\
1162	-0.44489\\
1163	-0.44489\\
1164	-0.44489\\
1165	-0.44489\\
1166	-0.44489\\
1167	-0.44489\\
1168	-0.44489\\
1169	-0.44489\\
1170	-0.44489\\
1171	-0.44489\\
1172	-0.44489\\
1173	-0.44489\\
1174	-0.44489\\
1175	-0.44489\\
1176	-0.44489\\
1177	-0.44489\\
1178	-0.44489\\
1179	-0.44489\\
1180	-0.44489\\
1181	-0.44489\\
1182	-0.44489\\
1183	-0.44489\\
1184	-0.44489\\
1185	-0.44489\\
1186	-0.44489\\
1187	-0.44489\\
1188	-0.44489\\
1189	-0.44489\\
1190	-0.44489\\
1191	-0.44489\\
1192	-0.44489\\
1193	-0.44489\\
1194	-0.44489\\
1195	-0.44489\\
1196	-0.44489\\
1197	-0.44489\\
1198	-0.44489\\
1199	-0.44489\\
1200	-0.2567\\
1201	-0.25011\\
1202	-0.17893\\
1203	-0.18543\\
1204	-0.16933\\
1205	-0.1913\\
1206	-0.20575\\
1207	-0.23508\\
1208	-0.25676\\
1209	-0.27836\\
1210	-0.29016\\
1211	-0.29612\\
1212	-0.29376\\
1213	-0.28646\\
1214	-0.27488\\
1215	-0.26175\\
1216	-0.24835\\
1217	-0.23641\\
1218	-0.22671\\
1219	-0.21989\\
1220	-0.21594\\
1221	-0.21462\\
1222	-0.21531\\
1223	-0.21732\\
1224	-0.21987\\
1225	-0.22231\\
1226	-0.22413\\
1227	-0.22503\\
1228	-0.22493\\
1229	-0.22388\\
1230	-0.22209\\
1231	-0.21982\\
1232	-0.21734\\
1233	-0.21491\\
1234	-0.21272\\
1235	-0.2109\\
1236	-0.20951\\
1237	-0.20853\\
1238	-0.20791\\
1239	-0.20755\\
1240	-0.20736\\
1241	-0.20723\\
1242	-0.20707\\
1243	-0.20684\\
1244	-0.20649\\
1245	-0.20603\\
1246	-0.20548\\
1247	-0.20487\\
1248	-0.20424\\
1249	-0.20361\\
1250	-0.20302\\
1251	-0.20249\\
1252	-0.20203\\
1253	-0.20164\\
1254	-0.2013\\
1255	-0.20102\\
1256	-0.20077\\
1257	-0.20054\\
1258	-0.20032\\
1259	-0.2001\\
1260	-0.19987\\
1261	-0.19964\\
1262	-0.19941\\
1263	-0.19917\\
1264	-0.19894\\
1265	-0.19868\\
1266	-0.19843\\
1267	-0.19819\\
1268	-0.19797\\
1269	-0.19777\\
1270	-0.19758\\
1271	-0.1974\\
1272	-0.19724\\
1273	-0.19708\\
1274	-0.19694\\
1275	-0.19681\\
1276	-0.19668\\
1277	-0.19656\\
1278	-0.19644\\
1279	-0.19632\\
1280	-0.19621\\
1281	-0.1961\\
1282	-0.196\\
1283	-0.1959\\
1284	-0.19585\\
1285	-0.1958\\
1286	-0.19575\\
1287	-0.19571\\
1288	-0.19568\\
1289	-0.19565\\
1290	-0.19561\\
1291	-0.19558\\
1292	-0.19555\\
1293	-0.19551\\
1294	-0.19548\\
1295	-0.19544\\
1296	-0.19541\\
1297	-0.19537\\
1298	-0.19534\\
1299	-0.19532\\
1300	-0.19529\\
1301	-0.19527\\
1302	-0.19524\\
1303	-0.19522\\
1304	-0.19521\\
1305	-0.19519\\
1306	-0.19517\\
1307	-0.19515\\
1308	-0.19514\\
1309	-0.19512\\
1310	-0.19511\\
1311	-0.19509\\
1312	-0.19508\\
1313	-0.19506\\
1314	-0.19505\\
1315	-0.19504\\
1316	-0.19503\\
1317	-0.19502\\
1318	-0.19501\\
1319	-0.195\\
1320	-0.19499\\
1321	-0.19498\\
1322	-0.19497\\
1323	-0.19497\\
1324	-0.19496\\
1325	-0.19495\\
1326	-0.19494\\
1327	-0.19494\\
1328	-0.19493\\
1329	-0.19493\\
1330	-0.19492\\
1331	-0.19491\\
1332	-0.19491\\
1333	-0.1949\\
1334	-0.1949\\
1335	-0.1949\\
1336	-0.19489\\
1337	-0.19489\\
1338	-0.19488\\
1339	-0.19488\\
1340	-0.19488\\
1341	-0.19487\\
1342	-0.19487\\
1343	-0.19487\\
1344	-0.19487\\
1345	-0.19486\\
1346	-0.19486\\
1347	-0.19486\\
1348	-0.19486\\
1349	-0.19485\\
1350	-0.19485\\
1351	-0.19485\\
1352	-0.19485\\
1353	-0.19485\\
1354	-0.19484\\
1355	-0.19484\\
1356	-0.19484\\
1357	-0.19484\\
1358	-0.19484\\
1359	-0.19484\\
1360	-0.19484\\
1361	-0.19483\\
1362	-0.19483\\
1363	-0.19483\\
1364	-0.19483\\
1365	-0.19483\\
1366	-0.19483\\
1367	-0.19483\\
1368	-0.19483\\
1369	-0.19483\\
1370	-0.19483\\
1371	-0.19482\\
1372	-0.19482\\
1373	-0.19482\\
1374	-0.19482\\
1375	-0.19482\\
1376	-0.19482\\
1377	-0.19482\\
1378	-0.19482\\
1379	-0.19482\\
1380	-0.19482\\
1381	-0.19482\\
1382	-0.19482\\
1383	-0.19482\\
1384	-0.19482\\
1385	-0.19482\\
1386	-0.19482\\
1387	-0.19482\\
1388	-0.19482\\
1389	-0.19482\\
1390	-0.19482\\
1391	-0.19482\\
1392	-0.19482\\
1393	-0.19482\\
1394	-0.19481\\
1395	-0.19481\\
1396	-0.19481\\
1397	-0.19481\\
1398	-0.19481\\
1399	-0.19481\\
1400	-0.1183\\
1401	-0.12627\\
1402	-0.093302\\
1403	-0.10601\\
1404	-0.095489\\
1405	-0.10964\\
1406	-0.10895\\
1407	-0.1195\\
1408	-0.11924\\
1409	-0.11751\\
1410	-0.10265\\
1411	-0.095352\\
1412	-0.087059\\
1413	-0.063979\\
1414	-0.040627\\
1415	-0.017293\\
1416	0.0055006\\
1417	0.027282\\
1418	0.047313\\
1419	0.064719\\
1420	0.078785\\
1421	0.089084\\
1422	0.095501\\
1423	0.098192\\
1424	0.097515\\
1425	0.093955\\
1426	0.088047\\
1427	0.080324\\
1428	0.071284\\
1429	0.058031\\
1430	0.044377\\
1431	0.027603\\
1432	0.011292\\
1433	-0.0042518\\
1434	-0.018706\\
1435	-0.029614\\
1436	-0.03904\\
1437	-0.045469\\
1438	-0.050286\\
1439	-0.053267\\
1440	-0.05428\\
1441	-0.0533\\
1442	-0.050409\\
1443	-0.045778\\
1444	-0.039645\\
1445	-0.032297\\
1446	-0.024063\\
1447	-0.015299\\
1448	-0.0063794\\
1449	0.0023291\\
1450	0.010483\\
1451	0.017783\\
1452	0.023988\\
1453	0.028929\\
1454	0.03251\\
1455	0.034705\\
1456	0.035557\\
1457	0.035159\\
1458	0.033645\\
1459	0.031178\\
1460	0.027936\\
1461	0.0241\\
1462	0.019851\\
1463	0.01536\\
1464	0.010788\\
1465	0.0062795\\
1466	0.0019647\\
1467	-0.0020436\\
1468	-0.0056498\\
1469	-0.008776\\
1470	-0.011363\\
1471	-0.013368\\
1472	-0.014771\\
1473	-0.015566\\
1474	-0.015768\\
1475	-0.015409\\
1476	-0.014537\\
1477	-0.013213\\
1478	-0.011512\\
1479	-0.009518\\
1480	-0.0073192\\
1481	-0.0050076\\
1482	-0.0026733\\
1483	-0.00040214\\
1484	0.0017284\\
1485	0.0036512\\
1486	0.005312\\
1487	0.0066702\\
1488	0.0077\\
1489	0.0083898\\
1490	0.0087417\\
1491	0.0087698\\
1492	0.0084989\\
1493	0.007962\\
1494	0.0071988\\
1495	0.0062533\\
1496	0.005172\\
1497	0.0040017\\
1498	0.0027886\\
1499	0.0015764\\
1500	0.00040503\\
1501	-0.00068971\\
1502	-0.0016774\\
1503	-0.0025333\\
1504	-0.0032388\\
1505	-0.0037813\\
1506	-0.0041547\\
1507	-0.0043585\\
1508	-0.004398\\
1509	-0.0042837\\
1510	-0.0040304\\
1511	-0.0036565\\
1512	-0.0031834\\
1513	-0.0026345\\
1514	-0.0020341\\
1515	-0.0014067\\
1516	-0.00077586\\
1517	-0.00016378\\
1518	0.00040966\\
1519	0.00092737\\
1520	0.0013755\\
1521	0.0017436\\
1522	0.0020249\\
1523	0.0022161\\
1524	0.0023175\\
1525	0.0023322\\
1526	0.0022663\\
1527	0.0021281\\
1528	0.0019278\\
1529	0.0016769\\
1530	0.0013878\\
1531	0.0010733\\
1532	0.0007458\\
1533	0.00041752\\
1534	9.9622e-05\\
1535	-0.0001979\\
1536	-0.00046648\\
1537	-0.00069916\\
1538	-0.00089069\\
1539	-0.0010376\\
1540	-0.0011382\\
1541	-0.0011925\\
1542	-0.001202\\
1543	-0.0011696\\
1544	-0.0010996\\
1545	-0.00099705\\
1546	-0.00086787\\
1547	-0.00071843\\
1548	-0.00055533\\
1549	-0.00038517\\
1550	-0.00021429\\
1551	-4.8626e-05\\
1552	0.00010653\\
1553	0.00024662\\
1554	0.00036796\\
1555	0.00046775\\
1556	0.00054418\\
1557	0.00059636\\
1558	0.00062429\\
1559	0.00062882\\
1560	0.00061154\\
1561	0.00057466\\
1562	0.00052089\\
1563	0.00045332\\
1564	0.0003753\\
1565	0.00029026\\
1566	0.00020163\\
1567	0.00011269\\
1568	2.6503e-05\\
1569	-5.4198e-05\\
1570	-0.00012707\\
1571	-0.0001902\\
1572	-0.00024216\\
1573	-0.00028199\\
1574	-0.00030925\\
1575	-0.00032392\\
1576	-0.00032644\\
1577	-0.0003176\\
1578	-0.00029856\\
1579	-0.00027071\\
1580	-0.00023566\\
1581	-0.00019513\\
1582	-0.00015092\\
1583	-0.00010481\\
1584	-5.8513e-05\\
1585	-1.3634e-05\\
1586	2.8398e-05\\
1587	6.6356e-05\\
1588	9.9241e-05\\
1589	0.0001263\\
1590	0.00014705\\
1591	0.00016123\\
1592	0.00016886\\
1593	0.00017015\\
1594	0.00016553\\
1595	0.00015559\\
1596	0.00014107\\
1597	0.00012281\\
1598	0.0001017\\
1599	7.8684e-05\\
1600	0.02154\\
1601	0.041674\\
1602	0.061213\\
1603	0.08015\\
1604	0.0985\\
1605	0.11609\\
1606	0.13253\\
1607	0.14747\\
1608	0.16067\\
1609	0.17204\\
1610	0.18162\\
1611	0.19082\\
1612	0.1992\\
1613	0.20584\\
1614	0.21165\\
1615	0.21669\\
1616	0.22073\\
1617	0.22399\\
1618	0.22683\\
1619	0.22909\\
1620	0.23085\\
1621	0.2322\\
1622	0.2332\\
1623	0.23393\\
1624	0.23441\\
1625	0.23469\\
1626	0.23483\\
1627	0.23484\\
1628	0.23477\\
1629	0.23464\\
1630	0.23447\\
1631	0.23427\\
1632	0.23406\\
1633	0.23385\\
1634	0.23364\\
1635	0.23344\\
1636	0.23325\\
1637	0.23309\\
1638	0.23293\\
1639	0.2328\\
1640	0.23268\\
1641	0.23258\\
1642	0.2325\\
1643	0.23243\\
1644	0.23237\\
1645	0.23233\\
1646	0.23229\\
1647	0.23227\\
1648	0.23225\\
1649	0.23224\\
1650	0.23223\\
1651	0.23222\\
1652	0.23222\\
1653	0.23222\\
1654	0.23223\\
1655	0.23223\\
1656	0.23224\\
1657	0.23224\\
1658	0.23225\\
1659	0.23225\\
1660	0.23226\\
1661	0.23226\\
1662	0.23227\\
1663	0.23227\\
1664	0.23228\\
1665	0.23228\\
1666	0.23228\\
1667	0.23228\\
1668	0.23229\\
1669	0.23229\\
1670	0.23229\\
1671	0.23229\\
1672	0.23229\\
1673	0.23229\\
1674	0.23229\\
1675	0.23229\\
1676	0.23229\\
1677	0.23229\\
1678	0.23229\\
1679	0.23229\\
1680	0.23229\\
1681	0.23229\\
1682	0.23229\\
1683	0.23229\\
1684	0.23229\\
1685	0.23229\\
1686	0.23229\\
1687	0.23229\\
1688	0.23229\\
1689	0.23229\\
1690	0.23229\\
1691	0.23229\\
1692	0.23229\\
1693	0.23229\\
1694	0.23229\\
1695	0.23229\\
1696	0.23229\\
1697	0.23229\\
1698	0.23229\\
1699	0.23229\\
1700	0.23229\\
1701	0.23229\\
1702	0.23229\\
1703	0.23229\\
1704	0.23229\\
1705	0.23229\\
1706	0.23229\\
1707	0.23229\\
1708	0.23229\\
1709	0.23229\\
1710	0.23229\\
1711	0.23229\\
1712	0.23229\\
1713	0.23229\\
1714	0.23229\\
1715	0.23229\\
1716	0.23229\\
1717	0.23229\\
1718	0.23229\\
1719	0.23229\\
1720	0.23229\\
1721	0.23229\\
1722	0.23229\\
1723	0.23229\\
1724	0.23229\\
1725	0.23229\\
1726	0.23229\\
1727	0.23229\\
1728	0.23229\\
1729	0.23229\\
1730	0.23229\\
1731	0.23229\\
1732	0.23229\\
1733	0.23229\\
1734	0.23229\\
1735	0.23229\\
1736	0.23229\\
1737	0.23229\\
1738	0.23229\\
1739	0.23229\\
1740	0.23229\\
1741	0.23229\\
1742	0.23229\\
1743	0.23229\\
1744	0.23229\\
1745	0.23229\\
1746	0.23229\\
1747	0.23229\\
1748	0.23229\\
1749	0.23229\\
1750	0.23229\\
1751	0.23229\\
1752	0.23229\\
1753	0.23229\\
1754	0.23229\\
1755	0.23229\\
1756	0.23229\\
1757	0.23229\\
1758	0.23229\\
1759	0.23229\\
1760	0.23229\\
1761	0.23229\\
1762	0.23229\\
1763	0.23229\\
1764	0.23229\\
1765	0.23229\\
1766	0.23229\\
1767	0.23229\\
1768	0.23229\\
1769	0.23229\\
1770	0.23229\\
1771	0.23229\\
1772	0.23229\\
1773	0.23229\\
1774	0.23229\\
1775	0.23229\\
1776	0.23229\\
1777	0.23229\\
1778	0.23229\\
1779	0.23229\\
1780	0.23229\\
1781	0.23229\\
1782	0.23229\\
1783	0.23229\\
1784	0.23229\\
1785	0.23229\\
1786	0.23229\\
1787	0.23229\\
1788	0.23229\\
1789	0.23229\\
1790	0.23229\\
1791	0.23229\\
1792	0.23229\\
1793	0.23229\\
1794	0.23229\\
1795	0.23229\\
1796	0.23229\\
1797	0.23229\\
1798	0.23229\\
1799	0.23229\\
1800	0.23229\\
};
\addlegendentry{Sterowanie u}

\end{axis}
\end{tikzpicture}%
   \caption{Cztery regulatory lokalne DMC}
   \label{projekt:zad7:DMC:4:figure}
\end{figure}

\begin{figure}[H] 
   \centering
   % This file was created by matlab2tikz.
%
\definecolor{mycolor1}{rgb}{0.00000,0.44700,0.74100}%
\definecolor{mycolor2}{rgb}{0.85000,0.32500,0.09800}%
%
\begin{tikzpicture}

\begin{axis}[%
width=4.521in,
height=1.493in,
at={(0.758in,2.554in)},
scale only axis,
xmin=1,
xmax=1800,
xlabel style={font=\color{white!15!black}},
xlabel={k},
ymin=-4.5167,
ymax=0.10216,
ylabel style={font=\color{white!15!black}},
ylabel={y},
axis background/.style={fill=white},
title style={font=\bfseries, align=center},
title={E=220.2797\\[1ex]N= [70         18         25         25         25]\\[1ex]$\text{N}_\text{u}\text{= [5         16          6         10         10]}$\\[1ex]lambda= [18          7         10         10          1]},
xmajorgrids,
ymajorgrids,
legend style={legend cell align=left, align=left, draw=white!15!black}
]
\addplot [color=mycolor1]
  table[row sep=crcr]{%
1	0\\
2	0\\
3	0\\
4	0\\
5	0\\
6	0\\
7	0\\
8	0\\
9	0\\
10	0\\
11	0\\
12	0\\
13	0\\
14	0\\
15	0\\
16	0\\
17	0\\
18	0\\
19	0\\
20	0\\
21	0\\
22	0\\
23	0\\
24	0\\
25	-0.041775\\
26	-0.19537\\
27	-0.4731\\
28	-0.83396\\
29	-1.2324\\
30	-1.6377\\
31	-2.0301\\
32	-2.386\\
33	-2.685\\
34	-2.9122\\
35	-3.0597\\
36	-3.1252\\
37	-3.1132\\
38	-3.0333\\
39	-2.9005\\
40	-2.733\\
41	-2.5499\\
42	-2.3692\\
43	-2.2056\\
44	-2.0698\\
45	-1.968\\
46	-1.9024\\
47	-1.8716\\
48	-1.8715\\
49	-1.8961\\
50	-1.9382\\
51	-1.9902\\
52	-2.0448\\
53	-2.0957\\
54	-2.1376\\
55	-2.1672\\
56	-2.1828\\
57	-2.1842\\
58	-2.1727\\
59	-2.1509\\
60	-2.1218\\
61	-2.0889\\
62	-2.0555\\
63	-2.0245\\
64	-1.9982\\
65	-1.9781\\
66	-1.9649\\
67	-1.9586\\
68	-1.9585\\
69	-1.9636\\
70	-1.9724\\
71	-1.9835\\
72	-1.9954\\
73	-2.0066\\
74	-2.0162\\
75	-2.0235\\
76	-2.028\\
77	-2.0296\\
78	-2.0286\\
79	-2.0254\\
80	-2.0207\\
81	-2.0149\\
82	-2.0089\\
83	-2.0032\\
84	-1.9982\\
85	-1.9943\\
86	-1.9917\\
87	-1.9904\\
88	-1.9903\\
89	-1.9912\\
90	-1.993\\
91	-1.9952\\
92	-1.9976\\
93	-1.9999\\
94	-2.0019\\
95	-2.0035\\
96	-2.0046\\
97	-2.0051\\
98	-2.0051\\
99	-2.0046\\
100	-2.0038\\
101	-2.0028\\
102	-2.0017\\
103	-2.0006\\
104	-1.9996\\
105	-1.9989\\
106	-1.9983\\
107	-1.998\\
108	-1.998\\
109	-1.9981\\
110	-1.9985\\
111	-1.9989\\
112	-1.9993\\
113	-1.9998\\
114	-2.0002\\
115	-2.0006\\
116	-2.0008\\
117	-2.0009\\
118	-2.001\\
119	-2.0009\\
120	-2.0007\\
121	-2.0006\\
122	-2.0004\\
123	-2.0001\\
124	-2\\
125	-1.9998\\
126	-1.9997\\
127	-1.9996\\
128	-1.9996\\
129	-1.9996\\
130	-1.9997\\
131	-1.9998\\
132	-1.9998\\
133	-1.9999\\
134	-2\\
135	-2.0001\\
136	-2.0001\\
137	-2.0002\\
138	-2.0002\\
139	-2.0002\\
140	-2.0001\\
141	-2.0001\\
142	-2.0001\\
143	-2\\
144	-2\\
145	-2\\
146	-1.9999\\
147	-1.9999\\
148	-1.9999\\
149	-1.9999\\
150	-1.9999\\
151	-1.9999\\
152	-2\\
153	-2\\
154	-2\\
155	-2\\
156	-2\\
157	-2\\
158	-2\\
159	-2\\
160	-2\\
161	-2\\
162	-2\\
163	-2\\
164	-2\\
165	-2\\
166	-2\\
167	-2\\
168	-2\\
169	-2\\
170	-2\\
171	-2\\
172	-2\\
173	-2\\
174	-2\\
175	-2\\
176	-2\\
177	-2\\
178	-2\\
179	-2\\
180	-2\\
181	-2\\
182	-2\\
183	-2\\
184	-2\\
185	-2\\
186	-2\\
187	-2\\
188	-2\\
189	-2\\
190	-2\\
191	-2\\
192	-2\\
193	-2\\
194	-2\\
195	-2\\
196	-2\\
197	-2\\
198	-2\\
199	-2\\
200	-2\\
201	-2\\
202	-2\\
203	-2\\
204	-2\\
205	-2.0801\\
206	-2.2463\\
207	-2.4603\\
208	-2.6922\\
209	-2.9273\\
210	-3.1462\\
211	-3.3363\\
212	-3.4867\\
213	-3.5946\\
214	-3.6624\\
215	-3.6978\\
216	-3.7109\\
217	-3.7123\\
218	-3.712\\
219	-3.7181\\
220	-3.7361\\
221	-3.7694\\
222	-3.8188\\
223	-3.8833\\
224	-3.9601\\
225	-4.0437\\
226	-4.1281\\
227	-4.2096\\
228	-4.285\\
229	-4.3517\\
230	-4.4075\\
231	-4.4512\\
232	-4.4825\\
233	-4.5021\\
234	-4.5113\\
235	-4.5119\\
236	-4.5061\\
237	-4.496\\
238	-4.4838\\
239	-4.4714\\
240	-4.4601\\
241	-4.4513\\
242	-4.4456\\
243	-4.4433\\
244	-4.4443\\
245	-4.4484\\
246	-4.455\\
247	-4.4633\\
248	-4.4727\\
249	-4.4823\\
250	-4.4915\\
251	-4.4998\\
252	-4.5066\\
253	-4.5117\\
254	-4.5151\\
255	-4.5167\\
256	-4.5167\\
257	-4.5154\\
258	-4.513\\
259	-4.5099\\
260	-4.5065\\
261	-4.503\\
262	-4.4998\\
263	-4.4971\\
264	-4.4949\\
265	-4.4934\\
266	-4.4927\\
267	-4.4926\\
268	-4.493\\
269	-4.494\\
270	-4.4952\\
271	-4.4967\\
272	-4.4981\\
273	-4.4995\\
274	-4.5008\\
275	-4.5018\\
276	-4.5025\\
277	-4.5029\\
278	-4.5031\\
279	-4.503\\
280	-4.5026\\
281	-4.5022\\
282	-4.5016\\
283	-4.501\\
284	-4.5004\\
285	-4.4998\\
286	-4.4994\\
287	-4.4991\\
288	-4.4988\\
289	-4.4987\\
290	-4.4987\\
291	-4.4988\\
292	-4.499\\
293	-4.4993\\
294	-4.4995\\
295	-4.4998\\
296	-4.5\\
297	-4.5002\\
298	-4.5003\\
299	-4.5005\\
300	-4.5005\\
301	-4.5005\\
302	-4.5005\\
303	-4.5004\\
304	-4.5003\\
305	-4.5002\\
306	-4.5001\\
307	-4.5\\
308	-4.4999\\
309	-4.4999\\
310	-4.4998\\
311	-4.4998\\
312	-4.4998\\
313	-4.4998\\
314	-4.4998\\
315	-4.4998\\
316	-4.4999\\
317	-4.4999\\
318	-4.5\\
319	-4.5\\
320	-4.5\\
321	-4.5001\\
322	-4.5001\\
323	-4.5001\\
324	-4.5001\\
325	-4.5001\\
326	-4.5001\\
327	-4.5001\\
328	-4.5\\
329	-4.5\\
330	-4.5\\
331	-4.5\\
332	-4.5\\
333	-4.5\\
334	-4.5\\
335	-4.5\\
336	-4.5\\
337	-4.5\\
338	-4.5\\
339	-4.5\\
340	-4.5\\
341	-4.5\\
342	-4.5\\
343	-4.5\\
344	-4.5\\
345	-4.5\\
346	-4.5\\
347	-4.5\\
348	-4.5\\
349	-4.5\\
350	-4.5\\
351	-4.5\\
352	-4.5\\
353	-4.5\\
354	-4.5\\
355	-4.5\\
356	-4.5\\
357	-4.5\\
358	-4.5\\
359	-4.5\\
360	-4.5\\
361	-4.5\\
362	-4.5\\
363	-4.5\\
364	-4.5\\
365	-4.5\\
366	-4.5\\
367	-4.5\\
368	-4.5\\
369	-4.5\\
370	-4.5\\
371	-4.5\\
372	-4.5\\
373	-4.5\\
374	-4.5\\
375	-4.5\\
376	-4.5\\
377	-4.5\\
378	-4.5\\
379	-4.5\\
380	-4.5\\
381	-4.5\\
382	-4.5\\
383	-4.5\\
384	-4.5\\
385	-4.5\\
386	-4.5\\
387	-4.5\\
388	-4.5\\
389	-4.5\\
390	-4.5\\
391	-4.5\\
392	-4.5\\
393	-4.5\\
394	-4.5\\
395	-4.5\\
396	-4.5\\
397	-4.5\\
398	-4.5\\
399	-4.5\\
400	-4.5\\
401	-4.5\\
402	-4.5\\
403	-4.5\\
404	-4.5\\
405	-4.4111\\
406	-4.2403\\
407	-3.9844\\
408	-3.687\\
409	-3.3743\\
410	-3.0818\\
411	-2.8304\\
412	-2.6366\\
413	-2.507\\
414	-2.4427\\
415	-2.4387\\
416	-2.4865\\
417	-2.5744\\
418	-2.6895\\
419	-2.8183\\
420	-2.9481\\
421	-3.0679\\
422	-3.1688\\
423	-3.2448\\
424	-3.2926\\
425	-3.3119\\
426	-3.3048\\
427	-3.2753\\
428	-3.2289\\
429	-3.1718\\
430	-3.1103\\
431	-3.0502\\
432	-2.9963\\
433	-2.952\\
434	-2.9196\\
435	-2.9\\
436	-2.8928\\
437	-2.8966\\
438	-2.9095\\
439	-2.9289\\
440	-2.9522\\
441	-2.9768\\
442	-3.0003\\
443	-3.0209\\
444	-3.0371\\
445	-3.0482\\
446	-3.0539\\
447	-3.0545\\
448	-3.0506\\
449	-3.0432\\
450	-3.0334\\
451	-3.0223\\
452	-3.0111\\
453	-3.0007\\
454	-2.9918\\
455	-2.9851\\
456	-2.9807\\
457	-2.9786\\
458	-2.9787\\
459	-2.9807\\
460	-2.9841\\
461	-2.9884\\
462	-2.993\\
463	-2.9977\\
464	-3.0019\\
465	-3.0053\\
466	-3.0078\\
467	-3.0093\\
468	-3.0098\\
469	-3.0095\\
470	-3.0083\\
471	-3.0067\\
472	-3.0047\\
473	-3.0026\\
474	-3.0006\\
475	-2.9989\\
476	-2.9975\\
477	-2.9965\\
478	-2.996\\
479	-2.9959\\
480	-2.9961\\
481	-2.9967\\
482	-2.9974\\
483	-2.9983\\
484	-2.9992\\
485	-3\\
486	-3.0007\\
487	-3.0013\\
488	-3.0016\\
489	-3.0018\\
490	-3.0018\\
491	-3.0016\\
492	-3.0014\\
493	-3.001\\
494	-3.0006\\
495	-3.0002\\
496	-2.9999\\
497	-2.9996\\
498	-2.9994\\
499	-2.9993\\
500	-2.9992\\
501	-2.9992\\
502	-2.9993\\
503	-2.9995\\
504	-2.9996\\
505	-2.9998\\
506	-2.9999\\
507	-3.0001\\
508	-3.0002\\
509	-3.0003\\
510	-3.0003\\
511	-3.0003\\
512	-3.0003\\
513	-3.0003\\
514	-3.0002\\
515	-3.0001\\
516	-3.0001\\
517	-3\\
518	-2.9999\\
519	-2.9999\\
520	-2.9999\\
521	-2.9999\\
522	-2.9999\\
523	-2.9999\\
524	-2.9999\\
525	-2.9999\\
526	-2.9999\\
527	-3\\
528	-3\\
529	-3\\
530	-3\\
531	-3.0001\\
532	-3.0001\\
533	-3.0001\\
534	-3.0001\\
535	-3\\
536	-3\\
537	-3\\
538	-3\\
539	-3\\
540	-3\\
541	-3\\
542	-3\\
543	-3\\
544	-3\\
545	-3\\
546	-3\\
547	-3\\
548	-3\\
549	-3\\
550	-3\\
551	-3\\
552	-3\\
553	-3\\
554	-3\\
555	-3\\
556	-3\\
557	-3\\
558	-3\\
559	-3\\
560	-3\\
561	-3\\
562	-3\\
563	-3\\
564	-3\\
565	-3\\
566	-3\\
567	-3\\
568	-3\\
569	-3\\
570	-3\\
571	-3\\
572	-3\\
573	-3\\
574	-3\\
575	-3\\
576	-3\\
577	-3\\
578	-3\\
579	-3\\
580	-3\\
581	-3\\
582	-3\\
583	-3\\
584	-3\\
585	-3\\
586	-3\\
587	-3\\
588	-3\\
589	-3\\
590	-3\\
591	-3\\
592	-3\\
593	-3\\
594	-3\\
595	-3\\
596	-3\\
597	-3\\
598	-3\\
599	-3\\
600	-3\\
601	-3\\
602	-3\\
603	-3\\
604	-3\\
605	-2.9023\\
606	-2.7233\\
607	-2.4613\\
608	-2.1692\\
609	-1.8758\\
610	-1.6157\\
611	-1.4059\\
612	-1.2574\\
613	-1.1716\\
614	-1.1453\\
615	-1.1702\\
616	-1.2353\\
617	-1.3275\\
618	-1.4333\\
619	-1.5397\\
620	-1.6355\\
621	-1.712\\
622	-1.7637\\
623	-1.7882\\
624	-1.7865\\
625	-1.762\\
626	-1.7204\\
627	-1.6679\\
628	-1.6115\\
629	-1.5569\\
630	-1.5092\\
631	-1.4716\\
632	-1.4461\\
633	-1.4328\\
634	-1.431\\
635	-1.4386\\
636	-1.4534\\
637	-1.4725\\
638	-1.493\\
639	-1.5127\\
640	-1.5294\\
641	-1.5419\\
642	-1.5493\\
643	-1.5516\\
644	-1.5493\\
645	-1.5432\\
646	-1.5344\\
647	-1.5242\\
648	-1.5138\\
649	-1.5042\\
650	-1.4961\\
651	-1.4902\\
652	-1.4866\\
653	-1.4852\\
654	-1.4859\\
655	-1.4882\\
656	-1.4916\\
657	-1.4955\\
658	-1.4995\\
659	-1.503\\
660	-1.5059\\
661	-1.5079\\
662	-1.5089\\
663	-1.509\\
664	-1.5082\\
665	-1.5069\\
666	-1.5051\\
667	-1.5032\\
668	-1.5013\\
669	-1.4996\\
670	-1.4983\\
671	-1.4975\\
672	-1.497\\
673	-1.497\\
674	-1.4973\\
675	-1.4979\\
676	-1.4986\\
677	-1.4994\\
678	-1.5001\\
679	-1.5007\\
680	-1.5012\\
681	-1.5015\\
682	-1.5016\\
683	-1.5016\\
684	-1.5014\\
685	-1.5011\\
686	-1.5007\\
687	-1.5004\\
688	-1.5\\
689	-1.4998\\
690	-1.4996\\
691	-1.4994\\
692	-1.4994\\
693	-1.4994\\
694	-1.4995\\
695	-1.4996\\
696	-1.4998\\
697	-1.4999\\
698	-1.5001\\
699	-1.5002\\
700	-1.5002\\
701	-1.5003\\
702	-1.5003\\
703	-1.5003\\
704	-1.5002\\
705	-1.5002\\
706	-1.5001\\
707	-1.5\\
708	-1.5\\
709	-1.4999\\
710	-1.4999\\
711	-1.4999\\
712	-1.4999\\
713	-1.4999\\
714	-1.4999\\
715	-1.4999\\
716	-1.5\\
717	-1.5\\
718	-1.5\\
719	-1.5\\
720	-1.5001\\
721	-1.5001\\
722	-1.5001\\
723	-1.5\\
724	-1.5\\
725	-1.5\\
726	-1.5\\
727	-1.5\\
728	-1.5\\
729	-1.5\\
730	-1.5\\
731	-1.5\\
732	-1.5\\
733	-1.5\\
734	-1.5\\
735	-1.5\\
736	-1.5\\
737	-1.5\\
738	-1.5\\
739	-1.5\\
740	-1.5\\
741	-1.5\\
742	-1.5\\
743	-1.5\\
744	-1.5\\
745	-1.5\\
746	-1.5\\
747	-1.5\\
748	-1.5\\
749	-1.5\\
750	-1.5\\
751	-1.5\\
752	-1.5\\
753	-1.5\\
754	-1.5\\
755	-1.5\\
756	-1.5\\
757	-1.5\\
758	-1.5\\
759	-1.5\\
760	-1.5\\
761	-1.5\\
762	-1.5\\
763	-1.5\\
764	-1.5\\
765	-1.5\\
766	-1.5\\
767	-1.5\\
768	-1.5\\
769	-1.5\\
770	-1.5\\
771	-1.5\\
772	-1.5\\
773	-1.5\\
774	-1.5\\
775	-1.5\\
776	-1.5\\
777	-1.5\\
778	-1.5\\
779	-1.5\\
780	-1.5\\
781	-1.5\\
782	-1.5\\
783	-1.5\\
784	-1.5\\
785	-1.5\\
786	-1.5\\
787	-1.5\\
788	-1.5\\
789	-1.5\\
790	-1.5\\
791	-1.5\\
792	-1.5\\
793	-1.5\\
794	-1.5\\
795	-1.5\\
796	-1.5\\
797	-1.5\\
798	-1.5\\
799	-1.5\\
800	-1.5\\
801	-1.5\\
802	-1.5\\
803	-1.5\\
804	-1.5\\
805	-1.5807\\
806	-1.7509\\
807	-1.9826\\
808	-2.247\\
809	-2.5222\\
810	-2.7856\\
811	-3.021\\
812	-3.2141\\
813	-3.3573\\
814	-3.4481\\
815	-3.4899\\
816	-3.4898\\
817	-3.4578\\
818	-3.4052\\
819	-3.3432\\
820	-3.2817\\
821	-3.2287\\
822	-3.1899\\
823	-3.1687\\
824	-3.1662\\
825	-3.1815\\
826	-3.2123\\
827	-3.255\\
828	-3.3056\\
829	-3.3598\\
830	-3.4134\\
831	-3.4629\\
832	-3.5055\\
833	-3.5391\\
834	-3.5627\\
835	-3.5762\\
836	-3.5802\\
837	-3.5759\\
838	-3.5651\\
839	-3.5498\\
840	-3.5321\\
841	-3.5137\\
842	-3.4966\\
843	-3.4818\\
844	-3.4704\\
845	-3.4629\\
846	-3.4593\\
847	-3.4594\\
848	-3.4627\\
849	-3.4685\\
850	-3.4759\\
851	-3.4842\\
852	-3.4926\\
853	-3.5003\\
854	-3.5068\\
855	-3.5119\\
856	-3.5152\\
857	-3.5168\\
858	-3.5168\\
859	-3.5154\\
860	-3.513\\
861	-3.5098\\
862	-3.5063\\
863	-3.5028\\
864	-3.4996\\
865	-3.4968\\
866	-3.4947\\
867	-3.4934\\
868	-3.4928\\
869	-3.4928\\
870	-3.4935\\
871	-3.4946\\
872	-3.4959\\
873	-3.4974\\
874	-3.4989\\
875	-3.5003\\
876	-3.5014\\
877	-3.5023\\
878	-3.5029\\
879	-3.5031\\
880	-3.5031\\
881	-3.5028\\
882	-3.5023\\
883	-3.5017\\
884	-3.5011\\
885	-3.5004\\
886	-3.4998\\
887	-3.4994\\
888	-3.499\\
889	-3.4988\\
890	-3.4987\\
891	-3.4987\\
892	-3.4988\\
893	-3.499\\
894	-3.4993\\
895	-3.4996\\
896	-3.4998\\
897	-3.5001\\
898	-3.5003\\
899	-3.5004\\
900	-3.5005\\
901	-3.5006\\
902	-3.5006\\
903	-3.5005\\
904	-3.5004\\
905	-3.5003\\
906	-3.5002\\
907	-3.5001\\
908	-3.5\\
909	-3.4999\\
910	-3.4998\\
911	-3.4998\\
912	-3.4998\\
913	-3.4998\\
914	-3.4998\\
915	-3.4998\\
916	-3.4999\\
917	-3.4999\\
918	-3.5\\
919	-3.5\\
920	-3.5001\\
921	-3.5001\\
922	-3.5001\\
923	-3.5001\\
924	-3.5001\\
925	-3.5001\\
926	-3.5001\\
927	-3.5001\\
928	-3.5\\
929	-3.5\\
930	-3.5\\
931	-3.5\\
932	-3.5\\
933	-3.5\\
934	-3.5\\
935	-3.5\\
936	-3.5\\
937	-3.5\\
938	-3.5\\
939	-3.5\\
940	-3.5\\
941	-3.5\\
942	-3.5\\
943	-3.5\\
944	-3.5\\
945	-3.5\\
946	-3.5\\
947	-3.5\\
948	-3.5\\
949	-3.5\\
950	-3.5\\
951	-3.5\\
952	-3.5\\
953	-3.5\\
954	-3.5\\
955	-3.5\\
956	-3.5\\
957	-3.5\\
958	-3.5\\
959	-3.5\\
960	-3.5\\
961	-3.5\\
962	-3.5\\
963	-3.5\\
964	-3.5\\
965	-3.5\\
966	-3.5\\
967	-3.5\\
968	-3.5\\
969	-3.5\\
970	-3.5\\
971	-3.5\\
972	-3.5\\
973	-3.5\\
974	-3.5\\
975	-3.5\\
976	-3.5\\
977	-3.5\\
978	-3.5\\
979	-3.5\\
980	-3.5\\
981	-3.5\\
982	-3.5\\
983	-3.5\\
984	-3.5\\
985	-3.5\\
986	-3.5\\
987	-3.5\\
988	-3.5\\
989	-3.5\\
990	-3.5\\
991	-3.5\\
992	-3.5\\
993	-3.5\\
994	-3.5\\
995	-3.5\\
996	-3.5\\
997	-3.5\\
998	-3.5\\
999	-3.5\\
1000	-3.5\\
1001	-3.5\\
1002	-3.5\\
1003	-3.5\\
1004	-3.5\\
1005	-3.3259\\
1006	-3.0198\\
1007	-2.5701\\
1008	-2.0918\\
1009	-1.6373\\
1010	-1.2592\\
1011	-0.97404\\
1012	-0.78809\\
1013	-0.69716\\
1014	-0.69216\\
1015	-0.75873\\
1016	-0.87874\\
1017	-1.0316\\
1018	-1.1963\\
1019	-1.3535\\
1020	-1.4871\\
1021	-1.5854\\
1022	-1.642\\
1023	-1.6556\\
1024	-1.6302\\
1025	-1.5737\\
1026	-1.4966\\
1027	-1.41\\
1028	-1.3243\\
1029	-1.2478\\
1030	-1.1864\\
1031	-1.1433\\
1032	-1.1194\\
1033	-1.1133\\
1034	-1.1224\\
1035	-1.1427\\
1036	-1.1701\\
1037	-1.2\\
1038	-1.2287\\
1039	-1.253\\
1040	-1.2707\\
1041	-1.2807\\
1042	-1.283\\
1043	-1.2783\\
1044	-1.2681\\
1045	-1.2541\\
1046	-1.2383\\
1047	-1.2226\\
1048	-1.2085\\
1049	-1.197\\
1050	-1.1889\\
1051	-1.1843\\
1052	-1.1831\\
1053	-1.1847\\
1054	-1.1885\\
1055	-1.1935\\
1056	-1.199\\
1057	-1.2043\\
1058	-1.2088\\
1059	-1.2121\\
1060	-1.214\\
1061	-1.2145\\
1062	-1.2137\\
1063	-1.2118\\
1064	-1.2093\\
1065	-1.2065\\
1066	-1.2037\\
1067	-1.2012\\
1068	-1.1991\\
1069	-1.1977\\
1070	-1.1969\\
1071	-1.1967\\
1072	-1.197\\
1073	-1.1977\\
1074	-1.1986\\
1075	-1.1996\\
1076	-1.2006\\
1077	-1.2015\\
1078	-1.2021\\
1079	-1.2024\\
1080	-1.2025\\
1081	-1.2024\\
1082	-1.2021\\
1083	-1.2016\\
1084	-1.2011\\
1085	-1.2006\\
1086	-1.2001\\
1087	-1.1998\\
1088	-1.1995\\
1089	-1.1994\\
1090	-1.1993\\
1091	-1.1994\\
1092	-1.1995\\
1093	-1.1997\\
1094	-1.1999\\
1095	-1.2001\\
1096	-1.2002\\
1097	-1.2004\\
1098	-1.2004\\
1099	-1.2004\\
1100	-1.2004\\
1101	-1.2004\\
1102	-1.2003\\
1103	-1.2002\\
1104	-1.2001\\
1105	-1.2\\
1106	-1.2\\
1107	-1.1999\\
1108	-1.1999\\
1109	-1.1999\\
1110	-1.1999\\
1111	-1.1999\\
1112	-1.1999\\
1113	-1.2\\
1114	-1.2\\
1115	-1.2\\
1116	-1.2001\\
1117	-1.2001\\
1118	-1.2001\\
1119	-1.2001\\
1120	-1.2001\\
1121	-1.2\\
1122	-1.2\\
1123	-1.2\\
1124	-1.2\\
1125	-1.2\\
1126	-1.2\\
1127	-1.2\\
1128	-1.2\\
1129	-1.2\\
1130	-1.2\\
1131	-1.2\\
1132	-1.2\\
1133	-1.2\\
1134	-1.2\\
1135	-1.2\\
1136	-1.2\\
1137	-1.2\\
1138	-1.2\\
1139	-1.2\\
1140	-1.2\\
1141	-1.2\\
1142	-1.2\\
1143	-1.2\\
1144	-1.2\\
1145	-1.2\\
1146	-1.2\\
1147	-1.2\\
1148	-1.2\\
1149	-1.2\\
1150	-1.2\\
1151	-1.2\\
1152	-1.2\\
1153	-1.2\\
1154	-1.2\\
1155	-1.2\\
1156	-1.2\\
1157	-1.2\\
1158	-1.2\\
1159	-1.2\\
1160	-1.2\\
1161	-1.2\\
1162	-1.2\\
1163	-1.2\\
1164	-1.2\\
1165	-1.2\\
1166	-1.2\\
1167	-1.2\\
1168	-1.2\\
1169	-1.2\\
1170	-1.2\\
1171	-1.2\\
1172	-1.2\\
1173	-1.2\\
1174	-1.2\\
1175	-1.2\\
1176	-1.2\\
1177	-1.2\\
1178	-1.2\\
1179	-1.2\\
1180	-1.2\\
1181	-1.2\\
1182	-1.2\\
1183	-1.2\\
1184	-1.2\\
1185	-1.2\\
1186	-1.2\\
1187	-1.2\\
1188	-1.2\\
1189	-1.2\\
1190	-1.2\\
1191	-1.2\\
1192	-1.2\\
1193	-1.2\\
1194	-1.2\\
1195	-1.2\\
1196	-1.2\\
1197	-1.2\\
1198	-1.2\\
1199	-1.2\\
1200	-1.2\\
1201	-1.2\\
1202	-1.2\\
1203	-1.2\\
1204	-1.2\\
1205	-1.1481\\
1206	-1.0567\\
1207	-0.93281\\
1208	-0.80149\\
1209	-0.67678\\
1210	-0.57151\\
1211	-0.49105\\
1212	-0.43758\\
1213	-0.40945\\
1214	-0.403\\
1215	-0.41277\\
1216	-0.4327\\
1217	-0.45658\\
1218	-0.47902\\
1219	-0.49587\\
1220	-0.50465\\
1221	-0.50456\\
1222	-0.4963\\
1223	-0.48164\\
1224	-0.46293\\
1225	-0.44263\\
1226	-0.42291\\
1227	-0.40544\\
1228	-0.39125\\
1229	-0.38076\\
1230	-0.37386\\
1231	-0.37003\\
1232	-0.36848\\
1233	-0.36834\\
1234	-0.36878\\
1235	-0.36882\\
1236	-0.36776\\
1237	-0.36529\\
1238	-0.36135\\
1239	-0.35612\\
1240	-0.34988\\
1241	-0.343\\
1242	-0.33586\\
1243	-0.32856\\
1244	-0.3212\\
1245	-0.31375\\
1246	-0.30609\\
1247	-0.29829\\
1248	-0.29524\\
1249	-0.29643\\
1250	-0.2993\\
1251	-0.30264\\
1252	-0.30607\\
1253	-0.30954\\
1254	-0.31296\\
1255	-0.31614\\
1256	-0.31883\\
1257	-0.32075\\
1258	-0.32169\\
1259	-0.32147\\
1260	-0.31999\\
1261	-0.3172\\
1262	-0.31315\\
1263	-0.30797\\
1264	-0.30186\\
1265	-0.2951\\
1266	-0.28804\\
1267	-0.28105\\
1268	-0.27455\\
1269	-0.26893\\
1270	-0.26457\\
1271	-0.26183\\
1272	-0.26098\\
1273	-0.26222\\
1274	-0.26569\\
1275	-0.2714\\
1276	-0.27927\\
1277	-0.2891\\
1278	-0.30055\\
1279	-0.31319\\
1280	-0.32642\\
1281	-0.33955\\
1282	-0.3518\\
1283	-0.36233\\
1284	-0.37028\\
1285	-0.38162\\
1286	-0.39507\\
1287	-0.40678\\
1288	-0.41494\\
1289	-0.41927\\
1290	-0.42025\\
1291	-0.41844\\
1292	-0.41433\\
1293	-0.40846\\
1294	-0.40143\\
1295	-0.39386\\
1296	-0.38632\\
1297	-0.37927\\
1298	-0.37301\\
1299	-0.3677\\
1300	-0.36337\\
1301	-0.35993\\
1302	-0.3572\\
1303	-0.35468\\
1304	-0.35193\\
1305	-0.34848\\
1306	-0.34407\\
1307	-0.3387\\
1308	-0.33227\\
1309	-0.32485\\
1310	-0.31667\\
1311	-0.3079\\
1312	-0.29876\\
1313	-0.29414\\
1314	-0.29357\\
1315	-0.29462\\
1316	-0.29619\\
1317	-0.29802\\
1318	-0.30019\\
1319	-0.3027\\
1320	-0.30547\\
1321	-0.30832\\
1322	-0.31103\\
1323	-0.31341\\
1324	-0.31526\\
1325	-0.31641\\
1326	-0.31673\\
1327	-0.3161\\
1328	-0.31448\\
1329	-0.31187\\
1330	-0.30832\\
1331	-0.30397\\
1332	-0.29899\\
1333	-0.29363\\
1334	-0.28816\\
1335	-0.28288\\
1336	-0.27812\\
1337	-0.27419\\
1338	-0.27136\\
1339	-0.26989\\
1340	-0.26999\\
1341	-0.27177\\
1342	-0.27532\\
1343	-0.28059\\
1344	-0.28747\\
1345	-0.29575\\
1346	-0.30512\\
1347	-0.31517\\
1348	-0.3254\\
1349	-0.33524\\
1350	-0.34405\\
1351	-0.3512\\
1352	-0.35607\\
1353	-0.36447\\
1354	-0.37538\\
1355	-0.3851\\
1356	-0.3918\\
1357	-0.39502\\
1358	-0.39479\\
1359	-0.39182\\
1360	-0.3869\\
1361	-0.38066\\
1362	-0.3737\\
1363	-0.36658\\
1364	-0.35982\\
1365	-0.35381\\
1366	-0.34878\\
1367	-0.34447\\
1368	-0.3406\\
1369	-0.33673\\
1370	-0.33233\\
1371	-0.32716\\
1372	-0.32105\\
1373	-0.31404\\
1374	-0.30636\\
1375	-0.30302\\
1376	-0.30922\\
1377	-0.32176\\
1378	-0.33604\\
1379	-0.34973\\
1380	-0.36182\\
1381	-0.37171\\
1382	-0.37869\\
1383	-0.38212\\
1384	-0.38168\\
1385	-0.37756\\
1386	-0.37052\\
1387	-0.36181\\
1388	-0.35261\\
1389	-0.34384\\
1390	-0.33617\\
1391	-0.32966\\
1392	-0.32394\\
1393	-0.31847\\
1394	-0.3128\\
1395	-0.30682\\
1396	-0.30551\\
1397	-0.30824\\
1398	-0.31224\\
1399	-0.31611\\
1400	-0.31937\\
1401	-0.32187\\
1402	-0.32353\\
1403	-0.32415\\
1404	-0.32355\\
1405	-0.31149\\
1406	-0.28151\\
1407	-0.23543\\
1408	-0.17957\\
1409	-0.12137\\
1410	-0.06713\\
1411	-0.020721\\
1412	0.016426\\
1413	0.044667\\
1414	0.065283\\
1415	0.079836\\
1416	0.089748\\
1417	0.096179\\
1418	0.10001\\
1419	0.10193\\
1420	0.10216\\
1421	0.10064\\
1422	0.09704\\
1423	0.090853\\
1424	0.081669\\
1425	0.06908\\
1426	0.052703\\
1427	0.032284\\
1428	0.0082078\\
1429	-0.018305\\
1430	-0.045569\\
1431	-0.071845\\
1432	-0.095648\\
1433	-0.11569\\
1434	-0.13078\\
1435	-0.13986\\
1436	-0.14214\\
1437	-0.13894\\
1438	-0.13039\\
1439	-0.11641\\
1440	-0.097959\\
1441	-0.076568\\
1442	-0.053948\\
1443	-0.031648\\
1444	-0.010874\\
1445	0.0080666\\
1446	0.02492\\
1447	0.038865\\
1448	0.049737\\
1449	0.05772\\
1450	0.063149\\
1451	0.066378\\
1452	0.067725\\
1453	0.067453\\
1454	0.065776\\
1455	0.06287\\
1456	0.058754\\
1457	0.053307\\
1458	0.046575\\
1459	0.038687\\
1460	0.029987\\
1461	0.020901\\
1462	0.011722\\
1463	0.0026887\\
1464	-0.0059642\\
1465	-0.013993\\
1466	-0.021151\\
1467	-0.027208\\
1468	-0.031968\\
1469	-0.035284\\
1470	-0.037074\\
1471	-0.037325\\
1472	-0.036097\\
1473	-0.033522\\
1474	-0.029795\\
1475	-0.02516\\
1476	-0.01989\\
1477	-0.014266\\
1478	-0.008562\\
1479	-0.0030222\\
1480	0.0021494\\
1481	0.0067926\\
1482	0.010792\\
1483	0.014073\\
1484	0.016603\\
1485	0.018376\\
1486	0.019412\\
1487	0.019752\\
1488	0.019446\\
1489	0.018561\\
1490	0.017168\\
1491	0.015347\\
1492	0.01318\\
1493	0.010756\\
1494	0.0081609\\
1495	0.005485\\
1496	0.0028149\\
1497	0.00023405\\
1498	-0.00218\\
1499	-0.0043578\\
1500	-0.0062401\\
1501	-0.0077799\\
1502	-0.008944\\
1503	-0.0097143\\
1504	-0.010088\\
1505	-0.010078\\
1506	-0.0097088\\
1507	-0.0090201\\
1508	-0.00806\\
1509	-0.0068838\\
1510	-0.0055513\\
1511	-0.0041234\\
1512	-0.0026595\\
1513	-0.0012152\\
1514	0.00015954\\
1515	0.001422\\
1516	0.0025374\\
1517	0.0034796\\
1518	0.0042307\\
1519	0.0047809\\
1520	0.0051279\\
1521	0.0052765\\
1522	0.0052373\\
1523	0.0050261\\
1524	0.004663\\
1525	0.0041712\\
1526	0.0035764\\
1527	0.0029056\\
1528	0.0021862\\
1529	0.0014457\\
1530	0.00071008\\
1531	3.568e-06\\
1532	-0.00065209\\
1533	-0.0012383\\
1534	-0.0017398\\
1535	-0.0021456\\
1536	-0.0024482\\
1537	-0.0026447\\
1538	-0.0027356\\
1539	-0.0027253\\
1540	-0.0026216\\
1541	-0.0024348\\
1542	-0.0021776\\
1543	-0.0018641\\
1544	-0.0015095\\
1545	-0.0011292\\
1546	-0.0007383\\
1547	-0.00035124\\
1548	1.893e-05\\
1549	0.00036073\\
1550	0.00066457\\
1551	0.00092294\\
1552	0.0011305\\
1553	0.001284\\
1554	0.0013823\\
1555	0.0014262\\
1556	0.0014184\\
1557	0.0013631\\
1558	0.0012655\\
1559	0.0011321\\
1560	0.00097019\\
1561	0.00078722\\
1562	0.00059096\\
1563	0.00038903\\
1564	0.0001887\\
1565	-3.3446e-06\\
1566	-0.00018117\\
1567	-0.00033975\\
1568	-0.00047507\\
1569	-0.0005842\\
1570	-0.00066532\\
1571	-0.00071767\\
1572	-0.00074159\\
1573	-0.00073832\\
1574	-0.00070997\\
1575	-0.00065937\\
1576	-0.00058989\\
1577	-0.00050533\\
1578	-0.0004097\\
1579	-0.00030712\\
1580	-0.00020162\\
1581	-9.703e-05\\
1582	3.1185e-06\\
1583	9.573e-05\\
1584	0.0001782\\
1585	0.00024845\\
1586	0.00030501\\
1587	0.00034695\\
1588	0.00037394\\
1589	0.00038617\\
1590	0.00038431\\
1591	0.00036947\\
1592	0.00034312\\
1593	0.00030701\\
1594	0.0002631\\
1595	0.00021345\\
1596	0.00016018\\
1597	0.00010538\\
1598	5.1026e-05\\
1599	-1.0599e-06\\
1600	-4.9266e-05\\
1601	-9.2231e-05\\
1602	-0.00012887\\
1603	-0.0001584\\
1604	-0.00018034\\
1605	0.0011605\\
1606	0.0051649\\
1607	0.011682\\
1608	0.019985\\
1609	0.02923\\
1610	0.038682\\
1611	0.047788\\
1612	0.05618\\
1613	0.06365\\
1614	0.070117\\
1615	0.075587\\
1616	0.080129\\
1617	0.083843\\
1618	0.08684\\
1619	0.089236\\
1620	0.091155\\
1621	0.092703\\
1622	0.093956\\
1623	0.094967\\
1624	0.095792\\
1625	0.096469\\
1626	0.097028\\
1627	0.097491\\
1628	0.097875\\
1629	0.098195\\
1630	0.098465\\
1631	0.098696\\
1632	0.098892\\
1633	0.09906\\
1634	0.099204\\
1635	0.099326\\
1636	0.09943\\
1637	0.099519\\
1638	0.099595\\
1639	0.09966\\
1640	0.099715\\
1641	0.099761\\
1642	0.099801\\
1643	0.099834\\
1644	0.099862\\
1645	0.099885\\
1646	0.099905\\
1647	0.099921\\
1648	0.099935\\
1649	0.099947\\
1650	0.099956\\
1651	0.099964\\
1652	0.099971\\
1653	0.099976\\
1654	0.099981\\
1655	0.099984\\
1656	0.099987\\
1657	0.09999\\
1658	0.099992\\
1659	0.099993\\
1660	0.099995\\
1661	0.099996\\
1662	0.099997\\
1663	0.099997\\
1664	0.099998\\
1665	0.099999\\
1666	0.099999\\
1667	0.099999\\
1668	0.099999\\
1669	0.1\\
1670	0.1\\
1671	0.1\\
1672	0.1\\
1673	0.1\\
1674	0.1\\
1675	0.1\\
1676	0.1\\
1677	0.1\\
1678	0.1\\
1679	0.1\\
1680	0.1\\
1681	0.1\\
1682	0.1\\
1683	0.1\\
1684	0.1\\
1685	0.1\\
1686	0.1\\
1687	0.1\\
1688	0.1\\
1689	0.1\\
1690	0.1\\
1691	0.1\\
1692	0.1\\
1693	0.1\\
1694	0.1\\
1695	0.1\\
1696	0.1\\
1697	0.1\\
1698	0.1\\
1699	0.1\\
1700	0.1\\
1701	0.1\\
1702	0.1\\
1703	0.1\\
1704	0.1\\
1705	0.1\\
1706	0.1\\
1707	0.1\\
1708	0.1\\
1709	0.1\\
1710	0.1\\
1711	0.1\\
1712	0.1\\
1713	0.1\\
1714	0.1\\
1715	0.1\\
1716	0.1\\
1717	0.1\\
1718	0.1\\
1719	0.1\\
1720	0.1\\
1721	0.1\\
1722	0.1\\
1723	0.1\\
1724	0.1\\
1725	0.1\\
1726	0.1\\
1727	0.1\\
1728	0.1\\
1729	0.1\\
1730	0.1\\
1731	0.1\\
1732	0.1\\
1733	0.1\\
1734	0.1\\
1735	0.1\\
1736	0.1\\
1737	0.1\\
1738	0.1\\
1739	0.1\\
1740	0.1\\
1741	0.1\\
1742	0.1\\
1743	0.1\\
1744	0.1\\
1745	0.1\\
1746	0.1\\
1747	0.1\\
1748	0.1\\
1749	0.1\\
1750	0.1\\
1751	0.1\\
1752	0.1\\
1753	0.1\\
1754	0.1\\
1755	0.1\\
1756	0.1\\
1757	0.1\\
1758	0.1\\
1759	0.1\\
1760	0.1\\
1761	0.1\\
1762	0.1\\
1763	0.1\\
1764	0.1\\
1765	0.1\\
1766	0.1\\
1767	0.1\\
1768	0.1\\
1769	0.1\\
1770	0.1\\
1771	0.1\\
1772	0.1\\
1773	0.1\\
1774	0.1\\
1775	0.1\\
1776	0.1\\
1777	0.1\\
1778	0.1\\
1779	0.1\\
1780	0.1\\
1781	0.1\\
1782	0.1\\
1783	0.1\\
1784	0.1\\
1785	0.1\\
1786	0.1\\
1787	0.1\\
1788	0.1\\
1789	0.1\\
1790	0.1\\
1791	0.1\\
1792	0.1\\
1793	0.1\\
1794	0.1\\
1795	0.1\\
1796	0.1\\
1797	0.1\\
1798	0.1\\
1799	0.1\\
1800	0.1\\
};
\addlegendentry{Wyjście y}

\addplot [color=mycolor2, dashed]
  table[row sep=crcr]{%
1	0\\
2	0\\
3	0\\
4	0\\
5	0\\
6	0\\
7	0\\
8	0\\
9	0\\
10	0\\
11	0\\
12	0\\
13	0\\
14	0\\
15	0\\
16	0\\
17	0\\
18	0\\
19	0\\
20	-2\\
21	-2\\
22	-2\\
23	-2\\
24	-2\\
25	-2\\
26	-2\\
27	-2\\
28	-2\\
29	-2\\
30	-2\\
31	-2\\
32	-2\\
33	-2\\
34	-2\\
35	-2\\
36	-2\\
37	-2\\
38	-2\\
39	-2\\
40	-2\\
41	-2\\
42	-2\\
43	-2\\
44	-2\\
45	-2\\
46	-2\\
47	-2\\
48	-2\\
49	-2\\
50	-2\\
51	-2\\
52	-2\\
53	-2\\
54	-2\\
55	-2\\
56	-2\\
57	-2\\
58	-2\\
59	-2\\
60	-2\\
61	-2\\
62	-2\\
63	-2\\
64	-2\\
65	-2\\
66	-2\\
67	-2\\
68	-2\\
69	-2\\
70	-2\\
71	-2\\
72	-2\\
73	-2\\
74	-2\\
75	-2\\
76	-2\\
77	-2\\
78	-2\\
79	-2\\
80	-2\\
81	-2\\
82	-2\\
83	-2\\
84	-2\\
85	-2\\
86	-2\\
87	-2\\
88	-2\\
89	-2\\
90	-2\\
91	-2\\
92	-2\\
93	-2\\
94	-2\\
95	-2\\
96	-2\\
97	-2\\
98	-2\\
99	-2\\
100	-2\\
101	-2\\
102	-2\\
103	-2\\
104	-2\\
105	-2\\
106	-2\\
107	-2\\
108	-2\\
109	-2\\
110	-2\\
111	-2\\
112	-2\\
113	-2\\
114	-2\\
115	-2\\
116	-2\\
117	-2\\
118	-2\\
119	-2\\
120	-2\\
121	-2\\
122	-2\\
123	-2\\
124	-2\\
125	-2\\
126	-2\\
127	-2\\
128	-2\\
129	-2\\
130	-2\\
131	-2\\
132	-2\\
133	-2\\
134	-2\\
135	-2\\
136	-2\\
137	-2\\
138	-2\\
139	-2\\
140	-2\\
141	-2\\
142	-2\\
143	-2\\
144	-2\\
145	-2\\
146	-2\\
147	-2\\
148	-2\\
149	-2\\
150	-2\\
151	-2\\
152	-2\\
153	-2\\
154	-2\\
155	-2\\
156	-2\\
157	-2\\
158	-2\\
159	-2\\
160	-2\\
161	-2\\
162	-2\\
163	-2\\
164	-2\\
165	-2\\
166	-2\\
167	-2\\
168	-2\\
169	-2\\
170	-2\\
171	-2\\
172	-2\\
173	-2\\
174	-2\\
175	-2\\
176	-2\\
177	-2\\
178	-2\\
179	-2\\
180	-2\\
181	-2\\
182	-2\\
183	-2\\
184	-2\\
185	-2\\
186	-2\\
187	-2\\
188	-2\\
189	-2\\
190	-2\\
191	-2\\
192	-2\\
193	-2\\
194	-2\\
195	-2\\
196	-2\\
197	-2\\
198	-2\\
199	-2\\
200	-4.5\\
201	-4.5\\
202	-4.5\\
203	-4.5\\
204	-4.5\\
205	-4.5\\
206	-4.5\\
207	-4.5\\
208	-4.5\\
209	-4.5\\
210	-4.5\\
211	-4.5\\
212	-4.5\\
213	-4.5\\
214	-4.5\\
215	-4.5\\
216	-4.5\\
217	-4.5\\
218	-4.5\\
219	-4.5\\
220	-4.5\\
221	-4.5\\
222	-4.5\\
223	-4.5\\
224	-4.5\\
225	-4.5\\
226	-4.5\\
227	-4.5\\
228	-4.5\\
229	-4.5\\
230	-4.5\\
231	-4.5\\
232	-4.5\\
233	-4.5\\
234	-4.5\\
235	-4.5\\
236	-4.5\\
237	-4.5\\
238	-4.5\\
239	-4.5\\
240	-4.5\\
241	-4.5\\
242	-4.5\\
243	-4.5\\
244	-4.5\\
245	-4.5\\
246	-4.5\\
247	-4.5\\
248	-4.5\\
249	-4.5\\
250	-4.5\\
251	-4.5\\
252	-4.5\\
253	-4.5\\
254	-4.5\\
255	-4.5\\
256	-4.5\\
257	-4.5\\
258	-4.5\\
259	-4.5\\
260	-4.5\\
261	-4.5\\
262	-4.5\\
263	-4.5\\
264	-4.5\\
265	-4.5\\
266	-4.5\\
267	-4.5\\
268	-4.5\\
269	-4.5\\
270	-4.5\\
271	-4.5\\
272	-4.5\\
273	-4.5\\
274	-4.5\\
275	-4.5\\
276	-4.5\\
277	-4.5\\
278	-4.5\\
279	-4.5\\
280	-4.5\\
281	-4.5\\
282	-4.5\\
283	-4.5\\
284	-4.5\\
285	-4.5\\
286	-4.5\\
287	-4.5\\
288	-4.5\\
289	-4.5\\
290	-4.5\\
291	-4.5\\
292	-4.5\\
293	-4.5\\
294	-4.5\\
295	-4.5\\
296	-4.5\\
297	-4.5\\
298	-4.5\\
299	-4.5\\
300	-4.5\\
301	-4.5\\
302	-4.5\\
303	-4.5\\
304	-4.5\\
305	-4.5\\
306	-4.5\\
307	-4.5\\
308	-4.5\\
309	-4.5\\
310	-4.5\\
311	-4.5\\
312	-4.5\\
313	-4.5\\
314	-4.5\\
315	-4.5\\
316	-4.5\\
317	-4.5\\
318	-4.5\\
319	-4.5\\
320	-4.5\\
321	-4.5\\
322	-4.5\\
323	-4.5\\
324	-4.5\\
325	-4.5\\
326	-4.5\\
327	-4.5\\
328	-4.5\\
329	-4.5\\
330	-4.5\\
331	-4.5\\
332	-4.5\\
333	-4.5\\
334	-4.5\\
335	-4.5\\
336	-4.5\\
337	-4.5\\
338	-4.5\\
339	-4.5\\
340	-4.5\\
341	-4.5\\
342	-4.5\\
343	-4.5\\
344	-4.5\\
345	-4.5\\
346	-4.5\\
347	-4.5\\
348	-4.5\\
349	-4.5\\
350	-4.5\\
351	-4.5\\
352	-4.5\\
353	-4.5\\
354	-4.5\\
355	-4.5\\
356	-4.5\\
357	-4.5\\
358	-4.5\\
359	-4.5\\
360	-4.5\\
361	-4.5\\
362	-4.5\\
363	-4.5\\
364	-4.5\\
365	-4.5\\
366	-4.5\\
367	-4.5\\
368	-4.5\\
369	-4.5\\
370	-4.5\\
371	-4.5\\
372	-4.5\\
373	-4.5\\
374	-4.5\\
375	-4.5\\
376	-4.5\\
377	-4.5\\
378	-4.5\\
379	-4.5\\
380	-4.5\\
381	-4.5\\
382	-4.5\\
383	-4.5\\
384	-4.5\\
385	-4.5\\
386	-4.5\\
387	-4.5\\
388	-4.5\\
389	-4.5\\
390	-4.5\\
391	-4.5\\
392	-4.5\\
393	-4.5\\
394	-4.5\\
395	-4.5\\
396	-4.5\\
397	-4.5\\
398	-4.5\\
399	-4.5\\
400	-3\\
401	-3\\
402	-3\\
403	-3\\
404	-3\\
405	-3\\
406	-3\\
407	-3\\
408	-3\\
409	-3\\
410	-3\\
411	-3\\
412	-3\\
413	-3\\
414	-3\\
415	-3\\
416	-3\\
417	-3\\
418	-3\\
419	-3\\
420	-3\\
421	-3\\
422	-3\\
423	-3\\
424	-3\\
425	-3\\
426	-3\\
427	-3\\
428	-3\\
429	-3\\
430	-3\\
431	-3\\
432	-3\\
433	-3\\
434	-3\\
435	-3\\
436	-3\\
437	-3\\
438	-3\\
439	-3\\
440	-3\\
441	-3\\
442	-3\\
443	-3\\
444	-3\\
445	-3\\
446	-3\\
447	-3\\
448	-3\\
449	-3\\
450	-3\\
451	-3\\
452	-3\\
453	-3\\
454	-3\\
455	-3\\
456	-3\\
457	-3\\
458	-3\\
459	-3\\
460	-3\\
461	-3\\
462	-3\\
463	-3\\
464	-3\\
465	-3\\
466	-3\\
467	-3\\
468	-3\\
469	-3\\
470	-3\\
471	-3\\
472	-3\\
473	-3\\
474	-3\\
475	-3\\
476	-3\\
477	-3\\
478	-3\\
479	-3\\
480	-3\\
481	-3\\
482	-3\\
483	-3\\
484	-3\\
485	-3\\
486	-3\\
487	-3\\
488	-3\\
489	-3\\
490	-3\\
491	-3\\
492	-3\\
493	-3\\
494	-3\\
495	-3\\
496	-3\\
497	-3\\
498	-3\\
499	-3\\
500	-3\\
501	-3\\
502	-3\\
503	-3\\
504	-3\\
505	-3\\
506	-3\\
507	-3\\
508	-3\\
509	-3\\
510	-3\\
511	-3\\
512	-3\\
513	-3\\
514	-3\\
515	-3\\
516	-3\\
517	-3\\
518	-3\\
519	-3\\
520	-3\\
521	-3\\
522	-3\\
523	-3\\
524	-3\\
525	-3\\
526	-3\\
527	-3\\
528	-3\\
529	-3\\
530	-3\\
531	-3\\
532	-3\\
533	-3\\
534	-3\\
535	-3\\
536	-3\\
537	-3\\
538	-3\\
539	-3\\
540	-3\\
541	-3\\
542	-3\\
543	-3\\
544	-3\\
545	-3\\
546	-3\\
547	-3\\
548	-3\\
549	-3\\
550	-3\\
551	-3\\
552	-3\\
553	-3\\
554	-3\\
555	-3\\
556	-3\\
557	-3\\
558	-3\\
559	-3\\
560	-3\\
561	-3\\
562	-3\\
563	-3\\
564	-3\\
565	-3\\
566	-3\\
567	-3\\
568	-3\\
569	-3\\
570	-3\\
571	-3\\
572	-3\\
573	-3\\
574	-3\\
575	-3\\
576	-3\\
577	-3\\
578	-3\\
579	-3\\
580	-3\\
581	-3\\
582	-3\\
583	-3\\
584	-3\\
585	-3\\
586	-3\\
587	-3\\
588	-3\\
589	-3\\
590	-3\\
591	-3\\
592	-3\\
593	-3\\
594	-3\\
595	-3\\
596	-3\\
597	-3\\
598	-3\\
599	-3\\
600	-1.5\\
601	-1.5\\
602	-1.5\\
603	-1.5\\
604	-1.5\\
605	-1.5\\
606	-1.5\\
607	-1.5\\
608	-1.5\\
609	-1.5\\
610	-1.5\\
611	-1.5\\
612	-1.5\\
613	-1.5\\
614	-1.5\\
615	-1.5\\
616	-1.5\\
617	-1.5\\
618	-1.5\\
619	-1.5\\
620	-1.5\\
621	-1.5\\
622	-1.5\\
623	-1.5\\
624	-1.5\\
625	-1.5\\
626	-1.5\\
627	-1.5\\
628	-1.5\\
629	-1.5\\
630	-1.5\\
631	-1.5\\
632	-1.5\\
633	-1.5\\
634	-1.5\\
635	-1.5\\
636	-1.5\\
637	-1.5\\
638	-1.5\\
639	-1.5\\
640	-1.5\\
641	-1.5\\
642	-1.5\\
643	-1.5\\
644	-1.5\\
645	-1.5\\
646	-1.5\\
647	-1.5\\
648	-1.5\\
649	-1.5\\
650	-1.5\\
651	-1.5\\
652	-1.5\\
653	-1.5\\
654	-1.5\\
655	-1.5\\
656	-1.5\\
657	-1.5\\
658	-1.5\\
659	-1.5\\
660	-1.5\\
661	-1.5\\
662	-1.5\\
663	-1.5\\
664	-1.5\\
665	-1.5\\
666	-1.5\\
667	-1.5\\
668	-1.5\\
669	-1.5\\
670	-1.5\\
671	-1.5\\
672	-1.5\\
673	-1.5\\
674	-1.5\\
675	-1.5\\
676	-1.5\\
677	-1.5\\
678	-1.5\\
679	-1.5\\
680	-1.5\\
681	-1.5\\
682	-1.5\\
683	-1.5\\
684	-1.5\\
685	-1.5\\
686	-1.5\\
687	-1.5\\
688	-1.5\\
689	-1.5\\
690	-1.5\\
691	-1.5\\
692	-1.5\\
693	-1.5\\
694	-1.5\\
695	-1.5\\
696	-1.5\\
697	-1.5\\
698	-1.5\\
699	-1.5\\
700	-1.5\\
701	-1.5\\
702	-1.5\\
703	-1.5\\
704	-1.5\\
705	-1.5\\
706	-1.5\\
707	-1.5\\
708	-1.5\\
709	-1.5\\
710	-1.5\\
711	-1.5\\
712	-1.5\\
713	-1.5\\
714	-1.5\\
715	-1.5\\
716	-1.5\\
717	-1.5\\
718	-1.5\\
719	-1.5\\
720	-1.5\\
721	-1.5\\
722	-1.5\\
723	-1.5\\
724	-1.5\\
725	-1.5\\
726	-1.5\\
727	-1.5\\
728	-1.5\\
729	-1.5\\
730	-1.5\\
731	-1.5\\
732	-1.5\\
733	-1.5\\
734	-1.5\\
735	-1.5\\
736	-1.5\\
737	-1.5\\
738	-1.5\\
739	-1.5\\
740	-1.5\\
741	-1.5\\
742	-1.5\\
743	-1.5\\
744	-1.5\\
745	-1.5\\
746	-1.5\\
747	-1.5\\
748	-1.5\\
749	-1.5\\
750	-1.5\\
751	-1.5\\
752	-1.5\\
753	-1.5\\
754	-1.5\\
755	-1.5\\
756	-1.5\\
757	-1.5\\
758	-1.5\\
759	-1.5\\
760	-1.5\\
761	-1.5\\
762	-1.5\\
763	-1.5\\
764	-1.5\\
765	-1.5\\
766	-1.5\\
767	-1.5\\
768	-1.5\\
769	-1.5\\
770	-1.5\\
771	-1.5\\
772	-1.5\\
773	-1.5\\
774	-1.5\\
775	-1.5\\
776	-1.5\\
777	-1.5\\
778	-1.5\\
779	-1.5\\
780	-1.5\\
781	-1.5\\
782	-1.5\\
783	-1.5\\
784	-1.5\\
785	-1.5\\
786	-1.5\\
787	-1.5\\
788	-1.5\\
789	-1.5\\
790	-1.5\\
791	-1.5\\
792	-1.5\\
793	-1.5\\
794	-1.5\\
795	-1.5\\
796	-1.5\\
797	-1.5\\
798	-1.5\\
799	-1.5\\
800	-3.5\\
801	-3.5\\
802	-3.5\\
803	-3.5\\
804	-3.5\\
805	-3.5\\
806	-3.5\\
807	-3.5\\
808	-3.5\\
809	-3.5\\
810	-3.5\\
811	-3.5\\
812	-3.5\\
813	-3.5\\
814	-3.5\\
815	-3.5\\
816	-3.5\\
817	-3.5\\
818	-3.5\\
819	-3.5\\
820	-3.5\\
821	-3.5\\
822	-3.5\\
823	-3.5\\
824	-3.5\\
825	-3.5\\
826	-3.5\\
827	-3.5\\
828	-3.5\\
829	-3.5\\
830	-3.5\\
831	-3.5\\
832	-3.5\\
833	-3.5\\
834	-3.5\\
835	-3.5\\
836	-3.5\\
837	-3.5\\
838	-3.5\\
839	-3.5\\
840	-3.5\\
841	-3.5\\
842	-3.5\\
843	-3.5\\
844	-3.5\\
845	-3.5\\
846	-3.5\\
847	-3.5\\
848	-3.5\\
849	-3.5\\
850	-3.5\\
851	-3.5\\
852	-3.5\\
853	-3.5\\
854	-3.5\\
855	-3.5\\
856	-3.5\\
857	-3.5\\
858	-3.5\\
859	-3.5\\
860	-3.5\\
861	-3.5\\
862	-3.5\\
863	-3.5\\
864	-3.5\\
865	-3.5\\
866	-3.5\\
867	-3.5\\
868	-3.5\\
869	-3.5\\
870	-3.5\\
871	-3.5\\
872	-3.5\\
873	-3.5\\
874	-3.5\\
875	-3.5\\
876	-3.5\\
877	-3.5\\
878	-3.5\\
879	-3.5\\
880	-3.5\\
881	-3.5\\
882	-3.5\\
883	-3.5\\
884	-3.5\\
885	-3.5\\
886	-3.5\\
887	-3.5\\
888	-3.5\\
889	-3.5\\
890	-3.5\\
891	-3.5\\
892	-3.5\\
893	-3.5\\
894	-3.5\\
895	-3.5\\
896	-3.5\\
897	-3.5\\
898	-3.5\\
899	-3.5\\
900	-3.5\\
901	-3.5\\
902	-3.5\\
903	-3.5\\
904	-3.5\\
905	-3.5\\
906	-3.5\\
907	-3.5\\
908	-3.5\\
909	-3.5\\
910	-3.5\\
911	-3.5\\
912	-3.5\\
913	-3.5\\
914	-3.5\\
915	-3.5\\
916	-3.5\\
917	-3.5\\
918	-3.5\\
919	-3.5\\
920	-3.5\\
921	-3.5\\
922	-3.5\\
923	-3.5\\
924	-3.5\\
925	-3.5\\
926	-3.5\\
927	-3.5\\
928	-3.5\\
929	-3.5\\
930	-3.5\\
931	-3.5\\
932	-3.5\\
933	-3.5\\
934	-3.5\\
935	-3.5\\
936	-3.5\\
937	-3.5\\
938	-3.5\\
939	-3.5\\
940	-3.5\\
941	-3.5\\
942	-3.5\\
943	-3.5\\
944	-3.5\\
945	-3.5\\
946	-3.5\\
947	-3.5\\
948	-3.5\\
949	-3.5\\
950	-3.5\\
951	-3.5\\
952	-3.5\\
953	-3.5\\
954	-3.5\\
955	-3.5\\
956	-3.5\\
957	-3.5\\
958	-3.5\\
959	-3.5\\
960	-3.5\\
961	-3.5\\
962	-3.5\\
963	-3.5\\
964	-3.5\\
965	-3.5\\
966	-3.5\\
967	-3.5\\
968	-3.5\\
969	-3.5\\
970	-3.5\\
971	-3.5\\
972	-3.5\\
973	-3.5\\
974	-3.5\\
975	-3.5\\
976	-3.5\\
977	-3.5\\
978	-3.5\\
979	-3.5\\
980	-3.5\\
981	-3.5\\
982	-3.5\\
983	-3.5\\
984	-3.5\\
985	-3.5\\
986	-3.5\\
987	-3.5\\
988	-3.5\\
989	-3.5\\
990	-3.5\\
991	-3.5\\
992	-3.5\\
993	-3.5\\
994	-3.5\\
995	-3.5\\
996	-3.5\\
997	-3.5\\
998	-3.5\\
999	-3.5\\
1000	-1.2\\
1001	-1.2\\
1002	-1.2\\
1003	-1.2\\
1004	-1.2\\
1005	-1.2\\
1006	-1.2\\
1007	-1.2\\
1008	-1.2\\
1009	-1.2\\
1010	-1.2\\
1011	-1.2\\
1012	-1.2\\
1013	-1.2\\
1014	-1.2\\
1015	-1.2\\
1016	-1.2\\
1017	-1.2\\
1018	-1.2\\
1019	-1.2\\
1020	-1.2\\
1021	-1.2\\
1022	-1.2\\
1023	-1.2\\
1024	-1.2\\
1025	-1.2\\
1026	-1.2\\
1027	-1.2\\
1028	-1.2\\
1029	-1.2\\
1030	-1.2\\
1031	-1.2\\
1032	-1.2\\
1033	-1.2\\
1034	-1.2\\
1035	-1.2\\
1036	-1.2\\
1037	-1.2\\
1038	-1.2\\
1039	-1.2\\
1040	-1.2\\
1041	-1.2\\
1042	-1.2\\
1043	-1.2\\
1044	-1.2\\
1045	-1.2\\
1046	-1.2\\
1047	-1.2\\
1048	-1.2\\
1049	-1.2\\
1050	-1.2\\
1051	-1.2\\
1052	-1.2\\
1053	-1.2\\
1054	-1.2\\
1055	-1.2\\
1056	-1.2\\
1057	-1.2\\
1058	-1.2\\
1059	-1.2\\
1060	-1.2\\
1061	-1.2\\
1062	-1.2\\
1063	-1.2\\
1064	-1.2\\
1065	-1.2\\
1066	-1.2\\
1067	-1.2\\
1068	-1.2\\
1069	-1.2\\
1070	-1.2\\
1071	-1.2\\
1072	-1.2\\
1073	-1.2\\
1074	-1.2\\
1075	-1.2\\
1076	-1.2\\
1077	-1.2\\
1078	-1.2\\
1079	-1.2\\
1080	-1.2\\
1081	-1.2\\
1082	-1.2\\
1083	-1.2\\
1084	-1.2\\
1085	-1.2\\
1086	-1.2\\
1087	-1.2\\
1088	-1.2\\
1089	-1.2\\
1090	-1.2\\
1091	-1.2\\
1092	-1.2\\
1093	-1.2\\
1094	-1.2\\
1095	-1.2\\
1096	-1.2\\
1097	-1.2\\
1098	-1.2\\
1099	-1.2\\
1100	-1.2\\
1101	-1.2\\
1102	-1.2\\
1103	-1.2\\
1104	-1.2\\
1105	-1.2\\
1106	-1.2\\
1107	-1.2\\
1108	-1.2\\
1109	-1.2\\
1110	-1.2\\
1111	-1.2\\
1112	-1.2\\
1113	-1.2\\
1114	-1.2\\
1115	-1.2\\
1116	-1.2\\
1117	-1.2\\
1118	-1.2\\
1119	-1.2\\
1120	-1.2\\
1121	-1.2\\
1122	-1.2\\
1123	-1.2\\
1124	-1.2\\
1125	-1.2\\
1126	-1.2\\
1127	-1.2\\
1128	-1.2\\
1129	-1.2\\
1130	-1.2\\
1131	-1.2\\
1132	-1.2\\
1133	-1.2\\
1134	-1.2\\
1135	-1.2\\
1136	-1.2\\
1137	-1.2\\
1138	-1.2\\
1139	-1.2\\
1140	-1.2\\
1141	-1.2\\
1142	-1.2\\
1143	-1.2\\
1144	-1.2\\
1145	-1.2\\
1146	-1.2\\
1147	-1.2\\
1148	-1.2\\
1149	-1.2\\
1150	-1.2\\
1151	-1.2\\
1152	-1.2\\
1153	-1.2\\
1154	-1.2\\
1155	-1.2\\
1156	-1.2\\
1157	-1.2\\
1158	-1.2\\
1159	-1.2\\
1160	-1.2\\
1161	-1.2\\
1162	-1.2\\
1163	-1.2\\
1164	-1.2\\
1165	-1.2\\
1166	-1.2\\
1167	-1.2\\
1168	-1.2\\
1169	-1.2\\
1170	-1.2\\
1171	-1.2\\
1172	-1.2\\
1173	-1.2\\
1174	-1.2\\
1175	-1.2\\
1176	-1.2\\
1177	-1.2\\
1178	-1.2\\
1179	-1.2\\
1180	-1.2\\
1181	-1.2\\
1182	-1.2\\
1183	-1.2\\
1184	-1.2\\
1185	-1.2\\
1186	-1.2\\
1187	-1.2\\
1188	-1.2\\
1189	-1.2\\
1190	-1.2\\
1191	-1.2\\
1192	-1.2\\
1193	-1.2\\
1194	-1.2\\
1195	-1.2\\
1196	-1.2\\
1197	-1.2\\
1198	-1.2\\
1199	-1.2\\
1200	-0.3\\
1201	-0.3\\
1202	-0.3\\
1203	-0.3\\
1204	-0.3\\
1205	-0.3\\
1206	-0.3\\
1207	-0.3\\
1208	-0.3\\
1209	-0.3\\
1210	-0.3\\
1211	-0.3\\
1212	-0.3\\
1213	-0.3\\
1214	-0.3\\
1215	-0.3\\
1216	-0.3\\
1217	-0.3\\
1218	-0.3\\
1219	-0.3\\
1220	-0.3\\
1221	-0.3\\
1222	-0.3\\
1223	-0.3\\
1224	-0.3\\
1225	-0.3\\
1226	-0.3\\
1227	-0.3\\
1228	-0.3\\
1229	-0.3\\
1230	-0.3\\
1231	-0.3\\
1232	-0.3\\
1233	-0.3\\
1234	-0.3\\
1235	-0.3\\
1236	-0.3\\
1237	-0.3\\
1238	-0.3\\
1239	-0.3\\
1240	-0.3\\
1241	-0.3\\
1242	-0.3\\
1243	-0.3\\
1244	-0.3\\
1245	-0.3\\
1246	-0.3\\
1247	-0.3\\
1248	-0.3\\
1249	-0.3\\
1250	-0.3\\
1251	-0.3\\
1252	-0.3\\
1253	-0.3\\
1254	-0.3\\
1255	-0.3\\
1256	-0.3\\
1257	-0.3\\
1258	-0.3\\
1259	-0.3\\
1260	-0.3\\
1261	-0.3\\
1262	-0.3\\
1263	-0.3\\
1264	-0.3\\
1265	-0.3\\
1266	-0.3\\
1267	-0.3\\
1268	-0.3\\
1269	-0.3\\
1270	-0.3\\
1271	-0.3\\
1272	-0.3\\
1273	-0.3\\
1274	-0.3\\
1275	-0.3\\
1276	-0.3\\
1277	-0.3\\
1278	-0.3\\
1279	-0.3\\
1280	-0.3\\
1281	-0.3\\
1282	-0.3\\
1283	-0.3\\
1284	-0.3\\
1285	-0.3\\
1286	-0.3\\
1287	-0.3\\
1288	-0.3\\
1289	-0.3\\
1290	-0.3\\
1291	-0.3\\
1292	-0.3\\
1293	-0.3\\
1294	-0.3\\
1295	-0.3\\
1296	-0.3\\
1297	-0.3\\
1298	-0.3\\
1299	-0.3\\
1300	-0.3\\
1301	-0.3\\
1302	-0.3\\
1303	-0.3\\
1304	-0.3\\
1305	-0.3\\
1306	-0.3\\
1307	-0.3\\
1308	-0.3\\
1309	-0.3\\
1310	-0.3\\
1311	-0.3\\
1312	-0.3\\
1313	-0.3\\
1314	-0.3\\
1315	-0.3\\
1316	-0.3\\
1317	-0.3\\
1318	-0.3\\
1319	-0.3\\
1320	-0.3\\
1321	-0.3\\
1322	-0.3\\
1323	-0.3\\
1324	-0.3\\
1325	-0.3\\
1326	-0.3\\
1327	-0.3\\
1328	-0.3\\
1329	-0.3\\
1330	-0.3\\
1331	-0.3\\
1332	-0.3\\
1333	-0.3\\
1334	-0.3\\
1335	-0.3\\
1336	-0.3\\
1337	-0.3\\
1338	-0.3\\
1339	-0.3\\
1340	-0.3\\
1341	-0.3\\
1342	-0.3\\
1343	-0.3\\
1344	-0.3\\
1345	-0.3\\
1346	-0.3\\
1347	-0.3\\
1348	-0.3\\
1349	-0.3\\
1350	-0.3\\
1351	-0.3\\
1352	-0.3\\
1353	-0.3\\
1354	-0.3\\
1355	-0.3\\
1356	-0.3\\
1357	-0.3\\
1358	-0.3\\
1359	-0.3\\
1360	-0.3\\
1361	-0.3\\
1362	-0.3\\
1363	-0.3\\
1364	-0.3\\
1365	-0.3\\
1366	-0.3\\
1367	-0.3\\
1368	-0.3\\
1369	-0.3\\
1370	-0.3\\
1371	-0.3\\
1372	-0.3\\
1373	-0.3\\
1374	-0.3\\
1375	-0.3\\
1376	-0.3\\
1377	-0.3\\
1378	-0.3\\
1379	-0.3\\
1380	-0.3\\
1381	-0.3\\
1382	-0.3\\
1383	-0.3\\
1384	-0.3\\
1385	-0.3\\
1386	-0.3\\
1387	-0.3\\
1388	-0.3\\
1389	-0.3\\
1390	-0.3\\
1391	-0.3\\
1392	-0.3\\
1393	-0.3\\
1394	-0.3\\
1395	-0.3\\
1396	-0.3\\
1397	-0.3\\
1398	-0.3\\
1399	-0.3\\
1400	0\\
1401	0\\
1402	0\\
1403	0\\
1404	0\\
1405	0\\
1406	0\\
1407	0\\
1408	0\\
1409	0\\
1410	0\\
1411	0\\
1412	0\\
1413	0\\
1414	0\\
1415	0\\
1416	0\\
1417	0\\
1418	0\\
1419	0\\
1420	0\\
1421	0\\
1422	0\\
1423	0\\
1424	0\\
1425	0\\
1426	0\\
1427	0\\
1428	0\\
1429	0\\
1430	0\\
1431	0\\
1432	0\\
1433	0\\
1434	0\\
1435	0\\
1436	0\\
1437	0\\
1438	0\\
1439	0\\
1440	0\\
1441	0\\
1442	0\\
1443	0\\
1444	0\\
1445	0\\
1446	0\\
1447	0\\
1448	0\\
1449	0\\
1450	0\\
1451	0\\
1452	0\\
1453	0\\
1454	0\\
1455	0\\
1456	0\\
1457	0\\
1458	0\\
1459	0\\
1460	0\\
1461	0\\
1462	0\\
1463	0\\
1464	0\\
1465	0\\
1466	0\\
1467	0\\
1468	0\\
1469	0\\
1470	0\\
1471	0\\
1472	0\\
1473	0\\
1474	0\\
1475	0\\
1476	0\\
1477	0\\
1478	0\\
1479	0\\
1480	0\\
1481	0\\
1482	0\\
1483	0\\
1484	0\\
1485	0\\
1486	0\\
1487	0\\
1488	0\\
1489	0\\
1490	0\\
1491	0\\
1492	0\\
1493	0\\
1494	0\\
1495	0\\
1496	0\\
1497	0\\
1498	0\\
1499	0\\
1500	0\\
1501	0\\
1502	0\\
1503	0\\
1504	0\\
1505	0\\
1506	0\\
1507	0\\
1508	0\\
1509	0\\
1510	0\\
1511	0\\
1512	0\\
1513	0\\
1514	0\\
1515	0\\
1516	0\\
1517	0\\
1518	0\\
1519	0\\
1520	0\\
1521	0\\
1522	0\\
1523	0\\
1524	0\\
1525	0\\
1526	0\\
1527	0\\
1528	0\\
1529	0\\
1530	0\\
1531	0\\
1532	0\\
1533	0\\
1534	0\\
1535	0\\
1536	0\\
1537	0\\
1538	0\\
1539	0\\
1540	0\\
1541	0\\
1542	0\\
1543	0\\
1544	0\\
1545	0\\
1546	0\\
1547	0\\
1548	0\\
1549	0\\
1550	0\\
1551	0\\
1552	0\\
1553	0\\
1554	0\\
1555	0\\
1556	0\\
1557	0\\
1558	0\\
1559	0\\
1560	0\\
1561	0\\
1562	0\\
1563	0\\
1564	0\\
1565	0\\
1566	0\\
1567	0\\
1568	0\\
1569	0\\
1570	0\\
1571	0\\
1572	0\\
1573	0\\
1574	0\\
1575	0\\
1576	0\\
1577	0\\
1578	0\\
1579	0\\
1580	0\\
1581	0\\
1582	0\\
1583	0\\
1584	0\\
1585	0\\
1586	0\\
1587	0\\
1588	0\\
1589	0\\
1590	0\\
1591	0\\
1592	0\\
1593	0\\
1594	0\\
1595	0\\
1596	0\\
1597	0\\
1598	0\\
1599	0\\
1600	0.1\\
1601	0.1\\
1602	0.1\\
1603	0.1\\
1604	0.1\\
1605	0.1\\
1606	0.1\\
1607	0.1\\
1608	0.1\\
1609	0.1\\
1610	0.1\\
1611	0.1\\
1612	0.1\\
1613	0.1\\
1614	0.1\\
1615	0.1\\
1616	0.1\\
1617	0.1\\
1618	0.1\\
1619	0.1\\
1620	0.1\\
1621	0.1\\
1622	0.1\\
1623	0.1\\
1624	0.1\\
1625	0.1\\
1626	0.1\\
1627	0.1\\
1628	0.1\\
1629	0.1\\
1630	0.1\\
1631	0.1\\
1632	0.1\\
1633	0.1\\
1634	0.1\\
1635	0.1\\
1636	0.1\\
1637	0.1\\
1638	0.1\\
1639	0.1\\
1640	0.1\\
1641	0.1\\
1642	0.1\\
1643	0.1\\
1644	0.1\\
1645	0.1\\
1646	0.1\\
1647	0.1\\
1648	0.1\\
1649	0.1\\
1650	0.1\\
1651	0.1\\
1652	0.1\\
1653	0.1\\
1654	0.1\\
1655	0.1\\
1656	0.1\\
1657	0.1\\
1658	0.1\\
1659	0.1\\
1660	0.1\\
1661	0.1\\
1662	0.1\\
1663	0.1\\
1664	0.1\\
1665	0.1\\
1666	0.1\\
1667	0.1\\
1668	0.1\\
1669	0.1\\
1670	0.1\\
1671	0.1\\
1672	0.1\\
1673	0.1\\
1674	0.1\\
1675	0.1\\
1676	0.1\\
1677	0.1\\
1678	0.1\\
1679	0.1\\
1680	0.1\\
1681	0.1\\
1682	0.1\\
1683	0.1\\
1684	0.1\\
1685	0.1\\
1686	0.1\\
1687	0.1\\
1688	0.1\\
1689	0.1\\
1690	0.1\\
1691	0.1\\
1692	0.1\\
1693	0.1\\
1694	0.1\\
1695	0.1\\
1696	0.1\\
1697	0.1\\
1698	0.1\\
1699	0.1\\
1700	0.1\\
1701	0.1\\
1702	0.1\\
1703	0.1\\
1704	0.1\\
1705	0.1\\
1706	0.1\\
1707	0.1\\
1708	0.1\\
1709	0.1\\
1710	0.1\\
1711	0.1\\
1712	0.1\\
1713	0.1\\
1714	0.1\\
1715	0.1\\
1716	0.1\\
1717	0.1\\
1718	0.1\\
1719	0.1\\
1720	0.1\\
1721	0.1\\
1722	0.1\\
1723	0.1\\
1724	0.1\\
1725	0.1\\
1726	0.1\\
1727	0.1\\
1728	0.1\\
1729	0.1\\
1730	0.1\\
1731	0.1\\
1732	0.1\\
1733	0.1\\
1734	0.1\\
1735	0.1\\
1736	0.1\\
1737	0.1\\
1738	0.1\\
1739	0.1\\
1740	0.1\\
1741	0.1\\
1742	0.1\\
1743	0.1\\
1744	0.1\\
1745	0.1\\
1746	0.1\\
1747	0.1\\
1748	0.1\\
1749	0.1\\
1750	0.1\\
1751	0.1\\
1752	0.1\\
1753	0.1\\
1754	0.1\\
1755	0.1\\
1756	0.1\\
1757	0.1\\
1758	0.1\\
1759	0.1\\
1760	0.1\\
1761	0.1\\
1762	0.1\\
1763	0.1\\
1764	0.1\\
1765	0.1\\
1766	0.1\\
1767	0.1\\
1768	0.1\\
1769	0.1\\
1770	0.1\\
1771	0.1\\
1772	0.1\\
1773	0.1\\
1774	0.1\\
1775	0.1\\
1776	0.1\\
1777	0.1\\
1778	0.1\\
1779	0.1\\
1780	0.1\\
1781	0.1\\
1782	0.1\\
1783	0.1\\
1784	0.1\\
1785	0.1\\
1786	0.1\\
1787	0.1\\
1788	0.1\\
1789	0.1\\
1790	0.1\\
1791	0.1\\
1792	0.1\\
1793	0.1\\
1794	0.1\\
1795	0.1\\
1796	0.1\\
1797	0.1\\
1798	0.1\\
1799	0.1\\
1800	0.1\\
};
\addlegendentry{$\text{Wartość zadana y}_{\text{zad}}$}

\end{axis}

\begin{axis}[%
width=4.521in,
height=1.493in,
at={(0.758in,0.481in)},
scale only axis,
xmin=1,
xmax=1800,
xlabel style={font=\color{white!15!black}},
xlabel={k},
ymin=-1,
ymax=0.5,
ylabel style={font=\color{white!15!black}},
ylabel={u},
axis background/.style={fill=white},
xmajorgrids,
ymajorgrids,
legend style={legend cell align=left, align=left, draw=white!15!black}
]
\addplot [color=mycolor1]
  table[row sep=crcr]{%
1	0\\
2	0\\
3	0\\
4	0\\
5	0\\
6	0\\
7	0\\
8	0\\
9	0\\
10	0\\
11	0\\
12	0\\
13	0\\
14	0\\
15	0\\
16	0\\
17	0\\
18	0\\
19	0\\
20	-0.42971\\
21	-0.83288\\
22	-1\\
23	-1\\
24	-1\\
25	-1\\
26	-1\\
27	-0.94049\\
28	-0.88125\\
29	-0.80814\\
30	-0.73973\\
31	-0.67124\\
32	-0.61185\\
33	-0.56107\\
34	-0.52342\\
35	-0.49875\\
36	-0.48836\\
37	-0.49092\\
38	-0.50509\\
39	-0.52788\\
40	-0.55606\\
41	-0.58583\\
42	-0.6138\\
43	-0.6371\\
44	-0.65392\\
45	-0.66343\\
46	-0.66574\\
47	-0.6617\\
48	-0.65266\\
49	-0.6402\\
50	-0.62596\\
51	-0.61145\\
52	-0.59798\\
53	-0.58656\\
54	-0.57789\\
55	-0.57231\\
56	-0.56987\\
57	-0.57032\\
58	-0.57319\\
59	-0.57784\\
60	-0.58356\\
61	-0.58963\\
62	-0.59539\\
63	-0.60033\\
64	-0.60409\\
65	-0.60646\\
66	-0.60743\\
67	-0.60711\\
68	-0.60572\\
69	-0.60356\\
70	-0.60093\\
71	-0.59815\\
72	-0.59549\\
73	-0.59319\\
74	-0.5914\\
75	-0.59021\\
76	-0.58964\\
77	-0.58965\\
78	-0.59015\\
79	-0.59101\\
80	-0.59211\\
81	-0.59329\\
82	-0.59444\\
83	-0.59545\\
84	-0.59624\\
85	-0.59678\\
86	-0.59704\\
87	-0.59705\\
88	-0.59683\\
89	-0.59646\\
90	-0.59597\\
91	-0.59544\\
92	-0.59493\\
93	-0.59447\\
94	-0.5941\\
95	-0.59385\\
96	-0.59372\\
97	-0.5937\\
98	-0.59378\\
99	-0.59393\\
100	-0.59413\\
101	-0.59436\\
102	-0.59459\\
103	-0.59479\\
104	-0.59496\\
105	-0.59507\\
106	-0.59513\\
107	-0.59515\\
108	-0.59512\\
109	-0.59505\\
110	-0.59496\\
111	-0.59486\\
112	-0.59476\\
113	-0.59467\\
114	-0.5946\\
115	-0.59454\\
116	-0.59451\\
117	-0.5945\\
118	-0.59451\\
119	-0.59454\\
120	-0.59458\\
121	-0.59462\\
122	-0.59467\\
123	-0.59471\\
124	-0.59474\\
125	-0.59476\\
126	-0.59478\\
127	-0.59478\\
128	-0.59478\\
129	-0.59477\\
130	-0.59475\\
131	-0.59473\\
132	-0.59471\\
133	-0.5947\\
134	-0.59468\\
135	-0.59467\\
136	-0.59466\\
137	-0.59466\\
138	-0.59466\\
139	-0.59467\\
140	-0.59467\\
141	-0.59468\\
142	-0.59469\\
143	-0.5947\\
144	-0.5947\\
145	-0.59471\\
146	-0.59471\\
147	-0.59471\\
148	-0.59471\\
149	-0.59471\\
150	-0.59471\\
151	-0.59471\\
152	-0.5947\\
153	-0.5947\\
154	-0.5947\\
155	-0.59469\\
156	-0.59469\\
157	-0.59469\\
158	-0.59469\\
159	-0.59469\\
160	-0.59469\\
161	-0.59469\\
162	-0.5947\\
163	-0.5947\\
164	-0.5947\\
165	-0.5947\\
166	-0.5947\\
167	-0.5947\\
168	-0.5947\\
169	-0.5947\\
170	-0.5947\\
171	-0.5947\\
172	-0.5947\\
173	-0.5947\\
174	-0.5947\\
175	-0.5947\\
176	-0.5947\\
177	-0.5947\\
178	-0.5947\\
179	-0.5947\\
180	-0.5947\\
181	-0.5947\\
182	-0.5947\\
183	-0.5947\\
184	-0.5947\\
185	-0.5947\\
186	-0.5947\\
187	-0.5947\\
188	-0.5947\\
189	-0.5947\\
190	-0.5947\\
191	-0.5947\\
192	-0.5947\\
193	-0.5947\\
194	-0.5947\\
195	-0.5947\\
196	-0.5947\\
197	-0.5947\\
198	-0.5947\\
199	-0.5947\\
200	-1\\
201	-1\\
202	-1\\
203	-0.98196\\
204	-1\\
205	-0.95238\\
206	-0.93105\\
207	-0.88249\\
208	-0.85749\\
209	-0.83254\\
210	-0.82629\\
211	-0.82671\\
212	-0.83949\\
213	-0.8578\\
214	-0.88226\\
215	-0.90843\\
216	-0.93522\\
217	-0.95976\\
218	-0.98095\\
219	-0.99738\\
220	-1\\
221	-1\\
222	-1\\
223	-0.99706\\
224	-0.99085\\
225	-0.98272\\
226	-0.97399\\
227	-0.96562\\
228	-0.95827\\
229	-0.95237\\
230	-0.94816\\
231	-0.94572\\
232	-0.94497\\
233	-0.94575\\
234	-0.94777\\
235	-0.9507\\
236	-0.95421\\
237	-0.95793\\
238	-0.96156\\
239	-0.96482\\
240	-0.96753\\
241	-0.96956\\
242	-0.97085\\
243	-0.9714\\
244	-0.97129\\
245	-0.97062\\
246	-0.96952\\
247	-0.96813\\
248	-0.96661\\
249	-0.96509\\
250	-0.96368\\
251	-0.96248\\
252	-0.96154\\
253	-0.9609\\
254	-0.96056\\
255	-0.96051\\
256	-0.96071\\
257	-0.96111\\
258	-0.96165\\
259	-0.96227\\
260	-0.9629\\
261	-0.96351\\
262	-0.96405\\
263	-0.96448\\
264	-0.96479\\
265	-0.96497\\
266	-0.96503\\
267	-0.96498\\
268	-0.96484\\
269	-0.96463\\
270	-0.96439\\
271	-0.96412\\
272	-0.96386\\
273	-0.96363\\
274	-0.96343\\
275	-0.96329\\
276	-0.96319\\
277	-0.96315\\
278	-0.96316\\
279	-0.9632\\
280	-0.96328\\
281	-0.96338\\
282	-0.96349\\
283	-0.96359\\
284	-0.9637\\
285	-0.96378\\
286	-0.96385\\
287	-0.9639\\
288	-0.96392\\
289	-0.96393\\
290	-0.96391\\
291	-0.96389\\
292	-0.96385\\
293	-0.9638\\
294	-0.96376\\
295	-0.96371\\
296	-0.96368\\
297	-0.96365\\
298	-0.96362\\
299	-0.96361\\
300	-0.9636\\
301	-0.96361\\
302	-0.96362\\
303	-0.96363\\
304	-0.96365\\
305	-0.96367\\
306	-0.96369\\
307	-0.9637\\
308	-0.96372\\
309	-0.96373\\
310	-0.96374\\
311	-0.96374\\
312	-0.96374\\
313	-0.96373\\
314	-0.96373\\
315	-0.96372\\
316	-0.96371\\
317	-0.96371\\
318	-0.9637\\
319	-0.96369\\
320	-0.96369\\
321	-0.96369\\
322	-0.96368\\
323	-0.96368\\
324	-0.96368\\
325	-0.96369\\
326	-0.96369\\
327	-0.96369\\
328	-0.96369\\
329	-0.9637\\
330	-0.9637\\
331	-0.9637\\
332	-0.9637\\
333	-0.96371\\
334	-0.96371\\
335	-0.96371\\
336	-0.96371\\
337	-0.9637\\
338	-0.9637\\
339	-0.9637\\
340	-0.9637\\
341	-0.9637\\
342	-0.9637\\
343	-0.9637\\
344	-0.9637\\
345	-0.9637\\
346	-0.9637\\
347	-0.9637\\
348	-0.9637\\
349	-0.9637\\
350	-0.9637\\
351	-0.9637\\
352	-0.9637\\
353	-0.9637\\
354	-0.9637\\
355	-0.9637\\
356	-0.9637\\
357	-0.9637\\
358	-0.9637\\
359	-0.9637\\
360	-0.9637\\
361	-0.9637\\
362	-0.9637\\
363	-0.9637\\
364	-0.9637\\
365	-0.9637\\
366	-0.9637\\
367	-0.9637\\
368	-0.9637\\
369	-0.9637\\
370	-0.9637\\
371	-0.9637\\
372	-0.9637\\
373	-0.9637\\
374	-0.9637\\
375	-0.9637\\
376	-0.9637\\
377	-0.9637\\
378	-0.9637\\
379	-0.9637\\
380	-0.9637\\
381	-0.9637\\
382	-0.9637\\
383	-0.9637\\
384	-0.9637\\
385	-0.9637\\
386	-0.9637\\
387	-0.9637\\
388	-0.9637\\
389	-0.9637\\
390	-0.9637\\
391	-0.9637\\
392	-0.9637\\
393	-0.9637\\
394	-0.9637\\
395	-0.9637\\
396	-0.9637\\
397	-0.9637\\
398	-0.9637\\
399	-0.9637\\
400	-0.65005\\
401	-0.63908\\
402	-0.52044\\
403	-0.53126\\
404	-0.50445\\
405	-0.54156\\
406	-0.56952\\
407	-0.62906\\
408	-0.68341\\
409	-0.74573\\
410	-0.79679\\
411	-0.8405\\
412	-0.86872\\
413	-0.8843\\
414	-0.88587\\
415	-0.87689\\
416	-0.85907\\
417	-0.83586\\
418	-0.80968\\
419	-0.78326\\
420	-0.75858\\
421	-0.73738\\
422	-0.72074\\
423	-0.70937\\
424	-0.70343\\
425	-0.70267\\
426	-0.70645\\
427	-0.71385\\
428	-0.72372\\
429	-0.73488\\
430	-0.74615\\
431	-0.75652\\
432	-0.7652\\
433	-0.77166\\
434	-0.77566\\
435	-0.77722\\
436	-0.77656\\
437	-0.77406\\
438	-0.77022\\
439	-0.76555\\
440	-0.76057\\
441	-0.75574\\
442	-0.75144\\
443	-0.74795\\
444	-0.74544\\
445	-0.74399\\
446	-0.74356\\
447	-0.74405\\
448	-0.74527\\
449	-0.74703\\
450	-0.74909\\
451	-0.75123\\
452	-0.75326\\
453	-0.75502\\
454	-0.7564\\
455	-0.75733\\
456	-0.7578\\
457	-0.75784\\
458	-0.75751\\
459	-0.75688\\
460	-0.75606\\
461	-0.75515\\
462	-0.75422\\
463	-0.75337\\
464	-0.75266\\
465	-0.75212\\
466	-0.75178\\
467	-0.75163\\
468	-0.75166\\
469	-0.75185\\
470	-0.75215\\
471	-0.75252\\
472	-0.75292\\
473	-0.75331\\
474	-0.75366\\
475	-0.75395\\
476	-0.75416\\
477	-0.75428\\
478	-0.75432\\
479	-0.75428\\
480	-0.75418\\
481	-0.75404\\
482	-0.75387\\
483	-0.7537\\
484	-0.75353\\
485	-0.75339\\
486	-0.75327\\
487	-0.7532\\
488	-0.75315\\
489	-0.75315\\
490	-0.75317\\
491	-0.75322\\
492	-0.75329\\
493	-0.75336\\
494	-0.75344\\
495	-0.75351\\
496	-0.75357\\
497	-0.75361\\
498	-0.75364\\
499	-0.75365\\
500	-0.75365\\
501	-0.75363\\
502	-0.75361\\
503	-0.75358\\
504	-0.75355\\
505	-0.75352\\
506	-0.75349\\
507	-0.75346\\
508	-0.75345\\
509	-0.75344\\
510	-0.75343\\
511	-0.75344\\
512	-0.75344\\
513	-0.75346\\
514	-0.75347\\
515	-0.75348\\
516	-0.7535\\
517	-0.75351\\
518	-0.75352\\
519	-0.75352\\
520	-0.75353\\
521	-0.75353\\
522	-0.75353\\
523	-0.75352\\
524	-0.75352\\
525	-0.75351\\
526	-0.7535\\
527	-0.7535\\
528	-0.75349\\
529	-0.75349\\
530	-0.75349\\
531	-0.75349\\
532	-0.75349\\
533	-0.75349\\
534	-0.75349\\
535	-0.75349\\
536	-0.7535\\
537	-0.7535\\
538	-0.7535\\
539	-0.7535\\
540	-0.7535\\
541	-0.7535\\
542	-0.7535\\
543	-0.7535\\
544	-0.7535\\
545	-0.7535\\
546	-0.7535\\
547	-0.7535\\
548	-0.7535\\
549	-0.7535\\
550	-0.7535\\
551	-0.7535\\
552	-0.7535\\
553	-0.7535\\
554	-0.7535\\
555	-0.7535\\
556	-0.7535\\
557	-0.7535\\
558	-0.7535\\
559	-0.7535\\
560	-0.7535\\
561	-0.7535\\
562	-0.7535\\
563	-0.7535\\
564	-0.7535\\
565	-0.7535\\
566	-0.7535\\
567	-0.7535\\
568	-0.7535\\
569	-0.7535\\
570	-0.7535\\
571	-0.7535\\
572	-0.7535\\
573	-0.7535\\
574	-0.7535\\
575	-0.7535\\
576	-0.7535\\
577	-0.7535\\
578	-0.7535\\
579	-0.7535\\
580	-0.7535\\
581	-0.7535\\
582	-0.7535\\
583	-0.7535\\
584	-0.7535\\
585	-0.7535\\
586	-0.7535\\
587	-0.7535\\
588	-0.7535\\
589	-0.7535\\
590	-0.7535\\
591	-0.7535\\
592	-0.7535\\
593	-0.7535\\
594	-0.7535\\
595	-0.7535\\
596	-0.7535\\
597	-0.7535\\
598	-0.7535\\
599	-0.7535\\
600	-0.43985\\
601	-0.42888\\
602	-0.31024\\
603	-0.32106\\
604	-0.29425\\
605	-0.33319\\
606	-0.36292\\
607	-0.42451\\
608	-0.47838\\
609	-0.53722\\
610	-0.58093\\
611	-0.61427\\
612	-0.63018\\
613	-0.63311\\
614	-0.62304\\
615	-0.60459\\
616	-0.58017\\
617	-0.55362\\
618	-0.52735\\
619	-0.50389\\
620	-0.48476\\
621	-0.4711\\
622	-0.46331\\
623	-0.46129\\
624	-0.46436\\
625	-0.47149\\
626	-0.48133\\
627	-0.49243\\
628	-0.50341\\
629	-0.5131\\
630	-0.52067\\
631	-0.52563\\
632	-0.52787\\
633	-0.52758\\
634	-0.52518\\
635	-0.52125\\
636	-0.51642\\
637	-0.51129\\
638	-0.50641\\
639	-0.50223\\
640	-0.49903\\
641	-0.49697\\
642	-0.49608\\
643	-0.49626\\
644	-0.49732\\
645	-0.49901\\
646	-0.50106\\
647	-0.5032\\
648	-0.5052\\
649	-0.50686\\
650	-0.50808\\
651	-0.50879\\
652	-0.509\\
653	-0.50876\\
654	-0.50817\\
655	-0.50735\\
656	-0.5064\\
657	-0.50545\\
658	-0.50458\\
659	-0.50388\\
660	-0.50338\\
661	-0.50311\\
662	-0.50305\\
663	-0.50317\\
664	-0.50344\\
665	-0.50381\\
666	-0.50421\\
667	-0.50461\\
668	-0.50496\\
669	-0.50524\\
670	-0.50543\\
671	-0.50552\\
672	-0.50551\\
673	-0.50544\\
674	-0.5053\\
675	-0.50513\\
676	-0.50495\\
677	-0.50478\\
678	-0.50463\\
679	-0.50452\\
680	-0.50444\\
681	-0.50441\\
682	-0.50442\\
683	-0.50446\\
684	-0.50452\\
685	-0.50459\\
686	-0.50467\\
687	-0.50474\\
688	-0.5048\\
689	-0.50485\\
690	-0.50488\\
691	-0.50488\\
692	-0.50488\\
693	-0.50486\\
694	-0.50483\\
695	-0.50479\\
696	-0.50476\\
697	-0.50473\\
698	-0.5047\\
699	-0.50469\\
700	-0.50468\\
701	-0.50467\\
702	-0.50468\\
703	-0.50469\\
704	-0.5047\\
705	-0.50472\\
706	-0.50473\\
707	-0.50474\\
708	-0.50475\\
709	-0.50476\\
710	-0.50476\\
711	-0.50476\\
712	-0.50476\\
713	-0.50476\\
714	-0.50475\\
715	-0.50474\\
716	-0.50474\\
717	-0.50473\\
718	-0.50473\\
719	-0.50473\\
720	-0.50473\\
721	-0.50473\\
722	-0.50473\\
723	-0.50473\\
724	-0.50473\\
725	-0.50473\\
726	-0.50474\\
727	-0.50474\\
728	-0.50474\\
729	-0.50474\\
730	-0.50474\\
731	-0.50474\\
732	-0.50474\\
733	-0.50474\\
734	-0.50474\\
735	-0.50474\\
736	-0.50474\\
737	-0.50474\\
738	-0.50473\\
739	-0.50473\\
740	-0.50473\\
741	-0.50473\\
742	-0.50473\\
743	-0.50474\\
744	-0.50474\\
745	-0.50474\\
746	-0.50474\\
747	-0.50474\\
748	-0.50474\\
749	-0.50474\\
750	-0.50474\\
751	-0.50474\\
752	-0.50474\\
753	-0.50474\\
754	-0.50474\\
755	-0.50474\\
756	-0.50474\\
757	-0.50474\\
758	-0.50474\\
759	-0.50474\\
760	-0.50474\\
761	-0.50474\\
762	-0.50474\\
763	-0.50474\\
764	-0.50474\\
765	-0.50474\\
766	-0.50474\\
767	-0.50474\\
768	-0.50474\\
769	-0.50474\\
770	-0.50474\\
771	-0.50474\\
772	-0.50474\\
773	-0.50474\\
774	-0.50474\\
775	-0.50474\\
776	-0.50474\\
777	-0.50474\\
778	-0.50474\\
779	-0.50474\\
780	-0.50474\\
781	-0.50474\\
782	-0.50474\\
783	-0.50474\\
784	-0.50474\\
785	-0.50474\\
786	-0.50474\\
787	-0.50474\\
788	-0.50474\\
789	-0.50474\\
790	-0.50474\\
791	-0.50474\\
792	-0.50474\\
793	-0.50474\\
794	-0.50474\\
795	-0.50474\\
796	-0.50474\\
797	-0.50474\\
798	-0.50474\\
799	-0.50474\\
800	-0.92293\\
801	-0.93756\\
802	-1\\
803	-0.98557\\
804	-1\\
805	-0.95843\\
806	-0.93347\\
807	-0.88038\\
808	-0.84061\\
809	-0.79704\\
810	-0.76668\\
811	-0.74192\\
812	-0.72907\\
813	-0.7241\\
814	-0.72856\\
815	-0.73949\\
816	-0.75613\\
817	-0.77579\\
818	-0.79684\\
819	-0.81708\\
820	-0.83508\\
821	-0.84955\\
822	-0.85988\\
823	-0.8658\\
824	-0.86753\\
825	-0.86557\\
826	-0.86068\\
827	-0.8537\\
828	-0.84553\\
829	-0.837\\
830	-0.82885\\
831	-0.82167\\
832	-0.81589\\
833	-0.81179\\
834	-0.80945\\
835	-0.80881\\
836	-0.80968\\
837	-0.8118\\
838	-0.81479\\
839	-0.81831\\
840	-0.82198\\
841	-0.82549\\
842	-0.82856\\
843	-0.83101\\
844	-0.83272\\
845	-0.83366\\
846	-0.83387\\
847	-0.83344\\
848	-0.83249\\
849	-0.83118\\
850	-0.82966\\
851	-0.82809\\
852	-0.82661\\
853	-0.82532\\
854	-0.8243\\
855	-0.82359\\
856	-0.82321\\
857	-0.82314\\
858	-0.82334\\
859	-0.82375\\
860	-0.82432\\
861	-0.82497\\
862	-0.82564\\
863	-0.82627\\
864	-0.82682\\
865	-0.82725\\
866	-0.82754\\
867	-0.82769\\
868	-0.82771\\
869	-0.82762\\
870	-0.82743\\
871	-0.82718\\
872	-0.8269\\
873	-0.82662\\
874	-0.82635\\
875	-0.82612\\
876	-0.82594\\
877	-0.82582\\
878	-0.82576\\
879	-0.82576\\
880	-0.8258\\
881	-0.82588\\
882	-0.82599\\
883	-0.82611\\
884	-0.82623\\
885	-0.82634\\
886	-0.82644\\
887	-0.82652\\
888	-0.82657\\
889	-0.82659\\
890	-0.82659\\
891	-0.82657\\
892	-0.82653\\
893	-0.82649\\
894	-0.82644\\
895	-0.82638\\
896	-0.82634\\
897	-0.82629\\
898	-0.82626\\
899	-0.82624\\
900	-0.82623\\
901	-0.82623\\
902	-0.82624\\
903	-0.82626\\
904	-0.82628\\
905	-0.8263\\
906	-0.82632\\
907	-0.82634\\
908	-0.82636\\
909	-0.82637\\
910	-0.82638\\
911	-0.82639\\
912	-0.82639\\
913	-0.82638\\
914	-0.82637\\
915	-0.82637\\
916	-0.82636\\
917	-0.82635\\
918	-0.82634\\
919	-0.82633\\
920	-0.82633\\
921	-0.82632\\
922	-0.82632\\
923	-0.82632\\
924	-0.82632\\
925	-0.82633\\
926	-0.82633\\
927	-0.82633\\
928	-0.82634\\
929	-0.82634\\
930	-0.82634\\
931	-0.82635\\
932	-0.82635\\
933	-0.82635\\
934	-0.82635\\
935	-0.82635\\
936	-0.82635\\
937	-0.82634\\
938	-0.82634\\
939	-0.82634\\
940	-0.82634\\
941	-0.82634\\
942	-0.82634\\
943	-0.82634\\
944	-0.82634\\
945	-0.82634\\
946	-0.82634\\
947	-0.82634\\
948	-0.82634\\
949	-0.82634\\
950	-0.82634\\
951	-0.82634\\
952	-0.82634\\
953	-0.82634\\
954	-0.82634\\
955	-0.82634\\
956	-0.82634\\
957	-0.82634\\
958	-0.82634\\
959	-0.82634\\
960	-0.82634\\
961	-0.82634\\
962	-0.82634\\
963	-0.82634\\
964	-0.82634\\
965	-0.82634\\
966	-0.82634\\
967	-0.82634\\
968	-0.82634\\
969	-0.82634\\
970	-0.82634\\
971	-0.82634\\
972	-0.82634\\
973	-0.82634\\
974	-0.82634\\
975	-0.82634\\
976	-0.82634\\
977	-0.82634\\
978	-0.82634\\
979	-0.82634\\
980	-0.82634\\
981	-0.82634\\
982	-0.82634\\
983	-0.82634\\
984	-0.82634\\
985	-0.82634\\
986	-0.82634\\
987	-0.82634\\
988	-0.82634\\
989	-0.82634\\
990	-0.82634\\
991	-0.82634\\
992	-0.82634\\
993	-0.82634\\
994	-0.82634\\
995	-0.82634\\
996	-0.82634\\
997	-0.82634\\
998	-0.82634\\
999	-0.82634\\
1000	-0.34541\\
1001	-0.32859\\
1002	-0.14667\\
1003	-0.16327\\
1004	-0.12215\\
1005	-0.18695\\
1006	-0.23934\\
1007	-0.34594\\
1008	-0.43759\\
1009	-0.53298\\
1010	-0.59806\\
1011	-0.64213\\
1012	-0.65602\\
1013	-0.64937\\
1014	-0.6239\\
1015	-0.58803\\
1016	-0.54588\\
1017	-0.50333\\
1018	-0.46369\\
1019	-0.43029\\
1020	-0.40487\\
1021	-0.38867\\
1022	-0.38178\\
1023	-0.38361\\
1024	-0.39269\\
1025	-0.407\\
1026	-0.42413\\
1027	-0.44168\\
1028	-0.45751\\
1029	-0.47006\\
1030	-0.4784\\
1031	-0.48228\\
1032	-0.48198\\
1033	-0.47823\\
1034	-0.47199\\
1035	-0.46429\\
1036	-0.45615\\
1037	-0.44847\\
1038	-0.44192\\
1039	-0.43699\\
1040	-0.4339\\
1041	-0.43267\\
1042	-0.4331\\
1043	-0.43488\\
1044	-0.43759\\
1045	-0.44077\\
1046	-0.444\\
1047	-0.4469\\
1048	-0.44921\\
1049	-0.45077\\
1050	-0.45152\\
1051	-0.45152\\
1052	-0.45087\\
1053	-0.44976\\
1054	-0.44836\\
1055	-0.44688\\
1056	-0.44547\\
1057	-0.44427\\
1058	-0.44337\\
1059	-0.44281\\
1060	-0.44259\\
1061	-0.44268\\
1062	-0.44301\\
1063	-0.44351\\
1064	-0.4441\\
1065	-0.44469\\
1066	-0.44522\\
1067	-0.44564\\
1068	-0.44593\\
1069	-0.44607\\
1070	-0.44607\\
1071	-0.44596\\
1072	-0.44575\\
1073	-0.4455\\
1074	-0.44523\\
1075	-0.44498\\
1076	-0.44476\\
1077	-0.4446\\
1078	-0.4445\\
1079	-0.44446\\
1080	-0.44448\\
1081	-0.44454\\
1082	-0.44463\\
1083	-0.44474\\
1084	-0.44485\\
1085	-0.44494\\
1086	-0.44502\\
1087	-0.44507\\
1088	-0.4451\\
1089	-0.4451\\
1090	-0.44508\\
1091	-0.44504\\
1092	-0.445\\
1093	-0.44495\\
1094	-0.4449\\
1095	-0.44486\\
1096	-0.44483\\
1097	-0.44481\\
1098	-0.44481\\
1099	-0.44481\\
1100	-0.44482\\
1101	-0.44484\\
1102	-0.44486\\
1103	-0.44488\\
1104	-0.4449\\
1105	-0.44491\\
1106	-0.44492\\
1107	-0.44492\\
1108	-0.44492\\
1109	-0.44492\\
1110	-0.44491\\
1111	-0.44491\\
1112	-0.4449\\
1113	-0.44489\\
1114	-0.44488\\
1115	-0.44488\\
1116	-0.44487\\
1117	-0.44487\\
1118	-0.44487\\
1119	-0.44487\\
1120	-0.44488\\
1121	-0.44488\\
1122	-0.44488\\
1123	-0.44489\\
1124	-0.44489\\
1125	-0.44489\\
1126	-0.44489\\
1127	-0.44489\\
1128	-0.44489\\
1129	-0.44489\\
1130	-0.44489\\
1131	-0.44489\\
1132	-0.44489\\
1133	-0.44489\\
1134	-0.44488\\
1135	-0.44488\\
1136	-0.44488\\
1137	-0.44488\\
1138	-0.44488\\
1139	-0.44488\\
1140	-0.44489\\
1141	-0.44489\\
1142	-0.44489\\
1143	-0.44489\\
1144	-0.44489\\
1145	-0.44489\\
1146	-0.44489\\
1147	-0.44489\\
1148	-0.44489\\
1149	-0.44489\\
1150	-0.44489\\
1151	-0.44489\\
1152	-0.44489\\
1153	-0.44489\\
1154	-0.44489\\
1155	-0.44489\\
1156	-0.44489\\
1157	-0.44489\\
1158	-0.44489\\
1159	-0.44489\\
1160	-0.44489\\
1161	-0.44489\\
1162	-0.44489\\
1163	-0.44489\\
1164	-0.44489\\
1165	-0.44489\\
1166	-0.44489\\
1167	-0.44489\\
1168	-0.44489\\
1169	-0.44489\\
1170	-0.44489\\
1171	-0.44489\\
1172	-0.44489\\
1173	-0.44489\\
1174	-0.44489\\
1175	-0.44489\\
1176	-0.44489\\
1177	-0.44489\\
1178	-0.44489\\
1179	-0.44489\\
1180	-0.44489\\
1181	-0.44489\\
1182	-0.44489\\
1183	-0.44489\\
1184	-0.44489\\
1185	-0.44489\\
1186	-0.44489\\
1187	-0.44489\\
1188	-0.44489\\
1189	-0.44489\\
1190	-0.44489\\
1191	-0.44489\\
1192	-0.44489\\
1193	-0.44489\\
1194	-0.44489\\
1195	-0.44489\\
1196	-0.44489\\
1197	-0.44489\\
1198	-0.44489\\
1199	-0.44489\\
1200	-0.2567\\
1201	-0.25011\\
1202	-0.17893\\
1203	-0.18543\\
1204	-0.16933\\
1205	-0.1913\\
1206	-0.20575\\
1207	-0.23508\\
1208	-0.25676\\
1209	-0.27836\\
1210	-0.29016\\
1211	-0.29612\\
1212	-0.29376\\
1213	-0.28646\\
1214	-0.27488\\
1215	-0.26175\\
1216	-0.24835\\
1217	-0.23641\\
1218	-0.22671\\
1219	-0.21989\\
1220	-0.21594\\
1221	-0.21462\\
1222	-0.21531\\
1223	-0.21732\\
1224	-0.21987\\
1225	-0.22231\\
1226	-0.22413\\
1227	-0.22503\\
1228	-0.22493\\
1229	-0.22388\\
1230	-0.22062\\
1231	-0.217\\
1232	-0.21321\\
1233	-0.2094\\
1234	-0.20576\\
1235	-0.20244\\
1236	-0.19959\\
1237	-0.19726\\
1238	-0.19387\\
1239	-0.19108\\
1240	-0.1875\\
1241	-0.18344\\
1242	-0.18024\\
1243	-0.20646\\
1244	-0.20379\\
1245	-0.20236\\
1246	-0.2022\\
1247	-0.20332\\
1248	-0.20478\\
1249	-0.20579\\
1250	-0.20607\\
1251	-0.20561\\
1252	-0.20443\\
1253	-0.2026\\
1254	-0.20015\\
1255	-0.19716\\
1256	-0.19375\\
1257	-0.19006\\
1258	-0.18629\\
1259	-0.18265\\
1260	-0.17936\\
1261	-0.17665\\
1262	-0.17473\\
1263	-0.17379\\
1264	-0.17397\\
1265	-0.17535\\
1266	-0.17796\\
1267	-0.18176\\
1268	-0.18663\\
1269	-0.19239\\
1270	-0.19878\\
1271	-0.20552\\
1272	-0.21224\\
1273	-0.21858\\
1274	-0.22415\\
1275	-0.22856\\
1276	-0.23145\\
1277	-0.23252\\
1278	-0.23151\\
1279	-0.22825\\
1280	-0.26022\\
1281	-0.25402\\
1282	-0.24763\\
1283	-0.24197\\
1284	-0.23794\\
1285	-0.23483\\
1286	-0.23121\\
1287	-0.22738\\
1288	-0.2239\\
1289	-0.22103\\
1290	-0.2189\\
1291	-0.21746\\
1292	-0.21661\\
1293	-0.21617\\
1294	-0.21597\\
1295	-0.21582\\
1296	-0.21559\\
1297	-0.21521\\
1298	-0.21287\\
1299	-0.21051\\
1300	-0.20666\\
1301	-0.2029\\
1302	-0.19922\\
1303	-0.19447\\
1304	-0.19001\\
1305	-0.18598\\
1306	-0.18141\\
1307	-0.17758\\
1308	-0.20255\\
1309	-0.19915\\
1310	-0.19718\\
1311	-0.19673\\
1312	-0.19783\\
1313	-0.19958\\
1314	-0.2012\\
1315	-0.20241\\
1316	-0.20319\\
1317	-0.20351\\
1318	-0.20337\\
1319	-0.20274\\
1320	-0.2016\\
1321	-0.19997\\
1322	-0.19788\\
1323	-0.19541\\
1324	-0.19266\\
1325	-0.18977\\
1326	-0.18689\\
1327	-0.1842\\
1328	-0.18185\\
1329	-0.18004\\
1330	-0.1789\\
1331	-0.17858\\
1332	-0.17915\\
1333	-0.18067\\
1334	-0.18313\\
1335	-0.18648\\
1336	-0.1906\\
1337	-0.19532\\
1338	-0.20043\\
1339	-0.20569\\
1340	-0.2108\\
1341	-0.21546\\
1342	-0.21938\\
1343	-0.22227\\
1344	-0.22385\\
1345	-0.22391\\
1346	-0.22228\\
1347	-0.2189\\
1348	-0.24982\\
1349	-0.24451\\
1350	-0.2383\\
1351	-0.23267\\
1352	-0.22833\\
1353	-0.22299\\
1354	-0.22008\\
1355	-0.2169\\
1356	-0.21398\\
1357	-0.21163\\
1358	-0.21007\\
1359	-0.20924\\
1360	-0.20899\\
1361	-0.20912\\
1362	-0.20745\\
1363	-0.20598\\
1364	-0.20305\\
1365	-0.19898\\
1366	-0.19517\\
1367	-0.19059\\
1368	-0.18636\\
1369	-0.18256\\
1370	-0.20816\\
1371	-0.23893\\
1372	-0.23607\\
1373	-0.23442\\
1374	-0.2341\\
1375	-0.23422\\
1376	-0.23293\\
1377	-0.22908\\
1378	-0.2233\\
1379	-0.21651\\
1380	-0.20996\\
1381	-0.20486\\
1382	-0.20222\\
1383	-0.2006\\
1384	-0.19998\\
1385	-0.20025\\
1386	-0.19913\\
1387	-0.19677\\
1388	-0.19347\\
1389	-0.18962\\
1390	-0.1865\\
1391	-0.21363\\
1392	-0.21054\\
1393	-0.20827\\
1394	-0.20689\\
1395	-0.20645\\
1396	-0.20605\\
1397	-0.20495\\
1398	-0.20296\\
1399	-0.20017\\
1400	-0.13733\\
1401	-0.077356\\
1402	-0.018658\\
1403	0.03853\\
1404	0.09401\\
1405	0.14582\\
1406	0.19098\\
1407	0.22692\\
1408	0.25228\\
1409	0.26746\\
1410	0.27264\\
1411	0.26974\\
1412	0.26089\\
1413	0.2479\\
1414	0.22601\\
1415	0.1971\\
1416	0.16252\\
1417	0.12304\\
1418	0.084118\\
1419	0.045754\\
1420	0.0082127\\
1421	-0.027932\\
1422	-0.061757\\
1423	-0.088276\\
1424	-0.10761\\
1425	-0.12047\\
1426	-0.12819\\
1427	-0.13195\\
1428	-0.13107\\
1429	-0.12513\\
1430	-0.1139\\
1431	-0.097632\\
1432	-0.090317\\
1433	-0.067645\\
1434	-0.043176\\
1435	-0.01811\\
1436	0.0067446\\
1437	0.030444\\
1438	0.052441\\
1439	0.071708\\
1440	0.10223\\
1441	0.11392\\
1442	0.12105\\
1443	0.12396\\
1444	0.12316\\
1445	0.11913\\
1446	0.11232\\
1447	0.10333\\
1448	0.092758\\
1449	0.081145\\
1450	0.068936\\
1451	0.052856\\
1452	0.037051\\
1453	0.021799\\
1454	0.007335\\
1455	-0.0034809\\
1456	-0.013401\\
1457	-0.022204\\
1458	-0.029674\\
1459	-0.03562\\
1460	-0.039919\\
1461	-0.042535\\
1462	-0.043487\\
1463	-0.042832\\
1464	-0.040664\\
1465	-0.037132\\
1466	-0.032396\\
1467	-0.026662\\
1468	-0.020217\\
1469	-0.0133\\
1470	-0.0062406\\
1471	0.00067742\\
1472	0.0071871\\
1473	0.013024\\
1474	0.018031\\
1475	0.02207\\
1476	0.025042\\
1477	0.026922\\
1478	0.027748\\
1479	0.027586\\
1480	0.026526\\
1481	0.024687\\
1482	0.022204\\
1483	0.019219\\
1484	0.015876\\
1485	0.012314\\
1486	0.0086648\\
1487	0.0050493\\
1488	0.0015765\\
1489	-0.0016582\\
1490	-0.0045738\\
1491	-0.0071043\\
1492	-0.0091994\\
1493	-0.010824\\
1494	-0.01196\\
1495	-0.012605\\
1496	-0.012769\\
1497	-0.01248\\
1498	-0.011777\\
1499	-0.010711\\
1500	-0.0093405\\
1501	-0.0077355\\
1502	-0.0059675\\
1503	-0.0041099\\
1504	-0.0022347\\
1505	-0.00041003\\
1506	0.0013024\\
1507	0.0028493\\
1508	0.0041874\\
1509	0.0052843\\
1510	0.0061192\\
1511	0.0066839\\
1512	0.006978\\
1513	0.0070124\\
1514	0.0068062\\
1515	0.0063855\\
1516	0.0057815\\
1517	0.0050292\\
1518	0.0041657\\
1519	0.0032287\\
1520	0.0022555\\
1521	0.0012812\\
1522	0.00033858\\
1523	-0.00054345\\
1524	-0.00134\\
1525	-0.002031\\
1526	-0.0026011\\
1527	-0.0030401\\
1528	-0.0033428\\
1529	-0.0035089\\
1530	-0.0035427\\
1531	-0.0034525\\
1532	-0.0032503\\
1533	-0.0029509\\
1534	-0.0025716\\
1535	-0.0021312\\
1536	-0.0016492\\
1537	-0.0011452\\
1538	-0.0006382\\
1539	-0.000146\\
1540	0.00031546\\
1541	0.00073247\\
1542	0.0010938\\
1543	0.0013912\\
1544	0.0016191\\
1545	0.0017747\\
1546	0.0018581\\
1547	0.0018719\\
1548	0.0018208\\
1549	0.0017115\\
1550	0.001552\\
1551	0.0013516\\
1552	0.0011201\\
1553	0.0008679\\
1554	0.00060498\\
1555	0.00034112\\
1556	8.5356e-05\\
1557	-0.00015424\\
1558	-0.00037072\\
1559	-0.00055846\\
1560	-0.00071321\\
1561	-0.00083213\\
1562	-0.00091382\\
1563	-0.00095825\\
1564	-0.00096662\\
1565	-0.00094131\\
1566	-0.00088564\\
1567	-0.00080374\\
1568	-0.00070034\\
1569	-0.00058054\\
1570	-0.00044966\\
1571	-0.000313\\
1572	-0.00017566\\
1573	-4.2407e-05\\
1574	8.2495e-05\\
1575	0.00019537\\
1576	0.00029325\\
1577	0.00037387\\
1578	0.00043575\\
1579	0.00047816\\
1580	0.00050109\\
1581	0.0005052\\
1582	0.00049175\\
1583	0.00046249\\
1584	0.00041961\\
1585	0.00036559\\
1586	0.00030308\\
1587	0.00023487\\
1588	0.00016369\\
1589	9.2208e-05\\
1590	2.2874e-05\\
1591	-4.2102e-05\\
1592	-0.00010083\\
1593	-0.00015176\\
1594	-0.00019373\\
1595	-0.00022599\\
1596	-0.00024813\\
1597	-0.00026015\\
1598	-0.0002624\\
1599	-0.0002555\\
1600	0.021245\\
1601	0.041426\\
1602	0.061016\\
1603	0.080007\\
1604	0.098412\\
1605	0.11605\\
1606	0.13255\\
1607	0.14753\\
1608	0.16076\\
1609	0.17216\\
1610	0.18176\\
1611	0.1897\\
1612	0.19617\\
1613	0.20139\\
1614	0.20611\\
1615	0.21029\\
1616	0.21363\\
1617	0.21635\\
1618	0.21882\\
1619	0.22088\\
1620	0.22263\\
1621	0.2241\\
1622	0.22534\\
1623	0.22639\\
1624	0.22739\\
1625	0.22823\\
1626	0.22894\\
1627	0.22955\\
1628	0.23007\\
1629	0.23051\\
1630	0.23089\\
1631	0.23122\\
1632	0.23149\\
1633	0.23172\\
1634	0.23192\\
1635	0.23208\\
1636	0.23222\\
1637	0.23234\\
1638	0.23243\\
1639	0.23252\\
1640	0.23258\\
1641	0.23264\\
1642	0.23268\\
1643	0.23272\\
1644	0.23275\\
1645	0.23278\\
1646	0.2328\\
1647	0.23282\\
1648	0.23283\\
1649	0.23285\\
1650	0.23286\\
1651	0.23286\\
1652	0.23287\\
1653	0.23287\\
1654	0.23288\\
1655	0.23288\\
1656	0.23288\\
1657	0.23289\\
1658	0.23289\\
1659	0.23289\\
1660	0.23289\\
1661	0.23289\\
1662	0.23289\\
1663	0.23289\\
1664	0.23289\\
1665	0.23289\\
1666	0.23289\\
1667	0.23289\\
1668	0.23289\\
1669	0.23289\\
1670	0.23289\\
1671	0.23289\\
1672	0.23289\\
1673	0.23289\\
1674	0.23289\\
1675	0.23289\\
1676	0.23289\\
1677	0.23289\\
1678	0.23289\\
1679	0.23289\\
1680	0.23289\\
1681	0.23289\\
1682	0.23289\\
1683	0.23289\\
1684	0.23289\\
1685	0.23289\\
1686	0.23289\\
1687	0.23289\\
1688	0.23289\\
1689	0.23289\\
1690	0.23289\\
1691	0.23289\\
1692	0.23289\\
1693	0.23289\\
1694	0.23289\\
1695	0.23289\\
1696	0.23289\\
1697	0.23289\\
1698	0.23289\\
1699	0.23289\\
1700	0.23289\\
1701	0.23289\\
1702	0.23289\\
1703	0.23289\\
1704	0.23289\\
1705	0.23289\\
1706	0.23289\\
1707	0.23289\\
1708	0.23289\\
1709	0.23289\\
1710	0.23289\\
1711	0.23289\\
1712	0.23289\\
1713	0.23289\\
1714	0.23289\\
1715	0.23289\\
1716	0.23289\\
1717	0.23289\\
1718	0.23289\\
1719	0.23289\\
1720	0.23289\\
1721	0.23289\\
1722	0.23289\\
1723	0.23289\\
1724	0.23289\\
1725	0.23289\\
1726	0.23289\\
1727	0.23289\\
1728	0.23289\\
1729	0.23289\\
1730	0.23289\\
1731	0.23289\\
1732	0.23289\\
1733	0.23289\\
1734	0.23289\\
1735	0.23289\\
1736	0.23289\\
1737	0.23289\\
1738	0.23289\\
1739	0.23289\\
1740	0.23289\\
1741	0.23289\\
1742	0.23289\\
1743	0.23289\\
1744	0.23289\\
1745	0.23289\\
1746	0.23289\\
1747	0.23289\\
1748	0.23289\\
1749	0.23289\\
1750	0.23289\\
1751	0.23289\\
1752	0.23289\\
1753	0.23289\\
1754	0.23289\\
1755	0.23289\\
1756	0.23289\\
1757	0.23289\\
1758	0.23289\\
1759	0.23289\\
1760	0.23289\\
1761	0.23289\\
1762	0.23289\\
1763	0.23289\\
1764	0.23289\\
1765	0.23289\\
1766	0.23289\\
1767	0.23289\\
1768	0.23289\\
1769	0.23289\\
1770	0.23289\\
1771	0.23289\\
1772	0.23289\\
1773	0.23289\\
1774	0.23289\\
1775	0.23289\\
1776	0.23289\\
1777	0.23289\\
1778	0.23289\\
1779	0.23289\\
1780	0.23289\\
1781	0.23289\\
1782	0.23289\\
1783	0.23289\\
1784	0.23289\\
1785	0.23289\\
1786	0.23289\\
1787	0.23289\\
1788	0.23289\\
1789	0.23289\\
1790	0.23289\\
1791	0.23289\\
1792	0.23289\\
1793	0.23289\\
1794	0.23289\\
1795	0.23289\\
1796	0.23289\\
1797	0.23289\\
1798	0.23289\\
1799	0.23289\\
1800	0.23289\\
};
\addlegendentry{Sterowanie u}

\end{axis}
\end{tikzpicture}%
   \caption{Pięć regulatorów lokalnych DMC}
   \label{projekt:zad7:DMC:5:figure}
\end{figure}

\subsection{Wnioski}
Ostatnim krokiem w dostrajaniu regulatora rozmytego było
wyznaczenie współczynnika kary lambda dla wszystkich regulatorów
lokalnych, za pomocą którego można zapewnić kompromis pomiędzy
szybkością regulacji a postacią sygnału sterującego. Ponownie był on
wyznaczany metodą testowania. Spośród wszystkich regulatorów
najlepszym względem współczynnika jakości regulacji okazał się być
rozmyty regulator DMC o 4 regulatorach lokalnych oraz nastawach:


\newpage
