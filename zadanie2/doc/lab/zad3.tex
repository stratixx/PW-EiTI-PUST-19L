\section{Wyznaczenie aproksymowanych wektorów $s$ i $s_z$ }
\label{lab:zad3}

-------POLECENIE--------

Przygotowac odpowiedzi skokowe wykorzystywane w algorytmie DMC, tzn. zestaw
liczb s1, s2, . . . oraz sz
1, sz
2, . . .. Zamiescic rysunki odpowiedzi skokowych. Nalezy wykonac
aproksymacje odpowiedzi skokowych. W celu mozna wykorzystac dowolne narzedzie.
Zamiescic rysunek porównujacy odpowiedz skokowa oryginalna i wersje aproksymowana.
Opisac zastosowana metode (pozwalajac na odtworzenie procesu aproksymacji)
oraz uzasadnic wybór wszystkich parametrów z tym zwiazanych.

-------POLECENIE--------

tutaj aproksymacja jak w proj 1 tylko zmiana na sz1

Do labki aproksymacja jeszcze
Wykresy
T1 = 4.59 T2 = 100.4 K = 0.345 Td = 7

\subsection{Odpowiedzi skokowe}
\label{lab:zad3:odpSkok}

\subsection{Aproksymacja odpowiedzi skokowych}
\label{lab:zad3:approxOdpSkok}
