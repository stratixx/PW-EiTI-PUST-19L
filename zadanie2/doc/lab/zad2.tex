\section{Wyznaczenie odpowiedzi skokowej toru zakłócenie-wyjście}


-------POLECENIE--------

Wyznaczyc odpowiedzi skokowe toru zakłócenie-wyjscie procesu dla trzech róznych
zmian sygnału zakłócajacego Z rozpoczynajac z punktu pracy. Narysowac otrzymane
przebiegi na jednym rysunku. Czy własciwosci statyczne obiektu mozna okreslic jako
(w przyblizeniu) liniowe? Jezeli tak, okreslic wzmocnienie statyczne tego toru procesu.

-------POLECENIE--------

Odpowiedzi skokowe toru zakłócenie-wyjście zostały wyznaczone symulacyjnie dla pięciu
zmian sygnału zakłócenia .

\subsection{Odpowiedzi skokowe obiektu}

Do uzyskania odpowiedzi skokowych dla tego toru ustawiono sygnał wejściowy na stałą
wartość U=32, 
przy ustabilizowanej temperaturze T1=35.4 stp C przeprowadzone zostały
skoki sterowania z Z =0 na 10, 15 oraz 30.
Dzięki uzyskanym odpowiedziom skokowym otrzymano charakterystykę statyczną
zakłócenia.

Wniosek: 

Na podstawie wykresu charakterystyki statycznej można ustalić, 
że właściwości statyczne procesu są liniowe. 
Wzmocnienie statyczne procesu określone zostało na K=0,1046.

\subsection{Właściwości statyczne obiektu}

\subsection{Wzmocnienie statyczne}

