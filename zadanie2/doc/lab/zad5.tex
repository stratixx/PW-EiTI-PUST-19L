\section{Wpływ skokowej zmiany sygnału zakłócenia}

Dobrac parametr Dz. Załozyc, ze oprócz zmian sygnału wartosci zadanej nastepuje
skokowa zmiana sygnału zakłócenia z wartosci 0 do ok. 30 (zmiana ta ma miejsce
po osiagnieciu przez proces wartosci zadanej wyjscia). Uwzglednic co najmniej
dwie zmiany sygnału zakłócenia. Zamiescic wybrane wyniki eksperymentu. Pokazac,
ze uwzglednienie pomiaru zakłócenia prowadzi do lepszej regulacji niz gdy brak jest
tego pomiaru – porównac wyniki eksperymentu z regulatorem nie uwzgledniajacym
pomiaru zakłócen.

\subsection{Dobór parametru $D_z$}

\subsection{Regulacja bez uwzględnienia zakłócenia}

\subsection{Regulacja z uwzględnieniem zakłócenia}
