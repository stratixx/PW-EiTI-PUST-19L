\section{Wpływ skokowej zmiany sygnału zakłócenia}
\label{lab:zad5}

-------POLECENIE--------

Dobrac parametr Dz. Załozyc, ze oprócz zmian sygnału wartosci zadanej nastepuje
skokowa zmiana sygnału zakłócenia z wartosci 0 do ok. 30 (zmiana ta ma miejsce
po osiagnieciu przez proces wartosci zadanej wyjscia). Uwzglednic co najmniej
dwie zmiany sygnału zakłócenia. Zamiescic wybrane wyniki eksperymentu. Pokazac,
ze uwzglednienie pomiaru zakłócenia prowadzi do lepszej regulacji niz gdy brak jest
tego pomiaru – porównac wyniki eksperymentu z regulatorem nie uwzgledniajacym
pomiaru zakłócen.

-------POLECENIE--------


\subsection{Dobór parametru $D_z$}
\label{lab:zad4:Dz}

Parametr Dz jest to liczba próbek, dla której następuje stabilizacja odpowiedzi
skokowych toru zakłóceń, 
dobrano go na podstawie analizy odpowiedzi skoku zakłócenia z 0 na 30.

Dz wynosi =

\subsection{Regulacja bez uwzględnienia zakłócenia}
\label{lab:zad4:regulacjaBezUwzgZ}

Po osiągnięciu przez proces wartości zadanej wyjścia następuje zmiana sygnału
zakłócenia z wartości 0 na 15 oraz na 30.

\subsection{Regulacja z uwzględnieniem zakłócenia}
\label{lab:zad4:regulacjaUwzgZ}

Po osiągnięciu przez proces wartości zadanej wyjścia następuje zmiana sygnału
zakłócenia z wartości 0 na 15 oraz na 30.

\subsection{Porównanie wskaźnika jakości}
\label{lab:zad4:porownanieWskaznika}

Dla symulacji regulowanego obiektu bez pomiaru zakłóceń wynosi on:

Dla symulacji regulowanego obiektu z pomiarem zakłóceń wynosi on:

Wnioski: 

Uwzględnienie mierzalnego zakłócenia w algorytmie regulacji jest bardzo
dobrym rozwiązaniem, 
odsprzęganie zakłócenia powoduje kompensację uchybu regulacji, 
a co za tym idzie wskaźnik jakości jest lepszy, 
a sama regulacja uznana jest za lepszą.
