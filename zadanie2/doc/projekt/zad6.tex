\section{Wpływ ciagłej sinusoidalnej zmiany sygnału zakłócenia}
\label{projekt:zad6}

-------POLECENIE--------

Sprawdzic działanie algorytmu przy zakłóceniu zmiennym sinusoidalnie. Zamiescic wybrane
wyniki symulacji przy uwzglednieniu i nie uwzglednieniu mierzonego zakłócenia
w algorytmie.

-------POLECENIE--------


\subsection{Regulacja bez uwzględnienia zakłócenia}
\label{projekt:zad6:regulacjaBezUwzg}

\subsection{Regulacja z uwzględnieniem zakłócenia}
\label{projekt:zad6:regulacjaZUwzg}

Wnioski: 

Zakłócenie sinusoidalne wpłynęło bardzo negatywnie na układ, 
w którym nie jest uwzględniona kompensacja zakłócenia co spowodowało, 
że wskaźnik jakości wynosi: …. . 

Uwzględniając mierzone zakłócenie w algorytmie udało się skompensować uchyb co poprawiło wskaźnik jakości do … :

Zakłócenie zmienne sinusoidalne jest trudniejsze do kompensacji niż zwykły skok zakłócenia, 
ponieważ wymaga ono innego modelu zakłóceń, którego nie zastosowano.
