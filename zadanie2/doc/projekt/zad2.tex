\section{Wyznaczenie symulacyjne odpowiedzi skokowych}

-------POLECENIE--------

Wyznaczyc symulacyjnie odpowiedzi skokowe torów wejscie-wyjscie i zakłócenie-wyjscie
procesu dla kilku zmian sygnału sterujacego. Narysowac te odpowiedzi, oddzielnie dla
obydwu torów. Narysowac charakterystyke statyczna procesu y(u, z). Czy własciwosci
statyczne i dynamiczne procesu sa (w przyblizeniu) liniowe? Jezeli tak, okreslic
wzmocnienie statyczne obu torów procesu.

-------POLECENIE--------


Odpowiedzi skokowe torów wejście-wyjście i zakłócenie-wyjście zostały wyznaczone
symulacyjnie dla pięciu zmian sygnału sterującego oraz pięciu zmian zakłócenia .

\subsection{Odpowiedź wyjścia na skok wejścia}

Do uzyskania odpowiedzi skokowych dla tego toru ustawiono zakłócenie na stałą
wartość Z=0 oraz przeprowadzone zostały skoki sterowania z Upp =0 na …

\subsection{Odpowiedź wyjścia na skok zakłócenia}

Do uzyskania odpowiedzi skokowych dla tego toru ustawiono sygnał wejściowy na stałą
wartość U=0 oraz przeprowadzone zostały skoki sterowania z Zpp =0 na …

Dzięki uzyskanym odpowiedziom skokowym otrzymano charakterystykę statyczną y(u,z)

Wniosek: 

Na podstawie wykresu charakterystyki statycznej można ustalić, że właściwości
statyczne procesu są liniowe. Wzmocnienie statyczne procesu określone zostało dzieki …,
wynosi on K=1,0305.
