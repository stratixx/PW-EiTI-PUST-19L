\section{Wyznaczenie wektorów $s$ i $s_z$ }
\label{projekt:zad3}

-------POLECENIE--------

Wyznaczyc odpowiedzi skokowe obu torów wykorzystywane w algorytmie DMC, tzn.
zestaw liczb s1, s2, . . . oraz sz
1, sz
2, . . . (przy skoku jednostkowym, odpowiednio sygnału
sterujacego i zakłócajacego: od chwili k = 0 włacznie sygnał wymuszenia ma wartosc
1, w przeszłosci jest zerowy). Zamiescic rysunki odpowiedzi skokowych obu torów.

-------POLECENIE--------



\subsection{Wyznaczenie wektora $s$}

Uzyskaną odpowiedź procesu na zmianę sygnału sterującego z punktu pracy Upp=32 na
Umax=55 przekształcono w następujący sposób: 
\begin{itemize}
    \item Ograniczono (przycięto) czas zmiany sterowania u oraz wyjścia y od chwili skoku do ustabilizowania, 
    \begin{itemize}
        \item subsec
    \end{itemize}
    \item Wykres sterowania u przesunięty został o wartość początkową Upp=? w dół, 
    \item Wykres wyjścia y przesunięty został o wartość początkową Ypp=? w dół, 
    \item Wykres sterowania u i wyjścia y podzielono przez delta u=23. 
\end{itemize}

Uzyskana odpowiedź skokowa daje nam zestaw liczb s1,s2. . . ,która wykorzystana będzie w algorytmie DMC.

\subsection{Wyznaczenie wektora $s_z$}

Uzyskaną odpowiedź procesu na zmianę sygnału zakłócenia z punktu pracy Zpp=0 na
Zmax =30 przekształcono w następujący sposób: 
\begin{itemize}
\item Ograniczono (przycięto) czas zmiany sterowania u oraz wyjścia y od chwili skoku do ustabilizowania, 
\item Wykres sterowania u przesunięty został o wartość początkową Upp=? w dół, 
\item Wykres wyjścia y przesunięty został o wartość początkową Ypp=? w dół, 
\item Wykres sterowania u i wyjścia y podzielono przez delta z= 30.
\end{itemize}


Uzyskana odpowiedź skokowa toru zakłócenie wyjście daje nam zestaw liczb s1z,s2z. . .
,która wykorzystana będzie w algorytmie DMC.
