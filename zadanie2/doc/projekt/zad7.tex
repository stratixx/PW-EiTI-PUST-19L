\section{Odporność algorytmu przy błędach pomiarowych sygnału zakłócenia}

-------POLECENIE--------

Dla dobranych parametrów algorytmu zbadac jego odpornosc przy błedach pomiaru
sygnału zakłócenia (szum pomiarowy). Rozwazyc kilka wartosci błedów. Zamiescic
wybrane wyniki symulacji.

-------POLECENIE--------

Szum pomiarowy wygenerowano za pomocą dodania do wartości sygnału zakłócenia
dodajemy funkcję MATLAB’a normrnd(), 
gdzie jako parametry podajemy 0 oraz sigma.

Dzięki tej funkcji sygnał zakłócenia zmienia się zgodnie z rozkładem Gauss’a.
Poprzez zwiększanie parametru simga, zwiększamy zmiany sygnału zakłócenia, 
a co za tym idzie - większe zakłócenia. 
Rozważono trzy różne wartości zakłócenia:

Wnioski: 

Im większe zakłócenia tym jakość regulacji jest mniejsza.
