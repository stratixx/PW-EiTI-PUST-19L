\section{Wpływ skokowej zmiany sygnału zakłócenia}

-------POLECENIE--------

Załozyc, ze oprócz zmian sygnału wartosci zadanej nastepuje skokowa zmiana sygnału
zakłócenia z wartosci 0 do 1 (zmiana ta ma miejsce po osiagnieciu przez proces wartosci
zadanej wyjscia). Dobrac parametr Dz. Zamiescic wybrane wyniki symulacji. Pokazac,
ze pomiar zakłócenia i jego uwzglednienie prowadzi do lepszej regulacji niz gdy brak
jest tego pomiaru.

-------POLECENIE--------



\subsection{Dobór parametru $D_z$}

Parametr Dz jest to liczba próbek, dla której następuje stabilizacja odpowiedzi skokowych toru zakłóceń, 
dobrano go na podstawie analizy odpowiedzi skoku
zakłócenia z 0 na 1.
Dz wynosi = 75

\subsection{Regulacja bez uwzględnienia zakłócenia}

Po osiągnięciu przez proces wartości zadanej wyjścia następuje zmiana sygnału
zakłócenia z wartości 0 na 1.

\subsection{Regulacja z uwzględnieniem zakłócenia}

Po osiągnięciu przez proces wartości zadanej wyjścia następuje zmiana sygnału
zakłócenia z wartości 0 na 1.

\subsection{Porównanie wskaźnika jakości}

Dla symulacji regulowanego obiektu bez pomiaru zakłóceń wynosi on:

Dla symulacji regulowanego obiektu z pomiarem zakłóceń wynosi on:

Wnioski: 

Uwzględnienie mierzalnego zakłócenia w algorytmie regulacji jest bardzo dobrym rozwiązaniem, 
odsprzęganie zakłócenia powoduje kompensację uchybu regulacji, 
a co za tym idzie wskaźnik jakości jest lepszy, 
a sama regulacja uznana jest za lepszą.

